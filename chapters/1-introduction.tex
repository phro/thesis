\chapter{Differentiable Manifolds}
\label{ch:intro}

In mathematics, a differentiable manifold\footnote{also differential manifold} is a type of manifold that is locally similar enough to a linear space to allow one to do calculus. Any manifold can be described by a collection of charts, also known as an atlas. One may then apply ideas from calculus while working within the individual charts, since each chart lies within a linear space to which the usual rules of calculus apply.

\section{Topological Manifolds}

Here are two definitions.

\begin{definition}[Locally Euclidean Space]\label{def:locallyeuclidean}
A topological space $X$ is called \emph{locally Euclidean} if there is a non-negative integer $n$ such that every point in $X$ has a neighbourhood which is homeomorphic to $\mathbb{R}^n$
\end{definition}

\begin{definition}[Topological Manifold]\label{def:topman}
	A \emph{topological manifold} is a locally Euclidean Hausdorff space.
\end{definition}

\section{Differentiable Manifolds}

Now let's define something else.

\begin{definition}[Differentiable Manifold]\label{def:diffman}
	A \emph{differentiable manifold} is a topological manifold equipped with an equivalence class of atlases whose transition maps are all differentiable.
\end{definition}

\section{Main Theorem of This Thesis}

\begin{theorem}[Theorema Egregium]
	The Gaussian curvature of a surface is invariant under local isometry.
\end{theorem}

\begin{proof}
	This is immediate from \autoref{def:diffman}. Details are left as an exercise to the reader.
\end{proof}