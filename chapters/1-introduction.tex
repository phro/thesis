\chapter{Tensor Products and Monoidal Categories}
\label{ch:intro}

\section{Tensor product notation}
In what follows, we will extensively use tensor products, and more generally
monoidal categories. As all our examples will involve trivial associators, it
will be more convenient to label tensor factors with elements of a finite set
rather than by their position in a linear order. We will put the label as a
subscript on each factor. For a finite set $S = \set{i_1, \dots, i_n}$:
\begin{equation}\label{eq:tensor_notation}
        V_S \defeq
        V^{\otimes S} \defeq
        V_{i_1}\otimes V_{i_2}\otimes\dots\otimes V_{{i_n}} \iso V^{\otimes n}
\end{equation}
In particular, $V_{\emptyset} = \K$ is the base field of $V$.
Pure tensors in this notation will be written with subscripts and the $\otimes$
will be suppressed:
$(v_1)_{i_1}(v_2)_{i_2}\cdots(v_n)_{i_n} \in V^{\otimes\set{i_1,\dots,i_n}}$.
Observe that with labelled factors, the order the factors are written in does
not matter: In $V^{\otimes{1,2}}$, we have $x_1y_2 = y_2x_1$.

Next, given a map
$\map {T} {V^{\otimes\set{j_1,\dots,j_s}}} {V^{\otimes\set{i_1,\dots,i_r}}}$
between tensor powers of $V$, we denote $T$ by
$T^{j_1,\dots,j_s}_{i_1,\dots, i_r}$. It is important to note that the order of
the indices in this notation matters for nonsymmetric tensors.
Furthermore, this same symbol will denote extensions by the identity to any
arbitrary finite set $S$ disjoint from the $j_k$'s and the $i_k$'s:
\begin{equation}\label{eq:tensor_extend}
        \mapdef {T^{j_1,\dots,j_s}_{i_1,\dots, i_r}}
                {V^{\otimes\set{j_1,\dots,j_s}\sqcup S}}
                {V^{\otimes\set{i_1,\dots,i_r}\sqcup S}}
                {v_{j_1,\dots,j_s}\otimes w_S} 
                {(Tv_{j_1,\dots,j_s})\otimes w_S} 
\end{equation}

\begin{remark}
        There are three special cases with this notation:
        \begin{itemize}
                \item We denote (multi)linear functionals with only superscripts:
                        $\map {ϕ} {V^{\otimes S}} {\K \iso V^{\otimes
                        \emptyset}}$ will be written $ϕ^S$ (with a linear order
                        put on $S$).
                \item Elements $v\in V^{\otimes S}$ will be interpreted as a map
                        $\map {v} {\K} {V^{\otimes S}}$ written with a (linearly
                        ordered) subscript: $v_S$ to denote which factors the
                        element lives in. Following \cref{eq:tensor_extend},
                        this notation allows us to embed $v$ into higher tensor
                        powers: $v_S \in V^{\otimes S\sqcup X}$ for a $X$ finite
                        set.
                \item When only one index is present in a subscript or
                        superscript, and its omission does not introduce an
                        ambiguity in an expression, then it may be omitted to
                        improve readability.
                      
\end{itemize}
\end{remark}

\subsection{Notation extension to non-Cartesian monoidal categories}
\label{sec:monoidal_notation}

\ProvideDocumentCommand{\CC}{}{\msc C}
\ProvideDocumentCommand{\MM}{}{A}

Some of the monoidal categories we will work with are not Cartesian.
Additionally, the ones we will work with are best suited to their factors being
labelled by finite sets. For the purpose of representing them clearly, we
introduce a notation put forth by Bar-Natan: consider an object $\MM$ in a
monoidal category $\CC$. Then:
\begin{itemize}
        \item Monoidal powers of $\MM$ are indexed by finite sets $S$, and
                are to be written $\MM_S$.
        \item Homsets in $\CC$ between monoidal powers of $\MM$ are indexed by
                pairs of finite sets $D$, $C$. Morphisms in these homsets will
                be denoted by $\map{ϕ^D_C}{\MM_D}{\MM_C}$.
        \item Composition of morphisms $ϕ^{D_1}_{C_1}$ and
                $ψ^{D_2}_{C_2}$ is defined when $D_2 = C_1$, and is
                written with the following concatenation operator: 
                $\map{ϕ^{D_1}_{C_1}\then ψ^{D_2}_{C_2}}{\msc
                C_{D_1}}{\CC_{C_2}}$.
        \item The monoidal product
                $\map {\otimes} {\CC \times \CC}{\CC}$ satisfies
                $\MM_{S} \otimes \MM_{T} = \MM_{S \sqcup T}$. Given
                morphisms $ϕ^{D_1}_{C_1}$ and $ψ^{D_2}_{C_2}$ such that
                $D_1 \cap D_2 = \emptyset = C_1 \cap C_2$, we have a product
                morphism $\map {ϕ \otimes ψ} {\CC_{D_1 \sqcup D_2}}
                {\CC_{C_1\sqcup C_2}}$.\footnote{This encodes the
                        data of a strict monoidal category with the
                        linear order of factors replaces with indices
                        from a finite set.
                } We will write such products with concatenation:
                $ϕ^{D_1}_{C_1}ψ^{D_2}_{C_2}$.
\end{itemize}

\begin{remark}
        Given a morphism $ϕ^D_C$ and a finite set $S$, there is an extension of $ϕ^D_C$
        to a morphism
        $\map {\overline{ϕ}} {\msc C_{D\sqcup S}} {\msc C_{C \sqcup S}}$ given
        by $ϕ^D_C \id^S_S$. To make expressions easier to read, in this paper we
        will introduce this extension implicitly in the following context: given
        morphisms $ϕ^{D_1}_{C_1}$ and $ψ^{D_2}_{C_2}$ such that
        $D_2 \subseteq C_1$ and
        $C_2 \cap (C_1\setminus D_2)=\emptyset=D_1\cap(D_2 \setminus C_1)$, we
        define:
        \begin{equation}\label{eq:composition_extension}
                ϕ^{D_1}_{C_1} \then ψ^{D_2}_{C_2}
                \defeq 
                        \pn*{\Id{{D_2\setminus C_1}}ϕ^{D_1}_{C_1}} \then
                        \pn*{ψ^{D_2}_{C_2}\Id{C_1\setminus D_2}}
        \end{equation}
\end{remark}
The two extreme cases of this definition are:
\begin{itemize}
        \item When $C_1 \cap D_2 = \emptyset$, the bifunctoriality of $\otimes$ reduces
                \cref{eq:composition_extension} to the monoidal product
                $ϕ^{D_1}_{C_1}ψ^{D_2}_{C_2}$ (here written with concatenation).
        \item When $C_1 = D_2$, \cref{eq:composition_extension} becomes the
                composition $ϕ^{D_1}_{C_1}\then ψ^{D_2}_{C_2}$ exactly.
\end{itemize}

\begin{remark}
        While the $\then$ operator is associative, care must be taken that the
        compositions are well-defined in the presence of duplicated indices.
        While it is sufficient for all the finite sets in a composition to be
        pairwise disjoint, this condition will prove too restrictive for clear
        communication of formulae.
\end{remark}
%TODO: add a proof of associativity? (Likely not necessary.)

\section{Meta-objects}

The notion of an \enquote{-object} in a category (for example, a monoid object
or an algebra object) is a useful way of generalizing an algebraic structure to
a different mathematical context. For instance, in the category of topological
spaces, one is able to talk about groups whose operations are continuous.
However, this notion only makes sense when we are working with a Cartesian
category (or more generally a monoidal category).

One of the main characters of this story does not fit into this mould, so we
will introduce a generalization of this notion by way of example:

Consider a group object. Traditionally, the data of a group object are the
following:
\begin{itemize}
        \item An object $G$ in a category $\CC$.
        \item A morphism $\map {\mult} {G\times G} {G}$ called
                \enquote{multiplication}.
        \item A \enquote{unit} morphism
                $\map {\unit} {\set{*}} {G}$.\footnote{When $\CC = \Set$, we
                usually write the unit as an element $e=\unit(*)\in G$
        }
        \item An \enquote{inversion} morphism $\map {\antipode} {G} {G}$.
        \item A collection of relations between the morphisms, written as
                equalities of morphisms between Cartesian powers of $G$. For
                example, associativity may be written:
                \begin{equation}\label{eq:cd_assoc}
                \begin{tikzcd}
                        G\times G\times G
                                \rar["\mult \times \id"]
                                \dar["\id \times\mult"']
                        &G \times G
                                \dar["\mult"] \\
                        G \times G
                                \rar["\mult"']
                        &G
                \end{tikzcd}
                \end{equation}
\end{itemize}
Further, the data of these relations is extended to higher powers of $G$ by
acting on other components by the identity:
\begin{equation}
\begin{tikzcd}[column sep=large]
        G^{n+3}
                \rar["\mult \times \id^{n+1}"]
                \dar["\id \times\mult\times \id^{n}"']
        &G^{n+2}
                \dar["\mult\times \id^{n}"] \\
        G^{n+2}
                \rar["\mult\times \id^{n}"']
        &G^{n+1}
\end{tikzcd}
\end{equation}

Consider now two changes in how we package these data:
\begin{enumerate}
        \item Instead of linear orders of factors $G \times \dots \times G$, we
                will index factors by a finite set $S$, writing the power $G_S$
                in the style of \cref{eq:tensor_notation}.
        \item Instead of implicitly including extensions of morphisms to higher
                powers by the identity, we will parametrize the extension by
                finite sets. For example, multiplication $\map {\mult^{ij}_{k}}
                {G_{\set{i,j}}} {G_{\set{k}}}$\footnote{This notation is defined
                        in \cref{eq:tensor_notation}, and is here extended from
                        tensor products to Cartesian products.
                } will be viewed as a family of maps
                $\map {\mult^{ij}_k[S]} {G_{\set{i,j}\sqcup S}}
                {G_{\set{k}\sqcup S}}$, each of which must satisfy the relations
                of the group object.
\end{enumerate}
This way of packaging the data leads us to a direct generalization: a
\defi{meta-group}-object in $\CC$ is a collection of objects $G_S\in \CC$
indexed by finite sets $S$, together with 
\begin{itemize}
        \item A family of objects $G_S\in \CC$, indexed over finite sets $S$.
        \item A family of morphisms $\map {\mult^{ij}_{k}[S]} {G_{\set{i,j}\sqcup S}}
                {G_{\set{k}\sqcup S}}$ called \enquote{multiplication}.
        \item A family of \enquote{unit} morphisms
                $\map {\unit_{i}[S]} {G_S} {G_{\set{i}\sqcup S}}$.
        \item An family of \enquote{inversion} morphisms $\map {\antipode^{i}_{j}[S]}
                {G_{\set{i}\sqcup S}} {G_{\set{j}\sqcup S}}$.
        \item A collection of relations between the morphisms, written as
                equalities of morphisms between the $G_X$'s. For
                example, associativity may be written:
                \begin{equation}
                \begin{tikzcd}[column sep=huge]
                        G_{\set{1,2,3}\sqcup S}
                                \rar["\mult^{1,2}_{1}\bk*{S\sqcup\set{3}}"]
                                \dar["\mult^{2,3}_{2}\bk*{S\sqcup\set{1}}"']
                        &G_{\set{1,3}\sqcup S}
                                \dar["\mult^{1,3}_{1}\bk{S}"] \\
                        G_{\set{1,2}\sqcup S}
                                \rar["\mult^{1,2}_{1}\bk{S}"']
                        &G_{\set{1}\sqcup S}
                \end{tikzcd}
                \end{equation}
\end{itemize}
We obtain traditional objects by further requiring that each family of morphisms
$ϕ[S]$ satisfy $ϕ[S] = ϕ\bk{\emptyset} \times \id_S$. 

Several examples of well-known algebraic structures presented as meta-objects
are given in \cref{sec:alg_defs}.

\section{Algebraic definitions}\label{sec:alg_defs}

We now introduce the algebraic structures which will be used to define the
tangle invariant. These definitions follow those given by Majid in \cite{SM},
although the ones presented below are given in a way that their corresponding
meta-structure is readily visible.

\begin{definition}[algebra]
        A \defi{algebra} is an object $A\in\CC$ together with an associative
        multiplication $\map {\mult^{i,j}_{k}} {A_{\set{i,j}}}
        {A_{\set{k}}}$ (satisfying \cref{eq:cd_mult}), and a unit
        $\map{\unit_{i}}{A_{\emptyset}}{A_{\set{i}}}$ satisfying
        \cref{eq:cd_unit}.\footnote{
                When $\CC = \Vect$, this is becomes the more familiar definition
                of an \defi{algebra}. When $A_\emptyset$ is a field, and it is
                more common think of the unit as an element $\one\in A$. The
                unit map is then defined by linearly extending the assignment
                $\unit_{i}(1) = \one_{i}$.
        }
\end{definition}

\begin{multicols}{2}\noindent
\begin{equation}\label{eq:cd_mult}
\begin{tikzcd}
        A_{\set{1,2,3}}
                \rar["\mult^{1,2}_{1}"]
                \dar["\mult^{2,3}_{2}"']
        &A_{\set{1,3}}
                \dar["\mult^{1,3}_{1}"] \\
        A_{\set{1,2}}
                \rar["\mult^{1,2}_{1}"']
        &A_{\set{1}}
\end{tikzcd}
\end{equation}
\columnbreak
\begin{equation}\label{eq:cd_unit}
\begin{tikzcd}[column sep=large]
        A_{\set{1}}
                \rar["\unit_{2}"]
                \drar["\id"']
        &A_{\set{1,2}}
                \dar["\mult^{1,2}_{1}", shift left]
                \dar["\mult^{2,1}_{1}"', shift right] \\
        &A_{\set{1}}
\end{tikzcd}
\end{equation}
\end{multicols}

\begin{remark}
        From now on, we will denote repeated multiplication as in
        \cref{eq:cd_mult} by using extra indices. For instance:
        $\mult^{i,j, k}_{\ell} \defeq \mult^{i,j}_{r}\then\mult^{r, k}_{\ell}
        = \mult^{j, k}_{s}\then\mult^{i, s}_{\ell}$.
\end{remark}

There is also the dual notion of a \emph{coalgebra}, which arises by reversing
the arrows in \cref{eq:cd_mult,eq:cd_unit}:

\begin{definition}[coalgebra]
        A \defi{colagebra} is a vector space $C$ over a field $\K$ with a
        \defi{comultiplication} $\map {\comult} {C} {C\otimes C}$ which is
        \defi{coassociative} \eqref{eq:cd_comult} and a \defi{counit}, which is
        a map $\counit\colon A\to k$ satisfying \eqref{eq:cd_counit}.
\end{definition}
\nopagebreak
\begin{multicols}{2}\noindent
\begin{equation}\label{eq:cd_comult}
\begin{tikzcd}
        C_{\set{1,2,3}}
        &C_{\set{2,3}}
                \lar["\comult^{1}_{{1,2}}"'] \\
        C_{\set{1,2}}
                \uar["\comult^{2}_{2,3}"]
        &C_{\set{1}}
                \lar["\comult^{1}_{1,2}"]
                \uar["\comult^{1}_{1,3}"']
\end{tikzcd}
\end{equation}
\columnbreak
\begin{equation}\label{eq:cd_counit}
\begin{tikzcd}
        C_{\set{1}}
        &C_{\set{1,2}}
                \lar["\counit^{2}"']\\
        &C_{\set{1}}
                \ular["\id", shift left]
                \uar["\comult^{1}_{1,2}", shift left]
                \uar["\comult^{1}_{2,1}"', shift right] \\
\end{tikzcd}
\end{equation}
\end{multicols}

\begin{remark}
        From now on, we will denote repeated comultiplication as in
        \cref{eq:cd_comult} by using extra indices. For instance:
        $\comult^{i}_{j, k, \ell}
        \defeq \comult^{i}_{j,r}\then\comult^{r}_{k, \ell}
        = \comult^{i}_{s,\ell}\then\comult^{s}_{j, j}$.
\end{remark}

If a vector space $B$ satisfies both definitions of an algebra and a coalgebra,
we introduce a definition for when the structures are compatible with each other
in the following way:

\begin{definition}[bialgebra]
        A \defi{bialgebra} is an algebra $(B,\mult,\unit)$ and a coalgebra
        $(B,\comult,\counit)$, such that $\comult$ and $\counit$ are algebra
        morphisms.\footnote{
                $B^{\otimes n}$ inherits a (co)algebra structure
                from $B$, given by
                component-wise operations. For instance, in the case of
                multiplication, this
                means $\mult\pn[\big]{(a_1\otimes b_1) \otimes (a_2 \otimes
                b_2)} = a_1a_2 \otimes b_1 b_2$. The bialgebra structure on
                $B_{\emptyset}$ is given by
                $\mult = \unit = \comult = \counit = \id$.
        }
\end{definition}

\ProvideDocumentCommand{\lift}{mm}{\curryIsolated{#1}^{(#2)}}

\begin{multicols}{2}\noindent
\begin{equation}\label{eq:cd_mult_comult}
\begin{tikzcd}[column sep=large]
        B_{\set{1,3}}
                \rar["\mult^{1,3}_{1}"]
                \dar["\comult^{1}_{1,2}\then\comult^{3}_{3,4}"']
        &B_{\set{1}}
                \dar["\comult^{1}_{1,2}"] \\
        B_{\set{1,2,3,4}}
                \rar["\mult^{1,3}_{1}\then\mult^{2,4}_{2}"']
        &B_{\set{1,2}}
\end{tikzcd}
\end{equation}\begin{equation}\label{eq:cd_unit_comult}
\begin{tikzcd}[row sep=tiny]
        &B_{\set{1}}
                \ar[dd,"\comult^{1}_{1,2}"] \\
        B_{\emptyset}
                \urar["\unit_{1}"]
                \drar["\unit_{1}\then\unit_{2}"',near end]\\
        &B_{\set{1,2}}
\end{tikzcd}
\end{equation}
\columnbreak
\begin{equation}\label{eq:cd_mult_counit}
\begin{tikzcd}[column sep=tiny]
        B_{\set{1,2}}
                \ar[rr,"\mult^{1,2}_{1}"]
                \drar["\counit^{1}\then\counit^{2}"']
        &&B_{\set{1}}
                \dlar["\counit^{1}"] \\
        &B_{\emptyset}
\end{tikzcd}
\end{equation}
\begin{equation}\label{eq:cd_unit_counit}
\begin{tikzcd}
        B_{\emptyset}
                \rar["\unit_{1}"]
                \drar["\id"']
        &B_{\set{1}}
                \dar["\counit^{1}"] \\
        &B_{\emptyset}
\end{tikzcd}
\end{equation}
\end{multicols}

\begin{remark}
        The conditions for $\comult$ being an algebra morphism are presented in
        \cref{eq:cd_mult_comult,eq:cd_unit_comult}, while those for $\counit$
        are in \cref{eq:cd_mult_counit,eq:cd_unit_counit}.\footnote{While
        notation explicitly naming each tensor factor appears cumbersome in
        these diagrams, it will prove invaluable later when used on tangle
        diagrams, so we leave it as is for the sake of consistency.} Observing
        invariance under arrow reversal, it may not come as a surprise that
        \cref{eq:cd_mult_comult,eq:cd_mult_counit} also are the conditions for
        $\mult$ being a coalgebra morphism, and
        \cref{eq:cd_unit_comult,eq:cd_unit_counit} tell us that $\unit$ is as
        well.
\end{remark}

Finally, we introduce the invertibility condition we would expect on a quantum
group.
\begin{definition}[Hopf algebra]
A \defi{Hopf algebra} is a bialgebra $H$ together with a map $\map {\antipode} {H}
{H}$ called the \defi{antipode}, which satisfies for all $h\in H$,
$\comult^{1}_{1,2}\then \antipode^1_1 \then \mult^{1,2}_1 =
\counit^{1}\then\unit_{1} = 
\comult^{1}_{1,2}\then \antipode^2_2 \then \mult^{1,2}_1$.
As a commutative diagram, this looks like \cref{eq:cd_antipode}
\begin{equation}
\begin{tikzcd}[column sep=tiny]\label{eq:cd_antipode}
        H_{\set{1}}
                \arrow[rr, "\counit^{1}"] \arrow[rd, "\comult^{1}_{1,2}"']
        && H_{\emptyset}
                \arrow[rr, "\unit_{1}"]
        && H_{\set{1}} \\
        & H_{\set{1,2}}
                \arrow[rr, "\antipode^{2}_{2}", shift left]
                \arrow[rr, "\antipode^{1}_{1}"', shift right]
        && H_{\set{1,2}} \arrow[ru, "\mult^{1,2}_{1}"']
\end{tikzcd}
\end{equation}
\end{definition}

In order to do knot theory, we need an algebraic way to represent a crossing of
two strands. This is accomplished by the so-called $\Rmat$-matrix:

\begin{definition}[quasitriangular Hopf algebra]
A \defi{quasitriangular Hopf algebra} is a Hopf algebra $H$, together with an
invertible element $\Rmat \in H\otimes H$, called the \defi{$\Rmat$-matrix},
which satisfies the following properties: (we will denote the inverse by
$\Rmati$)
\begin{align}
        \label{eq:Rmat_overstrand}
        \Rmat_{12}\then\comult^{2}_{23}&=\Rmat_{a2}\Rmat_{b3}\then\mult^{ab}_1\\
        \label{eq:Rmat_understrand}
        \Rmat_{13}\then\comult^{1}_{12}&=\Rmat_{1b}\Rmat_{2a}\then\mult^{ab}_3\\
        \label{eq:Rmat_comult}
        \comult^{1}_{21} &= 
                \comult^{1}_{12} \Rmat_{1_i,2_i}\Rmati_{1_f,2_f}\then
                \mult^{1_i,1,1_f}_{1}\then \mult^{2_i,2,2_f}_{2}
\end{align}
\end{definition}

\begin{definition}[Drinfeld element]
        In a quasitriangular Hopf algebra $H$, the \defi{Drinfeld element},
        $\dfe \in H$ is given by:
        \begin{equation}
                \dfe \defeq \Rmat_{21}\then\antipode^2_2 \then \mult^{12}
        \end{equation}
\end{definition}

\begin{definition}[monodromy]
        The \defi{monodromy}
        $\monodromy_{12} \defeq
        \Rmat_{12}\Rmat_{34}\then\mult^{14}_{1}\then\mult^{23}_{2}$. It's
        inverse will be denoted
        $\invb\monodromy =
        \Rmati_{12}\Rmati_{34}\then\mult^{14}_{1}\then\mult^{23}_{2}$.
\end{definition}

\begin{lemma}
        The Drinfeld element $\dfe$ satisfies for all $h\in H$:
        \begin{align}
                \dfe_{1}h_2\dfe_{3} \then \mult^{1,2,3}
                &= h \then S \then S\\
                \dfe \then \comult_{12} 
                &= \dfe_1\dfe_2\invb\monodromy_{34}
                \then\mult^{13}_{1}\then\mult^{24}_2
        \end{align}
\end{lemma}
\begin{proof}
        See \cite{SM} or \cite{ES} %or an original reference?
        for more details on this standard result. Note that the proof does not
        rely on the additive structure of the Hopf algebra, which allows us to
        extend this result to the realm of meta-Hopf algebras.
        %TODO: finish
\end{proof}

\begin{definition}[ribbon Hopf algebra]
        A quasitriangular Hopf algebra $H$ is called \defi{ribbon} if it has an
        element $\ribbon\in \centre(H)$ such that:
        \begin{align}
                \ribbon_1\ribbon_2\then\mult^{12}
                &= \dfe_1 \dfe_2 \then \antipode^{2}_{2} \then \mult^{12}\\
                \ribbon_1 \then \comult^1_{12}
                &=      \ribbon_1\ribbon_2
                        \then\invb\monodromy_{34}
                        \then\mult^{13}_{1}
                        \then\mult^{24}_{2} \\
                \ribbon \then \antipode &= \ribbon\\
                \ribbon \then \counit &= \unit \then \counit = 1
        \end{align}
\end{definition}

\begin{definition}[distinguished grouplike element (spinner)]
        A \defi{distinguished grouplike element} (or \defi{spinner}) in a
        quasitriangular Hopf algebra $H$ is an invertible element $\spin\in H$
        (with inverse $\invb\spin$) such that for all $x\in H$:
        \begin{align}
                \label{eq:spinner_ribbon}
                \spin_1\ribbon_2\spin_3 \then \antipode^2_2 \then \mult^{123} &=
                \ribbon\\
                \label{eq:spinner_comult}
                \spin_1\then\comult^{1}_{12} &=\spin_1\spin_2\\
                \label{eq:spinner_antipode}
                \spin \then\antipode &= \invb\spin\\
                \label{eq:spinner_conjugate}
                \spin_{1}x_2\invb\spin_{3}\then\mult^{1,2,3} &=
                x \then \antipode \then \antipode\\
                \label{eq:spinner_counit}
                \spin \then \counit &= \unit \then \counit = 1
        \end{align}
\end{definition}

\begin{lemma}[spinners and ribbon Hopf algebras]\label{lem:spinner_ribbon}
        If a Hopf algebra has either a ribbon element $\ribbon$ or a spinner
        $\spin$, then it must have the other as well, given by the formula:
        $\spin_1 \ribbon_2 \then \mult^{12} = \dfe$.
\end{lemma}

Both the ribbon and the spinner element relevant have topological
interpretations in the context of tangles, which are outlined in
\cref{sec:topological_interpretations}.

% TODO:
% braided categories

% define graphical calculus

\begin{remark}
        By insisting that the ends of all strands in the diagram point up, and
        that only upward-pointing portions of strands participate in crossings,
        we may replace the (co)evaluation operations with the \enquote{spinner}
        element (also called the distinguished grouplike element), derived from
        the ribbon element and the Drinfeld element.
\end{remark}
In \cite{BV}, Bar-Natan and Van der Veen define an invariant of tangles valued
in tensor powers of a certain Hopf algebra $\CU$. Their work expresses the
algebra operations as perturbed Gaußian generating functions so as to produce a
strong polynomial-time tangle invariant.
