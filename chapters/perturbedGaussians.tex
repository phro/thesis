% Note that depending on your settings in the table of contents, subsections and subsubsections might appear virtually identical.
% Make sure to set the ToC depth and the numbering depth in the ToC the way you want.
\chapter{Perturbed Gaußians}\label{ch:perturbed_gaussians}
We now summarize the work of Bar-Natan and van der Veen in \cite{BV}, which
develops uses perturbed Gaußians to compute $Z$ quickly.

\section{Expressing morphisms as generating functions}

In order to be able to compute $Z$ efficiently, we need an efficient way to
reduce expressions in $\CU$ to a closed form. In \cite{BV}, Bar-Natan and van der Veen
achieve this by compactly encoding operations. They use generating functions,
whose definition we reproduce below:

For $A$ and $B$ finite sets, consider the set $\hom(\polyring{\Q}{z_A},
\polyring{\Q}{z_B})$ of linear maps between multivariate polynomial rings. Such
a map is determined by its values on the monomials $z_A^\nn$ for each
multi-index $\nn \in \N^A$.

\begin{definition}[Exponential generating function]
        Given a linear map $\map {Φ} {\polyring{\Q}{z_A}} {\polyring{\Q}{z_B}}$
        between polynomial spaces, its \defi{exponential generating function}
        is:
        \begin{equation}\label{eq:genfunc}
                \Gen{Φ} \defeq
                \Sum[\nn\in\N^A] \frac{Φ(z_A^\nn)}{\nn!}ζ_A^\nn
                \in \powerseries{\polyring{\Q}{z_B}}{ζ_A}
        \end{equation}
        We also extend the domain of $Φ$ from $\polyring{\Q}{z_B}$ to 
        $\powerseries{\polyring{\Q}{z_B}}{ζ_A}$ via extension of scalars
        $Φ \mapsto Φ\otimes \id_{\powerseries{\Q}{ζ_A}}$. This extension allows
        us to write the exponential generating function in a cleaner way:
        \begin{equation}
                \Gen{Φ}
                = Φ\pn*{\Sum[\nn\in\N^A] \frac{(z_Aζ_A)^\nn}{\nn!}}
                = Φ\pn*{\Gen[\big]{\id_{\polyring{\Q}{z_A}}}}
        \end{equation}
\end{definition}

By the PBW theorem, we know that $\CU$ is isomorphic as a vector space to the
polynomial ring $\polyring{\Q}{y, b, a, x}$ by choosing an ordering of the
generators (following \cite{BV}, we use $(\yo, \bo, \ao, \xo)$):
\begin{equation}\label{eq:order_map}
        \mapdef {\Order} {\polyring{\Q}{y, b, a, x}} [\toiso] {\CU}
        {y^{n_1}b^{n_2}a^{n_3}x^{n_4}} {\yo^{n_1}\bo^{n_2}\ao^{n_3}\xo^{n_4}}
\end{equation}

\ProvideDocumentCommand{\UMaps}{}{\mcl{U}}
\ProvideDocumentCommand{\LinMaps}{}{\mcl{H}}
\ProvideDocumentCommand{\GenFunc}{}{\mcl{C}}
To efficiently describe $\K$-linear maps between tensor powers of the algebra
$\CU$, we define categories $\UMaps$, $\LinMaps$, and $\GenFunc$ with objects
finite sets and morphisms:
\begin{align}
        \label{eq:hom_umaps}\Hom[\UMaps]JK &\defeq
                \Hom[\K]{\CU_{J}}{\CU_{K}}\\
        \label{eq:hom_linmaps}\Hom[\LinMaps]JK &\defeq
                \Hom[\K]{\polyring\Q{z_J}}{\polyring\Q{z_K}}\\
        \label{eq:hom_genfunc}\Hom[\GenFunc]JK &\defeq
                \powerseries{\polyring\Q{z_K}}{ζ_J}
\end{align}
Where $z_X$ is a shorthand for all the variables $y_i$, $b_i$, $a_i$, and $x_i$
for all $i\in X$. \Cref{eq:hom_umaps,eq:hom_linmaps} explicitly denote vector
space maps, not just algebra or ring homomorphisms. The maps
$\map {\Order} {\LinMaps} [\toiso] {\UMaps}$
and $\map {\GenMap} {\LinMaps} [\toiso] {\GenFunc}$ as introduced in
\cref{eq:order_map,eq:genfunc} induce linear isomorphisms between the
corresponding homsets, which extend to monoidal isomorphisms.

We use this formulation because of the existence of a computationally amenable
subcategory of $\GenFunc$ which contains the image of this invariant. Instead of
considering all power series in $\GenFunc$, we may instead work with a subset
which is much easier to compute with.

\subsection{Formulating composition in other categories}
Composing operations in $\UMaps$ or $\LinMaps$ is straightforward to define, but
lacks a closed form. However, on $\GenFunc$, the pullback of composition through
$\inv{\GenMap}$ takes the following form (quoted from \cite[Lemma~3]{BV}):

\begin{lemma}[Composition of generating functions]
Suppose $J$, $K$, $L$ are finite sets and
$ϕ ∈ \Hom{\polyring{\Q}{z_J}}{\polyring{\Q}{z_K}}$ and
$ψ ∈ \Hom{\polyring{\Q}{z_K}}{\polyring{\Q}{z_L}}$.
We have
\begin{equation}
        \Gen{ϕ\then ψ}
        = \restrict[\bigg]{
                \pn*{
                        \restrict*{\Gen{ϕ}}[z_K\mapsto ∂_{ζ_K}]
                        \Gen{ψ}
                }
        }[ζ_K=0]
\end{equation}
\end{lemma}

Since the above notation will occur several times, we will use the notion of
\defi{contraction} used by Bar-Natan and van der Veen (taken from
\cite[Definition~4]{BV}):

\begin{definition}[Contraction]\label{def:contraction}
        Let $f\in\powerseries{\K}{r, s}$ be a powerseries. The
        \defi{contraction} of $f = \Sum[k, l]{c_{k, l}r^{k}s^{l}}$ along the
        pair $(r, s)$ is:
        \begin{equation}
                \contraction{f}[(r, s)]
                \defeq \Sum[k] c_{k, k}k!
                = \restrict[\bigg]{\Sum[k, l]{c_{k, l} \partial_{s}^{k}s^{l}}}[s=0]
        \end{equation}
        Further, this notation is to be extended to allow for multiple
        consecutive contractions for $f\in \powerseries{\K}{r_i, s_i}_{i≤n}$:
        \begin{equation}
                \contraction{f}[\pn*{(r_i)_i, (s_i)_i}]
                \defeq
                \contraction*{
                        \contraction*{
                                \contraction{f}[(r_1, s_1)]
                        }[(r_2, s_2)]
                        \cdots
                }[(r_n, s_n)]
        \end{equation}
\end{definition}
Using this notation, we may write the extended notation of
\cref{eq:composition_extension} via
\begin{equation}
        \Gen{ϕ^{D_1}_{C_1} \then ψ^{D_2}_{C_2}} 
        = \contraction*{
        \Gen*{ϕ^{D_1}_{C_1}}\Gen*{ψ^{D_2}_{C_2}} 
        }[(z_{C_1\cap D_2},ζ_{C_1\cap D_2})]
\end{equation}
where $z_X$ and $ζ_X$ are shorthand for $y_i$, $b_i$, $a_i$, $x_i$ and $η_i$,
$β_i$, $α_i$, $ξ_i$ respectively.

It is important to note that contraction does not always define a convergent
expression. We will focus our attention on cases when convergence is
well-defined, and especially those where the computation is accessible.

The theorem we will rely heavily on in this thesis is the following, taken from
\cite[Theorem~6]{BV}:

\newcommand{\Wt}{\tilde{W}}
\begin{theorem}[Contraction theorem]
        For any $n\in \N$, consider the ring $R_n =
        \powerseries{\polyring{\Q}{r_j, g_j}}{s_j, W_{ij}, f_j
        \SetSymbol[\big] 1 ≤ i, j ≤ n}$. Then
        \begin{equation}
                \contraction{\Exp{gs + rf + rWs}}[r, s] = \det(\Wt)\Exp{g\Wt f}
        \end{equation}
        where $\Wt = \inv{(1-W)}$.
\end{theorem}

The main takeaway of this theorem is this: morphisms whose generating functions
are Gaußians have a clean formula for composition. Furthermore, this formula is
computationally reasonable, growing only polynomially in complexity with $n$.
This is contrasted with the conventional approach of choosing a representation
$V$ of $\CU$. When one considers morphisms between large tensor powers
$V^{\otimes n}$, the computational complexity is exponential in $n$.

\subsection{Expressing Hopf algebra operations as perturbed Gaußians}

Using this vector space isomorphism, \cite{BV} expresses all Hopf algebra
operations as power series in a closed form, namely as perturbed Gaußians.

\begin{theorem}[The meta-Hopf structure of $\CU$ is Gaußian]
        Each of the meta-Hopf algebra operations for $\CU$ as defined in
        \cref{sec:algebraic-definitions} all have the form of a perturbed
        Gaußian. That is, when the generators $(y, b, a, x)$ are assigned
        weights of $(1, 0, 2, 1)$ respectively, and their dual variables are
        assigned complementary weights so that
        $\weight z + \weight \dual z = 2$, we have the following expressions
        which are either Gaußian or central:
        \begin{align}
                \label{eq:gen-mult}
                \Gen*{\mult^{ij}_{k}} &=
                \begin{multlined}[t]
                \exp\biggl(
                (α_i + α_j)a_k +
                        (β_i + β_j + ξ_iη_j)b_k \\
                +\pn*{\frac{ξ_i}{\A_j} + ξ_j} x_k+
                \pn*{\frac{η_j}{\A_i}+ η_i }y_k
                \biggr)
                \end{multlined}
                \\ \Gen*{\unit_{i}} &=1\label{eq:gen-unit}
                \\\Gen*{\comult^{i}_{jk}} &=
                \begin{multlined}[t]
                \exp\biggl(
                        η_i \frac{b_j+b_k}{1-B_jB_k} \pn*{
                                B_k\frac{1-B_j}{b_j}y_j+
                                \frac{1-B_k}{b_k}y_k
                        } \\+
                        β_i(b_j + b_k) +
                        α_i(a_j + a_k)+
                        ξ_i(x_j + x_k)
                \biggr)
                \end{multlined}
                \label{eq:gen-comult}
                \\\Gen*{\counit^{i}} &=1\label{eq:gen-counit}
                \\\Gen*{\antipode^{i}_{i}} &=\exp\pn*
                {
                        - η_i\A_iy_i
                        - β_ib_i
                        + η_i\A_iξ_ib_i
                        - a_iα_i
                        - \A_iξ_ix_i
                }
                \label{eq:gen-antipode}
                \\\Gen*{\Rmat_{ij}} &=\exp\pn*{
                        a_j b_i + \frac{1-B_i}{b_i} y_i x_j
                }\label{eq:gen-Rmat}
                \\\Gen*{\spin_{i}} &=\sqrt{B_i}\label{eq:gen-spin}
                \\\Gen*{\ribbon_{i}} &=\sqrt{B_i}\exp\pn*{
                        a_i b_i + \frac{1-B_i}{b_i}x_iy_i
                }\label{eq:gen-ribbon}
        \end{align}
\end{theorem}

\begin{proof}
        To prove \cref{eq:gen-mult}, we use
        \cref{eq:ax_commutation,eq:ay_commutation,eq:weyl_relation}, which
        allows us to commute exponentials past each other to
        bring expressions into $ybax$-order. Below we omit the index $k$
        for readability:
        \begin{equation}
                \begin{aligned}
                        \Gen{\mult^{ij}}
                        &= \pn*{\inv{\Order}\circ\mult^{ij}\circ\Order}\pn{\Exp{
                                        η_iy_i + β_ib_i + α_ia_i + ξ_ix_i
                                }
                                \Exp{
                                        η_jy_j + β_jb_j + α_ja_j + ξ_jx_j
                                }
                        }\\
                        &= \inv{\Order}\pn*{
                                \Exp{η_i\yo}
                                \Exp{β_i\bo}
                                \Exp{α_i\ao}
                                \Exp{ξ_i\xo}
                                \Exp{η_j\yo}
                                \Exp{β_j\bo}
                                \Exp{α_j\ao}
                                \Exp{ξ_j\xo}
                        }\\
                        &= \inv{\Order}\pn*{
                                \Exp{η_i\yo}
                                \Exp{β_i\bo}
                                \Exp{α_i\ao}
                                \pn*{\Exp{-ξ_iη_j\bo}\Exp{η_j\yo}\Exp{ξ_i\xo}}
                                \Exp{β_j\bo}
                                \Exp{α_j\ao}
                                \Exp{ξ_j\xo}
                        }\\
                        &= \inv{\Order}\pn*{
                                \Exp{(β_j+ β_i-ξ_iη_j)\bo}
                                \Exp{η_i\yo}
                                \Exp{\frac{η_j}{\A_i}\yo}\Exp{α_i\ao}
                                \Exp{ξ_i\xo}
                                \Exp{α_j\ao}
                                \Exp{ξ_j\xo}
                        }\\
                        &= \inv{\Order}\pn*{
                                \Exp{(β_j+ β_i-ξ_iη_j)\bo}
                                \Exp{\pn{η_i+\frac{η_j}{\A_i}}\yo}\Exp{α_i\ao}
                                \Exp{α_j\ao}\Exp{\frac{ξ_i}{\A_j}\xo}
                                \Exp{ξ_j\xo}
                        }\\
                        &= \inv{\Order}\pn*{
                                \Exp{\pn*{η_i+\frac{η_j}{\A_i}}\yo}
                                \Exp{(β_j+ β_i-ξ_iη_j)\bo}
                                \Exp{(α_i+α_j)\ao}
                                \Exp{\pn*{\frac{ξ_i}{\A_j}+ξ_j}\xo}
                        }\\
                        &=
                        \Exp{\pn*{η_i+\frac{η_j}{\A_i}}\yco}
                        \Exp{(β_j+ β_i-ξ_iη_j)\bco}
                        \Exp{(α_i+α_j)\aco}
                        \Exp{\pn*{\frac{ξ_i}{\A_j}+ξ_j}\xco}
        \end{aligned}
\end{equation}
Since this expression is now written in the $\yo\bo\ao\xo$-order, we conclude
that the corresponding generating function is this same expression, but
written with commuting variables.

The other computation we must verify is the antipode, which follows
similarly:
\begin{equation}
        \begin{aligned}
                \Gen{\antipode}
                        &= \pn*{\inv{\Order}\circ\antipode\circ\Order}\pn{\Exp{
                                        ηy + βb + αa + ξx
                                }
                        }\\
                        &= \inv{\Order}\pn*{
                                \Exp{-ξ\xo}
                                \Exp{-α\ao}
                                \Exp{-β\bo}
                                \Exp{-η\yo}
                        }\\
                        &= \inv{\Order}\pn*{
                                \Exp{-ξ\xo}
                                \Exp{-\Aη\yo}
                                \Exp{-β\bo}
                                \Exp{-α\ao}
                        }\\
                        &= \inv{\Order}\pn*{
                                \Exp{ξ\Aη\bo}
                                \Exp{-\Aη\yo}
                                \Exp{-ξ\xo}
                                \Exp{-α\ao}
                                \Exp{-β\bo}
                        }\\
                        &= \inv{\Order}\pn*{
                                \Exp{-\Aη\yo}
                                \Exp{ξ\Aη\bo}
                                \Exp{-β\bo}
                                \Exp{-ξ\xo}
                                \Exp{-α\ao}
                        }\\
                        &= \inv{\Order}\pn*{
                                \Exp{-\Aη\yo}
                                \Exp{(η\A ξ-β)\bo}
                                \Exp{-α\ao}
                                \Exp{-ξ\A \xo}
                        }\\
                        &= \Exp{
                                -\Aη y
                                +(η\A ξ-β)b
                                +-αa
                                +-ξ\A x
                        }
\end{aligned}
\end{equation}
Finally, \cref{eq:gen-unit,eq:gen-counit,eq:gen-Rmat,eq:gen-spin,eq:gen-ribbon}
follow immediately.
\end{proof}

\subsection{Notational conventions}

The generating function of a tangle is not the entirety of this definition, for
the additional data is the domain and codomain of the corresponding morphism.

We will thereby write a morphism with domain $D$, codomain $C$, and generating
function $f(ζ_{D}, z_{C})$ as $f(ζ_{D}, z_{C})^{D}_{C}$.

When applied to knots, this invariant computes the Alexander polynomial $Δ$:
\begin{equation}
        Z(\knot31) &= \pn*{
                \frac{1}{B_1^{-1} + 1 + B_1^1}
        }^{\emptyset}_{\set{1}}
        = \inv{Δ(\knot31)}
\end{equation}
Here is a second example, where we define $D_i^n \defeq B_i^n + B_i^{-n}$ to
emphasize the palindromic nature of the Alexander polynomial:
\begin{equation}
        Z(\knot[a]{11}{10}) &= \pn*{
                \frac{1}
                {2D_1^{3}-11D_1^{2}+25D_1^{1}-31}
        }^{\emptyset}_{\set{1}}\\
        &= \inv{Δ(\knot[a]{11}{10})}
\end{equation}

Since each tangle is expressed as an object, the domains in these examples are
empty.
