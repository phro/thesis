\chapter{Conclusions}\label{ch:conclusions}

\subsection{Limitations of this definition}\label{sec:limitations}
For some inputs to the trace, expressions involving the Lambert $W$-function
appear, which complicates attempts to keep the invariant valued in either the
space of perturbed Gaußians, or a manageable extension thereof.

\section{Comparison with the multivariable Alexander polynomial}
\label{sec:compare_MVA}

Given that the long knot (i.e. one-component) case of this invariant
encodes the Alexander Polynomial, it was suspected that the invariant on long
links (i.e. multiple components, one of which is long) formed by adding the
trace would encode the \ac{MVA}. However, there are links which the \ac{MVA}
separates which this invariant does not.

Using the Thistlethwaite table of prime links \cite{knotatlas}, on all prime
two-component links with at most $11$ crossings (a collection of size $914$),
the trace map attains $878$ distinct values, while the MVA attains only $778$.
However, the two invariants are incomparable in terms of their strength.

The links $\link[a]51$ and $\link[a]{10}{95}$ are not distinguished by their
partial traces, with both returning a value of:

\begin{equation}
        \pn*{
                \pn*{
                        \frac{B_1}{B_1^2 t_2 -2 B_1 t_2 +B_1+t_2}
                }_{\pn*{\set{1},\set{2}}},
                \pn*{
                        \frac{B_2}{B_2^2 t_1 -2 B_2 t_1 +B_2+t_1}
                }_{\pn*{\set{2},\set{1}}}
}
\end{equation}

However, the values of these links under the \ac{MVA} are
$\frac{\left(B_1-1\right) \left(B_2-1\right)}{\sqrt{B_1} \sqrt{B_2}}
$ and $-\frac{\left(B_1-1\right) \left(B_2-1\right) \left(B_1+B_2-1\right)
        \left(B_2 B_1-B_1-B_2\right)}{B_1^{3/2} B_2^{3/2}}$ respectively.

In the other direction, there are also pairs of links in the same fibre of the
\ac{MVA} which this traced invariant can distinguish. In particular
$\link[a]{5}{1}$ and $\link[n]{7}{2}$ both have the same value under the
\ac{MVA}, namely
$\frac{\left(B_1-1\right) \left(B_2-1\right)}{\sqrt{B_1} \sqrt{B_2}}$.
The trace yields the values
$ \pn[\bigg]{
        \left(
                \frac{B_1}{B_1^2 t_2  -2 B_1 t_2  +B_1+t_2}
        \right)_{\pn{\set{1},\set{2}}},
        \left(
                \frac{B_2}{B_2^2 t_1  -2 B_2 t_1  +B_2+t_1}
        \right)_{(\set{2},\set{1})}
        }$
and
$\pn[\bigg]{
        \left(
                \frac{B_1}{B_1^2 t_2  -2 B_1 t_2  +B_1+t_2}
        \right)_{\pn{\set{1},\set{2}}},
        \left(
                \frac{B_2}{B_2^2 t_1  -2 B_2 t_1  +B_2^2-B_2+t_1 +1}
        \right)_{\pn{\set{2},\set{1}}}
}$
respectively.

This example also serves to highlight that the information provided by leaving
one strand open is not enough to recover the value of a different strand being
left open.

\section{Further work}
While all other Hopf algebra operations in $U$ are expressed by \cite{BV} as
perturbed Gaußians, the form in \cref{eq:trace_formula} does not to conform to
the same structure. Further work is needed to either implement this operation
into the established framework, or to suitably extend the framework (perhaps
with the use of Lambert $W$-functions).

Another potential approach is to search for a trace within the framework of
perturbed Gaußians themselves. The coinvariants map defined here are universal
with respect to the algebra $\CU$ (meaning that any other cyclic map factors
through this one). Instead, one can probe the existence of a cyclic map
$\trace_{\text{PG}}$ in the space of perturbed Gaußians which is universal--
that is, all other perturbed Gaußians which are cyclic factor through
$\trace_{\text{PG}}$.
