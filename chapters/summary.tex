\chapter{Executive Summary}
\label{ch:summary}

\section{Context}
\subsection{Understanding Knotted Objects}
In the field of knot theory, distinguishing between two knots or links
has proven to be a difficult task. It is a popular endeavour to describe and
compute strong invariants of knotted objects.
\begin{figure}[h]
        \centering
        \includegraphics[width=0.8\textwidth]{figures/sample.png}%todo: fix
        \caption{Two equivalent knots. Can you see how they are the same?}
        \label{fig:unknot}
\end{figure}

Merely being able to distinguish between two knotted objects does not always
provide us with enough information about these topological structures. For
instance, one may ask if a particular link is a \emph{satellite} of another
(roughly: where one knot is embedded into a link by following one of its
components), whether a knot is \emph{slice} (i.e. it is the boundary of a disk in
$\R^4$), or whether it is \emph{ribbon} (i.e. the boundary of a disk in $\R^3$
with restricted types of singularities). Many interesting properties of knots
can be phrased in terms of certain topological properties, such as strand
doubling (taking a strand and replacing it with two copies of itself, as in
\cref{fig:strand_double}) or strand stitching (joining two open components
together to form one longer one).
\begin{figure}[h]
        \centering
        \includegraphics[width=0.4\textwidth]{figures/sample.png}% todo: fix
        \caption{An example of strand-doubling.}
        \label{fig:strand_double}
\end{figure}

Open problems such as the Ribbon Slice Conjecture (asking whether there exists a
slice knot which is not ribbon) may be advanced by the development of
\enquote{topologically aware} invariants---those which preserve topological data
in a retrievable way.

\subsection{Quantum Invariants}
One such class of topological invariants is derived from quantum groups, which
are algebraic structures whose relations mimic those of knotted objects. With
this approach, one takes a knotted object and decomposes it into a series of
topological operations (such as stitching strands or doubling strands), then
maps each of these operations to a corresponding algebraic operation. The
composition of these algebraic operations is the value of the invariant.

More specifically, a quantum group (also called a \defi{Hopf algebra}) is a
vector space $A$ together with several maps between various tensor powers (for
instance, a multiplication map $\map {\mult} {A \otimes A} {A}$). The value of
the invariant is a vector in some tensor power of $A$, with one factor assigned
to each strand of the knotted object. It is computed by associating to each
concatenation of strands with a multiplication of algebra elements. While this
formulation is elegant, it has a notable drawback: computing the value of an
invariant with many components requires manipulating large tensor powers of $A$.
One remedy is to instead perform the computation in a representation of $A$, say
a small vector space $V$, though the issue of exponential growth in complexity
remains.

\subsection{Images of the invariant}
To avoid the issue of exponential computational complexity, one can instead
investigate the nature of the set of all values of the invariant as a subset of
the algebra and its tensor powers. For a particular choice of algebra (namely
$\CUlong$, as investigated by Dror Bar-Natan and Roland van der Veen in
\cref{BV}) the space of values the corresponding invariant $Z$ can take is
significantly smaller than the whole space; the dimension of the space of values
grows only quadratically with the number of crossings in the knotted object. In
particular, by looking at the generating functions of the associated maps,
instead of a generic power series, the value of $Z$ on tangles always take the
form of a (perturbed) Gaußian. Computationally, this means one needs only to
keep track of the quadratic form and the perturbation. The invariant $Z$---which
dominates the $\Sl_2$-coloured Jones polynomial---provides an efficient method
of computing the Alexander polynomial.

\section{Extending the invariant to more knotted objects}
The research program outlined by Bar-Natan and van der Veen computes $Z$
only for (pure) tangles--that is, collections of open strands whose endpoints
are fixed to a boundary circle. (Note that this includes long knots, which are
exactly the one-component tangles.) This thesis is focused on extending $Z$ and
its computations to tangles with closed components.
\begin{figure}[h]
        \centering
        \includegraphics[width=0.2\textwidth]{figures/sample.png}% todo: fix
        \caption{A pure tangle.}
        \label{fig:pure_tangle}
\end{figure}
\begin{figure}[h]
        \centering
        \includegraphics[width=0.2\textwidth]{figures/sample.png}% todo: fix
        \caption{A tangle with a closed component.}
        \label{fig:impure_tangle}
\end{figure}        

\subsection{Computing the extended map}
The first task is to determine the space in which the extended invariant, which
we will call $\ptr$, lives. One may observe that in a matrix algebra, one is
able to contract two matrices together via matrix multiplication. When one
wishes to contract a matrix along itself, one uses the trace map. Analogously,
since two strands in a tangle corresponds to multiplication, closing a strand
into a loop should correspond algebraically to a trace map.

In a generic algebra $A$, the trace map is defined as the projection onto the
set of coinvariants: $\map {\trace} {A} {A_A = A/\liebk{A}{A}}$. In order to
extend the invariant in this framework, we must first compute the space of
coinvariants for $\CUlong$, then compute the coinvariants map, and write it as a
generating function. (This is accomplished in \cref{sec:coinv_comp}.)

\subsection{Performing computations}
Unfortunately, the resulting trace map does not take the form of a perturbed
Gaußian in a way that readily connects to the existing framework. In order to
determine whether further study in this direction is merited, we must find an
alternative computation method to get a preliminary sense of the strength of
$\ptr$.

For a subclass of links (which includes all two-component links), we compute an
explicit closed form for the trace map, then implement a computer program to
compute the value of $\ptr$ on all two-component links with up to $11$
crossings. When applied to knots, $Z$ computes the Alexander
polynomial. When applied to two-component
tangles, one may expect that $\ptr$ would produce the natural generalization to
multiple components: the \ac{MVA}. Surprisingly, the \ac{MVA} and $\ptr$ are
incomparable, with each being able to distinguish pairs of links the other
cannot. (See \cref{sec:compare_MVA} for more information.)

\subsection{Further study}
As $\ptr$ does not generalize $Z$ in the manner expected, several interesting
avenues of further research are opened. Firstly, determining what the
relationship between $\ptr$ and the \ac{MVA} is remains open. Second is the
challenge of finding an efficient method for computing $\ptr$ on links with more
than two components, which currently is mired in complications with
the presence of non-elementary functions where quadratic forms normally appear.
Third is the question of the existence of other viable trace candidates. In
particular, it may be worth exploring whether a universal trace with respect to
the perturbed Gaußian framework defines a sufficiently useful invariant.
