\chapter{Tensor Products and Meta-Objects}
\label{ch:intro}

\section{Tensor product notation}
In what follows, we will extensively use tensor products, tensor powers, and
generalizations thereof. We begin by introducing the notation that will be used
first for traditional tensor products, then for their generalizations.

Let $V$ be a $\K$-vector space (for the moment assumed to be finite
dimensional). When working with a large tensor power $V^{\otimes n}$ of $V$, it
will often be more convenient to label tensor factors with elements of a finite
set $S$ (with $\ord S = n$) rather than by their position in a linear order.

For example, consider the vector $u \otimes v \otimes w \in V^{\otimes 3}$. Let
us choose an index set $S = \set{1,2,3}$. We then may equivalently write this
vector by labelling each tensor factor with one of the elements of $S$, say
$u_1v_2w_3$. Since the labels serve to distinguish the separate factors, this
vector may equivalently be written as $u_1v_2w_3 = v_2u_1w_3 = w_3u_2u_1 \in
V^{\otimes S}$. We will write the set $V^{\otimes S}$ with a subscript: $V_S$.
We formalize the idea below:

\begin{definition}[indexed tensor powers]
        Let $V$ be a vector space and $S = \set{s_1,…, s_n}$ be a finite
        set. We define the \defi{indexed tensor power} of $V$ to be the
        collection of formal linear combinations of functions from $S$ to $V$
        \begin{equation}\label{eq:indexed_tensor}
                V_S \defeq \Span\set{\map {f} {S} {V}}/\sim
        \end{equation}
        subject to the standard multilinear relations, namely multi-additivity
        and the factoring of scalars.

        By multi-additivity, we mean that for each $i\in S$ and $f, g \in V_S$ satisfying
        $f(i) = g(i) = v$, we have:
        \begin{equation}
                \label{eq:tensor_additive}
                f + g = \pn*{
                        s \mapsto \begin{cases}
                                v & \text{if $s = i$}\\
                                f(i) + g(i) & \text{otherwise}
                        \end{cases}
                }
        \end{equation}
\end{definition}

In practice, we will write such functions $\map {f} {S} {V}$ with $f(s_i) = v_i$
in the following way:
\begin{equation}
        \pn{v_{1}}_{s_1}
        \pn{v_{2}}_{s_2} ⋯
        \pn{v_{n}}_{s_n}
        \defeq f
\end{equation}
With this notation, we may easily express the factoring of scalars as:
\begin{equation}
        \label{eq:tensor_scalar}
        (v_1)_{s_1}(v_2)_{s_2} ⋯(λv_i)_{s_i} ⋯(v_n)_{s_n} =
        λ\cdot(v_1)_{s_1}(v_2)_{s_2} ⋯(v_n)_{s_n}
\end{equation}

Next, we introduce notation for maps between tensor powers so that we may
unambiguously refer to appropriate tensor factors while defining morphisms. Let
$D$ and $C$ be finite sets, and $\map {T} {V_{D}} {V_{C}}$. We will denote $T$
by $T^{D}_{C}$. It is important to note that when $T$ is not symmetric in its
arguments, the order of the indices in this notation matters.

\begin{example}
        Let $V = \R^2$, and $T^{a, b}_{c}$ defined by
        \begin{equation}
                \begin{aligned}
                        T^{a, b}_{c}\pn*{\vec v_a(\vec e_1)_b} &= \vec 0_c\\
                        T^{a, b}_{c}\pn*{\vec v_a(\vec e_2)_b} &= \vec v_c
                \end{aligned}
        \end{equation}
        This function zeros out vectors whose $b$-component is $\vec e_1$. If we
        wish to define an analogous function for the $a$-component, we may
        simply reverse the order of the superscript: $T^{b, a}_{c}$, which sends
        $\vec v_b(\vec e_1)_a = (\vec e_1)_a\vec v_b$ to $\vec 0_c$ and
        $\vec v_b(\vec e_2)_a$ to $\vec v_c$.
\end{example}

Finally, we point out that any morphism $T^D_C$ may be extended to one with
larger domain and codomain. We introduce the notation
$T^{D}_{C}[S] \defeq T^D_C \otimes \Id{S}$ for this concept, though will also
overload the notation $T^D_C$ for the same map: for any $v_D \in V_{D}$ and
$w_S\in V_{S}$, we may also write
$\pn*{T^{D}_{C}}\pn{v_{D}\otimes w_S} \defeq \pn*{T^{D}_{C}v_{D}}\otimes w_S$.

\begin{remark}
There are three special cases with this notation:
\begin{itemize}
        \item Given a (multi)linear functional
                $\map {ϕ} {V_{S}} {\K \iso V_{\emptyset}}$, we will write $ϕ^S$
                instead of $ϕ^S_{\emptyset}$. The linear order on $S$ remains in
                this notation.
        \item Elements $v\in V_{S}$ will be interpreted as a map
                $\map {v} {\K = V_{\emptyset}} {V_S}$ written $v_S$ instead of
                $v^{\emptyset}_{S}$.
        \item When only one index is present in a subscript or superscript, and
                its omission does not introduce an ambiguity in an expression,
                then it may be omitted to improve readability. For instance, a
                map $\map {ϕ} {V_{\set{1,2}}} {V_{\set{3}}}$ may
                be written as $ϕ^{1,2}$ instead of $ϕ^{1,2}_{3}$, with the
                canonical isomorphism $V\iso V_{\set{3}}$ being suppressed.
\end{itemize}
\end{remark}

When taking the tensor product of two such tensor powers, we follow \cite{BS}
and use the notation \enquote{$\sqcup$} instead of \enquote{$\otimes$}:
\begin{equation}
        V_{X} \sqcup V_{Y} \defeq V_{X \sqcup Y}
\end{equation}
Additionally, given $ϕ^{D_1}_{C_1}$ and $ψ^{D_2}_{C_2}$ such that $D_1 \cap D_2 = \emptyset = C_1
\cap C_2$, we have a product morphism
$\map {ϕ^{D_1}_{C_1}ψ^{D_2}_{C_2} \defeq ϕ \otimes ψ} {V_{D_1 \sqcup D_2}}
{V_{C_1\sqcup C_2}}$, which we also write with concatenation.

\section{Meta-objects}\label{sec:meta-objects}

\subsection{Notation extension beyond vector spaces}
\label{sec:monoidal_notation}

\ProvideDocumentCommand{\CC}{}{\msc C}
\ProvideDocumentCommand{\MM}{}{A}

While the above notation is helpful when working with vector spaces, we are
interested in also using the same notation to describe a tangle. Our
formulation of tangles (introduced in
\cref{sec:topological_interpretations}) is neither a tensor product nor a
monoidal category, though it shares many similarities with both concepts. In
particular, the domains and codomains of the maps we have discussed so far have
only depended on the index set. With this observation, we replace the notation of
tensor powers with that of a so-called meta-object:

\begin{definition}[Meta-object]
        Let $\CC$ be a category. A \defi{meta-object} in $\CC$ is subcategory
        with objects indexed by the functor
        \begin{equation}
                \mapdef {\MM} {\FinSet} {\Object(\CC)}
                              {S}       {\MM_S}
        \end{equation}
        The homsets of this subcategory are indexed by pairs of finite sets $D$,
        $C$. Morphisms in these homsets will be denoted by
        $\map{ϕ^D_C}{\MM_D}{\MM_C}$.
        For each such morphism $ϕ^{D}_{C}$, there is a finite-set-indexed
        morphism $\map {ϕ^{D}_{C}[\curry{}]} {\FinSet} {\HomOp(\CC)}$
        such that
        \begin{enumerate}
                \item $\map {ϕ[\curry{S}]}{\MM_{C\sqcup S}} {\MM_{D\sqcup S}}$
                \item $ϕ[\emptyset] = ϕ$
                \item $\pn*{ϕ[S]}[T] = ϕ[S \sqcup T]$
        \end{enumerate}
\end{definition}

Note that the functor generalizes the map $S \mapsto V^{\otimes S}$, while the
morphisms generalize the extension-by-identity $ϕ \otimes \Id{S}$.

Composition of morphisms $ϕ^{D_1}_{C_1}$ and $ψ^{D_2}_{C_2}$ is defined when
$C_1 = D_2$, and is written with the following concatenation operator:
\footnote{
        We denote left-to-right composition with the \enquote{$\then$} symbol:
        $f \then g \defeq g \circ f$. Writing function composition in this order
        assists with readability when there are many functions to apply.
}
\begin{equation}
        \map{
                ϕ^{D_1}_{C_1}\then ψ^{D_2}_{C_2}
                \defeq
                ψ^{D_2}_{C_2}\circ ϕ^{D_1}_{C_1}
        }{\msc C_{D_1}}{\CC_{C_2}}
\end{equation}

In the general case, we still have the map
$\sqcup$ defined by $\MM_{S} \sqcup \MM_{T} = \MM_{S \sqcup T}$. Given morphisms
$ϕ^{D_1}_{C_1}$ and $ψ^{D_2}_{C_2}$ such that $D_1 \cap D_2 = \emptyset = C_1
\cap C_2$, we have a product morphism
$\map {ϕ^{D_1}_{C_1}ψ^{D_2}_{C_2} \defeq ϕ \otimes ψ} {\CC_{D_1 \sqcup D_2}}
{\CC_{C_1\sqcup C_2}}$, which we write with concatenation.

\begin{remark}
        To make expressions easier to read, in this paper we
        will introduce the domain extension implicitly in the following context:
        given morphisms $ϕ^{D_1}_{C_1}$ and $ψ^{D_2}_{C_2}$ such that
        $D_2 \subseteq C_1$ and
        $C_2 \cap (C_1\setminus D_2)=\emptyset=D_1\cap(D_2 \setminus C_1)$, we
        define:
        \begin{equation}\label{eq:composition_extension}
                ϕ^{D_1}_{C_1} \then ψ^{D_2}_{C_2}
                \defeq
                {ϕ^{D_1}_{C_1}[D_2\setminus C_1]} \then
                {ψ^{D_2}_{C_2}[C_1\setminus D_2]}
        \end{equation}
\end{remark}
The two extreme cases of this definition are:
\begin{itemize}
        \item When $C_1 \cap D_2 = \emptyset$, \cref{eq:composition_extension}
                becomes $ϕ^{D_1}_{C_1}ψ^{D_2}_{C_2}$.
        \item When $C_1 = D_2$, \cref{eq:composition_extension} becomes the
                composition $ϕ^{D_1}_{C_1}\then ψ^{D_2}_{C_2}$ exactly.
\end{itemize}

\begin{remark}
        While the $\then$ operator is associative, care must be taken that the
        compositions are well-defined in the presence of duplicated indices.
        While it is sufficient for all the finite sets in a composition to be
        pairwise disjoint, this condition will prove too restrictive for clear
        communication of formulae.
\end{remark}
%TODO: add a proof of associativity? (Likely not necessary.)

\subsection{Defining a meta-group}

To make the above definition more concrete, we will go through the process of
defining a meta-group, which is a generalization of a group object.
Traditionally, the data of a group object are the following:
\begin{itemize}
        \item An object $G$ in a category $\CC$.
        \item A morphism $\map {\mult} {G\times G} {G}$ called
                \enquote{multiplication}.
        \item A \enquote{unit} morphism
                $\map {\unit} {\set{1}} {G}$.\footnote{When $\CC = \Set$, we
                usually write the unit as an element $1=\unit(1)\in G$
        }
        \item An \enquote{inversion} morphism $\map {\antipode} {G} {G}$.
        \item A collection of relations between the morphisms, written as
                equalities of morphisms between Cartesian powers of $G$. For
                example, associativity may be written:
                \begin{equation}\label{eq:cd_assoc}
                \begin{tikzcd}
                        G\times G\times G
                                \rar["\mult \times \id"]
                                \dar["\id \times\mult"']
                        &G \times G
                                \dar["\mult"] \\
                        G \times G
                                \rar["\mult"']
                        &G
                \end{tikzcd}
                \end{equation}
\end{itemize}
Further, the data of these relations is extended to higher powers of $G$ by
acting on other components by the identity:
\begin{equation}\label{eq:extend_group_identity}
\begin{tikzcd}[column sep=large]
        G^{n+3}
                \rar["\mult \times \id^{n+1}"]
                \dar["\id \times\mult\times \id^{n}"']
        &G^{n+2}
                \dar["\mult\times \id^{n}"] \\
        G^{n+2}
                \rar["\mult\times \id^{n}"']
        &G^{n+1}
\end{tikzcd}
\end{equation}

Let us alter how we package these data so as to maximize the clarity of the
meta-group structure:
\begin{enumerate}
        \item Instead of linear orders of factors $G \times \dots \times G$, we
                will index factors by a finite set $X$, writing it $G_X \defeq
                \set{\map {f} {X} {G}}$ in the style of
                \cref{eq:indexed_tensor}.
        \item The indexed factors will determine how the group operations act.
                For instance, multiplication of factor $i$ and $j$ together,
                with the result labelled in factor $k$ is to be written
                $\map {\mult^{ij}_{k}} {G_{\set{i,j}}} {G_{\set{k}}}$.
        \item Instead of implicitly including extensions of morphisms to higher
                powers by the identity, we will parametrize the extension by
                finite sets by $ϕ^{D}_{C}[X] \defeq ϕ^{D}_{C}\times\Id{X}$.
                For example, multiplication $\map {\mult^{ij}_{k}}
                {G_{\set{i,j}}} {G_{\set{k}}}$ generates a family of maps
                $\map {\mult^{ij}_k[X]} {G_{\set{i,j}\sqcup X}}
                {G_{\set{k}\sqcup X}}$, each of which must satisfy the relations
                of the group object such as \cref{eq:extend_group_identity}.
\end{enumerate}
This way of packaging the data leads us to the following generalization:
\begin{definition}
A \defi{meta-group} in $\CC$ is the following data:
\begin{itemize}
        \item A family of objects $G_X\in \CC$, indexed over finite sets $X$.
        \item A family of morphisms $\map {\mult^{ij}_{k}[X]} {G_{\set{i,j}\sqcup X}}
                {G_{\set{k}\sqcup X}}$ called \enquote{multiplication}.
        \item A family of \enquote{unit} morphisms
                $\map {\unit_{i}[X]} {G_X} {G_{\set{i}\sqcup X}}$.
        \item An family of \enquote{inversion} morphisms $\map {\antipode^{i}_{j}[X]}
                {G_{\set{i}\sqcup X}} {G_{\set{j}\sqcup X}}$.
        \item A collection of relations between the morphisms, written as
                equalities of morphisms between the $G_X$'s. For
                example, associativity may be written:
                \begin{equation}
                        \begin{tikzcd}[column sep=huge]
                                G_{\set{1,2,3}\sqcup X}
                                \rar["\mult^{1,2}_{1}\bk*{X\sqcup\set{3}}"]
                                \dar["\mult^{2,3}_{2}\bk*{X\sqcup\set{1}}"']
                        &G_{\set{1,3}\sqcup X}
                        \dar["\mult^{1,3}_{1}\bk{X}"] \\
                        G_{\set{1,2}\sqcup X}
                        \rar["\mult^{1,2}_{1}\bk{X}"']
                        &G_{\set{1}\sqcup X}
                        \end{tikzcd}
                \end{equation}
\end{itemize}
\end{definition}

\section{Algebraic definitions}\label{sec:alg_defs}

We now introduce the algebraic structures which will be used to define the
tangle invariant. These definitions follow those given by Majid in \cite{SM},
although the ones presented below are given in a way that their corresponding
meta-structure is readily visible.

\begin{definition}[meta-algebra]\label{def:meta_algebra}
        A \defi{meta-algebra} (or \defi{meta-monoid}) is a collection of objects
        $\set{A_X}_X$ in $\CC$ together with an associative multiplication
        $\map {\mult^{i,j}_{k}} {A_{\set{i,j}}} {A_{\set{k}}}$ (satisfying
        \cref{eq:cd_mult}), and a unit
        $\map{\unit_{i}}{A_{\emptyset}}{A_{\set{i}}}$ satisfying
        \cref{eq:cd_unit}.
\end{definition}
\begin{remark}
When $\CC = \Vect$ and $A_X = V^{\otimes X}$ for some vector space $V$,
\cref{def:meta_algebra} becomes the more familiar definition of an
\defi{algebra}. When $A_\emptyset$ is a field, it is more common think of the
unit as an element $\one\in V$. The unit map is then defined by linearly
extending the assignment $\unit_{i}(1) = \one_{i}$.
\end{remark}

\begin{multicols}{2}\noindent
\begin{equation}\label{eq:cd_mult}
\begin{tikzcd}
        A_{\set{1,2,3}}
                \rar["\mult^{1,2}_{1}"]
                \dar["\mult^{2,3}_{2}"']
        &A_{\set{1,3}}
                \dar["\mult^{1,3}_{1}"] \\
        A_{\set{1,2}}
                \rar["\mult^{1,2}_{1}"']
        &A_{\set{1}}
\end{tikzcd}
\end{equation}
\columnbreak
\begin{equation}\label{eq:cd_unit}
\begin{tikzcd}[column sep=large]
        A_{\set{1}}
                \rar["\unit_{2}"]
                \drar["\id"']
        &A_{\set{1,2}}
                \dar["\mult^{1,2}_{1}", shift left]
                \dar["\mult^{2,1}_{1}"', shift right] \\
        &A_{\set{1}}
\end{tikzcd}
\end{equation}
\end{multicols}

\begin{remark}
        From now on, we will denote repeated multiplication as in
        \cref{eq:cd_mult} by using extra indices. For instance:
        $\mult^{i,j, k}_{\ell} \defeq \mult^{i,j}_{r}\then\mult^{r, k}_{\ell}
        = \mult^{j, k}_{s}\then\mult^{i, s}_{\ell}$.
\end{remark}

There is also the dual notion of a \emph{coalgebra}, which arises by reversing
the arrows in \cref{eq:cd_mult,eq:cd_unit}:

\begin{definition}[meta-coalgebra]
        A \defi{meta-colagebra} (or \defi{meta-comonoid}) is a collection
        $\set{C_X}_X$ together with a \defi{comultiplication} $\map
        {\comult^{i}_{jk}} {C_{\set{i}}} {C_{\set{j,k}}}$ which is
        \defi{coassociative} (\cref{eq:cd_comult}) and a \defi{counit}, which is a
        map $\map {\counit^{i}} {A_{i}} {A_{\emptyset}}$ satisfying
        \cref{eq:cd_counit}.
\end{definition}
\nopagebreak
\begin{multicols}{2}\noindent
\begin{equation}\label{eq:cd_comult}
\begin{tikzcd}
        C_{\set{1,2,3}}
        &C_{\set{2,3}}
                \lar["\comult^{1}_{{1,2}}"'] \\
        C_{\set{1,2}}
                \uar["\comult^{2}_{2,3}"]
        &C_{\set{1}}
                \lar["\comult^{1}_{1,2}"]
                \uar["\comult^{1}_{1,3}"']
\end{tikzcd}
\end{equation}
\columnbreak
\begin{equation}\label{eq:cd_counit}
\begin{tikzcd}
        C_{\set{1}}
        &C_{\set{1,2}}
                \lar["\counit^{2}"']\\
        &C_{\set{1}}
                \ular["\id", shift left]
                \uar["\comult^{1}_{1,2}", shift left]
                \uar["\comult^{1}_{2,1}"', shift right] \\
\end{tikzcd}
\end{equation}
\end{multicols}

\begin{remark}
        From now on, we will denote repeated comultiplication as in
        \cref{eq:cd_comult} by using extra indices. For instance:
        $\comult^{i}_{j, k, \ell}
        \defeq \comult^{i}_{j,r}\then\comult^{r}_{k, \ell}
        = \comult^{i}_{s,\ell}\then\comult^{s}_{j, j}$.
\end{remark}

If a meta-object $\set{B_X}_x$ satisfies both definitions of an algebra and a
coalgebra, we introduce a definition for when the structures are compatible with
each other in the following way:

\begin{definition}[meta-bialgebra]
        A \defi{meta-bialgebra} (or \defi{meta-bimonoid}) is a meta-algebra
        $(B,\mult,\unit)$ and a meta-coalgebra
        $(B,\comult,\counit)$, such that $\comult$ and $\counit$ are
        meta-algebra morphisms.
        \footnote{
                $B_{X}$ inherits a (co)algebra structure from $B$, given by
                $(B_X)_Y \defeq B_{X^Y}$ and component-wise operations. The
                bialgebra structure on $B_{\emptyset}$ is given by
                $\mult = \unit = \comult = \counit = \id$.
        }
\end{definition}

\ProvideDocumentCommand{\lift}{mm}{\curryIsolated{#1}^{(#2)}}

\begin{multicols}{2}\noindent
\begin{equation}\label{eq:cd_mult_comult}
\begin{tikzcd}[column sep=large]
        B_{\set{1,2}}
                \rar["\mult^{1,2}_{1}"]
                \dar["\comult^{1}_{1_1,1_2}\then\comult^{2}_{2_1,2_2}"]
        &B_{\set{1}}
                \dar["\comult^{1}_{1,2}"] \\
        B_{\set{1_1,1_2,2_1,2_2}}
                \rar["\mult^{1_1,1_2}_{1}\then\mult^{2_1,2_2}_{2}"']
        &B_{\set{1,2}}
\end{tikzcd}
\end{equation}\begin{equation}\label{eq:cd_unit_comult}
\begin{tikzcd}[row sep=tiny]
        &B_{\set{1}}
                \ar[dd,"\comult^{1}_{1,2}"] \\
        B_{\emptyset}
                \urar["\unit_{1}"]
                \drar["\unit_{1}\then\unit_{2}"',near end]\\
        &B_{\set{1,2}}
\end{tikzcd}
\end{equation}
\columnbreak
\begin{equation}\label{eq:cd_mult_counit}
\begin{tikzcd}[column sep=tiny]
        B_{\set{1,2}}
                \ar[rr,"\mult^{1,2}_{1}"]
                \drar["\counit^{1}\then\counit^{2}"']
        &&B_{\set{1}}
                \dlar["\counit^{1}"] \\
        &B_{\emptyset}
\end{tikzcd}
\end{equation}
\begin{equation}\label{eq:cd_unit_counit}
\begin{tikzcd}
        B_{\emptyset}
                \rar["\unit_{1}"]
                \drar["\id"']
        &B_{\set{1}}
                \dar["\counit^{1}"] \\
        &B_{\emptyset}
\end{tikzcd}
\end{equation}
\end{multicols}

\begin{remark}
        The conditions for $\comult$ being an algebra morphism are presented in
        \cref{eq:cd_mult_comult,eq:cd_unit_comult}, while those for $\counit$
        are in \cref{eq:cd_mult_counit,eq:cd_unit_counit}.\footnote{While
        notation explicitly naming each tensor factor appears cumbersome in
        these diagrams, it will prove invaluable later when used on tangle
        diagrams, so we leave it as is for the sake of consistency.} Observing
        invariance under arrow reversal, it may not come as a surprise that
        \cref{eq:cd_mult_comult,eq:cd_mult_counit} also are the conditions for
        $\mult$ being a coalgebra morphism, and
        \cref{eq:cd_unit_comult,eq:cd_unit_counit} tell us that $\unit$ is as
        well.
\end{remark}

Next, we introduce a notion of invertibility which extends a bialgebra to a Hopf
algebra.
\begin{definition}[meta-Hopf algebra]
A \defi{meta-Hopf algebra} (or \defi{meta-Hopf monoid})is a bialgebra $H$ together with a map $\map {\antipode} {H} {H}$ called the \defi{antipode}, which satisfies
$\comult^{1}_{1,2}\then \antipode^1_1 \then \mult^{1,2}_1 =
\counit^{1}\then\unit_{1} =
\comult^{1}_{1,2}\then \antipode^2_2 \then \mult^{1,2}_1$.
As a commutative diagram, this looks like \cref{eq:cd_antipode}
\begin{equation}
\begin{tikzcd}[column sep=tiny]\label{eq:cd_antipode}
        H_{\set{1}}
                \arrow[rr, "\counit^{1}"] \arrow[rd, "\comult^{1}_{1,2}"']
        && H_{\emptyset}
                \arrow[rr, "\unit_{1}"]
        && H_{\set{1}} \\
        & H_{\set{1,2}}
                \arrow[rr, "\antipode^{2}_{2}", shift left]
                \arrow[rr, "\antipode^{1}_{1}"', shift right]
        && H_{\set{1,2}} \arrow[ru, "\mult^{1,2}_{1}"']
\end{tikzcd}
\end{equation}
\end{definition}

In order to do knot theory, we need an algebraic way to represent a crossing of
two strands. This is accomplished by the so-called $\Rmat$-matrix:

\begin{definition}[quasitriangular meta-Hopf algebra]
A \defi{quasitriangular meta-Hopf algebra} (or \defi{quasitriangular meta-Hopf
monoid}) is a Hopf algebra $H$, together with an invertible element $\Rmat_{i,j} \in
H_{i,j}$, called the \defi{$\Rmat$-matrix}, which satisfies the following
properties: (we will denote the inverse by $\Rmati$)
\begin{align}
        \label{eq:Rmat_overstrand}
        \Rmat_{12}\then\comult^{2}_{23}&=\Rmat_{a2}\Rmat_{b3}\then\mult^{ab}_1\\
        \label{eq:Rmat_understrand}
        \Rmat_{13}\then\comult^{1}_{12}&=\Rmat_{1b}\Rmat_{2a}\then\mult^{ab}_3\\
        \label{eq:Rmat_comult}
        \comult^{1}_{21} &=
                \comult^{1}_{12} \Rmat_{1_i,2_i}\Rmati_{1_f,2_f}\then
                \mult^{1_i,1,1_f}_{1}\then \mult^{2_i,2,2_f}_{2}
\end{align}
\end{definition}

\begin{definition}[Drinfeld element]
        In a quasitriangular meta-Hopf algebra $H$, the \defi{Drinfeld element},
        $\dfe \in H$ is given by:
        \begin{equation}
                \dfe \defeq \Rmat_{21}\then\antipode^2_2 \then \mult^{12}
        \end{equation}
\end{definition}

\begin{definition}[monodromy]
        Each quasitriangular meta-Hopf algebra has a \defi{monodromy}
        $\monodromy_{12} \defeq
        \Rmat_{12}\Rmat_{34}\then\mult^{14}_{1}\then\mult^{23}_{2}$. Its
        inverse will be denoted
        $\invb\monodromy_{12} =
        \Rmati_{12}\Rmati_{34}\then\mult^{14}_{1}\then\mult^{23}_{2}$.
\end{definition}

\begin{lemma}
        The Drinfeld element $\dfe$ satisfies for all $h\in H$:
        \begin{align}
                \dfe_{1}h_2\dfe_{3} \then \mult^{1,2,3}
                &= h \then S \then S\\
                \dfe \then \comult_{12}
                &= \dfe_1\dfe_2\invb\monodromy_{34}
                \then\mult^{13}_{1}\then\mult^{24}_2
        \end{align}
\end{lemma}
\begin{proof}
        See Majid's work in \cite{SM} or Etingof and Schiffmann in \cite{ES}
        for more details on this standard result. Note that the proof does not
        rely on the additive structure of the Hopf algebra, which allows us to
        extend this result to the realm of meta-Hopf algebras.
        %TODO: finish
\end{proof}

\begin{definition}[ribbon meta-Hopf algebra]
        A quasitriangular meta-Hopf algebra $H$ is called \defi{ribbon} if it has an
        element $\ribbon\in \centre(H)$ such that:
        \begin{align}
                \ribbon_1\ribbon_2\then\mult^{12}
                &= \dfe_1 \dfe_2 \then \antipode^{2}_{2} \then \mult^{12}\\
                \ribbon_1 \then \comult^1_{12}
                &=      \ribbon_1\ribbon_2
                        \then\invb\monodromy_{34}
                        \then\mult^{13}_{1}
                        \then\mult^{24}_{2} \\
                \ribbon \then \antipode &= \ribbon\\
                \ribbon \then \counit &= \unit \then \counit = 1
        \end{align}
\end{definition}

\begin{definition}[distinguished grouplike element (spinner)]
        A \defi{distinguished grouplike element} (or \defi{spinner}) in a
        quasitriangular meta-Hopf algebra $H$ is an invertible element
        $\spin\in H$ (with inverse $\invb\spin$) such that for all $x\in H$:
        \begin{align}
                \label{eq:spinner_ribbon}
                \spin_1\ribbon_2\spin_3 \then \antipode^2_2 \then \mult^{123} &=
                \ribbon\\
                \label{eq:spinner_comult}
                \spin_1\then\comult^{1}_{12} &=\spin_1\spin_2\\
                \label{eq:spinner_antipode}
                \spin \then\antipode &= \invb\spin\\
                \label{eq:spinner_conjugate}
                \spin_{1}x_2\invb\spin_{3}\then\mult^{1,2,3} &=
                x \then \antipode \then \antipode\\
                \label{eq:spinner_counit}
                \spin \then \counit &= \unit \then \counit = 1
        \end{align}
\end{definition}

\begin{lemma}[spinners and ribbon Hopf algebras]\label{lem:spinner_ribbon}
        If a Hopf algebra has either a ribbon element $\ribbon$ or a spinner
        $\spin$, then it must have the other as well, given by the formula:
        $\spin_1 \ribbon_2 \then \mult^{12} = \dfe$.
\end{lemma}
% todo: is this lemma important for the thesis? If not, cut it out!

\section{The meta-algebra of tangle diagrams}
\label{sec:topological_interpretations}

The particular structures introduced were chosen for their ability to represent
the topological properties of knotted objects. We will now introduce the notion
of a tangle and demonstrate its meta-algebraic structure.

\subsection{Upright tangles}

For our purposes, a tangle will be visualised as follows: take a stiff circular
metal frame and attach a collection of strings to the wire, ensuring that the
strings always remain inside the circle, and that each string is tied to the
metal frame in two unique locations (that is, no two strings share an endpoint).
\begin{figure}[h]
        \centering
        \includegraphics[width=0.4\textwidth]{figures/sample.png}% todo: fix
        \caption{Example of a tangle}
        \label{fig:tangle_example}
\end{figure}

\begin{definition}[pure tangle]
        A \defi{pure tangle} is an embedding of line segments (called
        \defi{components} or \defi{strands}) into the thickened unit disk
        $D \times [-1,1]$ (or a disjoint union of such disks) such that the
        endpoints of the line segments are fixed along
        $\boundary D \times \set0$. Two pure tangles are considered equivalent
        if there exists an isotopy of the embedding which fixes the endpoints of
        the strands. The term \enquote{pure} refers to the absence of closed
        loops.
\end{definition}

\begin{definition}[framed tangle]
        The object which is more natural for us to deal with is the \defi{framed
        tangle}, which is a pure tangle together with a choice of section of the
        normal bundle for each component. One may equivalently think of
        replacing the strings in a pure tangle with thin ribbons. Unless
        otherwise mentioned, it will be assumed that all tangles are framed.
\end{definition}

In order to best capture the combinatorial properties of a tangle, observe that
a generic projection of a tangle to its \defi{central core} $D\times \set 0$
will result in the strands forming a graph, with each crossing of two strands in
the tangle producing a vertex in the graph. By assigning to each vertex the sign
of the corresponding crossing (either \enquote{positive} or \enquote{negative}),
we end up with a combinatorial object which is equivalent to the original
concept of a tangle.

\begin{definition}[pure tangle diagram]\label{def:pure_tangle_diagram}
        A \defi{pure tangle diagram} is a projection of a tangle onto its
        central core such that all the line segments are immersions which
        intersect both the boundary disk and the other strands transversally,
        together with an assignment of a sign to each strand intersection.
        Small open neighbourhoods of these intersections are called
        \defi{crossings}, while the complement of the crossings is a collection
        of embedded line segments called \defi{arcs}.

        Two pure tangle diagrams are considered equivalent if they differ by a
        finite sequence of Reidemeister moves, as outlined in
        \crefrange{fig:R1}{fig:R3}
\end{definition}

\begin{figure}[h]
        \centering
        \includegraphics[width=0.2\textwidth]{figures/sample.png}% todo: fix
        \caption{Reidemeister move $R1'$}
        \label{fig:R1p}
\end{figure}
\begin{figure}[h]
        \centering
        \includegraphics[width=0.2\textwidth]{figures/sample.png}% todo: fix
        \caption{Reidemeister move $R2$}
        \label{fig:R2}
\end{figure}
\begin{figure}[h]
        \centering
        \includegraphics[width=0.2\textwidth]{figures/sample.png}% todo: fix
        \caption{Reidemeister move $R3$}
        \label{fig:R3}
\end{figure}

The rotation numbers of arcs will play a role in this thesis, so we will capture
these data in the following way (as described in \cref{apai}):

\begin{definition}[upright pure tangle diagram]
        We will put a further requirement on our tangles: each arc must begin and end
        pointing upwards, and each crossing must involve only upward-pointing curves.
        This way, each arc will have a well-defined integer rotation number.
        Two tangles are considered equivalent if they agree under the
        \enquote{rotational Reidemeister moves}.
\end{definition}
\begin{figure}[h]
        \centering
        \includegraphics[width=0.2\textwidth]{figures/sample.png}% todo: fix
        \caption{Rotational Reidemeister move $R1'_{\text{rot}}$}
        \label{fig:R1prot}
\end{figure}
\begin{figure}[h]
        \centering
        \includegraphics[width=0.2\textwidth]{figures/sample.png}% todo: fix
        \caption{The two rotational Reidemeister moves $R2_{\text{rot}}$}
        \label{fig:R2rot}
\end{figure}
\begin{figure}[h]
        \centering
        \includegraphics[width=0.2\textwidth]{figures/sample.png}% todo: fix
        \caption{Rotational Reidemeister move $R3_{\text{rot}}$}
        \label{fig:R3rot}
\end{figure}
\begin{figure}[h]
        \centering
        \includegraphics[width=0.2\textwidth]{figures/sample.png}% todo: fix
        \caption{Whirling move $Wh$}
        \label{fig:whirl}
\end{figure}

Fortunately, ambient isotopy allows us to rotate any classical tangle into an
upward-pointing form. Additionally, there is only one way to do this. We
reproduce the proof of this fact by Bar-Natan and van der Veen in \cite{BV}
below:

\begin{lemma}[tangles inject into upright tangles]
        To each pure tangle diagram $D$ there exists an upright pure tangle
        diagram $D'$ obtained from $D$ by a planar isotopy. Further, if $D''$ is
        another such upright pure tangle diagram obtained from $D$, then $D'$
        and $D''$ differ by a finite sequence of rotational Reidemeister moves
        and a change of rotation number at the endpoints.
\end{lemma}
\begin{proof}
        Each arc and crossing in the diagram $D$ may be rotated so that its
        endpoints are pointing upwards, giving rise to a diagram $D'$. Two
        (nonupright) tangle diagrams are equivalent when they differ by a finite
        sequence of Reidemeister moves. Each of these Reidemeister may also be
        rotated to an equivalence of upright tangles, each of which is given as
        a rotational Reidemeister move \crefrange{fig:R1prot}{fig:R3rot}. The
        last possibility is the rotation of an entire crossing, which is covered
        by \cref{fig:whirl}.
\end{proof}

\subsection{The meta-algebra structure of upright tangle diagrams}

Let us now formally connect tangle diagrams with meta-algebras.

% TODO: Add a footnote or remark about the multiplication caveat.
\begin{theorem}[tangles form a ribbon meta-Hopf algebra]
        \label{thm:vt_qtmha}
        Define $\tangle_X$ to be the set of upright tangles with $\ord{X}$
        strands, each labelled by an element $i\in X$. Then the collection
        $\set{\tangle_X}_X$ forms a quasitriangular ribbon meta-Hopf algebra
        with the following operations:
        \begin{itemize}
                \item multiplication $\mult^{ij}_{k}[X]$ takes a tangle with
                        strands $X\sqcup\set{i,j}$ and glues the end of strand
                        $i$ to strand $j$, labelling the resulting strand $k$.
                \item the unit $\unit_{i}[X]$ takes a tangle diagram with
                        strands $X$ and introduces a new strand $i$ which does
                        not touch any of the other strands.
                \item the comultiplication $\comult^{i}_{jk}[X]$ takes a tangle
                        with strands $X\sqcup \set{i}$ and doubles strand $i$,
                        separating the two strands along the framing of strand
                        $i$, calling the right strand $j$ and the left one
                        $k$.\footnote{While this convention appears unfortunate,
                        we follow the notation laid out in \cite{BV} so that the
                        antipode and spinner have a more memorable
                        representation, namely looking like the letters they are
                        represented by (see \cref{thm:rvt_metaHopf} for more
                        details).
                }
                \item the counit $\counit^{i}[X]$ takes a tangle with strands
                        indexed by $X\sqcup \set{i}$ and returns the tangle with
                        strand labelled by $i$ deleted.
                \item The antipode $\antipode^{i}_{j}[X]$ takes a tangle with
                        strands labelled by $X \sqcup \set{i}$ and reverses the
                        direction of strand $i$, then adds a counter-clockwise
                        cap to the new beginning, and a clockwise cup to the
                        end. This new strand is called $j$. When applied to a
                        single vertical strand, the resulting tangle looks like
                        the letter \enquote{S}.
                \item the $\Rmat$-matrix $\Rmat_{ij}$ is given by the two-strand
                        tangle with a single positive crossing of strand $i$
                        over strand $j$. The inverse $\Rmat$-matrix
                        $\Rmati_{ij}$ is the two-strand tangle with a
                        \emph{negative} crossing of strand $i$ over strand $j$.
                \item The spinner $\spin_i[X]$ takes a tangle in $\tangle_X$ and
                        adds a new strand with rotation number $1$ which has no
                        interactions with any other strands. This new strand
                        looks like the letter \enquote{C}.
        \end{itemize}
\end{theorem}
\begin{figure}[h]
        \centering
        \includegraphics[width=0.8\textwidth]{figures/tangle_mult.pdf}
        \caption{
                Multiplication $\mult^{ij}_{k}$ stitches two strands in a tangle
                together.
        }
        \label{fig:tangle_mult}
\end{figure}
\begin{figure}[h]
        \centering
        \includegraphics[width=0.8\textwidth]{figures/tangle_unit.pdf}
        \caption{The unit $\unit_i$ introduces a new strand in a tangle.}
        \label{fig:tangle_unit}
\end{figure}
\begin{figure}[h]
        \centering
        \includegraphics[width=0.8\textwidth]{figures/tangle_counit.pdf}
        \caption{The counit $\counit^i$ deletes a strand in a tangle.}
        \label{fig:tangle_counit}
\end{figure}
\begin{figure}[h]
\centering
\includegraphics[width=0.8\textwidth]{figures/tangle_comult.pdf}
\caption{The comultiplication $\comult^{i}_{jk}$ doubles a strand in a tangle
along its framing. Notice the right-to-left strand labels.}
\label{fig:tangle_comult}
\end{figure}
\begin{figure}[h]
        \centering
        \begin{minipage}[c]{0.4\linewidth}
                \centering
                \includegraphics{figures/tangle_rmat.pdf}
                \caption{The $\Rmat$-matrix $\Rmat_{ij}$ represents a tangles with a
                single positive crossing.}
                \label{fig:tangle_rmat}
        \end{minipage}
        \begin{minipage}[c]{0.4\linewidth}
                \centering
                \includegraphics{figures/tangle_rmati.pdf}
                \caption{The inverse $\Rmat$-matrix $\Rmati_{ij}$ represents a
                        tangle with a single negative crossing.}
                \label{fig:tangle_rmati}
        \end{minipage}
\end{figure}
\begin{figure}[h]
        \centering
        \includegraphics[width=0.2\textwidth]{figures/sample.png}% todo: fix
        \caption{The Drinfeld element in the meta-Hopf algebra of tangles.}
        \label{fig:drinfeld_element}
\end{figure}
\begin{figure}[h]
        \centering
        \includegraphics{figures/monodromy.pdf}
        \caption{The monodromy in the meta-Hopf algebra of tangles.}
        \label{fig:monodromy}
\end{figure}
\begin{figure}[h]
        \centering
        \includegraphics{figures/tangle_spin.pdf}
        \caption{The spinner $\spin_i$ represents a strand with rotation number
        $1$.}
        \label{fig:tangle_spin}
\end{figure}
\begin{figure}[h]
        \centering
        \includegraphics{figures/tangle_spini.pdf}
        \caption{The inverse spinner $\spini_i$ represents a strand with
                rotation number $-1$.}
        \label{fig:tangle_spini}
\end{figure}
\begin{figure}[h]
        \centering
        \includegraphics[width=0.2\textwidth]{figures/sample.png}% todo: fix
        \caption{A ribbon element in the meta-Hopf algebra of tangles. One can
        use \cref{lem:spinner_ribbon} to verify this is compatible with the
        spinner.
        }
        \label{fig:ribbon_element}
\end{figure}

\begin{figure}[h]
        \centering
        \includegraphics[width=0.2\textwidth]{figures/sample.png}% todo: fix
        \caption{Example of a tangle satisfying \cref{eq:Rmat_overstrand}}
        \label{fig:Rmat_overstrand}
\end{figure}
\begin{figure}[h]
        \centering
        \includegraphics[width=0.2\textwidth]{figures/sample.png}% todo: fix
        \caption{Example of a tangle satisfying \cref{eq:Rmat_understrand}}
        \label{fig:Rmat_understrand}
\end{figure}
\begin{figure}[h]
        \centering
        \includegraphics[width=0.2\textwidth]{figures/sample.png}% todo: fix
        \caption{Example of a tangle satisfying \cref{eq:Rmat_comult}}
        \label{fig:Rmat_comult}
\end{figure}
\begin{proof}
        Associativity of multiplication (\cref{eq:cd_mult}) follows from the
        fact that stitching strands together amounts to concatenating the order
        of the crossings each strand interacts with. Since list concatenation is
        an associative operation, associativity follows in this case as well.

        Adding a non-interacting strand to a diagram, then stitching it to an
        existing strand (\cref{eq:cd_unit}) does not change any of the
        combinatorial data in the diagram, and results in identical diagrams.

        Establishing coassociativity (\cref{eq:cd_comult}) amount to the same
        argument that cutting a piece of paper into three strips does not depend
        on the order of cutting.

        The counit identity (\cref{eq:cd_counit}) states deleting a strand
        is the same operation as first doubling it, then deleting both resulting
        strands.

        The meta-bialgebra axioms we verify next:

        \Cref{eq:cd_mult_comult} states that if two strands are stitched
        together, then the resulting strand is doubled, this could have
        equivalently been achieved by doubling each of the original strands,
        then performing a stitching on both resulting pairs of strands.

        \Cref{eq:cd_mult_counit} simply states that stitching two strands
        together, then removing the resulting strand could have equally been
        achieved by removing both of the original strands without stitching them
        first.

        \Cref{eq:cd_unit_counit} states that introducing a strand, then
        immediately removing it is the identity operation.

        \Cref{eq:cd_unit_comult} says that doubling a newly-introduced (and
        therefore free of crossings) strand is the same operation as introducing
        two strands separately. (Recall that in the virtual case, proximity of
        strands is not accounted for)

        \Cref{eq:cd_antipode} states that when a strand is doubled, then one of
        the two strands is reversed, multiplying the two strands together
        results in a strand which can be rearranged to not interact with any of
        the other strands. This can be readily seen, as this newly-created
        strand looks like a snake weaving through the tangle diagram. One can
        remove the snake by applying a series of Reidemeister 2 moves, resulting
        in a strand disjoint from the rest of the diagram. This is the same as
        deleting the original strand, then introducing a new disjoint one.

        The quasitriangular axioms are equalities of pairs of three-strand
        tangles:
        \begin{itemize}
                \item \Cref{eq:Rmat_overstrand,eq:Rmat_understrand} tell us that
                        doubling a strand involved in a single crossing can also
                        be built by adjoining two crossings together.
                \item \Cref{eq:Rmat_comult} tells us that we can swap the order
                        of a doubled strand by adding crossings to either end
                        (reminiscent of a Reidemeister 2 move)
        \end{itemize}

        Finally, we observe that the quotient we introduce to tangle diagrams by
        the Reidemeister moves does not introduce any new relations.
        Reidemeister 2 follows from the invertibility of the $\Rmat$-matrix.
        Next, it is readily seen that the quasitriangular relations governing
        the $\Rmat$-matrix force it to solve the Yang-Baxter equation, which is
        one equivalent to the Reidemeister 3 in this case.

        Using \cref{lem:spinner_ribbon}, it is enough to verify the spinner
        axioms
        (\cref{eq:spinner_ribbon,eq:spinner_comult,eq:spinner_antipode,eq:spinner_conjugate,eq:spinner_counit}).
        All these axioms have corresponding pictures one can draw, keeping in
        mind the orientations in the definitions of the relevant operations.
\end{proof}

\section{The $ybax$ meta-algebra}

Here we define the ribbon Hopf algebra $\CU$.

Define the Lie algebra
\begin{equation}
\fg \defeq \Span\setbuilder[\Big]{\yo, \bo, \ao, \xo}{
                \liebk{\ao}{\xo} = \xo,
                \liebk{\ao}{\yo} = -\yo,
                \liebk{\xo}{\yo} = \bo,
                \liebk{\bo}{ } = 0
        }
\end{equation}
Then the algebra $\CU$ is defined to be the universal enveloping algebra
$\uea{\fg}$. The bialgebra structure of $\CU$ is: for any $\zo\in\set{\yo, \bo,
\ao, \xo}$, we have
\begin{align}
        \comult_{i,j}(\zo) &= \zo_i + \zo_j\\
        \counit(\zo) &= 0
\end{align}
Next, we define the Hopf algebra structure by defining the antipode
\begin{equation}
        \antipode(\zo) &= -\zo
\end{equation}

The ribbon structure of $\CU$ requires we introduce both the $\Rmat$-matrix and
the spinner $\spin$:
\begin{align}
        \Rmat_{i,j}
        &\defeq \exp\pn*{\ao_j\bo_i + \frac{1-\Bo_i}{\bo_i}\yo_i\xo_j}\\
        \spin &\defeq \sqrt{\Bo}
\end{align}
