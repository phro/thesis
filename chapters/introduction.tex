\chapter{Tensor Products and Meta-Objects}
\label{ch:intro}

\section{Tensor product notation}
In what follows, we will extensively use tensor products, tensor powers, and
generalizations thereof. We begin by introducing the notation that will
make working with these objects more straightforward, similar to the slot-naming
index notation introduced by Penrose \cite{penrose}.

Let $\K$ be a field and $V$ a $\K$-vector space (for the moment assumed to be
finite dimensional). When working with a large tensor power $V^{\otimes n}$ of
$V$, it will often be more convenient to label tensor factors with elements of a
finite set $S$ (with $\ord S = n$) rather than by their position in a linear
order.

For example, consider the vector $u \otimes v \otimes w \in V^{\otimes 3}$. Let
us choose an index set $S = \set{i,j,k}$. We then may equivalently write this
vector by labelling each tensor factor with one of the elements of $S$, say
$u_iv_jw_k$. Since the labels serve to distinguish the separate factors, this
vector may equivalently be written as $u_iv_jw_k = v_ju_iw_k = w_ku_ju_i \in
V^{\otimes S}$, where $V^{\otimes S}$ denotes the tensor power of $V$. An
additional notation which we will prefer is $V_S = V^{\otimes S}$. We formalize
the idea below:

\begin{definition}[indexed tensor powers]\label{def:indexed_tensor_powers}
        Let $V$ be a vector space and $S = \set{s_1,…, s_n}$ be a finite
        set. We define the \defi{indexed tensor power} of $V$ to be the
        collection of formal linear combinations of functions from $S$ to $V$
        \begin{equation}\label{eq:indexed_tensor}
                V_S \defeq V^{\otimes S} \defeq \Span\set{\map {f} {S} {V}}/\sim
        \end{equation}
        subject to the standard multilinear relations, namely multi-additivity
        and the factoring of scalars:

        By multi-additivity, we mean that for each $i\in S$ and $f, g \in V_S$
        satisfying $f(s) = g(s)$ for each $s \in S\setminus\set i$, we have:
        \begin{equation}
                \label{eq:tensor_additive}
                f + g\sim \pn*{
                        s \mapsto
                        \begin{cases}
                                f(s) = g(s) & \text{if $s \neq i$}\\
                                f(i) + g(i) & \text{if $s = i$}
                        \end{cases}
                }
        \end{equation}
\end{definition}

We will write such functions $\map {f} {S} {V}$ with $f(s_i) = v_i$
with the following notation:
\begin{equation}\label{eq:product_notation}
        \pn{v_{1}}_{s_1}
        \pn{v_{2}}_{s_2} ⋯
        \pn{v_{n}}_{s_n}
        \defeq f
\end{equation}
The factoring of scalars relation is:
\begin{equation}
        \label{eq:tensor_scalar}
        (v_1)_{s_1}(v_2)_{s_2} ⋯(λv_i)_{s_i} ⋯(v_n)_{s_n} =
        λ\cdot(v_1)_{s_1}(v_2)_{s_2} ⋯(v_n)_{s_n}
\end{equation}
Similarly, \cref{eq:tensor_additive} in the style of \cref{eq:product_notation}
becomes:
\begin{equation}\label{eq:tensor_additive_compact}
        \begin{multlined}[t]
                \pn*{(v_1)_{s_1}(v_2)_{s_2} ⋯x_{s_i} ⋯(v_n)_{s_n}} +
                \pn*{(v_1)_{s_1}(v_2)_{s_2} ⋯y_{s_i} ⋯(v_n)_{s_n}}
                \\
                =
                (v_1)_{s_1}(v_2)_{s_2} ⋯(x+y)_{s_i} ⋯(v_n)_{s_n}
        \end{multlined}
\end{equation}

Next, we introduce notation for maps between tensor powers so that we may
unambiguously refer to appropriate tensor factors while defining morphisms. We
accomplish this task by adding a convenient way of writing the domain and
codomain of a map. Let $D$ and $C$ be finite sets, and $\map {T} {V_{D}}
{V_{C}}$. We will denote $T$ alternatively by $T^{D}_{C}$, so that its domain
and codomain are easily read off. It is important to note that when $T$ is not
symmetric in its arguments, the order of the indices in this notation matters.

\begin{example}
        Let $V = \R^3$, and $T^{a, b}_{c}$ (equivalently, $\map {T} {V_{\set{a,
        b}}} {V_{\set{c}}}$) defined by $T^{a, b}_{c}\pn*{\vec v_a\vec w_b} =
        (\vec v \times \vec w)_c$ denote the cross product. Stating that the
        cross product is antisymmetric may be accomplished without referencing
        vectors by writing:
        \begin{equation}
                T^{a, b}_{c} = -T^{b, a}_{c}
        \end{equation}
\end{example}

\begin{remark}
There are three special cases with this notation:
\begin{itemize}
        \item Given a (multi)linear functional
                $\map {ϕ} {V_{S}} {\K \iso V_{\emptyset}}$, we will write $ϕ^S$
                instead of $ϕ^S_{\emptyset}$. The linear order on $S$ remains in
                this notation.
        \item Elements $v\in V_{S}$ will be interpreted as a map
                $\map {v} {\K = V_{\emptyset}} {V_S}$ written $v_S$ instead of
                $v^{\emptyset}_{S}$.
        \item When only one index is present in a subscript or superscript, and
                its omission does not introduce an ambiguity in an expression,
                then it may be omitted to improve readability. For instance, a
                map $\map {ϕ} {V_{\set{1,2}}} {V_{\set{3}}}$ may
                be written as $ϕ^{1,2}$ instead of $ϕ^{1,2}_{3}$, with the
                canonical isomorphism $V\iso V_{\set{3}}$ being suppressed.
\end{itemize}
\end{remark}

When taking the tensor product of two tensor powers, we follow \cite{BS}
and use the notation \enquote{$\sqcup$} instead of \enquote{$\otimes$}:
\begin{equation}
        V_{X} \sqcup V_{Y} \defeq V_{X \sqcup Y}
\end{equation}
Additionally, given $ϕ^{D_1}_{C_1}$ and $ψ^{D_2}_{C_2}$ such that $D_1 \cap D_2 = \emptyset = C_1
\cap C_2$, we have a product morphism
$\map {ϕ^{D_1}_{C_1}ψ^{D_2}_{C_2} \defeq ϕ \otimes ψ} {V_{D_1 \sqcup D_2}}
{V_{C_1\sqcup C_2}}$, which we also write with concatenation.

Finally, we point out that any morphism $T^D_C$ may be extended to one with
larger domain and codomain. We introduce the notation
$T[S] \defeq T^{D}_{S} \Id{S}$ (recalling the concatenation means
$T \otimes \Id{S}$) for this concept. When no ambiguity arises, we will also
suppress the \enquote{$[S]$} so that $T^D_C$ represents more generally:
\begin{equation}
        \pn*{T^{D}_{C}}\pn{v_{D} w_S}
                \defeq \pn[\big]{T^{D}_{C}v_{D}} w_S
\end{equation}
for any $v_D \in V_{D}$ and $w_S\in V_{S}$.

\section{Meta-objects}\label{sec:meta-objects}

\ProvideDocumentCommand{\CC}{}{\msc C}

While the above notation is helpful when working with vector spaces, we are
interested in also using the same notation to describe a tangle. Our formulation
of tangles (introduced in \cref{sec:topological_interpretations}) is not a
tensor product, though it shares many similarities. In particular, the domains
and codomains of the maps we have discussed so far have only depended on the
index set. With this observation, we replace the notation of tensor powers with
that of a so-called \defi{meta-object}. We introduce this concept by starting
with monoids.

We now go through the process of defining a meta-monoid, which is a
generalization of a monoid object. Traditionally, the data of a monoid object
are the following:
\begin{itemize}
        \item An object $M$ in a category $\CC$.
        \item A morphism $\map {\mult} {M\times M} {M}$ called
                the \enquote{multiplication} operation.
        \item A \enquote{unit} morphism
                $\map {\unit} {\set{1}} {M}$.\footnote{When $\CC = \Set$, we
                usually write the unit as an element $1=\unit(1)\in M$
        }
        \item A collection of relations between the operations, written as
                equalities of morphisms between Cartesian powers of $M$. For
                example, associativity may be written:
                \begin{equation}\label{eq:cd_assoc}
                \begin{tikzcd}
                        M\times M\times M
                                \rar["\mult \times \id"]
                                \dar["\id \times\mult"']
                        &M \times M
                                \dar["\mult"] \\
                        M \times M
                                \rar["\mult"']
                        &M
                \end{tikzcd}
                \end{equation}
\end{itemize}
Further, the data of these relations is extended to higher powers of $M$ by
acting on other components by the identity:
\begin{equation}\label{eq:extend_monoid_identity}
\begin{tikzcd}[column sep=large]
        M^{n+3}
                \rar["\mult \times \id^{n+1}"]
                \dar["\id \times\mult\times \id^{n}"']
        &M^{n+2}
                \dar["\mult\times \id^{n}"] \\
        M^{n+2}
                \rar["\mult\times \id^{n}"']
        &M^{n+1}
\end{tikzcd}
\end{equation}

Let us alter how we package these data so as to maximize the clarity of the
meta-monoid structure:
\begin{enumerate}
        \item Instead of linear orders of factors $M \times \dots \times M$, we
                will index factors by a finite set $X$, writing it $M_X \defeq
                \set{\map {f} {X} {M}}$ in the style of
                \cref{eq:indexed_tensor}.\footnote{Indeed, when $\CC = \Vect$,
                these definitions are identical when the monoidal product is
                $\otimes$. In this case, they are called \defi{algebras}.%
        }
        \item The indexed factors will determine how the monoid operations act.
                For instance, multiplication of factor $i$ and $j$ together,
                with the result labelled in factor $k$ is to be written
                $\map {\mult^{ij}_{k}} {M_{\set{i,j}}} {M_{\set{k}}}$.
        \item Instead of implicitly including extensions of operations to higher
                powers by the identity, we will parametrize the extension by
                finite sets by $ϕ^{D}_{C}[X] \defeq ϕ^{D}_{C}\times\Id{X}$.
                For example, multiplication $\map {\mult^{ij}_{k}}
                {M_{\set{i,j}}} {M_{\set{k}}}$ generates a family of maps
                $\map {\mult^{ij}_k[X]} {M_{\set{i,j}\sqcup X}}
                {M_{\set{k}\sqcup X}}$, each of which must satisfy the relations
                of the monoid object such as \cref{eq:extend_monoid_identity}.
                (Again, the \enquote{$[X]$} is frequently omitted from writing.)
\end{enumerate}
This way of packaging the data leads us to the following generalization:
\begin{definition}\label{def:meta_monoid}
A \defi{meta-monoid} in $\CC$ is the following data:
\begin{itemize}
        \item A family of objects $M_X\in \CC$, indexed over finite sets $X$,
                with set bijections $\map {ψ} {X} [\toiso] {Y}$ inducing
                isomorphisms $M_X \iso M_Y$.
        \item A family of morphisms $\map {\mult^{ij}_{k}[X]} {M_{\set{i,j}\sqcup X}}
                {M_{\set{k}\sqcup X}}$ called \enquote{multiplication}.
        \item A family of \enquote{unit} morphisms
                $\map {\unit_{i}[X]} {M_X} {M_{\set{i}\sqcup X}}$.
        \item A collection of relations between the morphisms, written as
                equalities of morphisms between the $M_X$'s. In particular,
                associativity:
                \begin{equation}
                        \begin{tikzcd}[column sep=huge]
                                M_{\set{1,2,3}\sqcup X}
                                \rar["\mult^{1,2}_{1}\bk*{X\sqcup\set{3}}"]
                                \dar["\mult^{2,3}_{2}\bk*{X\sqcup\set{1}}"']
                        &M_{\set{1,3}\sqcup X}
                        \dar["\mult^{1,3}_{1}\bk{X}"] \\
                        M_{\set{1,2}\sqcup X}
                        \rar["\mult^{1,2}_{1}\bk{X}"']
                        &M_{\set{1}\sqcup X}
                        \end{tikzcd}
                \end{equation}
                and the identity:
                \begin{equation}\label{eq:monoid_unit}
                        \begin{tikzcd}[column sep=large]
                                M_{\set{1}\sqcup X}
                                \rar["\unit_{2}\bk{X}"]
                                \drar["\id"', bend right]
                                &M_{\set{1,2}\sqcup X}
                                \dar["\mult^{1,2}_{1}\bk{X}", shift left]
                                \dar["\mult^{2,1}_{1}\bk{X}"', shift right] \\
                                &M_{\set{1}\sqcup X}
                        \end{tikzcd}
                \end{equation}
\end{itemize}
\end{definition}

\begin{example}[monoid objects are meta-monoids]
Any monoid object $M$ in a strict, symmetric monoidal category
$(\CC,\otimes,\set{1})$ has the structure of a meta-monoid $\set{M_X}_X$ via
$M_X \defeq M^{\otimes X}$\footnote{%
        By $M^{\otimes X}$ we mean $M^{\otimes
        \ord{X}}$ together with a choice of assigning each factor an index,
        analogous to \cref{def:indexed_tensor_powers}.%
}, $\mult^{ij}_k[X] \defeq \mult^{ij}_k \otimes \Id{X}$, and
$\unit_i[X](v) \defeq 1_i \otimes v$ for any $v\in M^{\otimes X}$.
\end{example}

Consider the following structure, which satisfies the definition of a
meta-monoid, but is not a monoid in the traditional sense:
\begin{example}[the meta-monoid of square matrices]
Let $\K$ be a field and $M_X \defeq \Mat_{X\times X}(\K)$ be the set of square
matrices whose rows and columns are indexed by the finite set $X=\set{x_i}_i$.
Define
$\map {\mult^{ij}_{k}[X]}
        {M_{X\sqcup\set{i, j}}}
        {M_{X\sqcup\set{k}}}$
by
$\mult^{ij}_{k}[X]\pn[\big]{(a_{rs})_{rs}} \defeq
\pn*{a_{rs} + δ_{rk}(a_{is} + a_{js}) + δ_{sk}(a_{si} + a_{sj})}_{rs}$. That is,
the multiplication of two indices corresponds to the summation of their
respective rows and columns, the result of which is stored in row and column
$k$:
\begin{equation}
\begin{bmatrix}
a_{x_1,x_1} & \cdots & a_{x_1,i} & a_{x_1,j} \\
\vdots         & \ddots & \vdots         & \vdots         \\
a_{i,x_1} & \cdots & a_{ii} & a_{ij} \\
a_{j,x_1} & \cdots & a_{ji} & a_{jj}
\end{bmatrix}
\xrightarrow{\mult^{ij}_{k}}
\begin{bmatrix}
a_{x_1,x_1} & \cdots & a_{x_1,i}+a_{x_1,j} \\
\vdots         & \ddots & \vdots         & \\
a_{i,x_1}+a_{j,x_1} & \cdots & a_{ii}+a_{ji}+a_{ij}+a_{jj} \\
\end{bmatrix}
\end{equation}
Where the last column and row on the right-hand-side is indexed by $k$.
The unit $\unit_i[X]\pn[\big]{(a_{rs})_{rs}}$ extends $(a_{rs})_{rs}$ to
include a row and column of $0$'s, each labelled by the index $i$.
\end{example}

\begin{example}[tangles form a meta-algebra]
        Tangles are the main example of a meta-algebra which is not an algebra
        in the traditional sense. We go into more detail in
        \cref{sec:topological_interpretations}.
\end{example}

\ProvideDocumentCommand{\MM}{}{A}

In order to define other meta-objects (such as a meta-colagebra or a
meta-semigroup) we provide the following more general definition:

\begin{definition}[meta-object]
        Let $\CC$ be a category. A \defi{meta-object} in $\CC$ is four things:
        \begin{enumerate}
                \item A collection of objects $\MM_X$, one for each choice of
                        finite set $X$. (This serves as the analogue to monoidal
                        powers.)
                \item  For each bijection $\map {ψ} {X} [\toiso] {Y}$ of finite
                        sets $X$ and $Y$, a \defi{reindexing} isomorphism
                        $\map {ι_{ψ}} {\MM_X} [\toiso] {\MM_Y}$.
                \item A collection of \defi{operations} $ϕ_1, ϕ_2, \dots, ϕ_n$
                        each with a \defi{signature}
                        $\signature{ϕ_i}\in \Z_{\ge 0} \times \Z_{\ge 0}$. For
                        any pair of finite sets $(D, C)$ satisfying
                        $\pn*{\ord D, \ord C} = \signature{ϕ}$, we have a
                        morphism:
                        \begin{equation}
                                \map {ϕ^D_C} {\MM_D} {\MM_C}
                        \end{equation}
                \item For each operation $ϕ^{D}_{C}$, there is a
                        collection of morphisms $ϕ^{D}_{C}[\curry{}]$ indexed by
                        finite sets such that for each finite set $S$, $T$:
                        \begin{enumerate}
                                \item $\map {ϕ[\curry{S}]}
                                        {\MM_{C\sqcup S}} {\MM_{D\sqcup S}}$
                                \item $ϕ[\emptyset] = ϕ$
                                \item $\pn*{ϕ[S]}[T] = ϕ[S \sqcup T]$
                        \end{enumerate}
        \end{enumerate}

        When no ambiguity arises, we will omit the portion written in square
        brackets, so that $ϕ$ will stand for $ϕ[X]$, with the set $X$ determined
        from context.

        Finally, we may define the product of two spaces $\MM_{S}$ and $\MM_{T}$
        by $\MM_{S} \MM_{T} = \MM_{S \sqcup T}$. Given operations
        $ϕ^{D_1}_{C_1}$ and $ψ^{D_2}_{C_2}$ such that $D_1 \cap D_2 = \emptyset
        = C_1 \cap C_2$, we have a product morphism
        $\map {ϕ^{D_1}_{C_1}ψ^{D_2}_{C_2}}
                {\CC_{D_1 \sqcup D_2}}
                {\CC_{C_1\sqcup C_2}}$. This is visualized in
                \cref{fig:composition_example}.
\end{definition}

Composition of operators $ϕ^{D_1}_{C_1}$ and $ψ^{D_2}_{C_2}$ is defined when
$C_1 = D_2$:
\begin{equation}
        \map{ψ^{D_2}_{C_2}\circ ϕ^{D_1}_{C_1}}{\msc C_{D_1}}{\CC_{C_2}}
\end{equation}

\begin{remark}
        In this text, we will denote left-to-right composition with the
        symbol \enquote{$\then$} (pronounced \enquote{then}):
        $f \then g \defeq g \circ f$. Writing function
        composition in this order assists with readability when there are many
        functions to apply.
\end{remark}

\begin{remark}
        To make expressions easier to read, we introduce the domain extension
        implicitly in the following context:
        given morphisms $ϕ^{D_1}_{C_1}$ and $ψ^{D_2}_{C_2}$ such that
        % $D_2 \subseteq C_1$ and
        $C_2 \cap (C_1\setminus D_2)=\emptyset=D_1\cap(D_2 \setminus C_1)$, we
        define:
        \begin{equation}\label{eq:composition_extension}
                ϕ^{D_1}_{C_1} \then ψ^{D_2}_{C_2}
                \defeq
                {ϕ^{D_1}_{C_1}[D_2\setminus C_1]} \then
                {ψ^{D_2}_{C_2}[C_1\setminus D_2]}
        \end{equation}
        \Cref{fig:composition_extension} visualizes this extension.
\end{remark}
\begin{figure}[h]
        \centering
        \includegraphics{figures/composition_example.pdf}
        \caption{We may visualize a composition of morphisms with a graphical
                calculus. This graphic represents $(ϕ_1)^{D_1}_{C_1}\then
                (ϕ_2)^{D_2}_{C_2}$ when $C_1 = D_2$. Each arrow represents one
                factor. Indices are written in grey boxes.
        }
        \label{fig:composition_example}
\end{figure}
\begin{figure}[h]
        \centering
        \includegraphics{figures/composition_extension.pdf}
        \caption{Visual mnemonic for extending morphisms. This graphic
        represents \cref{eq:composition_extension}.}
        \label{fig:composition_extension}
\end{figure}
The two extreme cases of this definition are:
\begin{itemize}
        \item When $C_1 \cap D_2 = \emptyset$, \cref{eq:composition_extension}
                becomes $ϕ^{D_1}_{C_1}ψ^{D_2}_{C_2}$.
        \item When $C_1 = D_2$, \cref{eq:composition_extension} becomes the
                composition $ϕ^{D_1}_{C_1}\then ψ^{D_2}_{C_2}$ exactly.
\end{itemize}

\section{Algebraic definitions}\label{sec:alg_defs}

We now introduce the algebraic structures which will be used to define the
tangle invariant. These definitions follow those given by Majid in \cite{SM},
presented in a way that their corresponding meta-structure are readily visible.

\begin{definition}[meta-algebra]\label{def:meta_algebra}
        A \defi{meta-algebra} (or \defi{meta-monoid}\footnote{%
                This is a repeat of \cref{def:meta_monoid}. The only difference
                between an algebra object and a monoid object is the presence of
                a linear structure.%
        }) is a collection of objects
        $\set{A_X}_X$ in $\CC$ together with an associative multiplication
        $\map {\mult^{i,j}_{k}} {A_{\set{i,j}}} {A_{\set{k}}}$ (satisfying
        \cref{eq:cd_mult}), and a unit
        $\map{\unit_{i}}{A_{\emptyset}}{A_{\set{i}}}$ satisfying
        \cref{eq:cd_unit}.
\end{definition}
\begin{remark}
When $\CC = (\Vect,\otimes)$ and $A_X = V_X = V^{\otimes X}$ for some vector
space $V$, \cref{def:meta_algebra} becomes the more familiar definition of an
\defi{algebra}. Then $A_\emptyset$ is a field. It is more common think of the
unit as an element $\one\in V$. The unit map is then defined by linearly
extending the assignment $\unit_{i}(1) = \one_{i}$.
\end{remark}

\begin{multicols}{2}\noindent
\begin{equation}\label{eq:cd_mult}
\begin{tikzcd}
        A_{\set{1,2,3}}
                \rar["\mult^{1,2}_{1}"]
                \dar["\mult^{2,3}_{2}"']
        &A_{\set{1,3}}
                \dar["\mult^{1,3}_{1}"] \\
        A_{\set{1,2}}
                \rar["\mult^{1,2}_{1}"']
        &A_{\set{1}}
\end{tikzcd}
\end{equation}
\columnbreak
\begin{equation}\label{eq:cd_unit}
\begin{tikzcd}[column sep=large]
        A_{\set{1}}
                \rar["\unit_{2}"]
                \drar["\id"']
        &A_{\set{1,2}}
                \dar["\mult^{1,2}_{1}", shift left]
                \dar["\mult^{2,1}_{1}"', shift right] \\
        &A_{\set{1}}
\end{tikzcd}
\end{equation}
\end{multicols}

\begin{remark}
        Associativity allows us to denote repeated multiplication by using extra
        indices. For instance:
        $\mult^{i,j, k}_{\ell} \defeq \mult^{i,j}_{r}\then\mult^{r, k}_{\ell}
        = \mult^{j, k}_{s}\then\mult^{i, s}_{\ell}$.
\end{remark}

There is also the dual notion of a \emph{coalgebra}, which arises by reversing
the arrows in \cref{eq:cd_mult,eq:cd_unit}:

\begin{definition}[meta-coalgebra]
        A \defi{meta-colagebra} (or \defi{meta-comonoid}) is a collection
        $\set{C_X}_X$ together with a \defi{comultiplication} $\map
        {\comult^{i}_{jk}} {C_{\set{i}}} {C_{\set{j,k}}}$ which is
        \defi{coassociative} (\cref{eq:cd_comult}) and a \defi{counit}, which is a
        map $\map {\counit^{i}} {A_{i}} {A_{\emptyset}}$ satisfying
        \cref{eq:cd_counit}.
\end{definition}
\nopagebreak
\begin{multicols}{2}\noindent
\begin{equation}\label{eq:cd_comult}
\begin{tikzcd}
        C_{\set{1,2,3}}
        &C_{\set{2,3}}
                \lar["\comult^{1}_{{1,2}}"'] \\
        C_{\set{1,2}}
                \uar["\comult^{2}_{2,3}"]
        &C_{\set{1}}
                \lar["\comult^{1}_{1,2}"]
                \uar["\comult^{1}_{1,3}"']
\end{tikzcd}
\end{equation}
\columnbreak
\begin{equation}\label{eq:cd_counit}
\begin{tikzcd}
        C_{\set{1}}
        &C_{\set{1,2}}
                \lar["\counit^{2}"']\\
        &C_{\set{1}}
                \ular["\id", shift left]
                \uar["\comult^{1}_{1,2}", shift left]
                \uar["\comult^{1}_{2,1}"', shift right] \\
\end{tikzcd}
\end{equation}
\end{multicols}

\begin{remark}
        Coassociativity allows us to denote repeated comultiplication by using
        extra indices. For instance:
        $\comult^{i}_{j, k, \ell}
        \defeq \comult^{i}_{j,r}\then\comult^{r}_{k, \ell}
        = \comult^{i}_{s,\ell}\then\comult^{s}_{j, j}$.
\end{remark}

If a meta-object $\set{B_X}_x$ has the structure of both an algebra and a
coalgebra, we introduce a definition for when the structures are compatible with
each other:

\begin{definition}[meta-bialgebra]
        A \defi{meta-bialgebra} (or \defi{meta-bimonoid}) is a meta-algebra
        $(B,\mult,\unit)$ and a meta-coalgebra
        $(B,\comult,\counit)$, such that $\comult$ and $\counit$ are
        meta-algebra morphisms.
        \footnote{
                $B_{X}$ inherits a meta-(co)algebra structure from $B$, given by
                $(B_X)_Y \defeq B_{X^Y}$ and component-wise operations. The
                bialgebra structure on $B_{\emptyset}$ is given by
                $\mult = \unit = \comult = \counit = \id$.
        }
\end{definition}

\ProvideDocumentCommand{\lift}{mm}{\curryIsolated{#1}^{(#2)}}

\begin{multicols}{2}\noindent
\begin{equation}\label{eq:cd_mult_comult}
\begin{tikzcd}[column sep=large]
        B_{\set{1,2}}
                \rar["\mult^{1,2}_{1}"]
                \dar["\comult^{1}_{1,3}\then\comult^{2}_{2,4}"]
        &B_{\set{1}}
                \dar["\comult^{1}_{1,2}"] \\
        B_{\set{1,2,3,4}}
                \rar["\mult^{1,2}_{1}\then\mult^{3,4}_{2}"']
        &B_{\set{1,2}}
\end{tikzcd}
\end{equation}\begin{equation}\label{eq:cd_unit_comult}
\begin{tikzcd}[row sep=tiny]
        &B_{\set{1}}
                \ar[dd,"\comult^{1}_{1,2}"] \\
        B_{\emptyset}
                \urar["\unit_{1}"]
                \drar["\unit_{1}\then\unit_{2}"',near end]\\
        &B_{\set{1,2}}
\end{tikzcd}
\end{equation}
\columnbreak
\begin{equation}\label{eq:cd_mult_counit}
\begin{tikzcd}[column sep=tiny]
        B_{\set{1,2}}
                \ar[rr,"\mult^{1,2}_{1}"]
                \drar["\counit^{1}\then\counit^{2}"']
        &&B_{\set{1}}
                \dlar["\counit^{1}"] \\
        &B_{\emptyset}
\end{tikzcd}
\end{equation}
\begin{equation}\label{eq:cd_unit_counit}
\begin{tikzcd}
        B_{\emptyset}
                \rar["\unit_{1}"]
                \drar["\id"']
        &B_{\set{1}}
                \dar["\counit^{1}"] \\
        &B_{\emptyset}
\end{tikzcd}
\end{equation}
\end{multicols}

\begin{remark}
        The conditions for $\comult$ being an algebra morphism are presented in
        \cref{eq:cd_mult_comult,eq:cd_unit_comult}, while those for $\counit$
        are in \cref{eq:cd_mult_counit,eq:cd_unit_counit}. Observing invariance
        under arrow reversal, it may not come as a surprise that
        \cref{eq:cd_mult_comult,eq:cd_mult_counit} also are the conditions for
        $\mult$ being a coalgebra morphism, and
        \cref{eq:cd_unit_comult,eq:cd_unit_counit} tell us that $\unit$ is as
        well.
\end{remark}

Next, we introduce a notion of invertibility which extends a (meta-)bialgebra to
a (meta-)Hopf algebra:
\begin{definition}[meta-Hopf algebra]
A \defi{meta-Hopf algebra} (or \defi{meta-Hopf monoid}) is a meta-bialgebra $H$
together with a map $\map {\antipode} {H} {H}$ called the \defi{antipode}, which
satisfies
$\comult^{1}_{1,2}\then \antipode^1_1 \then \mult^{1,2}_1 =
\counit^{1}\then\unit_{1} =
\comult^{1}_{1,2}\then \antipode^2_2 \then \mult^{1,2}_1$.
As a commutative diagram, this looks like \cref{eq:cd_antipode}
\begin{equation}
\begin{tikzcd}[column sep=tiny]\label{eq:cd_antipode}
        H_{\set{1}}
                \arrow[rr, "\counit^{1}"] \arrow[rd, "\comult^{1}_{1,2}"']
        && H_{\emptyset}
                \arrow[rr, "\unit_{1}"]
        && H_{\set{1}} \\
        & H_{\set{1,2}}
                \arrow[rr, "\antipode^{2}_{2}", shift left]
                \arrow[rr, "\antipode^{1}_{1}"', shift right]
        && H_{\set{1,2}} \arrow[ru, "\mult^{1,2}_{1}"']
\end{tikzcd}
\end{equation}
\end{definition}

In order to do knot theory, we need an algebraic way to represent a crossing of
two strands. This is accomplished by the $\Rmat$-matrix:

\begin{definition}[quasitriangular meta-Hopf algebra]
A \defi{quasitriangular meta-Hopf algebra} (or \defi{quasitriangular meta-Hopf
monoid}) is a meta-Hopf algebra $H$, together with an invertible element
$\Rmat_{i,j} \in H_{i,j}$, called the \defi{$\Rmat$-matrix}, which satisfies the
following properties: (we will denote the inverse by $\Rmati$)
\begin{align}
        \label{eq:Rmat_understrand}
        \Rmat_{13}\then\comult^{1}_{12}&=\Rmat_{13}\Rmat_{24}\then\mult^{34}_3\\
        \label{eq:Rmat_overstrand}
        \Rmat_{13}\then\comult^{3}_{23}&=\Rmat_{13}\Rmat_{42}\then\mult^{14}_1\\
        \label{eq:Rmat_comult}
        \comult^{1}_{21} &=
                \comult^{1}_{12} \Rmat_{a_1,a_2}\Rmati_{p_1,p_2}\then
                \mult^{a_1,1,p_1}_{1}\then \mult^{a_2,2,p_2}_{2}
\end{align}
\end{definition}

\begin{definition}[Drinfeld element]
        In a quasitriangular meta-Hopf algebra $H$, the \defi{Drinfeld element},
        $\dfe \in H$ is:
        \begin{equation}
                \dfe \defeq \Rmat_{21}\then\antipode^1_1 \then \mult^{12}
        \end{equation}
\end{definition}

\begin{definition}[monodromy]
        Each quasitriangular meta-Hopf algebra has a \defi{monodromy}
        $\monodromy_{12} \defeq
        \Rmat_{12}\Rmat_{34}\then\mult^{14}_{1}\then\mult^{23}_{2}$. Its
        inverse will be denoted
        $\invb\monodromy_{12} =
        \Rmati_{12}\Rmati_{34}\then\mult^{14}_{1}\then\mult^{23}_{2}$.
\end{definition}

\begin{definition}[ribbon meta-Hopf algebra]
        A quasitriangular meta-Hopf algebra $H$ is called \defi{ribbon} if it has an
        element $\ribbon\in \centre(H)$ such that:
        \begin{align}
                \ribbon_1\ribbon_2\then\mult^{12}
                &= \dfe_1 \dfe_2 \then \antipode^{2}_{2} \then \mult^{12}\\
                \ribbon_1 \then \comult^1_{12}
                &=      \ribbon_1\ribbon_2
                        \then\invb\monodromy_{34}
                        \then\mult^{13}_{1}
                        \then\mult^{24}_{2} \\
                \ribbon \then \antipode &= \ribbon\\
                \ribbon \then \counit &= \unit \then \counit = 1
        \end{align}
\end{definition}

\begin{definition}[spinner]
        A \defi{spinner}\footnote{%
                These are more commonly referred to as \defi{distinguished
                grouplike elements}. The term we use is inspired by
                \cref{fig:spinner}
        }) in a ribbon meta-Hopf algebra $H$ is an invertible element
        $\spin\in H$ (with inverse $\invb\spin$) such that for all $x\in H$:
        \begin{align}
                \label{eq:spinner_ribbon}
                \spin_1\ribbon_2\spin_3 \then \antipode^2_2 \then \mult^{123} &=
                \ribbon\\
                \label{eq:spinner_comult}
                \spin_1\then\comult^{1}_{12} &=\spin_1\spin_2\\
                \label{eq:spinner_antipode}
                \spin \then\antipode &= \invb\spin\\
                \label{eq:spinner_conjugate}
                \spin_{1}x_2\invb\spin_{3}\then\mult^{123} &=
                x \then \antipode \then \antipode\\
                \label{eq:spinner_counit}
                \spin \then \counit &= \unit \then \counit = 1
        \end{align}
\end{definition}

\begin{lemma}[spinners and ribbon Hopf algebras]\label{lem:spinner_ribbon}
        If a (meta-)Hopf algebra has either a ribbon element $\ribbon$ or a spinner
        $\spin$, then it must have the other as well, given by the formula:
        $\spin_1 \ribbon_2 \then \mult^{12} = \dfe$.
\end{lemma}
\begin{proof}
        See Majid's work in \cite{SM} or Etingof and Schiffmann in \cite{ES}
        for more details on this standard result. Note that the proof does not
        rely on the additive structure of the Hopf algebra, which allows us to
        extend this result to the realm of meta-Hopf algebras.
        %TODO: finish
\end{proof}
% todo: is this lemma important for the thesis? If not, cut it out!

\section{The meta-algebra of tangle diagrams}
\label{sec:topological_interpretations}

The particular structures introduced were chosen for their ability to represent
the topological properties of knotted objects. We will now introduce the notion
of a tangle and demonstrate its meta-algebraic structure.

\subsection{Upright tangles}

For our purposes, a tangle will be visualised as follows: take a stiff
(topologically) circular metal frame forming a Jordan curve (i.e. with a defined
inside and outside), then attach a collection of strings to the wire, ensuring
that the strings always remain inside the wire, and that each string is tied to
the metal frame in two unique locations (that is, no two strings share an
endpoint).
\begin{figure}[h]
        \centering
        \includegraphics{figures/tangle_example.pdf}
        \caption{Example of a tangle with strands labelled $1$ and $2$.}
        \label{fig:tangle_example}
\end{figure}

\begin{definition}[open tangle]
        An \defi{open tangle} is an embedding of line segments (called
        \defi{components} or \defi{strands}) into the thickened unit disk
        $D \times [-1,1]$ (or a disjoint union of such disks) such that the
        endpoints of the line segments are fixed along
        $\boundary D \times \set0$. Each strand is labelled with elements of a
        set $X$. Two open tangles are considered equivalent if there exists an
        isotopy of the embedding which fixes the endpoints of the strands. The
        set of all tangles with strands indexed by $X$ will be denoted
        $\tangleDown X$. (The term \enquote{open} refers to the absence of
        closed loops.)
\end{definition}

The objects which are more natural for us to study are tangles with a framing,
which one may thing of as open tangles with the strings replaced with thin
ribbons.

\begin{definition}[framed tangle]
        A \defi{framed tangle} is an open tangle together with a choice of
        section of the normal bundle for each component, with endpoints of the
        section fixed pointing to the right of the tangent vector. This choice
        is taken up to endpoint-fixing homotopy. Unless otherwise mentioned, it
        will be assumed that all tangles are framed.
\end{definition}

Observe that a generic projection of a tangle to its \defi{central core}
$D\times \set 0$ results in the strands forming a graph, with each crossing of
two strands in the tangle producing a vertex in the graph. By assigning to each
vertex the sign of the corresponding crossing (either \enquote{positive} or
\enquote{negative}), we end up with a combinatorial object which is equivalent
to the original tangle.

\begin{definition}[open tangle diagram]\label{def:open_tangle_diagram}
        An \defi{open tangle diagram} is a projection of a tangle onto its
        central core such that all the line segments are immersions which
        intersect both the boundary disk and the other strands transversally,
        together with an assignment of a sign to each strand intersection.
        Small open neighbourhoods of these intersections are called
        \defi{crossings}, while the complement of the crossings is a collection
        of embedded line segments called \defi{arcs}.

        Two open tangle diagrams are considered equivalent if they differ by a
        finite sequence of Reidemeister moves, as outlined in
        \crefrange{fig:R1p}{fig:R3}
\end{definition}

\begin{figure}[h]
        \centering
        \includegraphics{figures/R1p.pdf}
        \caption{(Framed) Reidemeister move $R1'$}
        \label{fig:R1p}
\end{figure}
\begin{figure}[h]
        \centering
        \includegraphics{figures/R2.pdf}
        \caption{Reidemeister move $R2$}
        \label{fig:R2}
\end{figure}
\begin{figure}[h]
        \centering
        \includegraphics{figures/R3.pdf}
        \caption{Reidemeister move $R3$}
        \label{fig:R3}
\end{figure}

The rotation numbers of arcs will play a role in this thesis, so we will capture
these data in the following way (as described in \cite{BV23}):

\begin{definition}[upright open tangle diagram]
        An \defi{upright tangle diagrams} is a tangle diagram with the further
        requirement that the endpoints of each arc must have a vertical tangent
        vector, and each crossing must involve only curves with tangent vectors
        that point (diagonally) upwards. Here, each arc has well-defined integer
        rotation number. Two tangles are considered equivalent if they agree
        under the \enquote{rotational Reidemeister moves}, which are
        \crefrange{fig:R1p}{fig:whirl}. Given a finite set $X$, the set of
        $X$-indexed upright tangle diagrams will be denoted $\tangle X$.
\end{definition}
\begin{remark}
        The concept of upright tangles was first introduced by Louis Kauffman in
        \cite{LK} under the name \emph{rotational virtual knot theory}. In the
        formulation here, we insist that all strands end pointing upwards
        instead of merely requiring that endpoint vectors are vertical, so we
        will use the term \enquote{upright} to remind the reader of this
        difference.
\end{remark}
\begin{figure}[h]
        \centering
        \includegraphics{figures/R2rot.pdf}
        \caption{The (cyclic) rotational Reidemeister move $R2_{\text{rot}}$}
        \label{fig:R2rot}
\end{figure}
\begin{figure}[h]
        \centering
        \includegraphics{figures/whirl.pdf}
        \caption{The whirling move}
        \label{fig:whirl}
\end{figure}

Fortunately, ambient isotopy allows us to rotate any classical tangle into an
upward-pointing form. Additionally, there is only one way to do this. We
reproduce the proof of this fact by Bar-Natan and van der Veen in \cite{BV}
below:

\begin{lemma}[tangles inject into upright tangles]
        To each open tangle diagram $D$ there exists an upright open tangle
        diagram $D'$ obtained from $D$ by a planar isotopy. Further, if $D''$ is
        another such upright open tangle diagram obtained from $D$, then $D'$
        and $D''$ differ by a finite sequence of rotational Reidemeister moves
        and a change of rotation number at the endpoints.
\end{lemma}
\begin{proof}
        Each arc and crossing in the diagram $D$ may be rotated so that its
        endpoints are pointing upwards, giving rise to a diagram $D'$. Two
        (nonupright) tangle diagrams are equivalent when they differ by a finite
        sequence of Reidemeister moves. Each of these Reidemeister may also be
        rotated to an equivalence of upright tangles, each of which is given as
        a rotational Reidemeister move \crefrange{fig:R1p}{fig:R2rot}. The
        last possibility is the rotation of an entire crossing, which is covered
        by \cref{fig:whirl}.
\end{proof}

\subsection{The meta-algebra structure of upright tangle diagrams}

We now formally connect tangle diagrams with meta-algebras.

\begin{theorem}[tangles form a ribbon meta-Hopf algebra]
        \label{thm:tangles_meta_algebra}
        The collection $\set{\tangle X}_X$ forms a quasitriangular ribbon
        meta-Hopf algebra (in the category $\Set$) with the following
        operations:
        \begin{itemize}
                \item multiplication $\mult^{ij}_{k}[X]$ takes a tangle with
                        strands $X\sqcup\set{i,j}$ and glues the end of strand
                        $i$ to strand $j$, labelling the resulting strand
                        $k$.\footnote{Strictly speaking, this operation is only
                        defined when the end of strand $i$ is adjacent to strand
                        $j$. See \cref{rem:virtual_caveat} for more details.%
                }
                \item the unit $\unit_{i}[X]$ takes a tangle diagram with
                        strands $X$ and introduces a new strand $i$ which does
                        not touch any of the other strands.
                \item the comultiplication $\comult^{i}_{jk}[X]$ takes a tangle
                        with strands $X\sqcup \set{i}$ and doubles strand $i$,
                        separating the two strands along the framing of strand
                        $i$, calling the right strand $j$ and the left one
                        $k$.\footnote{While this convention appears unfortunate,
                        we follow the notation laid out in \cite{BV} so that the
                        antipode and spinner have a more memorable
                        representation, namely looking like the letters they are
                        represented by (see \cref{thm:rvt_metaHopf} for more
                        details).
                }
                \item the counit $\counit^{i}[X]$ takes a tangle with strands
                        indexed by $X\sqcup \set{i}$ and returns the tangle with
                        strand labelled by $i$ deleted.
                \item The antipode $\antipode^{i}_{j}[X]$ takes a tangle with
                        strands labelled by $X \sqcup \set{i}$ and reverses the
                        direction of strand $i$, then adds a counter-clockwise
                        cap to the new beginning, and a clockwise cup to the
                        end. This new strand is called $j$. When applied to a
                        single vertical strand, the resulting tangle looks like
                        the letter \enquote{S}.
                \item the $\Rmat$-matrix $\Rmat_{ij}$ is given by the two-strand
                        tangle with a single positive crossing of strand $i$
                        over strand $j$. The inverse $\Rmat$-matrix
                        $\Rmati_{ij}$ is the two-strand tangle with a
                        \emph{negative} crossing of strand $i$ over strand $j$.
                \item The spinner $\spin_i[X]$ takes a tangle in $\tangle X$ and
                        adds a new strand with rotation number $1$ which has no
                        interactions with any other strands. This new strand
                        looks like the letter \enquote{C}.
        \end{itemize}
\end{theorem}
\begin{remark}\label{rem:virtual_caveat}
        One may object that strand-stitching $\mult^{ij}_k$ is not defined when
        the endpoint of strand $i$ is not adjacent to the starting point of
        strand $j$. This issue is resolved in multiple ways:
        \begin{enumerate}
                \item Extend the collection of tangles we work with to include
                        \defi{virtual tangles}. This generalization of tangles
                        deals exactly with the issue that multiplication need
                        not produce a planar tangle diagram. In fact, virtual
                        tangles can be thought of as merely non-planar tangle
                        diagrams.
                \item Commit to only apply multiplication when doing so would
                        result in a valid (classical) tangle. This is the
                        approach we will take when performing computations on
                        tangles.
        \end{enumerate}
\end{remark}
\begin{figure}[h]
        \centering
        \includegraphics[width=0.8\textwidth]{figures/tangle_mult.pdf}
        \caption{
                Multiplication $\mult^{ij}_{k}$ stitches two strands in a tangle
                together.
        }
        \label{fig:tangle_mult}
\end{figure}
\begin{figure}[h]
        \centering
        \includegraphics[width=0.8\textwidth]{figures/tangle_unit.pdf}
        \caption{The unit $\unit_i$ introduces a new strand in a tangle.}
        \label{fig:tangle_unit}
\end{figure}
\begin{figure}[h]
        \centering
        \includegraphics[width=0.8\textwidth]{figures/tangle_counit.pdf}
        \caption{The counit $\counit^i$ deletes a strand in a tangle.}
        \label{fig:tangle_counit}
\end{figure}
\begin{figure}[h]
\centering
\includegraphics[width=0.8\textwidth]{figures/tangle_comult.pdf}
\caption{The comultiplication $\comult^{i}_{jk}$ doubles a strand in a tangle
along its framing. Notice the right-to-left strand labels.}
\label{fig:tangle_comult}
\end{figure}
\begin{figure}[h]
        \centering
        \begin{subfigure}[b]{0.4\textwidth}
                \centering
                \includegraphics{figures/tangle_rmat.pdf}
                \caption{A positive crossing, represented by $\Rmat_{ij}$}
                \label{fig:tangle_rmat}
        \end{subfigure}
        \begin{subfigure}[b]{0.4\textwidth}
                \centering
                \includegraphics{figures/tangle_rmati.pdf}
                \caption{A negative crossing, represented by $\Rmati_{ij}$}
                \label{fig:tangle_rmati}
        \end{subfigure}
        \caption{The $\Rmat$-matrix and its inverse represent a tangle with a
        single crossing.}
\end{figure}
\begin{figure}[h]
        \centering
        \includegraphics{figures/tangle_drinfeld.pdf}
        \caption{The Drinfeld element $\dfe_i$ in the meta-Hopf algebra of
        tangles.}
        \label{fig:tangle_drinfeld}
\end{figure}
\begin{figure}[h]
        \centering
        \includegraphics{figures/monodromy.pdf}
        \caption{The monodromy in the meta-Hopf algebra of tangles.}
        \label{fig:monodromy}
\end{figure}
\begin{figure}[h]
        \centering
        \begin{subfigure}[b]{0.4\textwidth}
                \centering
                \includegraphics{figures/tangle_spin.pdf}
                \caption{The spinner $\spin_i$ has rotation number $1$.}
                \label{fig:tangle_spin}
        \end{subfigure}
        \begin{subfigure}[b]{0.4\textwidth}
                \centering
                \includegraphics{figures/tangle_spini.pdf}
                \caption{The inverse spinner $\spini_i$ has rotation number
                $-1$.}
                \label{fig:tangle_spini}
        \end{subfigure}
        \caption{The spinners represent strands with a unit rotation number.}
        \label{fig:spinner}
\end{figure}
\begin{figure}[h]
        \centering
        \includegraphics{figures/tangle_ribbon.pdf}
        \caption{A ribbon element $\ribbon_i$ in the meta-Hopf algebra of
                tangles. One can use \cref{lem:spinner_ribbon} to verify this is
                compatible with the spinner.
        }
        \label{fig:tangle_ribbon}
\end{figure}

\begin{figure}[h]
        \centering
        \includegraphics{figures/Rmat_overstrand.pdf}
        \caption{Example of a tangle satisfying \cref{eq:Rmat_overstrand}}
        \label{fig:Rmat_overstrand}
\end{figure}
\begin{figure}[h]
        \centering
        \includegraphics{figures/Rmat_understrand.pdf}
        \caption{Example of a tangle satisfying \cref{eq:Rmat_understrand}}
        \label{fig:Rmat_understrand}
\end{figure}
\begin{proof}
        Associativity of multiplication (\cref{eq:cd_mult}) follows from the
        fact that stitching strands together amounts to concatenating the order
        of the crossings each strand interacts with. Since list concatenation is
        an associative operation, associativity follows in this case as well.

        Adding a non-interacting strand to a diagram, then stitching it to an
        existing strand (\cref{eq:cd_unit}) does not change any of the
        combinatorial data in the diagram, and results in identical diagrams.

        Establishing coassociativity (\cref{eq:cd_comult}) amount to the same
        argument that cutting a piece of paper into three strips does not depend
        on the order of cutting.

        The counit identity (\cref{eq:cd_counit}) states deleting a strand
        is the same operation as first doubling it, then deleting both resulting
        strands.

        The meta-bialgebra axioms we verify next:

        \Cref{eq:cd_mult_comult} states that if two strands are stitched
        together, then the resulting strand is doubled, this could have
        equivalently been achieved by doubling each of the original strands,
        then performing a stitching on both resulting pairs of strands.

        \Cref{eq:cd_mult_counit} simply states that stitching two strands
        together, then removing the resulting strand could have equally been
        achieved by removing both of the original strands without stitching them
        first.

        \Cref{eq:cd_unit_counit} states that introducing a strand, then
        immediately removing it is the identity operation.

        \Cref{eq:cd_unit_comult} says that doubling a newly-introduced (and
        therefore free of crossings) strand is the same operation as introducing
        two strands separately. (For those worried that this equation depends on
        the location of the separately introduced strands, this is one place
        that the use of virtual tangles will be used, which does not heed the
        relative locations of disjoint strands.)

        \Cref{eq:cd_antipode} states that when a strand is doubled, then one of
        the two strands is reversed, multiplying the two strands together
        results in a strand which can be rearranged to not interact with any of
        the other strands. This can be readily seen, as this newly-created
        strand looks like a snake weaving through the tangle diagram. One can
        remove the snake by applying a series of Reidemeister 2 moves, resulting
        in a strand disjoint from the rest of the diagram. This is the same as
        deleting the original strand, then introducing a new disjoint one.

        The quasitriangular axioms are equalities of pairs of three-strand
        tangles:
        \begin{itemize}
                \item \Cref{eq:Rmat_overstrand,eq:Rmat_understrand} tell us that
                        doubling a strand involved in a single crossing can also
                        be built by adjoining two crossings together.
                \item \Cref{eq:Rmat_comult} tells us that we can swap the order
                        of a doubled strand by adding crossings to either end
                        (reminiscent of a Reidemeister 2 move)
        \end{itemize}

        Finally, we observe that the quotient we introduce to tangle diagrams by
        the Reidemeister moves does not introduce any new relations.
        Reidemeister 2 follows from the invertibility of the $\Rmat$-matrix.
        Next, it is readily seen that the quasitriangular relations governing
        the $\Rmat$-matrix force it to solve the Yang-Baxter equation, which is
        one equivalent to the Reidemeister 3 in this case.

        Using \cref{lem:spinner_ribbon}, it is enough to verify the spinner
        axioms
        (\cref{eq:spinner_ribbon,eq:spinner_comult,eq:spinner_antipode,eq:spinner_conjugate,eq:spinner_counit}).
        All these axioms have corresponding pictures one can draw, keeping in
        mind the orientations in the definitions of the relevant operations.
\end{proof}

\section{The $ybax$ meta-algebra}\label{sec:algebraic-definitions}

Here we define the ribbon Hopf algebra $\CU$, and point out some of its
properties.

\begin{definition}[The ribbon Hopf algebra $\CU$]
Define the Lie algebra
\begin{equation}
\fg \defeq \Span\setbuilder[\Big]{\yo, \bo, \ao, \xo}{
        \liebk{\ao}{\xo} = \xo,
        \liebk{\ao}{\yo} = -\yo,
        \liebk{\xo}{\yo} = \bo,
        \liebk{\bo}{ } = 0
}
\end{equation}
Then the algebra $\CU$ is defined to be the universal enveloping algebra
$\uea{\fg}$. The bialgebra structure of $\CU$ is: \begin{equation}\begin{aligned}
        \comult_{i,j}(\yo) &=
        \frac{\bo_i+\bo_j}{1-\Bo_i\Bo_j} \pn*{
                \Bo_j\frac{1-\Bo_i}{\bo_i}\yo_i+
                \frac{1-\Bo_j}{\bo_j}\yo_j
        }\\
        \comult_{i,j}(\bo) &= \bo_i + \bo_j\\
        \comult_{i,j}(\ao) &= \ao_i + \ao_j\\
        \comult_{i,j}(\xo) &= \xo_i + \xo_j\\
\end{aligned}\end{equation}
For any $\zo\in\set{\yo, \bo, \ao, \xo}$, we have $\counit(\zo)=0$ (extended
multiplicatively by \cref{eq:cd_mult_counit}).

Next, we define the Hopf algebra structure by defining the antipode, which is
defined as $\antipode(\zo) \defeq -\zo$ for each
$\zo\in\set{\yo, \bo, \ao, \xo}$, extended antimultiplicatively.

Next, we introduce the ribbon structure of $\CU$ with an $\Rmat$-matrix and
the spinner $\spin$:
\begin{align}
        \Rmat_{i,j}
                &\defeq \exp\pn*{\bo_i\ao_j}
                        \exp\pn*{\frac{1-\Bo_i}{\bo_i}\yo_i\xo_j}\\
        \spin &\defeq \sqrt{\Bo}\\
        \ribbon &\defeq \Rmati_{31}\spini_{2} \then \mult^{123}
\end{align}
\end{definition}

\begin{lemma}[Commutation relations in $\CU$]\label{lem:xay_relations}
        Given a polynomial $f$, we have the
        following relations in $U$:
        \begin{align}\label{eq:xay_relations}
                f(a)y^r &= y^rf(a-r) &
                x^rf(a) &= f(a-r)x^r
        \end{align}
\end{lemma}
\begin{proof}
        We begin by commuting a single $a$ past a power of $y$:
        \begin{equation}
                ay^r
                =y(a-1)y^{r-1}
                =y^2(a-2)y^{r-2}
                =\dots
                =y^{r}(a-r)
        \end{equation}
        Repeating the above process for multiple copies of $a$:
        \begin{equation}
                a^ky^r
                =a^{k-1}(ay^r)
                =a^{k-1}y^r(a-r)
                =a^{k-2}y^r(a-r)^2
                =\dots
                =y^r(a-r)^k
        \end{equation}
        Finally, we conclude for any linear combination of powers of $a$, the
        same relation holds, so this holds for any polynomial. Extending this
        result to formal power series is straightforward.

        A similar argument exists to show $x^rf(a) = f(a-r)x^r$.
\end{proof}

Next, we observe that since $\liebk{x}{y}$ is central, the Weyl canonical
commutation relation holds:
\begin{equation}\label{eq:weyl_relation}
        \Exp{ξ\xo}\Exp{η\yo} = \Exp{η\yo}\Exp{ξ\xo}\Exp{ξη\bo}
\end{equation}
Secondly, using \cref{eq:xay_relations} and setting $\A\defeq \Exp{α}$, we
notice
\begin{equation}\label{eq:ay_commutation}
        \Exp{α\ao}\Exp{η\yo}
        = \Exp{α\ao}\Sum[n]\frac{(η\yo)^n}{n!}
        = \Sum[n]\frac{(η\yo)^n}{n!}\Exp{α(\ao-n)}
        = \Sum[n]\frac{(η\yo)^n}{n!}\Exp{α\ao}\A^{-n}
        = \Exp{\frac{η}{\A}\yo}\Exp{α\ao}
\end{equation}
similarly,
\begin{equation}\label{eq:ax_commutation}
        \Exp{ξ\xo}\Exp{α\ao}
        = \Sum[n]\frac{(ξ\xo)^n}{n!}\Exp{α\ao}
        = \Sum[n]\Exp{α(\ao-n)}\frac{(ξ\xo)^n}{n!}
        = \Sum[n]\Exp{α\ao}\frac{\pn{\frac{ξ\xo}{\A}}^n}{n!}
        = \Exp{α\ao}\Exp{\frac{ξ}{\A}\xo}
\end{equation}

\begin{lemma}[the algebra $\CU$ is ribbon]
        The algebra $\CU$ has a ribbon structure given by the above
        $\Rmat$-matrix and spinner $\spin$.
\end{lemma}
\begin{proof}
        The Hopf algebra structure of $\CU$ is straightforward, and is left to
        the reader to verify. We will focus our attention on verifying
        quasitriangularity and the ribbon structure.

        Let us verify \cref{eq:Rmat_overstrand} first. The left-hand side is:
        \begin{equation}
                \Rmat_{12} \then \comult^{2}_{23} =
                \exp\pn*{\bo_1(\ao_2+\ao_3)}
                \exp\pn*{\frac{1-\Bo_1}{\bo_1}\yo_1(\xo_2+\xo_3)}
        \end{equation}
        Equality with the right-hand side follows by commutativity of $\bo_1$
        and $\yo_1$:
        \begin{equation}\begin{aligned}
                \Rmat_{13} \Rmat_{42} \then \mult^{14}_{1}
                &=
                \exp\pn*{\bo_1\ao_3}
                \exp\pn*{\frac{1-\Bo_1}{\bo_1}\yo_1\xo_3}
                \exp\pn*{\bo_1\ao_2}
                \exp\pn*{\frac{1-\Bo_1}{\bo_1}\yo_1\xo_2}
                \\&=
                \exp\pn*{\bo_1(\ao_2+\ao_3)}
                \exp\pn*{\frac{1-\Bo_1}{\bo_1}\yo_1(\xo_2+\xo_3)}
        \end{aligned}\end{equation}
        Next we verify \cref{eq:Rmat_understrand}, whose left-hand side is:
        \begin{equation}
                \Rmat_{13}\then\comult^{1}_{12} =
                \exp\pn[\big]{(\bo_1+\bo_2)\ao_3}
                \exp\pn*{
                        \pn*{
                                \Bo_2\frac{1-\Bo_1}{\bo_1}y_1+
                                \frac{1-\Bo_2}{\bo_2}y_2
                        }
                        \xo_3
                }
        \end{equation}
        On the right-hand side, we have
        \begin{equation}
        \begin{aligned}
                \Rmat_{13}\Rmat_{24}\then\mult^{34}_3
                &=
                \exp\pn*{\bo_1\ao_3}
                \exp\pn*{\frac{1-\Bo_1}{\bo_1}\yo_1\xo_3}
                \exp\pn*{\bo_2\ao_3}
                \exp\pn*{\frac{1-\Bo_2}{\bo_2}\yo_2\xo_3}
              \\&=
                \exp\pn[\big]{(\bo_1+\bo_2)\ao_3}
                \exp\pn*{\frac{1-\Bo_1}{\bo_1}\Bo_2\yo_1\xo_3}
                \exp\pn*{\frac{1-\Bo_2}{\bo_2}\yo_2\xo_3}
        \end{aligned}
        \end{equation}
        We use \cref{eq:ax_commutation} to write the expression in a canonical
        order. Finally, the right two exponentials may be combined since each
        variable commutes with the others, either by belonging to separate
        tensor factors, or in the case of $\bo$, being central.
        The verifications of \cref{eq:Rmat_comult} and
        \crefrange{eq:spinner_ribbon}{eq:spinner_counit} follow with similar
        computations.
\end{proof}

\section{Morphisms between meta-objects}

When equipped with meta-structures on both tangles and an algebraic object, we
can define a tangle invariant by considering a morphism between the
meta-objects.

\begin{definition}[morphism of meta-objects]
        Let $\set{A_X}_X$ and $\set{B_X}_X$ be compatible meta-objects (i.e.
        ones with the same operations and relations between the operations). A
        \defi{morphism} $ϕ$ between these meta-objects is map $\map {ϕ}
        {\set{A_X}_X} {\set{B_X}_X}$ sending $A_X \mapsto B_X$ such that for
        each operation $f^{X}_{Y}$ in $A$, $ϕ(f^{X}_{Y}) = f^X_Y$ in $B$.
\end{definition}

\subsection{Upright tangle invariants from a ribbon meta-Hopf algebra}
We define a $\CU$-valued tangle invariant in the following way:
\begin{enumerate}
        \item Given a open tangle, disconnect each crossing from its neighbours,
                as well as each arc with a nonzero rotation number.
        \item Replace each crossing with an $\Rmat$-matrix
                $\Rmat_{ij}\in\CU_{\set{i,j}}$, and each rotation of an arc with
                a spinner $\spin_i\in \CU_{\set{i}}$.
        \item For each disconnection, there is a corresponding stitching
                operation required to bring the tangle back to its original
                state. Replace each stitching operation with a multiplication
                operation in $\CU$.
\end{enumerate}

\begin{figure}[h]
        \centering
        \includegraphics{figures/invariant_example.pdf}
        \caption{Breaking up a tangle into its constituent components. The
        left-hand tangle is obtainable by the right-hand one by applying the map
        $\mult^{1,2,\dots,7}_{i}$.}
        \label{fig:invariant_example}
\end{figure}
