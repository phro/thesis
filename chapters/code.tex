\chapter{Code}
\section{Implementation of the invariant $Z$}
This is a Mathematica™ implementation by Bar-Natan and van der Veen in
\cite{BV}, modified by the author. The full source code is available at
\url{https://github.com/phro/GDO}.

\subsection{Use case}
For example, to compute the value of $Z$ on the Whitehead link, one need only
run the following:

\begin{mathematica}
In[0]:= Z@Knot[5, Alternating, 1]
\end{mathematica}
\begin{mathematica}
Out[0]= GDO["do" -> {}, "dc" -> {}, "co" -> {1, 5}, "cc" -> {}, "PG" -> PG["L" -> 0, "Q" -> -(((-1 + B[1])*(-1 + B[5])*(B[5]*x[5] + B[1]*((-1 + B[5])*x[1] - B[5]*x[5]))*(b[5]*y[1] - b[1]*y[5]))/ (b[1]*b[5]*B[1]*(B[1]*(-1 + B[5]) - B[5])*B[5])), "P" -> Sqrt[B[5]]/(B[1] + B[5] - B[1]*B[5])]]
\end{mathematica}
Written in a more readable format, this becomes:
\begin{equation}
\left(
        \frac{\sqrt{B_5}}{-B_5 B_1+B_1+B_5}
        \Exp{
                -\frac{
                        \left(B_1-1\right) \left(B_5-1\right)
                        \left(b_5 y_1-b_1 y_5\right)
                        \left(
                                B_5 x_5+
                                B_1 \left((B_5-1) x_1 - B_5 x_5\right)
                        \right)
                }{
                        b_1 b_5 B_1 \left(B_1 \left(B_5-1\right)-B_5\right) B_5
                }
        }
\right)_{\set{1,5},\set{}}^{\set{},\set{}} 
\end{equation}

\subsection{Implementation}

\input{./GDO/src/newVersion.tex}

\section{Implementation of rotation number algorithm}

%\section{Generating \acp{RVT} for links}

\subsection{Description of algorithm for knots}

Bar-Natan and van der Veen develop an algorithm to convert a classical long knot
into an upright tangle. It involves passing a line segment, called the
\emph{front}, over the knot, requiring that everything behind the front is in
upright form. For example, consider the link in \cref{fig:figures-upright_41_1}.
\begin{figure}[h]
        \centering
        \includegraphics{figures/upright_41_1.pdf}
        \caption{A knot which is not in upright form. The front is written in
        dark grey.}
        \label{fig:figures-upright_41_1}
\end{figure}
By pulling the crossings along the arc which touches the front, we can bring the
knot to upright form. The crossings are absorbed into the front in the order the
knot's strand interacts with them.
\begin{figure}[h]
        \centering
        \includegraphics{figures/upright_41_2a.pdf}
        \includegraphics{figures/upright_41_2b.pdf}
        \caption{By advancing the front over a crossing, we bring a crossing
        into upright form. A dashed line indicates where the front is advancing
        to.
        }
        \label{fig:figures-upright_41_2}
\end{figure}
\begin{figure}[h]
        \centering
        \includegraphics{figures/upright_41_3a.pdf}
        \includegraphics{figures/upright_41_3b.pdf}
        \caption{By advancing the front over a crossing, we bring a crossing
        into upright form. A dashed front indicates where the front is advancing
        to.
        }
        \label{fig:figures-upright_41_3}
\end{figure}

\subsection{Extending the algorithm to multiple components}

An algorithm to convert a classical knot diagram into an upright knot diagram
was implemented by Bar-Natan and van der Veen. Here we generalize the algorithm
to convert a classical tangle with one open component to an upright tangle
diagram. This generalization allows us to consider tangle diagrams with multiple
components.

\begin{lemma}[\cite{BV}, Lemma~43]
        For each classical tangle with one open component, there exists a unique
        upright tangle whose unbounded arcs have rotation numbers $0$.
\end{lemma}

This is a Haskell implementation\footnote{The full source code is available at
\url{https://github.com/phro/KnotTheory}.} of the algorithm \hs{toRVT},\footnote{Here, the
acronym RVT stands for \enquote{Rotational Virtual Tangle}, which is
another term for \enquote{Upright Tangle}.} which takes a classical tangle and
produces an upright tangle by computing a compatible choice of rotation numbers
for each arc. This follows largely the same logic as above, except the leftmost
strand is always prioritized for absorption, regardless of which component it
belongs to.

\subsection{Use case}

For example, to query the \hs{SX}-form of a link (i.e. its skeleton-crossing
form), one writes:
\begin{code}
>>> link 4 True 1
\end{code}
to obtain
\begin{code}
SX [Loop [1,2,3,4],Loop [5,6,7,8]]
   [Xm 1 6,Xm 3 8,Xm 5 2,Xm 7 4]
\end{code}
To convert from the \hs{SX} form to an upright tangle form (here written
\hs{RVT}), we must first replace one of the closed \hs{Loop}s with an open
\hs{Strand} (accomplished by \hs{openFirstStrand}):
\begin{code}
>>> toRVT . openFirstStrand $ link 4 True 1
\end{code}
to obtain
\begin{code}
RVT [Strand [1,2,3,4],Loop [5,6,7,8]]
    [Xm 1 6,Xm 3 8,Xm 5 2,Xm 7 4]
    [(5,-1),(6,1),(8,1)]
\end{code}
Reading off the final line, we see that arc $5$ has rotation number $-1$, arcs
$6$ and $8$ have rotation number $1$, and the rest of the arcs have rotation
number $0$.

\subsection{Implementation}

\input{./KnotTheory/src/KnotTheory/PD.lhs}
