\chapter{Code}
\section{Implementation of the invariant $Z$}
% todo: include a link to the source code. Ensure it is all on github (i.e. a
% stable website not likely to disappear in a few years).
This is a Mathematica™ implementation by Bar-Natan and van der Veen in
\cite{BV}, modified by the author.
\begin{minted}{wolfram}
setValue[value_,obj_,coord_]:=Module[
        {b=Association@@obj},
        b[coord] = value; Head[obj]@@Normal@b
]

(* PG[L, Q, P] = Perturbed Gaußian Pe^(L + Q) *)
\end{minted}

\section{Implementation of the trace}

This is a Mathematica™ implementation of the trace.
\begin{minted}[linenos=true]{wolfram}
getConstLCoef::usage = "getConstLCoef[i][gdo] returns the terms in the L-portion of a GDO expression which are not a function of y[i], b[i], a[i], nor x[i]."
getConstLCoef[i_][gdo_] :=
        (SeriesCoefficient[#, {b[i],0,0}]&) @*
        (Coefficient[#, y[i], 0]&) @*
        (Coefficient[#, a[i], 0]&) @*
        (Coefficient[#, x[i], 0]&) @*
        ReplaceAll[U2l] @*
        getL@
        gdo
\end{minted}

\section{Implementation of rotation number algorithm}

%\section{Generating \acp{RVT} for links}

\subsection{\acp{RVT} for knots}
Describe algorithm previously developed for knots

\subsection{Extending the algorithm to multiple components}

Given a classical link, there is a unique \ac{RVL} corresponding to it. Given a
classical link diagram, one may obtain the corresponding \ac{RVL} by attaching
an appropriate rotation number to each arc. However, there is not a unique way
to do so.

The situation becomes more complicated when one considers the case where the
tangle has an open component. In this case, two \ac{RVT} diagrams which
correspond to the same classical link exactly when they differ only by a
sequence of rotational Reidemeister moves \emph{and} a modification of the
rotation numbers of the (two) unbounded arcs. Equivalently, we have the
statement:

\begin{lemma}
        For each classical tangle with one open component, there exists a unique
        \ac{RVT} whose unbounded arcs have rotation numbers $0$.
\end{lemma}
\begin{proof}
        See \cite{BV}.
\end{proof}

Bar-Natan and van der Veen develop an algorithm to convert a classical long knot
into an \ac{RVT}. As we are interested in links, we must extend this algorithm
to include so-called \enquote{long links}, which we outline below:
\begin{verbatim}
        1. Pass a front over the beginning of the open strand.
        2. Progressively absorb the leftmost crossings
                2a. As crossings are absorbed,
                    take into account any rotations of arcs.
        3. If an arc passes through the front twice, absorb it,
           taking into account any rotations of that arc as a
           result.
\end{verbatim}

% If the code is to be presented within this section, it should be "clean,
% short, and match the mathematics" according to Dror. We'll see how much the
% Haskell code reflects that description.


This is a Haskell implementation of the algorithm \hs{toRVT} which takes a
classical tangle and produces a rotational tangle by computing a compatible
choice of rotation numbers for each arc.

\input{./KnotTheory/src/KnotTheory/PD.lhs}
