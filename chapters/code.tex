\chapter{Code}
\section{Implementation of the invariant $Z$}
% todo: include a link to the source code. Ensure it is all on github (i.e. a
% stable website not likely to disappear in a few years).
This is a Mathematica™ implementation by Bar-Natan and van der Veen in
\cite{BV}, modified by the author.
\begin{minted}{wolfram}
setValue[value_,obj_,coord_]:=Module[
        {b=Association@@obj},
        b[coord] = value; Head[obj]@@Normal@b
]

(* PG[L, Q, P] = Perturbed Gaußian Pe^(L + Q) *)
\end{minted}

\section{Implementation of the trace}

This is a Mathematica™ implementation of the trace.
\begin{minted}[linenos=true]{wolfram}
getConstLCoef::usage = "getConstLCoef[i][gdo] returns the terms in the L-portion of a GDO expression which are not a function of y[i], b[i], a[i], nor x[i]."
getConstLCoef[i_][gdo_] :=
        (SeriesCoefficient[#, {b[i],0,0}]&) @*
        (Coefficient[#, y[i], 0]&) @*
        (Coefficient[#, a[i], 0]&) @*
        (Coefficient[#, x[i], 0]&) @*
        ReplaceAll[U2l] @*
        getL@
        gdo
\end{minted}

\section{Implementation of rotation number calculator}

This is a Haskell implementation of the algorithm \hs{toRVT} which takes a
classical tangle and produces a rotational tangle by computing a compatible
choice of rotation numbers for each arc.

\input{./KnotTheory/src/KnotTheory/PD.lhs}
