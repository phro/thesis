\chapter{Code}
\section{Implementation of the invariant $Z$}
% todo: include a link to the source code. Ensure it is all on github (i.e. a
% stable website not likely to disappear in a few years).

\input{./GDO/src/newVersion.tex}

\section{Implementation of rotation number algorithm}

%\section{Generating \acp{RVT} for links}

\subsection{Converting knots to upright tangles}

Bar-Natan and van der Veen develop an algorithm to convert a classical long knot
into an upright tangle. It involves passing a line segment, called the
\emph{front}, over the knot, requiring that everything behind the front is in
upright form. For example, consider the link:
\begin{figure}[h]
        \centering
        \includegraphics{figures/upright_41_1.pdf}
        \caption{A knot which is not in upright form. The front is written in
        grey.}
        \label{fig:figures-upright_41_1}
\end{figure}
By pulling the crossings along the arc which touches the front, we can bring the
knot to upright form.
\begin{figure}[h]
        \centering
        \includegraphics{figures/upright_41_2a.pdf}
        \includegraphics{figures/upright_41_2b.pdf}
        \caption{By advancing the front over a crossing, we bring a crossing
        into upright form. A dashed front indicates where the front is advancing
        to.
        }
        \label{fig:figures-upright_41_2}
\end{figure}
\begin{figure}[h]
        \centering
        \includegraphics{figures/upright_41_3a.pdf}
        \includegraphics{figures/upright_41_3b.pdf}
        \caption{By advancing the front over a crossing, we bring a crossing
        into upright form. A dashed front indicates where the front is advancing
        to.
        }
        \label{fig:figures-upright_41_3}
\end{figure}

\subsection{Extending the algorithm to multiple components}

An algorithm to convert a classical knot diagram into an upright knot diagram
was implemented by Bar-Natan and van der Veen. Here we generalize the algorithm
to convert a classical tangle with one open component to an upright tangle
diagram. This generalization allows us to consider tangle diagrams with multiple
components.

\begin{lemma}[\cite{BV}, Lemma~43]
        For each classical tangle with one open component, there exists a unique
        upright tangle whose unbounded arcs have rotation numbers $0$.
\end{lemma}

This is a Haskell implementation\footnote{The full source code is available at
\url{https://github.com/phro/KnotTheory}.} of the algorithm \hs{toRVT},\footnote{Here, the
acronym RVT stands for \enquote{Rotational Virtual Tangle}, which is
another term for \enquote{Upright Tangle}.} which takes a classical tangle and
produces an upright tangle by computing a compatible choice of rotation numbers
for each arc.

\input{./KnotTheory/src/KnotTheory/PD.lhs}
