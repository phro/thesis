% Note that depending on your settings in the table of contents, subsections and subsubsections might appear virtually identical.
% Make sure to set the ToC depth and the numbering depth in the ToC the way you want.
\chapter{Perturbed Gaußians}\label{ch:perturbed_gaussians}
We now summarize the work of Bar Natan and van der Veen in \cite{BV}, which
develops a universal knot invariant using perturbed Gaußians. 

\section{Algebraic definitions}\label{sec:algebraic-definitions}
\subsection{Defining the algebra}
Here we define the Hopf algebra $\CU$, it's quasitriangular structure, and its
ribbon structure:

We begin by defining the algebra $\CU$. Denote by $\fa$ the non-commutative
$2$-dimensional cocommutative Lie bialgebra spanned by $a$ and $x$ with relation
$\liebk{a}{x} = x$. (This is also a Borel subalgebra of $\Sl_2$.)

Next, we use the Drinfeld double construction (outlined in \cite{ES}) to obtain
a quasitriangular Lie algebra $\fg$. As a vector space,
$\fg = \fa \oplus \dual\fa$. Given $u \in \fa$ and $v\in \dual \fa$, we have
$\liebk{u}{v}_{\fg} \defeq \dual\ad_u(v) - \dual\ad_v(u)$, extended bilinearly
and anticommutatively to all of $\fg$.
Then the algebra $\CU$ is defined to be the universal enveloping algebra
$\uea{\fg}$.

\begin{remark}
        For convenience, we define $b \defeq \dual a \in \dual\fa$ and
        $y \defeq \dual x \in \dual\fa$, so that
        \begin{equation}
                \CU = \genbuilder*{y, b, a, x}{
                        \liebk{a}{x} = x,
                        \liebk{a}{y} = -y,
                        \liebk{x}{y} = b,
                        \liebk{b}{ } = 0
                }
        \end{equation}
        as an algebra.
\end{remark}


\subsection{Expressing morphisms as generating functions}

When defining a morphism-valued tangle invariant, one needs a compact way of
encoding the morphism. In \cite{BV} this is achieved through the use of
generating functions, whose definition we reproduce below:

For $A$ and $B$ finite sets, consider the set $\hom(\polyring{\Q}{z_A},
\polyring{\Q}{z_B})$ of linear maps between multivariate polynomial rings. Such
a map is determined by its values on the monomials $z_A^\nn$ for each
multi-index $\nn \in \N^A$.

\begin{definition}[Exponential generating function]
        The \defi{exponential generating function} of a map
        $\map {Φ} {\polyring{\Q}{z_A}} {\polyring{\Q}{z_B}}$ between polynomial
        spaces is
        \begin{equation}\label{eq:genfunc}
                \Gen{Φ} \defeq
                \Sum[\nn\in\N^A] \frac{Φ(z_A^\nn)}{\nn!}ζ_A^\nn
                \in \powerseries{\polyring{\Q}{z_B}}{ζ_A}
        \end{equation}
\end{definition}
\begin{remark}
        Extending the definition of $Φ$ to
        $\powerseries{\polyring{\Q}{z_B}}{ζ_A}$ by the extending scalars to
        $\powerseries{\Q}{ζ_A}$ gives us an equivalent formulation:
        \begin{equation}
                \Gen{Φ}
                = Φ\pn*{\Sum[\nn\in\N^A] \frac{(z_Aζ_A)^\nn}{\nn!}}
                = Φ\pn*{\Gen[\big]{\id_{\polyring{\Q}{z_A}}}}
        \end{equation}
\end{remark}

By the PBW theorem, we know that $\CU$ is isomorphic as a vector space to the
polynomial ring $\polyring{\Q}{y, b, a, x}$ by choosing an ordering of the
generators (following \cite{BV}, we use $(y, b, a, x)$):
\begin{equation}\label{eq:order_map}
        \mapdef {\Order} {\polyring{\Q}{y, b, a, x}} [\toiso] {\CU}
        {y^{n_1}b^{n_2}a^{n_3}x^{n_4}} {y^{n_1}b^{n_2}a^{n_3}x^{n_4}}
\end{equation}

Using this vector space isomorphism, \cite{BV} expresses all Hopf algebra
operations as perturbed Gaußians. To extend the resulting tangle invariant to
one on links, one would need to define a trace operator on $\CU$. The first
natural place to look is the coinvariants,
$\CU_\CU = \fracl{\CU}{\liebk{\CU}{\CU}}$. In what follows, we will compute
$\CU_\CU$, determine a vector space isomorphism to a suitable polynomial ring,
and compute the corresponding generating function of the quotient map $\map
{\trace} {\CU} {\CU_\CU}$.

\subsection{Pure tangles as a meta-Hopf algebra}
\label{sec:topological_interpretations}

Tangled objects also have the structure of a meta-Hopf algebra. In this section,
we follow the definitions laid out by Bar-Natan and van der Veen in \cite{BV}.

\begin{definition}[pure tangle]
        A \defi{pure tangle} is an embedding of line segments (called
        \defi{strands}) into the thickened unit disk $D \times [-1,1]$ (or a
        disjoint union of such disks) such that the endpoints of the line
        segments are fixed along $\boundary D \times \set0$. Two pure tangles
        are considered equivalent if there exists an isotopy of the embedding
        which fixes the endpoints of the strands.
\end{definition}

To work with pure tangles, we define an equivalent notion with a combinatorial
flair:

\begin{definition}[pure tangle diagram]\label{def:pure_tangle_diagram}
        A \defi{pure tangle diagram} is a finite planar graph with distinguished
        circles called \defi{boundary circles}. The remainder of the edges will
        be called \defi{arcs}, and are contained inside the boundary circles,
        either meeting a boundary circle at a trivalent vertex, or meeting other
        internal edges at a tetravalent vertex (called a \defi{crossing}). Each
        crossing is marked with a sign---positive or negative. Collections of
        edges which meet at opposite sides of an edge are called \defi{strands}.
        Each strand must meet a boundary circle twice (that is, strands may not
        form loops). Two pure tangle diagrams are equivalent if one can be
        transformed into the other by a sequence of Reidemeister moves on the
        crossings.
\end{definition}

\begin{theorem}[pure tangles are tangle diagrams]\label{thm:pure_tangle}
        To each pure tangle one associates exactly one pure tangle diagram
        equivalence class. Further, a generic projection of a pure tangle to the
        flattened disks $\Dunion D\times\set0$ allows one to construct a tangle
        diagram corresponding to it.
\end{theorem}
\begin{proof}
        The projection of a generic perturbation of a pure tangle has the
        following properties:
        \begin{itemize}
                \item all intersections of strands with
                                themselves,
                                other strands, or
                                a boundary circle
                are transverse.
                \item all intersections involve at most two strand components,
                        so that no triple intersections appear.
                \item each projected strand is an immersion, so that no cusps
                        appear.
        \end{itemize}
        From these data, we may construct a pure tangle diagram, assigning one
        crossing to each double intersection, with the sign selected based on
        which strand lay above the other before projecting. Extending
        Reidemeister's theorem to objects of this form is straightforward.
\end{proof}

For the sake of algebraic closure, the notion of virtual tangles will be useful:
\begin{definition}[virtual tangle diagram]
        A \defi{virtual tangle diagram} is a diagram satisfying all conditions
        laid out in \cref{def:pure_tangle_diagram} except planarity, taken again
        up to a sequence of Reidemeister moves. We denote by $\tangle_S$ the
        collection of virtual tangle diagrams with strands indexed by the finite
        set $S$.
\end{definition}

\begin{theorem}[virtual tangles are just crossings]
        The data of a virtual tangle is equivalent to:
        \begin{itemize}
                \item a collection of labels for strands
                \item a collection of crossings between strands
                \item the orders in which each strand interacts with each
                  crossing.
        \end{itemize}
\end{theorem}
\begin{proof}
        
\end{proof}

\begin{theorem}[virtual tangles form a quasitriangular meta-Hopf algebra]
        \label{thm:vt_qtmha}
        The collection $\tangle_X$ forms a quasitriangular meta-Hopf algebra
        with the following operations:
        \begin{itemize}
                \item multiplication $\mult^{ij}_{k}[X]$ takes a tangle with
                        strands $X\sqcup\set{i,j}$ and glues the end of strand
                        $i$ to strand $j$, labelling the resulting strand $k$.
                \item the unit $\unit_{i}[X]$ takes a tangle diagram with
                        strands $X$ and introduces a new strand $i$ which does
                        not touch any of the other strands.
                \item the comultiplication $\comult^{i}_{jk}[X]$ takes a tangle
                        with strands $X\sqcup \set{i}$ and doubles strand $i$,
                        separating the two strands along the framing of strand
                        $i$, calling the right strand $j$ and the left one
                        $k$.\footnote{While this convention appears unfortunate,
                        we follow the notation laid out in \cite{BV} so that the
                        antipode and spinner have a more memorable
                        representation, namely looking like the letters they are
                        represented by (see \cref{thm:rvt_metaHopf} for more
                        details).
                }
                \item the counit $\counit^{i}[X]$ takes a tangle with strands
                        indexed by $X\sqcup \set{i}$ and returns the tangle with
                        strand labelled by $i$ deleted.
                \item the antipode $\antipode^{i}_{j}[X]$ takes a tangle with
                        strands labelled by $X \sqcup \set{i}$ and reverses the
                        direction of strand $i$ (calling the new strand $j$).
                \item the $\Rmat$-matrix $\Rmat_{ij}$ is given by the two-strand
                        tangle with a single positive crossing of strand $i$
                        over strand $j$. The inverse $\Rmat$-matrix
                        $\Rmati_{ij}$ is the two-strand tangle with a
                        \emph{negative} crossing of strand $i$ over strand $j$.
        \end{itemize}
\end{theorem}
\begin{proof}
        Associativity of multiplication (\cref{eq:cd_mult}) follows from the
        fact that stitching strands together amounts to concatenating the order
        of the crossings each strand interacts with. Since list concatenation is
        an associative operation, associativity follows in this case as well.

        Adding a non-interacting strand to a diagram, then stitching it to an
        existing strand (\cref{eq:cd_unit}) does not change any of the
        combinatorial data in the diagram, and results in identical diagrams.

        Establishing coassociativity (\cref{eq:cd_comult}) amount to the same
        argument that cutting a piece of paper into three strips does not depend
        on the order of cutting.

        The counit identity (\cref{eq:cd_counit}) states deleting a strand
        is the same operation as first doubling it, then deleting both resulting
        strands.

        The meta-bialgebra axioms we verify next:
        \Cref{eq:cd_mult_comult} states that if two strands are stitched
        together, then the resulting strand is doubled, this could have
        equivalently been achieved by doubling each of the original strands,
        then performing a stitching on both resulting pairs of strands.

        \Cref{eq:cd_mult_counit} simply states that stitching two strands
        together, then removing the resulting strand could have equally been
        achieved by removing both of the original strands without stitching them
        first.

        \Cref{eq:cd_unit_counit} states that introducing a strand, then
        immediately removing it is the identity operation.

        \Cref{eq:cd_unit_comult} says that doubling a newly-introduced (and
        therefore free of crossings) strand is the same operation as introducing
        two strands separately. (Recall that in the virtual case, proximity of
        strands is not accounted for)

        \Cref{eq:cd_antipode} states that when a strand is doubled, then one of
        the two strands is reversed, multiplying the two strands together
        results in a strand which can be rearranged to not interact with any of
        the other strands. This can be readily seen, as this newly-created
        strand looks like a snake weaving through the tangle diagram. One can
        remove the snake by applying a series of Reidemeister 2 moves, resulting
        in a strand disjoint from the rest of the diagram. This is the same as
        deleting the original strand, then introducing a new disjoint one.
        
        The quasitriangular axioms are equalities of pairs of three-strand
        tangles:
        \begin{itemize}
                \item \Cref{eq:Rmat_overstrand,eq:Rmat_understrand} tell us that
                        doubling a strand involved in a single crossing can also
                        be built by adjoining two crossings together.
                \item \Cref{eq:Rmat_comult} tells us that we can swap the order
                        of a doubled strand by adding crossings to either end
                        (reminiscent of a Reidemeister 2 move)
        \end{itemize}

        Finally, we observe that the quotient we introduce to tangle diagrams by
        the Reidemeister moves does not introduce any new relations.
        Reidemeister 2 follows from the invertibility of the $\Rmat$-matrix.
        Next, it is readily seen that the quasitriangular relations governing
        the $\Rmat$-matrix force it to solve the Yang-Baxter equation, which is
        one equivalent to the Reidemeister 3 in this case.
\end{proof}

The invariants we deal with keep track not only of crossing data, but also
rotation of strands between crossings. We introduce an object which monitors
these additional data:
\begin{definition}[\acf{RVT} diagrams]
        A \defi{\acf{RVT} diagram} is a virtual tangle diagram, together with an
        assignment of an integer to each arc, called the \defi{rotation number}
        of the arc. This is visualised by requiring that each strand's
        intersection with the boundary is pointing upwards, and that each
        crossing is between curves whose tangent deviates less than $π/2$ from
        the vertical direction.

        Equivalence between \ac{RVT} diagrams is determined by extending the
        traditional Reidemeister moves with the whirling relation: any crossing
        may be rotated by full rotations. This amounts to increasing both
        outgoing strands' rotation number by some $n\in\Z$, and adding $-n$ to
        the rotation number of the incoming strands.
        %todo: add picture of whirling relation
        Additionally, we must take care that framed Reidemeister 1 and the
        cyclic Reidemeister 2 include appropriate rotation numbers on their
        arcs. %todo: add picture of situation
        The set of \acp{RVT} with strands indexed by a set $X$ will be denoted
        $\RVT_X$.
\end{definition}

\begin{theorem}[\acp{RVT} form a meta-ribbon Hopf algebra]
        \label{thm:rvt_metaHopf}
        The collections $\RVT_X$ form a meta-ribbon Hopf algebra, with the same
        definitions as in \cref{thm:vt_qtmha}, except:
        \begin{itemize}
                \item The antipode $\antipode^{i}_{j}[X]$ takes a tangle with
                        strands labelled by $X \sqcup \set{i}$ and reverses the
                        direction of strand $i$, then adds a counter-clockwise
                        cap to the new beginning, and a clockwise cup to the
                        end. This new strand is called $j$. When applied to a
                        single vertical strand, the resulting tangle looks like
                        the letter \enquote{S}.
                \item The spinner $\spin_i[X]$ takes a tangle in $\RVT_X$ and
                        adds a new strand with rotation number $1$ which has no
                        interactions with any other strands. This new strand
                        looks like the letter \enquote{C}.
        \end{itemize}
\end{theorem}
\begin{proof}
        This proof follows that of \cref{thm:vt_qtmha} almost exactly. We need
        only take note of the modifications:
        
        The antipode now has corrections to the ends of the strands so that all
        components continue pointing upwards. The same argument of generating a
        snake in \cref{eq:cd_antipode}, then sliding it out of the diagram with
        Reidemeister 2 moves still applies (though more care must be taken with
        the rotation numbers of the involved arcs).

        Using \cref{lem:spinner_ribbon}, it is enough to verify the spinner
        axioms
        (\cref{eq:spinner_ribbon,eq:spinner_comult,eq:spinner_antipode,eq:spinner_conjugate,eq:spinner_counit}).
        All these axioms have corresponding pictures one can draw, keeping in
        mind the orientations in the definitions of the relevant operations.
\end{proof}

\subsection{Rotational tangle invariants from a ribbon Hopf algebra}
Here we describe the morphism from the category of pure rotational virtual
tangles to a ribbon Hopf algebra, as outlined by Bar-Natan and van der Veen in
\cite{BV}.

We define the morphism of meta-ribbon Hopf algebras in two steps:
\begin{enumerate}
        \item Given a pure tangle, we write out a sequence of meta-ribbon Hopf
                algebra operations which produce the tangle. Each operation is
                then mapped to the corresponding operation on the algebra, with
                a sequence of operations mapped to composition.
        \item Since computing compositions of operations is an essential
                component to this invariant, we then define an equivalent
                category which allows for the more efficient computation. This
                is done by replacing morphisms 
\end{enumerate}

\ProvideDocumentCommand{\UMaps}{}{\mcl{U}}
\ProvideDocumentCommand{\LinMaps}{}{\mcl{H}}
\ProvideDocumentCommand{\GenFunc}{}{\mcl{C}}
To efficiently describe $\K$-linear maps between tensor powers of the algebra
$\CU$, we define categories $\UMaps$, $\LinMaps$, and $\GenFunc$ with objects
finite sets and morphisms:
\begin{align}
        \Hom[\UMaps]JK &\defeq \Hom[\K]{\CU^{\otimes J}}{\CU^{\otimes K}}\\
        \Hom[\LinMaps]JK &\defeq \Hom[\K]{\polyring\Q{z_J}}{\polyring\Q{z_K}}\\
        \Hom[\GenFunc]JK &\defeq \powerseries{\polyring\Q{z_K}}{ζ_J}
\end{align}

There exist monoidal isomorphisms between these categories, namely $\map
{\Order} {\LinMaps} [\toiso] {\UMaps}$ and $\map {\GenMap} {\LinMaps} [\toiso]
{\GenFunc}$ as introduced in \cref{eq:order_map,eq:genfunc}.

We use this formulation because of the existence of a computationally amenable
subcategory of $\GenFunc$ which contains the image of this invariant.

\subsection{Formulating composition in other categories}
Composing operations in $\UMaps$ or $\LinMaps$ is straightforward to define, but
lacks a closed form. However, on $\GenFunc$, the corresponding definition of
composition takes the following form (quoted from \cite[Lemma~3]{BV}):

\begin{lemma}[Composition of generating functions]
Suppose $J$, $K$, $L$ are finite sets and
$ϕ ∈ \Hom{\polyring{\Q}{z_J}}{\polyring{\Q}{z_K}}$ and
$ψ ∈ \Hom{\polyring{\Q}{z_K}}{\polyring{\Q}{z_L}}$.
We have
\begin{equation}
        \Gen{ϕ\then ψ}
        = \restrict[\bigg]{
                \pn*{
                        \restrict*{\Gen{ϕ}}[z_K\to ∂ζ_K]
                        \Gen{ψ}
                }
        }[ζ_K=0]
\end{equation}
\end{lemma}

Since the above notation will occur several times, we will use the notion of
\defi{contraction} used by Bar-Natan and van der Veen (taken from
\cite[Definition~4]{BV}):

\begin{definition}[Contraction]
        Let $f\in\powerseries{\K}{r, s}$ be a powerseries. The
        \defi{contraction} of $f = \Sum[k, l]{c_{k, l}r^{k}s^{l}}$ along the
        pair $(r, s)$ is:
        \begin{equation}
                \contraction*{f}[(r, s)]
                \defeq \Sum[k] c_{k, k}k!
                = \restrict[\bigg]{\Sum[k, l]{c_{k, l} \partial_{s}^{k}s^{l}}}[s=0]
        \end{equation}
        Further, this notation is to be extended to allow for multiple
        consecutive contractions for $f\in \powerseries{\K}{r_i, s_i}_{i≤n}$:
        \begin{equation}
                \contraction{f}[\pn*{(r_i)_i, (s_i)_i}]
                \defeq
                \contraction*{
                        \contraction*{
                                \contraction{f}[(r_1, s_1)]
                        }[(r_2, s_2)]
                        \cdots
                }[(r_n, s_n)]
        \end{equation}
\end{definition}
It is important to note that contraction does not always define a convergent
expression.

The theorem we will rely heavily on in this thesis is the following, taken from
\cite[Theorem~6]{BV}:

\newcommand{\Wt}{\widetilde{W}} 
\begin{theorem}[Contraction theorem]
        For any $n\in \N$, consider the ring $R_n =
        \powerseries{\polyring{\Q}{r_j, g_j}}{s_j, W_{ij}, f_j
        \SetSymbol[\big] 1 ≤ i, j ≤ n}$. Then
        \begin{equation}
                \contraction{\Exp{gs + rf + rWs}}[r, s] = \det(\Wt)\Exp{g\Wt f}
        \end{equation}
        where $\Wt = \inv{(1-W)}$.
\end{theorem}

The main takeaway of this theorem is this: morphisms whose generating functions
are Gaußians have a clean formula for composition. Furthermore, this formula is
computationally reasonable, growing only polynomially in complexity with $n$.
This is contrasted with the conventional approach of choosing a representation
$V$ of $\CU$. When one considers morphisms between large tensor powers
$V^{\otimes n}$, the computational complexity is exponential in $n$.

\subsection{Expressing Hopf algebra operations as perturbed Gaußians}

We will now observe that the meta-Hopf algebra operations for $\CU$ as defined
in \cref{sec:algebraic-definitions} all have the form of a perturbed Gaußian.
Namely:

\begin{align}
        \Gen*{\mult^{ij}_{k}} &=\exp\pn*
        {
                (α_i + α_j)a_k +
                (β_i + β_j + ξ_iη_j)b_k +
                \pn*{\frac{ξ_i}{\A_j} + ξ_j} x_k+
                \pn*{\frac{η_j}{\A_i}+ η_i }y_k
        } \label{eq:gen-mult}
        \\ \Gen*{\unit_{i}} &=1\label{eq:gen-unit}
        \\\Gen*{\comult^{i}_{jk}} &=\exp\pn*{
                β_i(b_j + b_k) +
                α_i(a_j + a_k) +
                η_i(y_j + y_k) +
                ξ_i(x_j + x_k)
        } \label{eq:gen-comult}
        \\\Gen*{\counit^{i}} &=1\label{eq:gen-counit}
        \\\Gen*{\antipode^{i}_{i}} &=\exp\pn*
        {
                - a_iα_i
                - b_iβ_i 
                - η_i\A_iy_i
                - \A_iξ_ix_i
                + η_i\A_iξ_ib_i
        }
        \label{eq:gen-antipode}
        \\\Gen*{\Rmat_{ij}} &=\exp\pn*{
                a_j b_i + \frac{B_i-1}{-b_i} y_i x_j 
        }\label{eq:gen-Rmat}
        \\\Gen*{\spin_{i}} &=\sqrt{B_i}\label{eq:gen-spin}
        \\\Gen*{\ribbon_{i}} &=\sqrt{B_i}\exp\pn*{
                a_i b_i + \frac{1-B_i}{b_i}x_iy_i
        }\label{eq:gen-ribbon}
\end{align}

% TODO: complete
