\chapter{Conclusions}\label{ch:conclusions}

\section{Comparison with the multivariable Alexander polynomial}

Given that the long knot (i.e. one-component) case of this invariant
encodes the Alexander Polynomial, it was suspected that the invariant on long
links (i.e. multiple components, one of which is long) formed by adding the
trace would encode the \ac{MVA}. However, there are links which the \ac{MVA}
separates which this invariant does not.

On all two-component links with at most $11$ crossings (a collection of size
$914$), the trace map attains $878$ distinct values, while the MVA attains only
$778$. However, the two invariants are incomparable in terms of their strength.

\section{Further work}
While all other Hopf algebra operations in $U$ are expressed by \cite{BV} as
perturbed Gaußians, the form in \cref{eq:trace_formula} does not to conform to
the same structure. Further work is needed to either implement this operation
into the established framework, or to suitably extend the framework.
