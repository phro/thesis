\chapter{Constructing the trace}\label{ch:trace}
\section{Extending an invariant of open tangles to mixed tangles}
Thus far, the algebraic framework we have defined allows us to describe
invariants of tangles with no closed components. We now provide an extension to
include closed components.

% To extend the resulting tangle invariant to one on links, one would need to
% define a trace operator on $\CU$. The first natural place to look is the
% coinvariants, $\CU_\CU = \fracl{\CU}{\liebk{\CU}{\CU}}$. In what follows, we
% will compute $\CU_\CU$, determine a vector space isomorphism to a suitable
% polynomial ring, and compute the corresponding generating function of the
% quotient map $\map {\trace} {\CU} {\CU_\CU}$.

\begin{definition}[traced meta-algebra]
        A \defi{traced meta-algebra} (or \defi{traced meta-monoid}) is two
        things:
        \begin{enumerate}
                \item A collection of meta-algebras: for each finite set $L$, we
                        assign one meta-algebra $\set{A_{S, L}}_S$. \footnote{%
                                These sets index the \enquote{strands} $S$ and
                                the \enquote{loops} $L$.%
                        } The multiplication maps $\mult^{i,j}_k[L]$ then take
                        the form:
                        \begin{equation}
                                \map {\mult^{i,j}_k[L][S]}
                                {A_{\set{i,j}\sqcup S, L}}
                                {A_{\set{k}\sqcup S, L}}
                        \end{equation}
                        for $i$, $j$, $k$ disjoint from both $S$ and $L$.
                \item For each pair of disjoint finite sets $S$, $L$ and index
                        $i\notin S\cup L$, a \defi{trace}
                        $\map {\trace^i}
                                {A_{\set{i}\sqcup S, L}}
                                {A_{S,\set{i}\sqcup L}}$
                        which governs the compatibility of the families of
                        meta-algebras in the following way:
                        \begin{equation}\label{eq:cyclic}
                                \mult^{i,j}_{k}\then\trace^k
                                =
                                \mult^{j,i}_{k}\then\trace^k
                        \end{equation}
                        \Cref{eq:cyclic} is called the \defi{cyclic} property of
                        the trace.

                        Furthermore, $\trace^i$ commutes with operations which
                        do not involve the same strands. That is, for any
                        operation $ϕ^{D}_{C}$ and $i \notin D \cup C$:
                        \begin{equation}
                                ϕ^{D}_{C}\then \trace^i
                                = \trace^i \then ϕ^{D}_{C}
                        \end{equation}
        \end{enumerate}
\end{definition}
% Dror finds the notation cumbersome; say instead of \mult[L][S], he would write
% \mult[L, S] which is a meta-algebra with respect to S

% Dror does not like thinking of (traced) meta-algebras as a category. He
% prefers thinking about it as a collection of sets together with operations,
% more akin to an "algebraic structure".
% What is the value of saying a meta-algebra is a category?
% "Categories are overrated." -- Dror

The first example we give is that of mixed tangles.

\begin{definition}[mixed upright tangles]
        Let $\tangleMix{L}{S}$ be the set of upright tangles with open
        strands indexed by $S$ and closed strands indexed by $L$. The operations
        $ϕ[L][S]$ are defined analogously to the $ϕ[S]$ given in
        \cref{thm:tangles_meta_algebra}. (Here $ϕ$ varies over $\mult$, $\unit$,
        $\comult$, $\counit$, $\antipode$, $\Rmat$, and $\spin$.)
\end{definition}

\begin{lemma}[tangles are a traced algebra]
        The collection of all $\tangleMix{L}{S}$ is a traced ribbon meta-Hopf
        algebra, with trace map given by closing a strand into a loop.
\end{lemma}
\begin{proof}
        When $L = \emptyset$, the situation is exactly the case of
        \cref{thm:tangles_meta_algebra}, so
        $\tangleMix{\emptyset}{S} = \tangle S$ is a
        meta-Hopf algebra. Furthermore, since the Reidemeister moves are local
        operations, the presence of closed components does not affect our
        ability to verify the identities on the Hopf-algebra operations.

        The last point to verify is that closing a strand into a loop is a
        cyclic operation. Given two strands, we must verify that stitching one
        end together, then tracing the other yields the same diagram as
        stitching the other ends together, then taking the trace. However, by
        definition of trace, these two actions yield identical diagrams: the two
        strands are replaced by the same closed loop.
\end{proof}

\begin{remark}
        As in \cref{rem:virtual_caveat_mult}, the use of virtual tangles allows
        us to ensure this result holds regardless of the diagram in question. In
        the computations to be done in this paper, we will ensure that the two
        endpoints being joined are situated so that joining them results in a
        planar diagram.
\end{remark}

\begin{lemma}[coinvariants are a trace map]
        Let $A$ be an algebra, and denote by $A_A\defeq A/\liebk{A}{A}$ its set
        of coinvariants. \footnote{Here, $\liebk AA =
        \Span\setbuilder{\liebk{x}{y}}{x, y\in A}$ refers to the vector space of
        Lie brackets, not the ideal generated thereby. The space $A_A$ does not have an
        algebra structure in general.%
        } Then define $A_{S, L} \defeq A_{S} \otimes \pn{A_A}_{L}$. Then $A$
        defines a traced meta-algebra with trace map given by
        $\map {\trace^i[S][L]} {A_{S\sqcup\set{i}, L}} {A_{S, L\sqcup\set{i}}}$.
\end{lemma}
\begin{proof}
       Observe that for any choice of $L$, extending morphisms by the identity
       yield an isomorphism of traced meta-Hopf algebras:
       \begin{equation}
               \mapdef {ϕ_L} {\set*{A_{S}}} [\toiso] {\set{A_{S, L}}}
                       {A_{S}} {A_{S, L} \\
                       f^{C}_{D} &\mapsto f^{C}_{D} \otimes \id_{(A_A)_L}
               }
       \end{equation}
       Next, we must show that
       $\mult^{ij}_k\then\trace^{k} =\mult^{ji}_k\then\trace^{k}$.
       This amounts to showing that, given $u, v\in A$, that
       $\overline{uv} = \overline{vu} \in A_A$.
       However, by the construction of the coinvariants,
       $\overline{uv}-\overline{vu} = \overline{uv-vu} = \overline{0} \in A$.

       Finally, we must show that $\trace$ intertwines the $S$-indexed
       collection of algebras. Since an operation $ϕ^{D}_{C}$ manipulates on a
       disjoint set of tensor factors than $\trace^{i}$, we notice that both are
       equivalent to a tensor product:
       \begin{equation}
               ϕ^{D}_{C} \then \trace^{i}
               = ϕ^{D}_{C} \otimes \trace^{i}
               = \trace^{i} \then ϕ^{D}_{C}
       \end{equation}
       We conclude that vector-space algebras have a traced algebraic structure.
\end{proof}

\section{The space of coinvariants of $\CU$}\label{sec:coinv_comp}

We start with a result which simplifies working with coinvariants:

\begin{lemma}[Coinvariant simplification]\label{lem:coinvLieAlg}
        Let $\fh$ be a Lie algebra. Then $\uea{\fh}_{\uea{\fh}} =
        \uea{\fh}_{\fh}$.
\end{lemma}
\begin{proof}
First, observe that for any $u$, $v$, $f\in\uea{\fh}$,
$\ad_{uv}(f) = \ad_u(vf) + \ad_v(fu)$. Proceeding inductively, for any monomial
$μ\in\uea{\fh}$, $\ad_{μ}(f)$ is a linear combination of elements of
$\liebk*{\fh}{\uea{\fh}}$. By linearity of $\ad$, we conclude
$\liebk*{\uea{\fh}}{\uea{\fh}} = \liebk*{\fh}{\uea{\fh}}$.
\end{proof}

\begin{theorem}\label{thm:CU_coinvariants_basis}
        The space of coinvariants of $U$, $U_U$, has basis
        $\set{\yo^n \ao^k \xo^n}_{n, k\ge 0}$.
\end{theorem}
\begin{proof}
By \cref{lem:coinvLieAlg}, we need only compute $\liebk{\Alg}{\CU}$ to
determine $\CU_\CU$. Using \cref{lem:xay_relations}, we compute the adjoint
actions of $\yo$, $\ao$, and $\xo$. (Recall $\bo$ is central.)
\begin{align}
  \ad_{\ao} f(\xo) &= \xo f'(\xo)&
  \ad_{\ao} f(\yo) &= -\yo f'(\yo)\label{eq:ada}\\
  \ad_{\xo} f(\yo) &= \bo f'(\yo) &
  \ad_{\xo} f(\ao) &= -\nabla[f](\ao)\xo\label{eq:adx}\\
  \ad_{\yo} f(\xo) &= -\bo f'(\xo) &
  \ad_{\yo} f(\ao) &= \yo\nabla[f](\ao)\label{eq:ady}
\end{align}
(Here $\nabla$ is the backwards finite difference operator $\nabla[f](x) \defeq
f(x) - f(x-1)$.) Observe for any $n$, $m$, $k$, and polynomials $f$ and $g$:
\begin{align}
        \ad_{\ao} \pn*{\yo^m g(\bo, \ao) \xo^n } &= (n-m)\yo^mg(\bo, \ao) \xo^n
        \label{eq:ada_rel}\\
        \ad_{\xo}\pn*{\yo^{n+1}\bo^{m-1}f(\ao)\xo^{k}} &=
                (n+1)\yo^{n}\bo^{m}f(\ao)\xo^{k}
                - \yo^{n+1}\bo^{m-1}\nabla[f](\ao)\xo^{k+1}
        \label{eq:adx_rel}\\
        \ad_{\yo}\pn*{\yo^n \bo^{m-1} f(\ao) \xo^{k+1}} &=
                - (k+1)\yo^n \bo^m f(\ao) \xo^k
                + \yo^{n+1} \bo^{m-1} \nabla[f](\ao)\xo^{k+1}
        \label{eq:ady_rel}
\end{align}
By \cref{eq:ada_rel}, any monomial whose powers of $\yo$ and $\xo$ differ vanish in
$\CU_{\Alg}$. As a consequence, in \cref{eq:adx_rel,eq:ady_rel}, the only
nontrivial case is when $n=k$, resulting in the same relation. By induction on
$n$, we conclude that:
\begin{equation}\label{eq:coinv_reduction}
        \yo^n \bo^m f(\ao) \xo^k
        \sim δ_{nk}\frac{n!}{(n+m)!}\yo^{n+m}\nabla^m[f](\ao)\xo^{n+m}
\end{equation}
where $\sim$ refers to equivalence in the set of coinvariants. Observing when
$f$ is a monomial in \cref{eq:coinv_reduction}, we see $\CU_{\Alg}$ is spanned
by $\set{\yo^n \ao^k \xo^n}_{n, k \ge 0}$.

Finally, all that remains to show is this set is linearly independent. We do so
by defining the trace map
\begin{equation}
        \mapdef {\trace} {\CU} {\CU_{\CU}}
                {\yo^n \bo^m f(\ao) \xo^k}
                {δ_{nk}\frac{n!}{(n+m)!}\yo^{n+m}\nabla^m[f](\ao)\xo^{n+m}}
\end{equation}
We will now show that the trace sends each relation defined in
\cref{eq:ada_rel,eq:adx_rel,eq:ady_rel} to $0$. For the relations coming from
$\ad_{\ao}$, notice that:
\begin{equation}
        \trace\pn*{\ad_{\ao}(\yo^n \bo^m f(\ao) \xo^k)}
        = \trace\pn*{
                (n-k)δ_{nk}\frac{n!}{(n+m)!}\yo^{n+m}\nabla^m[f](\ao)\xo^{n+m}
        }
        = 0
\end{equation}
For the relations from $\ad_{\yo}$, observe:
\begin{equation}
        \begin{aligned}
                &\trace\pn*{\ad_{\yo}\pn*{\yo^n \bo^{m-1} f(\ao) \xo^{k+1}}}
                \\&=
                \trace\pn[\big]{
                        - (k+1)\yo^n \bo^m f(\ao) \xo^k
                        + \yo^{n+1} \bo^{m-1} \nabla[f](\ao)\xo^{k+1}
                }
                \\&=
                \begin{multlined}[t]
                -(k+1)δ_{nk}\frac{n!}{(n+m)!}\yo^{n+m}\nabla^m[f](\ao)\xo^{n+m}
                \\
                        + δ_{n+1,k+1}\frac{(n+1)!}{(n+1+m-1)!}
                        \yo^{n+m}\nabla^{m-1}[\nabla f](\ao)\xo^{n+m}
                \end{multlined}
                \\&=0
\end{aligned}
\end{equation}
The relations from $\ad_{\xo}$ follow almost identically as those of
$\ad_{\yo}$. As $\bo$ is central, $\ad_{\bo}$ generates no more relations, so
linear independence is established.
\end{proof}

\subsection{A generating function for the coinvariants}

\ProvideDocumentCommand{\COrder}{}{\Order}
In order to define a generating function, we need to choose a basis for the
space of coinvariants. We define an isomorphism from the space of coinvariants
to a polynomial space, tweaking the basis defined in
\cref{thm:CU_coinvariants_basis} by scalar multiples. Since it plays the role of
the ordering map, we also name it $\COrder$.
\begin{equation}
        \mapdef {\COrder} {\polyring{\Q}{a, z}} [\toiso] {\CU_\CU}
        {a^{n}z^{k}}
        {\frac{1}{k!}\yo^{k}\ao^{n}\xo^{k} \\
                k!\nabla^m[f](a)z^{k+m} &\mapsfrom
                \yo^k \bo^mf(\ao)\xo^k
        }
\end{equation}

This defines a commutative square upon whose bottom edge
($τ = \Order \then {\trace} \then {\inv{\COrder}}$) we compute the generating
function:
\begin{equation}
\begin{tikzcd}
        \CU
                \rar["\trace"]
        & \CU_\CU \\
        \polyring{\Q}{y, b, a, x}
                \uar["\Order"]
                \rar["τ"]
        &
        \polyring{\Q}{a, z}
                \uar["\COrder"]
\end{tikzcd}
\end{equation}

We begin with a result on finite differences:
\begin{lemma}[finite differences of exponentials]\label{eq:findiffexp}
        The finite difference operator acts in the following way on
        exponentials:
        \begin{equation}
                \nabla^n[\Exp{αa}](a) = (1-\Exp{-α})^n\Exp{αa}
        \end{equation}
\end{lemma}
\begin{proof}
Using the fact that $\nabla^n[f](x) = \Sum[k=0][n]\binom n k(-1)^kf(x-k)$, we
see that
$\nabla^n[\Exp{αa}](a)
        = \Sum[k=0][n]\binom n k(-1)^k\Exp{αa-αk}
        % = \Exp{αa}\Sum[k=0][n]\binom n k(-1)^k\Exp{-αk}
        = (1-\Exp{-α})^n\Exp{αa}$.
\end{proof}
We now are ready to compute the generating function for the trace:
\begin{theorem}[Generating function for the trace of $\CU$]
\begin{equation}\label{eq:trace_formula}
        \Gen\trace = \exp\pn[\Big]{αa+\pn[\big]{ηξ+β(1-\Exp{-α})}z}
\end{equation}
\end{theorem}
\begin{proof}
        Using \cref{eq:findiffexp} and the extension of scalars of $\trace$ to
        $\powerseries{\Q}{η, β, α, ξ}$, we see
        \begin{equation}
        \begin{aligned}
                &\Gen[\big]{\Order \then {\trace} \then {\inv{\COrder}}}
                = \pn[\big]{\Exp{η y} \Exp{β b} \Exp{α a} \Exp{ξ x}} \then
                        {\trace} \then {\inv\COrder}\\
                &= \inv\COrder\Sum[i, j, k]\trace\pn*{
                        \frac{(η y)^i}{i!}
                        \frac{(β b)^j}{j!}
                        \Exp{α a}
                        \frac{(ξ x)^k}{k!}
                }\\
                % &=\Sum[i, j, k]\frac{η^i β^j ξ^k}{i! j! k!}
                        % \trace\pn*{ y^i b^j \Exp{α a} x^k }\\
                % &=\Sum[i, j, k]\frac{η^i β^j ξ^k}{i! j! k!}
                        % δ_{ik}i!\nabla^j[\Exp{αa}](a)t^{i+j}\\
                &=\Sum[i, j]\frac{η^i β^j ξ^i}{i! j!}
                        (1-\Exp{-α})^j\Exp{αa}z^{i+j}
                =\Exp{αa+\pn*{ηξ+β(1-\Exp{-α})}z}\qedhere
        \end{aligned}
        \end{equation}
\end{proof}

\subsection{Evaluation of the trace on a generic element}
Here we will outline a computation involving the trace by using Bar-Natan and
van der Veen's Contraction Theorem.

A typical value for a tangle invariant that arises is of the form:
\begin{equation}\label{eq:sample_GDO}
        P\Exp{\cc + \ca a_i + \cb b_i + σa_ib_i
        + \cx(b_i) x_i + \cy(b_i)y_i + λ(b_i)x_i y_i}
\end{equation}
Here, $\cc$, $\ca$, $\cb$, and $σ$ denote constants with respect to the variables
$y_i$, $b_i$, $a_i$, and $x_i$ (collectively referred to as \enquote{$v_i$}s),
while $\cx$, $\cy$, and $λ$ are potentially $b_i$-dependent, and $P$ is a (rational)
function in (the square root of) $B_i$.

\ProvideDocumentCommand{\ba}{}{\bar{a}}
\ProvideDocumentCommand{\bz}{}{\bar{z}}

\begin{theorem}[The trace of a Gaußian]\label{thm:trace_gaussian}
        With symbols as defined above, let $f(y_i, b_i, a_i, x_i) = P(B_i)
        \Exp{\cc + \ca a_i + \cb b_i + σa_ib_i
                + \cx(b_i) x_i + \cy(b_i)y_i + λ(b_i)x_i y_i}$. Then when $σ=0$,
\begin{equation}\begin{aligned}\label{eq:trace_on_gaussian}
\contraction*{ f(y_i, b_i, a_i, x_i) \trace^i }_{v_i}
&= \frac{P(\Exp{-μ})}
        {1-λ(μ)\bz_i}\Exp{\cc+\ca\ba_i+\cb μ+\frac{\cy(μ)\cx(μ)\bz_i}{1-λ(μ)\bz_i}}
\end{aligned}\end{equation}
where $μ\defeq (1-\Exp{-\ca})\bz_i$.
\end{theorem}
\begin{proof}
        Let us compute the trace of \cref{eq:sample_GDO}. For clarity, we will
        put bars over the coinvariants variables $a_i$ and $z_i$, as they do not
        play a role in the contraction.
        \begin{equation}\begin{aligned}\label{eq:trace_on_gaussian_computation}
                &\contraction{
                        P(B_i)
                        \Exp{\cc + \ca a_i + \cb b_i + \cx(b_i) x_i + \cy(b_i)y_i + λ(b_i)x_i y_i}
                        \trace^i
                }_{v_i}\\
                &= \contraction{
                        P(B_i)
                        \Exp{\cc + \cb b_i + \cx(b_i) x_i + \cy(b_i)y_i + λ(b_i)x_i y_i
                        +η_iξ_i\bz_i+β_i(1-\Exp{-α_i})\bz_i}
                        \Exp{\ca a_i + α_i\ba_i}
                }_{v_i}\\
                &= \Exp{\ca\ba_i}\contraction{
                        P(B_i)
                        \Exp{\cc + \cx(b_i) x_i + \cy(b_i)y_i + λ(b_i)x_i y_i
                        +η_iξ_i\bz_i}
                        \Exp{\cb b_i +β_i(1-\Exp{-\ca})\bz_i}
                }_{b_i, x_i, y_i}\\
                \alignedintertext{
                        In what follows, we let $μ\defeq (1-\Exp{-\ca})\bz_i$:
                }
                &= \Exp{\cc+\ca\ba_i+\cb μ}P(\Exp{-μ})\contraction{
                        \Exp{\cy(μ)y_i}
                        \Exp{(\cx(μ) + λ(μ)y_i)x_i + ξ_iη_i\bz_i}
                }_{x_i, y_i}\\
                &= \Exp{\cc+\ca\ba_i+\cb μ}P(\Exp{-μ})\contraction{
                        \Exp{\cy(μ)y_i+\cx(μ)\bz_iη_i + λ(μ)\bz_iη_iy_i}
                }_{y_i}\\
                &= \frac{P(\Exp{-μ})}
                {1-λ(μ)\bz_i}\Exp{\cc+\ca\ba_i+\cb μ+\frac{\cy(μ)\cx(μ)\bz_i}{1-λ(μ)\bz_i}}
        \end{aligned}\end{equation}
\end{proof}

\begin{remark}
        \Cref{thm:trace_gaussian} requires that the coefficient $σ$ of the
        $a_ib_i$-term in the exponential is $0$. This restriction allows the
        computed value to be a composition of elementary functions. We
        demonstrate this by considering the case where $\cc = \cb = \cx = \cy = λ =
        0$. Expanding \cref{eq:trace_on_gaussian} yields the following
        expression, which we name $S$:
        \begin{equation}
                \contraction*{\Exp{β_i(1-\Exp{-\ca-σb_i})\bz_i}}_{b_i} = S
        \end{equation}
        The quantity $S$ satisfies the relation $S = \Exp{-\ca-σS}$, whose
        solution is $S = \frac{W(\Exp{-\ca}σ)}{σ}$ where $W$ denotes the Lambert
        $W$ function. This expression now lies far outside of the Gaußian
        framework that $Z$ is computed in.
\end{remark}
\begin{proof}[Proof that $S$ satisfies $S = \Exp{-\ca-σS}$]
        Denote by $γ_n$ the $b_i$-coefficients of the expression
        $1-\Exp{-\ca-σb_i}$, so that $β_i(1-\Exp{-\ca-σb_i}) =
        \Sum[n=0][\infty]β_iγ_nb_i^n$. Visualizing the $γ_n$-term as a node of a
        directed tree with one incoming edge (corresponding to $β_i$) and $n$
        outgoing edges (corresponding to each of the $b_i$ factors), we get the
        relation
        \begin{equation}
                S = γ_1S + γ_2S^2 + γ_3S^3 + \dots
        \end{equation}
        With the observation that $γ_n = \frac{\Exp{-\ca}(-μ)^m)}{m!}$, we
        conclude that $S = \Exp{-\ca-μS}$.
\end{proof}

The $a_ib_i$-coefficient associated to an open tangle is exactly the
self-linking number of component $i$. By adding correcting writhe terms, one can
ensure this coefficient is $0$.

When applying \cref{eq:trace_on_gaussian} to a tangle with multiple closed
components, the $a_ib_i$-coefficient no longer has the interpretation of a
self-linking number, so the writhe-correction technique fails. The author did
not find a workable closed form for the trace in this case.

We point out that the outcome of this computation is not guaranteed to be a
Gaußian. This puts a limitation on the applicability of this formula to links
with more than two components, explored in \cref{sec:limitations}.

\section{Computational example}

Using the formula given in \cref{eq:trace_on_gaussian}, let us work through an
example.
Consider the Hopf link (\cref{fig:hopf_closed}). In order to compute $\ptr$, we
first open one of the components, ensuring the rotation number at the endpoints
is $0$ (\cref{fig:hopf_halfOpen}). We do this because the total trace of many
links we consider is a trivial $1$. The use of a partial trace provides a
stronger invariant. We then break open all the closed components (here, there is
just one), so that an open tangle remains, as in \cref{fig:hopf_pre}, which we
call $H$.

\begin{figure}[ht]
        \centering
        \includegraphics{figures/hopf_closed.pdf}
        \caption{The (positive) Hopf link.}
        \label{fig:hopf_closed}
\end{figure}
\begin{figure}[ht]
        \begin{subfigure}[b]{0.5\textwidth}
                \centering
                \includegraphics{figures/hopf_halfOpen.pdf}
                \caption{Opening the Hopf link to a tangle with one open
                component.}
                \label{fig:hopf_halfOpen}
        \end{subfigure}
        \begin{subfigure}[b]{0.5\textwidth}
                \centering
                \includegraphics{figures/hopf_pre.pdf}
                \caption{A tangle $H$ representing the Hopf link with both
                tangles open.}
                \label{fig:hopf_pre}
        \end{subfigure}
        \label{fig:hopf_decompose}
        \caption{Opening the strands of the closed Hopf link.}
\end{figure}

Computing $Z$ on the open tangle $H$ yields:
\begin{equation}
        Z(H) = \Exp{
                a_2 b_1+a_1 b_2+\frac{(B_1-1) y_1
                ((B_2-1) x_1-x_2)}{b_1}-\frac{B_1
                (B_2-1) x_1 y_2}{b_2}
        }
        \sqrt{B_2}
\end{equation}

Applying $\trace^2$ then provides us with the invariant corresponding to
\cref{fig:hopf_halfOpen}: (Here, we use $w_i$ to represent $a_i$ to be
consistent with the computer-generated table of values in
\cref{ch:table_of_values}, which allows for easier distinguishing between
variables corresponding to $\CU$ and those to $\CU_{\CU}$.)
\begin{equation}
        \trace^{2}(Z(H)) = \Exp{
                a_1 \left(z_2-B_1 z_2\right)+b_1 w_2+\frac{e^{-z_2} \left(x_1
                y_1 e^{z_2}-x_1 y_1 e^{B_{1} z_2}\right)}{b_1}
        }
        \Exp{\frac{1}{2} B_{1} z_2-\frac{z_2}{2}}
\end{equation}

We repeat this procedure with the other component, opening and straightening
component $2$, then opening component $1$ without straightening to get an open
tangle digaram $H'$, then computing $\trace^1(Z(H'))$. The symmetry of the Hopf
link results in the same computation with indices $1$ and $2$ exchanged:
\begin{equation}
        \trace^{1}(Z(H')) = \Exp{
                a_2 \left(z_1-B_2 z_1\right)+b_2 w_1+\frac{e^{-z_1} \left(x_2
                y_2 e^{z_1}-x_2 y_2 e^{B_2 z_1}\right)}{b_2}
        }
        \Exp{\frac{1}{2} B_2 z_1-\frac{z_1}{2}}
\end{equation}

Finally, we keep track of which indices correspond to open strands and closed
strands. We will extend the notation from the previous section to differentiate
between open and closed indices. We write a morphism with domain
$D = D_{\text{o}}\sqcup D_{\text{c}}$, codomain
$C = C_{\text{o}}\sqcup C_{\text{c}}$
(here $D_{\text{o}}$ denotes domain indices which are open, while $D_{\text{c}}$
those which are closed, with the same convention for $C$) and generating
function $f(ζ_{D}, z_{C})$ as $f(ζ_{D},
z_{C})^{(D_{\text{o}},D_{\text{c}})}_{(C_{\text{o}},C_{\text{c}})}$.

We package these data into a multiset:
\begin{equation}
        Z(L_{\text{Hopf}})=
        \begin{multlined}[t]
        \pn[\Bigg]{
                \pn[\bigg]{
                        \Exp{
                                a_1 \left(z_2-B_1 z_2\right)+b_1
                                w_2+\frac{e^{-z_2} \left(x_1
                                y_1 e^{z_2}-x_1 y_1 e^{B_{1} z_2}\right)}{b_1}
                        }
                        \Exp{\frac{1}{2} B_{1} z_2-\frac{z_2}{2}}
                }_{\set{1},\set{2}}
                ,\\
                \pn[\bigg]{
                        \Exp{
                                a_2 \left(z_1-B_2 z_1\right)+b_2
                                w_1+\frac{e^{-z_1} \left(x_2
                                y_2 e^{z_1}-x_2 y_2 e^{B_2 z_1}\right)}{b_2}
                        }
                        \Exp{\frac{1}{2} B_2 z_1-\frac{z_1}{2}}
                }_{\set{2},\set{1}}
        }
\end{multlined}
\end{equation}
