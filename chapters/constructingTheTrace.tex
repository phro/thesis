\chapter{Constructing the Trace}
\section{Extending a pure tangle invariant to links and general tangles}
Thus far, the algebraic setting we have defined allows us to describe invariants
of tangles with no closed components. We now extend the notion of a meta-Hopf
algebra to include closed components.

% To extend the resulting tangle invariant to one on links, one would need to
% define a trace operator on $\CU$. The first natural place to look is the
% coinvariants, $\CU_\CU = \fracl{\CU}{\liebk{\CU}{\CU}}$. In what follows, we
% will compute $\CU_\CU$, determine a vector space isomorphism to a suitable
% polynomial ring, and compute the corresponding generating function of the
% quotient map $\map {\trace} {\CU} {\CU_\CU}$.

\begin{definition}[traced meta-algebra]
        A \defi{traced meta-algebra} is a family of meta-algebras: for each
        finite set $L$, we assign one meta-algebra $\set{A_{L,
        S}}_S$.\footnote{These sets index the \enquote{strands} $S$ and the
        \enquote{loops} $L$.} The multiplication maps $\mult^{i,j}_k[L]$ then
        take the form:
        \begin{equation}
                \map {\mult^{i,j}_k[L][S]}
                        {A_{\set{i,j}\sqcup S, L}}
                        {A_{\set{k}\sqcup S, L}} 
        \end{equation}
        for $i$, $j$, $k$ disjoint from both $S$ and $L$.

        There is an additional structure, called a \defi{trace}. The
        compatibility of the families of meta-algebras is governed this trace in
        the following way:
        $\map {\trace^i} {A_{\set{i}\sqcup S, L}} {A_{S,\set{i}\sqcup L}}$ which
        is universal with respect to the cyclic property:
        \begin{equation}
                \mult^{i,j}_{k}\then\trace^k
                =
                \mult^{j,i}_{k}\then\trace^k
        \end{equation}
        Furthermore, $\trace^i$ is a morphism of meta-coalgebras. That is:
        \begin{align}
                \label{eq:tr_comult}
                \comult^{i}_{jk}\then\trace^j\then\trace^k
                &= \trace^i \then \comult^{i}_{jk}
                \\\label{eq:tr_counit}
                \trace^i \then \counit^i &= \counit^i
        \end{align}
\end{definition}
% Dror finds the notation cumbersome; say instead of \mult[L][S], he would write
% \mult[L, S] which is a meta-algebra with respect to S

% Dror does not like thinking of (traced) meta-algebras as a category. He
% prefers thinking about it as a collection of sets together with operations,
% more akin to an "algebraic structure".
% What is the value of saying a meta-algebra is a category?
% "Categories are overrated." -- Dror

The first example we give is that of mixed tangles.

\begin{definition}[mixed upright tangles]
        Let $\tangleMix{L}{S}$ be the set of upright tangles with open
        strands indexed by $S$ and closed strands indexed by $L$. The operations
        $ϕ[L][S]$ are defined analogously to the $ϕ[S]$ given in
        \cref{thm:vt_qtmha}. (Here $ϕ$ varies over $\mult$, $\unit$, $\comult$,
        $\counit$, $\antipode$, $\Rmat$, and $\spin$.)
\end{definition}

\begin{lemma}[tangles as a traced algebra]
        The collection of all $\tangleMix{L}{S}$ is a traced ribbon meta-Hopf
        algebra, with trace map given by closing a strand into a loop.
\end{lemma}
\begin{proof}
        When $L = \emptyset$, the situation is exactly the case of
        \cref{thm:tangles_meta_algebra}, so
        $\tangleMix{\emptyset}{S} = \tangle S$ is a
        meta-Hopf algebra. Furthermore, since the Reidemeister moves are local
        operations, the presence of closed components does not affect our
        ability to verify the identities on the Hopf-algebra operations.

        The last point to verify is that closing a strand into a loop is a
        cyclic operation. Given two strands, we must verify that stitching one
        end together, then tracing the other yields the same diagram as
        stitching the other ends together, then taking the trace. However, by
        definition of trace, these two actions yield identical diagrams, the two
        strands replaced by the same closed loop.
\end{proof}

\begin{lemma}[coinvariants as a trace map]
        Let $A$ be an algebra, and denote by $A_A\defeq A/\liebk{A}{A}$ its set
        of coinvariants. Then define
        $A_{S, L} \defeq A^{\otimes S} \otimes A_A^{\otimes L}$. Then $A$
        defines a traced meta-algebra with trace map given by
        $\map {\trace^i_j} {A_i} {(A_A)_j}$.
\end{lemma}
\begin{proof}
       Observe that for any choice of $L$, extending morphisms by the identity
       yield an isomorphism of traced meta-Hopf algebras:
       \begin{equation}
               \mapdef {ϕ_L}
                       {\set[\big]{A^{\otimes S}}_S} [\toiso]
                       {\set[\big]{A^{\otimes S}\otimes A_A^{\otimes L}}_S}
                       {A^{\otimes S}}
                       {A^{\otimes S}\otimes A_A^{\otimes L} \\
                       f &\mapsto f \otimes \id_{A_A}^{\otimes L}
               }
       \end{equation}
       Next, we must show that
       $\mult^{ij}_k\then\trace^{k} =\mult^{ji}_k\then\trace^{k}$.
       This amounts to showing that, given $u, v\in A$, that
       $\overline{uv} = \overline{vu} \in A_A$.
       However, by the construction of the coinvariants,
       $\overline{uv}-\overline{vu} = \overline{uv-vu} = \overline{0} \in A$,
       and we are done.
\end{proof}

\section{The coinvariants of $U$}\label{sec:coinv_comp}

We start with a result which simplifies working with coinvariants:

\begin{lemma}[Coinvariant simplification]\label{lem:coinvLieAlg}
        Let $\fh$ be a Lie algebra. Then $\uea{\fh}_{\uea{\fh}} =
        \uea{\fh}_{\fh}$.
\end{lemma}
\begin{proof}
First, observe that for any $u$, $v$, $f\in\uea{\fh}$,
$\ad_{uv}(f) = \ad_u(vf) + \ad_v(fu)$. Proceeding inductively, for any monomial
$μ\in\uea{\fh}$, $\ad_{μ}(u)$ is a linear combination of elements of
$\liebk*{\fh}{\uea{\fh}}$. By linearity of $\ad$, we conclude
$\liebk*{\uea{\fh}}{\uea{\fh}} = \liebk*{\fh}{\uea{\fh}}$.
\end{proof}

\begin{theorem}
        The coinvariants of $U$, $U_U$, has basis
        $\set{y^n a^k x^n}_{n, k\ge 0}$.
\end{theorem}
\begin{proof}
By \cref{lem:coinvLieAlg}, we need only compute $\liebk{\Alg}{\CU}$ to
determine $\CU_\CU$. Using \cref{lem:xay_relations}, we compute the adjoint
actions of $y$, $a$, and $x$. (Recall $b$ is central.)
\begin{align}
  \ad_a f(x) &= xf'(x)&
  \ad_a f(y) &= -yf'(y)\label{eq:ada}\\
  \ad_x f(y) &= bf'(y) &
  \ad_x f(a) &= -\nabla[f](a)x\label{eq:adx}\\
  \ad_y f(x) &= -bf'(x) &
  \ad_y f(a) &= y\nabla[f](a)\label{eq:ady}
\end{align}
(Here $\nabla$ is the backwards finite difference operator $\nabla[f](x) \defeq
f(x) - f(x-1)$.) Observe for any $n$, $m$, $k$, and polynomials $f$ and $g$:
\begin{align}
        \ad_a \pn*{y^m g(b, a) x^n } &= (n-m)y^mg(b, a) x^n
        \label{eq:ada_rel}\\
        \ad_{x}\pn*{y^{n+1}b^{m-1}f(a)x^{k}} &=
                (n+1)y^{n}b^{m}f(a)x^{k} - y^{n+1}b^{m-1}\nabla[f](a)x^{k+1}
        \label{eq:adx_rel}\\
        \ad_{y}\pn*{y^n b^{m-1} f(a) x^{k+1}} &=
                - (k+1)y^n b^m f(a) x^k + y^{n+1} b^{m-1} \nabla[f](a)x^{k+1}
        \label{eq:ady_rel}
\end{align}
By \cref{eq:ada_rel}, any monomial whose powers of $y$ and $x$ differ vanish in
$\CU_{\Alg}$. As a consequence, in \cref{eq:adx_rel,eq:ady_rel}, the only
nontrivial case is when $n=k$, resulting in the same relation. By induction on
$n$, we conclude that:
\begin{equation}\label{eq:coinv_reduction}
        y^n b^m f(a) x^k \sim δ_{nk}\frac{n!}{(n+m)!}y^{n+m}\nabla^m[f](a)x^{n+m}
\end{equation}
where $\sim$ refers to equivalence in the set of coinvariants. Observing when
$f$ is a monomial in \cref{eq:coinv_reduction}, we see $\CU_{\Alg}$ is spanned
by $\set{y^n a^k x^n}_{n, k \ge 0}$.

Finally, all that remains to show is this set is linearly independent. This is
equivalent to no two $y^na^kx^n$'s with distinct choices of exponents being
related under a sequence of relations. Since \cref{lem:coinvLieAlg} allows us to
only consider sequences of relations of the form $\ad_z$ for $z$ a one-letter
word in $\set{y, b, a, x}$, inspection of the above comprehensive summary of all
one-letter relations (in particular, \cref{eq:coinv_reduction} with $m=0$)
allows us to conclude that this set is indeed linearly independent.
\end{proof}

\subsection{A generating function for the coinvariants}

\ProvideDocumentCommand{\COrder}{}{\Order}
In order to define a generating function, we need to choose an appropriate basis
for the space of coinvariants. We define an isomorphism from the space of
coinvariants to a polynomial space, tweaking the earlier-defined basis by scalar
multiples. Since it plays the role of the ordering map, we also name it
$\COrder$.
\begin{equation}
        \mapdef {\COrder} {\polyring{\Q}{a, z}} [\toiso] {\CU_\CU}
        {a^{n}z^{k}} {\frac{1}{k!}y^{k}a^{n}x^{k} \\
                k!\nabla^m[f](a)z^{k+m} &\mapsfrom y^kb^mf(a)x^k
        }
\end{equation}

This defines a commutative square upon whose bottom edge
$τ = \Order \then {\trace} \then {\inv{\COrder}}$ we compute the generating
function:
\begin{equation}
\begin{tikzcd}
        \CU
                \rar["\trace"]
        & \CU_\CU \\
        \polyring{\Q}{y, b, a, x}
                \uar["\Order"]
                \rar["τ"]
        &
        \polyring{\Q}{a, z}
                \uar["\COrder"]
\end{tikzcd}
\end{equation}

We begin with a result on finite differences:
\begin{lemma}[finite differences of exponentials]\label{eq:findiffexp}
        The finite difference operator acts in the following way on
        exponentials:
        \begin{equation}
                \nabla^n[\Exp{αa}](a) = (1-\Exp{-α})^n\Exp{αa}
        \end{equation}
\end{lemma}
\begin{proof}
Using the fact that $\nabla^n[f](x) = \Sum[k=0][n]\binom n k(-1)^kf(x-k)$, we
see that
$\nabla^n[\Exp{αa}](a)
        = \Sum[k=0][n]\binom n k(-1)^k\Exp{αa-αk}
        % = \Exp{αa}\Sum[k=0][n]\binom n k(-1)^k\Exp{-αk}
        = (1-\Exp{-α})^n\Exp{αa}$.
\end{proof}
We now are ready to compute the generating function for the trace:
\begin{theorem}[Generating function for the trace of $\CU$]
\begin{equation}\label{eq:trace_formula}
        \Gen\trace = \exp\pn[\Big]{αa+\pn[\big]{ηξ+β(1-\Exp{-α})}z}
\end{equation}
\end{theorem}
\begin{proof}
        Using \cref{eq:findiffexp} and the extension of scalars of $\trace$ to
        $\powerseries{\Q}{η, β, α, ξ}$, we see
        \begin{equation}
        \begin{aligned}
                &\Gen[\big]{\Order \then {\trace} \then {\inv{\COrder}}}
                = \pn[\big]{\Exp{η y} \Exp{β b} \Exp{α a} \Exp{ξ x}} \then
                        {\trace} \then {\inv\COrder}\\
                &= \inv\COrder\Sum[i, j, k]\trace\pn*{
                        \frac{(η y)^i}{i!}
                        \frac{(β b)^j}{j!}
                        \Exp{α a}
                        \frac{(ξ x)^k}{k!}
                }\\
                % &=\Sum[i, j, k]\frac{η^i β^j ξ^k}{i! j! k!}
                        % \trace\pn*{ y^i b^j \Exp{α a} x^k }\\
                % &=\Sum[i, j, k]\frac{η^i β^j ξ^k}{i! j! k!}
                        % δ_{ik}i!\nabla^j[\Exp{αa}](a)t^{i+j}\\
                &=\Sum[i, j]\frac{η^i β^j ξ^i}{i! j!}
                        (1-\Exp{-α})^j\Exp{αa}z^{i+j}
                =\Exp{αa+\pn*{ηξ+β(1-\Exp{-α})}z}\qedhere
        \end{aligned}
        \end{equation}
\end{proof}

\subsection{Evaluation of the trace on a generic element}
Here we will outline a computation involving the trace by using Bar-Natan and
van der Veen's Contraction Theorem.

A typical value for a tangle invariant that arises is of the form:
\begin{equation}\label{eq:sample_GDO}
        P\Exp{c + α a_i + β b_i + ξ(b_i) x_i + η(b_i)y_i + λ(b_i)x_i y_i}
\end{equation}
Here, $c$, $α$, and $β$ denote constants with respect to the variables $y_i$,
$b_i$, $a_i$, and $x_i$ (collectively referred to as \enquote{$v_i$}s), while
$ξ$, $η$, and $λ$ are potentially $b_i$-dependent, and $P$ is a (rational)
function in (the square root of) $B_i$.

\ProvideDocumentCommand{\ba}{}{\bar{a}}
\ProvideDocumentCommand{\bz}{}{\bar{z}}

\begin{theorem}[The trace of a Gaußian]
        With symbols as defined above, let $f(y_i, b_i, a_i, x_i) = P(B_i)
        \Exp{c + α a_i + β b_i + ξ(b_i) x_i + η(b_i)y_i + λ(b_i)x_i y_i}$. Then
\begin{equation}\begin{aligned}\label{eq:trace_on_gaussian}
\contraction*{ f(y_i, b_i, a_i, x_i) \trace^i }_{v_i}
&= \frac{P(\Exp{-μ})}
        {1-λ(μ)\bz_i}\Exp{c+α\ba_i+βμ+\frac{η(μ)ξ(μ)\bz_i}{1-λ(μ)\bz_i}}
\end{aligned}\end{equation}
where $μ\defeq (1-\Exp{-α})\bz_i$.
\end{theorem}
\begin{proof}
        Let us compute the trace of \cref{eq:sample_GDO}. For clarity, we will
        put bars over the coinvariants variables $a_i$ and $z_i$, as they do not
        play a role in the contraction.
        \begin{equation}\begin{aligned}\label{eq:trace_on_gaussian_computation}
                &\contraction{
                        P(B_i)
                        \Exp{c + α a_i + β b_i + ξ(b_i) x_i + η(b_i)y_i + λ(b_i)x_i y_i}
                        \trace^i
                }_{v_i}\\
                &= \contraction{
                        P(B_i)
                        \Exp{c + β b_i + ξ(b_i) x_i + η(b_i)y_i + λ(b_i)x_i y_i
                        +η_iξ_i\bz_i+β_i(1-\Exp{-α_i})\bz_i}
                        \Exp{α a_i + α_i\ba_i}
                }_{v_i}\\
                &= \Exp{α\ba_i}\contraction{
                        P(B_i)
                        \Exp{c + ξ(b_i) x_i + η(b_i)y_i + λ(b_i)x_i y_i
                        +η_iξ_i\bz_i}
                        \Exp{β b_i +β_i(1-\Exp{-α})\bz_i}
                }_{b_i, x_i, y_i}\\
                \alignedintertext{
                        In what follows, we let $μ\defeq (1-\Exp{-α})\bz_i$:
                }
                &= \Exp{c+α\ba_i+βμ}P(\Exp{-μ})\contraction{
                        \Exp{η(μ)y_i}
                        \Exp{(ξ(μ) + λ(μ)y_i)x_i + ξ_iη_i\bz_i}
                }_{x_i, y_i}\\
                &= \Exp{c+α\ba_i+βμ}P(\Exp{-μ})\contraction{
                        \Exp{η(μ)y_i+ξ(μ)\bz_iη_i + λ(μ)\bz_iη_iy_i}
                }_{y_i}\\
                &= \frac{P(\Exp{-μ})}
                {1-λ(μ)\bz_i}\Exp{c+α\ba_i+βμ+\frac{η(μ)ξ(μ)\bz_i}{1-λ(μ)\bz_i}}
        \end{aligned}\end{equation}
\end{proof}

We point out that the outcome of this computation is not guaranteed to be a
Gaußian. This puts a limitation on the applicability of this formula to links
with more than two components, explored in \cref{sec:limitations}.

\subsection{Computational examples}

Using the formula given in \cref{eq:trace_on_gaussian}, let us do some
preliminary examples:

\begin{align}
        \trace^i(R_{ij}) &= 1\label{eq:trR}\\
        \trace^j(R_{ij}) &= \Exp{b_i\ba_j}\label{eq:trR2}\\
        \begin{split}\label{eq:trHopf}
                \trace^2\pn[\Big]{\sqrt{B_2}\Exp{
                                -a_2 b_1-a_1 b_2 +
                                \frac{(B_1-1) x_2 y_1}{b_1 B_1}+
                        \frac{(B_2-1) x_1 y_2}{b_2 B_2}}
                } &= \\
                \Exp{
                        \frac{a_1 (\bz_2-B_1 \bz_2)}{B_1}-b_1 \ba_2+
                        \frac{e^{B_1 \bz_2}
                        (x_1 y_1 e^{\inv{B_1} \bz_2}-x_1 y_1 e^{\bz_2})}{b_1}+
                        \frac{1}{2} \inv{B_1} \bz_2-\frac{\bz_2}{2}
                }&%\\
        \end{split}
        % \trace^i(
                % P\Exp{c + α a_i + β b_i + ξ(b_i) x_i + η(b_i)y_i + λ(b_i)x_i y_i}
        % )
\end{align}

\Cref{eq:trR,eq:trR2} are the values one obtains for the two (virtual)
one-crossing, two-component link, while \cref{eq:trHopf} is the value of the
invariant on the Hopf link.

When computing this on a link, however, it is important to keep track of which
strands are open, and which are closed. We will extend the notation from the
previous section to differentiate between open and closed indices. We write a
morphism with domain $D = D_{\text{o}}\sqcup D_{\text{c}}$, codomain
$C = C_{\text{o}}\sqcup C_{\text{c}}$ (here $D_{\text{o}}$ denotes domain
indices which are open, while $D_{\text{c}}$ those which are closed, with the
same convention for $C$) and generating function $f(ζ_{D}, z_{C})$ as
$f(ζ_{D}, z_{C})^{(D_{\text{o}},D_{\text{c}})}_{(C_{\text{o}},C_{\text{c}})}$. 
