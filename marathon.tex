\documentclass{article}
\usepackage{math}
\usepackage[
        backend=biber,
        style=alphabetic
]{biblatex}
\addbibresource{\jobname.bib}
\usepackage{acronym}
\usepackage{listings}
        \lstset{basicstyle=\ttfamily}

\title{Computing the generating function of a coinvariants map}
\author{Jesse Frohlich}

\ProvideDocumentCommand{\fa}{}{\mfk{a}}
\ProvideDocumentCommand{\Alg}{}{\fg}
\ProvideDocumentCommand{\CU}{}{U}
\ProvideDocumentCommand{\nn}{}{\mathbf{n}}
\ProvideDocumentCommand{\Order}{}{\mathbb O}
\DeclarePairedDelimiterXPP{\Gen}[1]{\msc G}\lparen\rparen{}{#1}
\DeclarePairedDelimiter{\contraction}\langle\rangle
\newcommand{\alignedintertext}[1]{%
  \noalign{%
    \vskip\belowdisplayshortskip
    \vtop{\hsize=\linewidth#1\par
    \expandafter}%
    \expandafter\prevdepth\the\prevdepth
  }%
}

\begin{document}
\maketitle

\begin{abstract}
        A well-known source of strong link invariants comes from quantum groups.
        Typically, one uses a representation of a quantum group to build a 
        computable invariant, though these computations require exponential time
        in the number of crossings. Recent work has allowed for direct and
        efficient computations within the quantum groups themselves, through the
        use of perturbed Gaussian differential operators. This thesis introduces
        and explores a partial expansion of the tangle-theoretic computations
        performed by Bar-Natan and van der Veen \cite{BV} in the quantum group
        $\uea{\Sl_{2+}^0}$ to its space of coinvariants, providing an extension
        of this computational method from open tangles to links.

\end{abstract}

\newpage

\tableofcontents

\section{Introduction}

\subsection{Quantum link invariants}
Here we describe the use of quantum groups in knot theory, and the role they and
their representations play in building a knot invariant.

One defines braided categories, ribbon categories, and associates to the
fundamental operations (evaluation, coevaluation, the braiding (and its
inverse), the identity map) a graphical calculus which matches the behaviour of
framed tangles (that is, embeddings up to isotopy).

\begin{remark}
        By insisting that the ends of all strands in the diagram point up, and
        that only upward-pointing portions of strands participate in crossings,
        we may replace the (co)evaluation operations with the \enquote{spinner}
        element (also called the distinguished grouplike element), derived from
        the ribbon element and the Drinfeld element.
\end{remark}
In \cite{BV}, Bar-Natan and Van der Veen define an invariant of tangles valued
in tensor powers of a certain Hopf algebra $\CU$. Their work expresses the
algebra operations as perturbed Gaußian generating functions so as to produce a
strong polynomial-time tangle invariant.

\subsection{Defining the algebra}
Here we define the Hopf algebra $U$, it's quasitriangular structure, and its
ribbon structure:

We begin by defining the algebra $\CU$. Denote by $\fa$ the non-commutative
$2$-dimensional cocommutative Lie bialgebra spanned by $a$ and $x$ with relation
$\liebk{a}{x} = x$. (This is also a Borel subalgebra of $\Sl_2$.)

Next, we use the Drinfeld double construction (outlined in \cite{ES}) to obtain
a quasitriangular Lie algebra $\fg$. As a vector space,
$\fg = \fa \oplus \dual\fa$. Given $u \in \fa$ and $v\in \dual \fa$, we have
$\liebk{u}{v}_{\fg} \defeq \dual\ad_u(v) - \dual\ad_v(u)$, extended bilinearly
and anticommutatively to all of $\fg$.
Then the algebra $\CU$ is defined to be the universal enveloping algebra
$\uea{\fg}$.

\begin{remark}
        For convenience, we define $b \defeq \dual a \in \dual\fa$ and
        $y \defeq \dual x \in \dual\fa$, so that
        \begin{equation}
                U = \genbuilder*{y, b, a, x}{
                        \liebk{a}{x} = x,
                        \liebk{a}{y} = -y,
                        \liebk{x}{y} = b,
                        \liebk{b}{ } = 0
                }
        \end{equation}
        as an algebra.
\end{remark}


\subsection{Expressing morphisms as generating functions}

When defining a morphism-valued tangle invariant, one needs a compact way of
encoding the morphism. In \cite{BV} this is achieved through the use of
generating functions, whose definition we reproduce below:

For $A$ and $B$ finite sets, consider the set $\hom(\polyring{\Q}{z_A},
\polyring{\Q}{z_B})$ of linear maps between multivariate polynomial rings. Such
a map is determined by its values on the monomials $z_A^\nn$ for each
multi-index $\nn \in \N^A$.

\begin{definition}[Exponential generating function]
        The \defi{exponential generating function} of a map
        $\map {Φ} {\polyring{\Q}{z_A}} {\polyring{\Q}{z_B}}$ between polynomial
        spaces is
        \begin{equation}
                \Gen{Φ} \defeq
                \Sum[\nn\in\N^A] \frac{Φ(z_A^\nn)}{\nn!}ζ_A^\nn
                \in \powerseries{\polyring{\Q}{z_B}}{ζ_A}
        \end{equation}
\end{definition}
\begin{remark}
        Extending the definition of $Φ$ to
        $\powerseries{\polyring{\Q}{z_B}}{ζ_A}$ by the extending scalars to
        $\powerseries{\Q}{ζ_A}$ gives us an equivalent formulation:
        \begin{equation}
                \Gen{Φ}
                = Φ\pn*{\Sum[\nn\in\N^A] \frac{(z_Aζ_A)^\nn}{\nn!}}
                = Φ\pn*{\Gen[\big]{\id_{\polyring{\Q}{z_A}}}}
        \end{equation}
\end{remark}

By the PBW theorem, we know that $\CU$ is isomorphic as a vector space to the
polynomial ring $\polyring{\Q}{y, b, a, x}$ by choosing an ordering of the
generators (following \cite{BV}, we use $(y, b, a, x)$):
\begin{equation}
        \mapdef {\Order} {\polyring{\Q}{y, b, a, x}} [\toiso] {\CU}
        {y^{n_1}b^{n_2}a^{n_3}x^{n_4}} {y^{n_1}b^{n_2}a^{n_3}x^{n_4}}
\end{equation}

Using this vector space isomorphism, \cite{BV} expresses all Hopf algebra
operations as perturbed Gaußians. To extend the resulting tangle invariant to
one on links, one would need to define a trace operator on $\CU$. The first
natural place to look is the coinvariants,
$\CU_\CU = \fracl{\CU}{\liebk{\CU}{\CU}}$. In what follows, we will compute
$\CU_\CU$, determine a vector space isomorphism to a suitable polynomial ring,
and compute the corresponding generating function of the quotient map $\map
{\trace} {\CU} {\CU_\CU}$.

\subsection{Rotational tangle invariants from a ribbon Hopf algebra}
Here we describe the morphism from the category of pure rotational virtual
tangles to a ribbon Hopf algebra.

\section{Extending a pure tangle invariant to links and general tangles}
To build define a invariant on tangles with potentially closed components, we
need a notion of trace which is cyclic under stitching of strands. Here is a
description of such a trace algebraically and topologically:

\subsection{Algebraic trace}
Here we talk about coinvariant maps, and (maybe) traced monoidal categories.

\subsection{Topological trace}
Here we talk about stitching of strands topologically.

\subsection{Extending the categories}
Here we discuss the context in which the invariant acts, namely the extended
category of impure tangles, as well as an extension of the algebraic tensor
category to include coinvariants factors.

In the next section, we propose a candidate trace on the algebra $U$.

\section{The coinvariants of $U$}

We start with a result which simplifies working with coinvariants:

\begin{lemma}\label{lem:coinvLieAlg}
        Let $\fh$ be a Lie algebra. Then $\uea{\fh}_{\uea{\fh}} =
        \uea{\fh}_\fh$.
\end{lemma}
\begin{proof}
First, observe that for any $u$, $v$, $f\in\uea{\fh}$,
$\ad_{uv}(f) = \ad_u(vf) + \ad_v(fu)$. Proceeding inductively, for any monomial
$μ\in\uea{\fh}$, $\ad_{μ}(u)$ is a linear combination of elements of
$\liebk*{\fh}{\uea{\fh}}$. By linearity of $\ad$, we conclude
$\liebk*{\uea{\fh}}{\uea{\fh}} = \liebk*{\fh}{\uea{\fh}}$.
\end{proof}

\begin{theorem}
        The coinvariants of $U$, $U_U$, has basis
        $\set{y^n a^k x^n}_{n, k\ge 0}$.
\end{theorem}
\begin{proof}
Using \cref{lem:coinvLieAlg}, we need only compute $\liebk{\Alg}{\CU}$ to
determine $\CU_\CU$. Given a polynomial $f$, we have the
following relations in $U$:
\begin{align}
        f(a)y^r &= y^rf(a-r) &
        x^rf(a) &= f(a-r)x^r
\end{align}
Next we compute the adjoint actions of $y$, $a$, and $x$. (Recall $b$ is
central.)
\begin{align}
  \ad_a f(x) &= xf'(x)&
  \ad_a f(y) &= -yf'(y)\label{eq:ada}\\
  \ad_x f(y) &= bf'(y) &
  \ad_x f(a) &= -\nabla[f](a)x\label{eq:adx}\\
  \ad_y f(x) &= -bf'(x) &
  \ad_y f(a) &= y\nabla[f](a)\label{eq:ady}
\end{align}
(Here $\nabla$ is the backwards finite difference operator $\nabla[f](x) \defeq
f(x) - f(x-1)$.) Observe for any $n$, $m$, $k$, and polynomials $f$ and $g$:
\begin{align}
        \ad_a \pn*{y^m g(b, a) x^n } &= (n-m)y^mg(b, a) x^n
        \label{eq:ada_rel}\\
        \ad_{x}\pn*{y^{n+1}b^{m-1}f(a)x^{k}} &=
                (n+1)y^{n}b^{m}f(a)x^{k} - y^{n+1}b^{m-1}\nabla[f](a)x^{k+1}
        \label{eq:adx_rel}\\
        \ad_{y}\pn*{y^n b^{m-1} f(a) x^{k+1}} &=
                - (k+1)y^n b^m f(a) x^k + y^{n+1} b^{m-1} \nabla[f](a)x^{k+1}
        \label{eq:ady_rel}
\end{align}
By \cref{eq:ada_rel}, any monomial whose powers of $y$ and $x$ differ vanish in
$\CU_{\Alg}$. As a consequence, in \cref{eq:adx_rel,eq:ady_rel}, the only
nontrivial case is when $n=k$, resulting in the same relation. By induction on
$n$, we conclude that:
\begin{equation}\label{eq:coinv_reduction}
        y^n b^m f(a) x^k \sim δ_{nk}\frac{n!}{(n+m)!}y^{n+m}\nabla^m[f](a)x^{n+m}
\end{equation}
Observing when $f$ is a monomial in \cref{eq:coinv_reduction}, we see
$\CU_{\Alg}$ is spanned by $\set{y^n a^k x^n}_{n, k \ge 0}$. Since all relations are
accounted for, setting $m=0$  demonstrates this set is linearly independent, and
we have a basis.
\end{proof}

\subsection{A generating function for the coinvariants}

\ProvideDocumentCommand{\COrder}{}{\Order}
In order to define a generating function, we need to choose an appropriate basis
for the space of coinvariants. We define an isomorphism from the space of
coinvariants to a polynomial space, tweaking the earlier-defined basis by scalar
multiples. Since it plays the role of the ordering map, we also name it
$\COrder$.
\begin{equation}
        \mapdef {\COrder} {\polyring{\Q}{a, z}} [\toiso] {\CU_\CU}
        {a^{n}z^{k}} {\frac{1}{k!}y^{k}a^{n}x^{k} \\
                k!\nabla^m[f](a)z^{k+m} &\mapsfrom y^kb^mf(a)x^k
        }
\end{equation}

This defines a commutative square upon whose bottom edge
$τ = \Order \then {\trace} \then {\inv{\COrder}}$ we compute the generating
function:
\begin{equation}
\begin{tikzcd}
        \CU
                \rar["\trace"]
        & \CU_\CU \\
        \polyring{\Q}{y, b, a, x}
                \uar["\Order"]
                \rar["τ"]
        &
        \polyring{\Q}{a, z}
                \uar["\COrder"]
\end{tikzcd}
\end{equation}

We begin with a result on finite differences:
\begin{lemma}\label{eq:findiffexp}
        The finite difference operator acts in the following way on
        exponentials:
        \begin{equation}
                \nabla^n[\Exp{αa}](a) = (1-\Exp{-α})^n\Exp{αa}
        \end{equation}
\end{lemma}
\begin{proof}
Using the fact that $\nabla^n[f](x) = \Sum[k=0][n]\binom n k(-1)^kf(x-k)$, we
see that
$\nabla^n[\Exp{αa}](a)
        = \Sum[k=0][n]\binom n k(-1)^k\Exp{αa-αk}
        % = \Exp{αa}\Sum[k=0][n]\binom n k(-1)^k\Exp{-αk}
        = (1-\Exp{-α})^n\Exp{αa}$.
\end{proof}
We now are ready to compute the generating function for the trace:
\begin{theorem}[Generating function for the trace of $\CU$]
\begin{equation}\label{eq:trace_formula}
        \Gen\trace = \Exp{αa+\pn*{ηξ+β(1-\Exp{-α})}z}
\end{equation}
\end{theorem}
\begin{proof}
        Using \cref{eq:findiffexp} and the extension of scalars of $\trace$ to
        $\powerseries{\Q}{η, β, α, ξ}$, we see
        \begin{equation}
        \begin{aligned}
                &\Gen[\big]{\Order \then {\trace} \then {\inv{\COrder}}}
                = \pn[\big]{\Exp{η y} \Exp{β b} \Exp{α a} \Exp{ξ x}} \then
                        {\trace} \then {\inv\COrder}\\
                &= \inv\COrder\Sum[i, j, k]\trace\pn*{
                        \frac{(η y)^i}{i!}
                        \frac{(β b)^j}{j!}
                        \Exp{α a}
                        \frac{(ξ x)^k}{k!}
                }\\
                % &=\Sum[i, j, k]\frac{η^i β^j ξ^k}{i! j! k!}
                        % \trace\pn*{ y^i b^j \Exp{α a} x^k }\\
                % &=\Sum[i, j, k]\frac{η^i β^j ξ^k}{i! j! k!}
                        % δ_{ik}i!\nabla^j[\Exp{αa}](a)t^{i+j}\\
                &=\Sum[i, j]\frac{η^i β^j ξ^i}{i! j!}
                        (1-\Exp{-α})^j\Exp{αa}z^{i+j}
                =\Exp{αa+\pn*{ηξ+β(1-\Exp{-α})}z}\qedhere
        \end{aligned}
        \end{equation}
\end{proof}

\subsection{Evaluation of the trace on a generic element}
Here we will outline a computation involving the trace by using Bar-Natan and
van der Veen's Contraction Theorem.

A typical value for a tangle invariant that arises is of the form:
\begin{equation}\label{eq:sample_GDO}
        P\Exp{c + α a_i + β b_i + ξ(b_i) x_i + η(b_i)y_i + λ(b_i)x_i y_i}
\end{equation}
Here, $c$, $α$, and $β$ denotes a constant with respect to the variables $y_i$,
$b_i$, $a_i$, and $x_i$ (collectively referred to as \enquote{$v_i$}s), while
$ξ$, $η$, and $λ$ are potentially $b_i$-dependent, and $P$ is a (rational)
function in (the square root of) $B_i$.

Let us compute the trace of \cref{eq:sample_GDO}. For clarity, we will put bars
over the coinvariants variables $a_i$ and $z_i$, as they do not play a role in
the contraction.
\ProvideDocumentCommand{\ba}{}{\bar{a}}
\ProvideDocumentCommand{\bz}{}{\bar{z}}
\begin{equation}\begin{aligned}
        &\contraction{
                P(B_i)
                \Exp{c + α a_i + β b_i + ξ(b_i) x_i + η(b_i)y_i + λ(b_i)x_i y_i}
                \trace^i
        }_{v_i}\\
        &= \contraction{
                P(B_i)
                \Exp{c + β b_i + ξ(b_i) x_i + η(b_i)y_i + λ(b_i)x_i y_i
                +η_iξ_i\bz_i+β_i(1-\Exp{-α_i})\bz_i}
                \Exp{α a_i + α_i\ba_i}
        }_{v_i}\\
        &= \Exp{α\ba_i}\contraction{
                P(B_i)
                \Exp{c + ξ(b_i) x_i + η(b_i)y_i + λ(b_i)x_i y_i
                +η_iξ_i\bz_i}
                \Exp{β b_i +β_i(1-\Exp{-α})\bz_i}
        }_{b_i, x_i, y_i}\\
        \alignedintertext{In what follows, we let $μ\defeq (1-\Exp{-α})\bz_i$:}
        &= \Exp{c+α\ba_i+βμ}P(\Exp{-μ})\contraction{
                \Exp{η(μ)y_i}
                \Exp{(ξ(μ) + λ(μ)y_i)x_i + ξ_iη_i\bz_i}
        }_{x_i, y_i}\\
        &= \Exp{c+α\ba_i+βμ}P(\Exp{-μ})\contraction{
                \Exp{η(μ)y_i+ξ(μ)\bz_iη_i + λ(μ)\bz_iη_iy_i}
        }_{y_i}\\
        &= \frac{P(\Exp{-μ})}
                {1-λ(μ)\bz_i}\Exp{c+α\ba_i+βμ+\frac{η(μ)ξ(μ)\bz_i}{1-λ(μ)\bz_i}}
\end{aligned}\end{equation}

\subsection{Computational examples}

Using the formula given in \cref{eq:trace_on_gaussian}, let us do some
preliminary examples:

\begin{align}
        \trace^i(R_{ij}) &= 1\label{eq:trR}\\
        \trace^j(R_{ij}) &= \Exp{b_i\ba_j}\label{eq:trR2}\\
        \begin{split}\label{eq:trHopf}
                \trace^2\pn[\Big]{\sqrt{B_2}\Exp{
                                -a_2 b_1-a_1 b_2 +
                                \frac{(B_1-1) x_2 y_1}{b_1 B_1}+
                        \frac{(B_2-1) x_1 y_2}{b_2 B_2}}
                } &= \\
                \Exp{
                        \frac{a_1 (\bz_2-B_1 \bz_2)}{B_1}-b_1 \ba_2+
                        \frac{e^{B_1 \bz_2}
                        (x_1 y_1 e^{\inv{B_1} \bz_2}-x_1 y_1 e^{\bz_2})}{b_1}+
                        \frac{1}{2} \inv{B_1} \bz_2-\frac{\bz_2}{2}
                }&%\\
        \end{split}
        % \trace^i(
                % P\Exp{c + α a_i + β b_i + ξ(b_i) x_i + η(b_i)y_i + λ(b_i)x_i y_i}
        % )
\end{align}

\Cref{eq:trR,eq:trR2} are the values one obtains for the two (virtual)
one-crossing, two-component link, while \cref{eq:trHopf} is the value of the
invariant on the Hopf link.

\subsection{Limitations of this definition}
For some inputs to the trace, expressions involving the Lambert $W$-function
appear, which complicates attempts to keep the invariant valued in the space of
perturbed Gaußians.

\section{Generating \acp{RVT} for links}

\subsection{\acp{RVT} for knots}
Describe algorithm previously developed for knots

\subsection{Extending the algorithm to multiple components}

\begin{definition}[\ac{RVT}]
        \Iac{RVT} is a virtual tangle diagram together with an integral rotation
        number attached to each arc, subject to the following so-called
        rotational Reidemeister moves:
        \ProvideDocumentCommand{\Reid}{sm}{\mathrm{R}\IfBooleanT{#1}{'}_{#2}}
        \begin{equation}
                \Reid1, \Reid2, \Reid3, \text{Wh} %todo: replace with diagrams.
        \end{equation}
\end{definition}

Given a classical link, there is a unique \ac{RVL} corresponding to it. Given a
classical link diagram, one may obtain the corresponding \ac{RVL} by attaching
an appropriate rotation number to each arc. However, there is not a unique way
to do so.

The situation becomes more complicated when one considers the case where the
tangle has an open component. In this case, two \ac{RVT} diagrams which
correspond to the same classical link exactly when they differ only by a
sequence of rotational Reidemeister moves \emph{and} a modification of the
rotation numbers of the (two) unbounded arcs. Equivalently, we have the
statement:

\begin{lemma}
        For each classical tangle with one open component, there exists a unique
        \ac{RVT} whose unbounded arcs have rotation numbers $0$.
\end{lemma}
\begin{proof}
        See \cite{BV}.
\end{proof}

Bar-Natan and Van der Veen develop an algorithm to convert a classical long knot
into an \ac{RVT}. As we are interested in links, we must extend this algorithm
to include so-called \enquote{long links}, which we outline below:
\begin{verbatim}
        1. Pass a front over the beginning of the open strand.
        2. Progressively absorb the leftmost crossings 
                2a. As crossings are absorbed,
                    take into account any rotations of arcs.
        3. If an arc passes through the front twice, absorb it,
           taking into account any rotations of that arc as a result.
\end{verbatim}

\section{Comparison with the \ac{MVA}}

Given that the long knot (i.e. one-component) case of this invariant
encodes the Alexander Polynomial, it was suspected that the invariant on long
links (i.e. multiple components, one of which is long) formed by adding the
trace would encode the \ac{MVA}. However, there are links which the \ac{MVA}
separates which this invariant does not.

On all two-component links with at most $11$ crossings (a collection of size
$914$), the trace map attains $878$ distinct values, while the MVA attains only
$778$. However, the two invariants are incomparable in terms of their strength.

\section{Further work}
While all other Hopf algebra operations in $U$ are expressed by \cite{BV} as
perturbed Gaußians, the form in \cref{eq:trace_formula} does not to conform to
the same structure. Further work is needed to either implement this operation
into the established framework, or to suitably extend the framework.

\appendix

\section{Code implementation}

\subsection{Implementation of $Z$ invariant (from \cite{BV})}

This is a Mathematica™ implementation by Bar-Natan and van der Veen, modified by
the author.

\subsection{Implementation of $\trace$}

This is a Mathematica™ implementation.

\subsection{Implementation of \lstinline|toRVT|}

This is a Haskell implementation.

\section{Acronyms}
\begin{acronym}
        \acro{RVT}{Rotational Virtual Tangle}
        \acro{RVK}{Rotational Virtual Knot}
        \acro{RVL}{Rotational Virtual Link}
        \acroindefinite{RVT}{an}{a}
        \acroindefinite{RVK}{an}{a}
        \acroindefinite{RVL}{an}{a}
        \acro{MVA}{Multivariable Alexander polynomial}
\end{acronym}

\printbibliography
\end{document}
