\documentclass{article}
\usepackage{math}
\usepackage[
        backend=biber,
        style=alphabetic
]{biblatex}
\addbibresource{\jobname.bib}
\usepackage{acronym}
\usepackage{listings}
        \lstset{basicstyle=\ttfamily}

\title{Computing the generating function of a coinvariants map}
\author{Jesse Frohlich}

\ProvideDocumentCommand{\invb}{m}{\overline{#1}}

\ProvideDocumentCommand{\tangle}{}{\msc T}
\ProvideDocumentCommand{\RVT}{}{\msc{T}^{\text{rv}}}

\ProvideDocumentCommand{\Id}{m}{\id^{#1}_{#1}}
\ProvideDocumentCommand{\Rmati}{}{\invb\Rmat}
\ProvideDocumentCommand{\spin}{}{C}
\ProvideDocumentCommand{\ribbon}{}{ν}
\ProvideDocumentCommand{\dfe}{}{\mfk u}
\ProvideDocumentCommand{\monodromy}{}{Q}
\ProvideDocumentCommand{\fa}{}{\mfk{a}}
\ProvideDocumentCommand{\Alg}{}{\fg}
\ProvideDocumentCommand{\CU}{}{U}
\ProvideDocumentCommand{\nn}{}{\mathbf{n}}
\ProvideDocumentCommand{\Order}{}{\mathbb O}
\RenewDocumentCommand{\k}{}{\mbf k}
\DeclarePairedDelimiterXPP{\Gen}[1]{\msc G}\lparen\rparen{}{#1}
\DeclarePairedDelimiter{\contraction}\langle\rangle
\newcommand{\alignedintertext}[1]{%
  \noalign{%
    \vskip\belowdisplayshortskip
    \vtop{\hsize=\linewidth#1\par
    \expandafter}%
    \expandafter\prevdepth\the\prevdepth
  }%
}

\begin{document}
\maketitle

\begin{abstract}
        A well-known source of strong link invariants comes from quantum groups.
        Typically, one uses a representation of a quantum group to build a
        computable invariant, though these computations require exponential time
        in the number of crossings. Recent work has allowed for direct and
        efficient computations within the quantum groups themselves, through the
        use of perturbed Gaussian differential operators. This thesis introduces
        and explores a partial expansion of the tangle-theoretic computations
        performed by Bar-Natan and van der Veen \cite{BV} in the quantum group
        $\uea{\Sl_{2+}^0}$ to its space of coinvariants, providing an extension
        of this computational method from open tangles to links.

\end{abstract}

\newpage

\tableofcontents

\section{Introduction}

\subsection{Tensor product notation}
In what follows, we will extensively use tensor products, and more generally
monoidal categories. As all our examples will involve trivial associators, it
will be more convenient to label tensor factors with elements of a finite set
rather than by their position in a linear order. We will put the label as a
subscript on each factor. For a finite set $S = \set{i_1, \dots, i_n}$:
\begin{equation}\label{eq:tensor_notation}
        V_S \defeq
        V^{\otimes S} \defeq
        V_{i_1}\otimes V_{i_2}\otimes\dots\otimes V_{{i_n}} \iso V^{\otimes n}
\end{equation}
In particular, $V_{\emptyset} = \k$ is the base field of $V$.
Pure tensors in this notation will be written with subscripts and the $\otimes$
will be suppressed:
$(v_1)_{i_1}(v_2)_{i_2}\cdots(v_n)_{i_n} \in V^{\otimes\set{i_1,\dots,i_n}}$.
Observe that with labelled factors, the order the factors are written in does
not matter: In $V^{\otimes{1,2}}$, we have $x_1y_2 = y_2x_1$.

Next, given a map
$\map {T} {V^{\otimes\set{j_1,\dots,j_s}}} {V^{\otimes\set{i_1,\dots,i_r}}}$
between tensor powers of $V$, we denote $T$ by
$T^{j_1,\dots,j_s}_{i_1,\dots, i_r}$. It is important to note that the order of
the indices in this notation matters for nonsymmetric tensors.
Furthermore, this same symbol will denote extensions by the identity to any
arbitrary finite set $S$ disjoint from the $j_k$'s and the $i_k$'s:
\begin{equation}\label{eq:tensor_extend}
        \mapdef {T^{j_1,\dots,j_s}_{i_1,\dots, i_r}}
                {V^{\otimes\set{j_1,\dots,j_s}\sqcup S}}
                {V^{\otimes\set{i_1,\dots,i_r}\sqcup S}}
                {v_{j_1,\dots,j_s}\otimes w_S} 
                {(Tv_{j_1,\dots,j_s})\otimes w_S} 
\end{equation}

\begin{remark}
        There are three special cases with this notation:
        \begin{itemize}
                \item We denote (multi)linear functionals with only superscripts:
                        $\map {ϕ} {V^{\otimes S}} {\k \iso V^{\otimes
                        \emptyset}}$ will be written $ϕ^S$ (with a linear order
                        put on $S$).
                \item Elements $v\in V^{\otimes S}$ will be interpreted as a map
                        $\map {v} {\k} {V^{\otimes S}}$ written with a (linearly
                        ordered) subscript: $v_S$ to denote which factors the
                        element lives in. Following \cref{eq:tensor_extend},
                        this notation allows us to embed $v$ into higher tensor
                        powers: $v_S \in V^{\otimes S\sqcup X}$ for a $X$ finite
                        set.
                \item When only one index is present in a subscript or
                        superscript, and its omission does not introduce an
                        ambiguity in an expression, then it may be omitted to
                        improve readability.
                      
\end{itemize}
\end{remark}

\subsubsection{Notation extension to non-Cartesian monoidal categories}
\label{sec:monoidal_notation}

\ProvideDocumentCommand{\CC}{}{\msc C}
\ProvideDocumentCommand{\MM}{}{A}

Some of the monoidal categories we will work with are not Cartesian.
Additionally, the ones we will work with are best suited to their factors being
labelled by finite sets. For the purpose of representing them clearly, we
introduce a notation put forth by Bar-Natan: consider an object $\MM$ in a
monoidal category $\CC$. Then:
\begin{itemize}
        \item Monoidal powers of $\MM$ are indexed by finite sets $S$, and
                are to be written $\MM_S$.
        \item Homsets in $\CC$ between monoidal powers of $\MM$ are indexed by
                pairs of finite sets $D$, $C$. Morphisms in these homsets will
                be denoted by $\map{ϕ^D_C}{\MM_D}{\MM_C}$.
        \item Composition of morphisms $ϕ^{D_1}_{C_1}$ and
                $ψ^{D_2}_{C_2}$ is defined when $D_2 = C_1$, and is
                written with the following concatenation operator: 
                $\map{ϕ^{D_1}_{C_1}\then ψ^{D_2}_{C_2}}{\msc
                C_{D_1}}{\CC_{C_2}}$.
        \item The monoidal product
                $\map {\otimes} {\CC \times \CC}{\CC}$ satisfies
                $\MM_{S} \otimes \MM_{T} = \MM_{S \sqcup T}$. Given
                morphisms $ϕ^{D_1}_{C_1}$ and $ψ^{D_2}_{C_2}$ such that
                $D_1 \cap D_2 = \emptyset = C_1 \cap C_2$, we have a product
                morphism $\map {ϕ \otimes ψ} {\CC_{D_1 \sqcup D_2}}
                {\CC_{C_1\sqcup C_2}}$.\footnote{This encodes the
                        data of a strict monoidal category with the
                        linear order of factors replaces with indices
                        from a finite set.
                } We will write such products with concatenation:
                $ϕ^{D_1}_{C_1}ψ^{D_2}_{C_2}$.
\end{itemize}

\begin{remark}
        Given a morphism $ϕ^D_C$ and a finite set $S$, there is an extension of $ϕ^D_C$
        to a morphism
        $\map {\overline{ϕ}} {\msc C_{D\sqcup S}} {\msc C_{C \sqcup S}}$ given
        by $ϕ^D_C \id^S_S$. To make expressions easier to read, in this paper we
        will introduce this extension implicitly in the following context: given
        morphisms $ϕ^{D_1}_{C_1}$ and $ψ^{D_2}_{C_2}$ such that
        $D_2 \subseteq C_1$ and
        $C_2 \cap (C_1\setminus D_2)=\emptyset=D_1\cap(D_2 \setminus C_1)$, we
        define:
        \begin{equation}\label{eq:composition_extension}
                ϕ^{D_1}_{C_1} \then ψ^{D_2}_{C_2}
                \defeq 
                        \pn*{\Id{{D_2\setminus C_1}}ϕ^{D_1}_{C_1}} \then
                        \pn*{ψ^{D_2}_{C_2}\Id{C_1\setminus D_2}}
        \end{equation}
\end{remark}
The two extreme cases of this definition are:
\begin{itemize}
        \item When $C_1 \cap D_2 = \emptyset$, the bifunctoriality of $\otimes$ reduces
                \cref{eq:composition_extension} to the monoidal product
                $ϕ^{D_1}_{C_1}ψ^{D_2}_{C_2}$ (here written with concatenation).
        \item When $C_1 = D_2$, \cref{eq:composition_extension} becomes the
                composition $ϕ^{D_1}_{C_1}\then ψ^{D_2}_{C_2}$ exactly.
\end{itemize}

\begin{remark}
        While the $\then$ operator is associative, care must be taken that the
        compositions are well-defined in the presence of duplicated indices.
        While it is sufficient for all the finite sets in a composition to be
        pairwise disjoint, this condition will prove too restrictive for clear
        communication of formulae.
\end{remark}
%TODO: add a proof of associativity? (Likely not necessary.)

\subsection{Meta-objects}

The notion of an \enquote{-object} in a category (for example, a monoid object
or an algebra object) is a useful way of generalizing an algebraic structure to
a different mathematical context. For instance, in the category of topological
spaces, one is able to talk about groups whose operations are continuous.
However, this notion only makes sense when we are working with a Cartesian
category (or more generally a monoidal category).

One of the main characters of this story does not fit into this mould, so we
will introduce a generalization of this notion by way of example:

Consider a group object. Traditionally, the data of a group object are the
following:
\begin{itemize}
        \item An object $G$ in a category $\CC$.
        \item A morphism $\map {\mult} {G\times G} {G}$ called
                \enquote{multiplication}.
        \item A \enquote{unit} morphism
                $\map {\unit} {\set{*}} {G}$.\footnote{When $\CC = \Set$, we
                usually write the unit as an element $e=\unit(*)\in G$
        }
        \item An \enquote{inversion} morphism $\map {\antipode} {G} {G}$.
        \item A collection of relations between the morphisms, written as
                equalities of morphisms between Cartesian powers of $G$. For
                example, associativity may be written:
                \begin{equation}\label{eq:cd_assoc}
                \begin{tikzcd}
                        G\times G\times G
                                \rar["\mult \times \id"]
                                \dar["\id \times\mult"']
                        &G \times G
                                \dar["\mult"] \\
                        G \times G
                                \rar["\mult"']
                        &G
                \end{tikzcd}
                \end{equation}
\end{itemize}
Further, the data of these relations is extended to higher powers of $G$ by
acting on other components by the identity:
\begin{equation}\label{eq:cd_assoc_extend}
\begin{tikzcd}[column sep=large]
        G^{n+3}
                \rar["\mult \times \id^{n+1}"]
                \dar["\id \times\mult\times \id^{n}"']
        &G^{n+2}
                \dar["\mult\times \id^{n}"] \\
        G^{n+2}
                \rar["\mult\times \id^{n}"']
        &G^{n+1}
\end{tikzcd}
\end{equation}

Consider now two changes in how we package these data:
\begin{enumerate}
        \item Instead of linear orders of factors $G \times \dots \times G$, we
                will index factors by a finite set $S$, writing the power $G_S$
                in the style of \cref{eq:tensor_notation}.
        \item Instead of implicitly including extensions of morphisms to higher
                powers by the identity, we will parametrize the extension by
                finite sets. For example, multiplication $\map {\mult^{ij}_{k}}
                {G_{\set{i,j}}} {G_{\set{k}}}$\footnote{This notation is defined
                        in \cref{eq:tensor_notation}, and is here extended from
                        tensor products to Cartesian products.
                } will be viewed as a family of maps
                $\map {\mult^{ij}_k[S]} {G_{\set{i,j}\sqcup S}}
                {G_{\set{k}\sqcup S}}$, each of which must satisfy the relations
                of the group object.
\end{enumerate}
This way of packaging the data leads us to a direct generalization: a
\defi{meta-group}-object in $\CC$ is a collection of objects $G_S\in \CC$
indexed by finite sets $S$, together with 
\begin{itemize}
        \item A family of objects $G_S\in \CC$, indexed over finite sets $S$.
        \item A family of morphisms $\map {\mult^{ij}_{k}[S]} {G_{\set{i,j}\sqcup S}}
                {G_{\set{k}\sqcup S}}$ called \enquote{multiplication}.
        \item A family of \enquote{unit} morphisms
                $\map {\unit_{i}[S]} {G_S} {G_{\set{i}\sqcup S}}$.
        \item An family of \enquote{inversion} morphisms $\map {\antipode^{i}_{j}[S]}
                {G_{\set{i}\sqcup S}} {G_{\set{j}\sqcup S}}$.
        \item A collection of relations between the morphisms, written as
                equalities of morphisms between the $G_X$'s. For
                example, associativity may be written:
                \begin{equation}\label{eq:cd_assoc}
                \begin{tikzcd}[column sep=huge]
                        G_{\set{1,2,3}\sqcup S}
                                \rar["\mult^{1,2}_{1}\bk*{S\sqcup\set{3}}"]
                                \dar["\mult^{2,3}_{2}\bk*{S\sqcup\set{1}}"']
                        &G_{\set{1,3}\sqcup S}
                                \dar["\mult^{1,3}_{1}\bk{S}"] \\
                        G_{\set{1,2}\sqcup S}
                                \rar["\mult^{1,2}_{1}\bk{S}"']
                        &G_{\set{1}\sqcup S}
                \end{tikzcd}
                \end{equation}
\end{itemize}
We obtain traditional objects by further requiring that each family of morphisms
$ϕ[S]$ satisfy $ϕ[S] = ϕ\bk{\emptyset} \times \id_S$. 

Several examples of well-known algebraic structures presented as meta-objects
are given in \cref{sec:alg_defs}.

\subsection{Algebraic definitions}\label{sec:alg_defs}

We now introduce the algebraic structures which will be used to define the
tangle invariant. These definitions follow those given by Majid in \cite{SM},
although the ones presented below are given in a way that their corresponding
meta-structure is readily visible.

\begin{definition}[algebra]
        A \defi{algebra} is an object $A\in\CC$ together with an associative
        multiplication $\map {\mult^{i,j}_{k}} {A_{\set{i,j}}}
        {A_{\set{k}}}$ (satisfying \cref{eq:cd_mult}), and a unit
        $\map{\unit_{i}}{A_{\emptyset}}{A_{\set{i}}}$ satisfying
        \cref{eq:cd_unit}.\footnote{
                When $\CC = \Vect$, this is becomes the more familiar definition
                of an \defi{algebra}. When $A_\emptyset$ is a field, and it is
                more common think of the unit as an element $\one\in A$. The
                unit map is then defined by linearly extending the assignment
                $\unit_{i}(1) = \one_{i}$.
        }
\end{definition}

\begin{multicols}{2}\noindent
\begin{equation}\label{eq:cd_mult}
\begin{tikzcd}
        A_{\set{1,2,3}}
                \rar["\mult^{1,2}_{1}"]
                \dar["\mult^{2,3}_{2}"']
        &A_{\set{1,3}}
                \dar["\mult^{1,3}_{1}"] \\
        A_{\set{1,2}}
                \rar["\mult^{1,2}_{1}"']
        &A_{\set{1}}
\end{tikzcd}
\end{equation}
\columnbreak
\begin{equation}\label{eq:cd_unit}
\begin{tikzcd}[column sep=large]
        A_{\set{1}}
                \rar["\unit_{2}"]
                \drar["\id"']
        &A_{\set{1,2}}
                \dar["\mult^{1,2}_{1}", shift left]
                \dar["\mult^{2,1}_{1}"', shift right] \\
        &A_{\set{1}}
\end{tikzcd}
\end{equation}
\end{multicols}

\begin{remark}
        From now on, we will denote repeated multiplication as in
        \cref{eq:cd_mult} by using extra indices. For instance:
        $\mult^{i,j, k}_{\ell} \defeq \mult^{i,j}_{r}\then\mult^{r, k}_{\ell}
        = \mult^{j, k}_{s}\then\mult^{i, s}_{\ell}$.
\end{remark}

There is also the dual notion of a \emph{coalgebra}, which arises by reversing
the arrows in \cref{eq:cd_mult,eq:cd_unit}:

\begin{definition}[coalgebra]
        A \defi{colagebra} is a vector space $C$ over a field $\k$ with a
        \defi{comultiplication} $\map {\comult} {C} {C\otimes C}$ which is
        \defi{coassociative} \eqref{eq:cd_comult} and a \defi{counit}, which is
        a map $\counit\colon A\to k$ satisfying \eqref{eq:cd_counit}.
\end{definition}
\nopagebreak
\begin{multicols}{2}\noindent
\begin{equation}\label{eq:cd_comult}
\begin{tikzcd}
        C_{\set{1,2,3}}
        &C_{\set{2,3}}
                \lar["\comult^{1}_{{1,2}}"'] \\
        C_{\set{1,2}}
                \uar["\comult^{2}_{2,3}"]
        &C_{\set{1}}
                \lar["\comult^{1}_{1,2}"]
                \uar["\comult^{1}_{1,3}"']
\end{tikzcd}
\end{equation}
\columnbreak
\begin{equation}\label{eq:cd_counit}
\begin{tikzcd}
        C_{\set{1}}
        &C_{\set{1,2}}
                \lar["\counit^{2}"']\\
        &C_{\set{1}}
                \ular["\id", shift left]
                \uar["\comult^{1}_{1,2}", shift left]
                \uar["\comult^{1}_{2,1}"', shift right] \\
\end{tikzcd}
\end{equation}
\end{multicols}

\begin{remark}
        From now on, we will denote repeated comultiplication as in
        \cref{eq:cd_comult} by using extra indices. For instance:
        $\comult^{i}_{j, k, \ell}
        \defeq \comult^{i}_{j,r}\then\comult^{r}_{k, \ell}
        = \comult^{i}_{s,\ell}\then\comult^{s}_{j, j}$.
\end{remark}

If a vector space $B$ satisfies both definitions of an algebra and a coalgebra,
we introduce a definition for when the structures are compatible with each other
in the following way:

\begin{definition}[bialgebra]
        A \defi{bialgebra} is an algebra $(B,\mult,\unit)$ and a coalgebra
        $(B,\comult,\counit)$, such that $\comult$ and $\counit$ are algebra
        morphisms.\footnote{
                $B^{\otimes n}$ inherits a (co)algebra structure
                from $B$, given by
                component-wise operations. For instance, in the case of
                multiplication, this
                means $\mult\pn[\big]{(a_1\otimes b_1) \otimes (a_2 \otimes
                b_2)} = a_1a_2 \otimes b_1 b_2$. The bialgebra structure on
                $B_{\emptyset}$ is given by
                $\mult = \unit = \comult = \counit = \id$.
        }
\end{definition}

\ProvideDocumentCommand{\lift}{mm}{\curryIsolated{#1}^{(#2)}}

\begin{multicols}{2}\noindent
\begin{equation}\label{eq:cd_mult_comult}
\begin{tikzcd}[column sep=large]
        B_{\set{1,3}}
                \rar["\mult^{1,3}_{1}"]
                \dar["\comult^{1}_{1,2}\then\comult^{3}_{3,4}"']
        &B_{\set{1}}
                \dar["\comult^{1}_{1,2}"] \\
        B_{\set{1,2,3,4}}
                \rar["\mult^{1,3}_{1}\then\mult^{2,4}_{2}"']
        &B_{\set{1,2}}
\end{tikzcd}
\end{equation}\begin{equation}\label{eq:cd_unit_comult}
\begin{tikzcd}[row sep=tiny]
        &B_{\set{1}}
                \ar[dd,"\comult^{1}_{1,2}"] \\
        B_{\emptyset}
                \urar["\unit_{1}"]
                \drar["\unit_{1}\then\unit_{2}"',near end]\\
        &B_{\set{1,2}}
\end{tikzcd}
\end{equation}
\columnbreak
\begin{equation}\label{eq:cd_mult_counit}
\begin{tikzcd}[column sep=tiny]
        B_{\set{1,2}}
                \ar[rr,"\mult^{1,2}_{1}"]
                \drar["\counit^{1}\then\counit^{2}"']
        &&B_{\set{1}}
                \dlar["\counit^{1}"] \\
        &B_{\emptyset}
\end{tikzcd}
\end{equation}
\begin{equation}\label{eq:cd_unit_counit}
\begin{tikzcd}
        B_{\emptyset}
                \rar["\unit_{1}"]
                \drar["\id"']
        &B_{\set{1}}
                \dar["\counit^{1}"] \\
        &B_{\emptyset}
\end{tikzcd}
\end{equation}
\end{multicols}

\begin{remark}
        The conditions for $\comult$ being an algebra morphism are presented in
        \cref{eq:cd_mult_comult,eq:cd_unit_comult}, while those for $\counit$
        are in \cref{eq:cd_mult_counit,eq:cd_unit_counit}.\footnote{While
        notation explicitly naming each tensor factor appears cumbersome in
        these diagrams, it will prove invaluable later when used on tangle
        diagrams, so we leave it as is for the sake of consistency.} Observing
        invariance under arrow reversal, it may not come as a surprise that
        \cref{eq:cd_mult_comult,eq:cd_mult_counit} also are the conditions for
        $\mult$ being a coalgebra morphism, and
        \cref{eq:cd_unit_comult,eq:cd_unit_counit} tell us that $\unit$ is as
        well.
\end{remark}

Finally, we introduce the invertibility condition we would expect on a quantum
group.
\begin{definition}[Hopf algebra]
A \defi{Hopf algebra} is a bialgebra $H$ together with a map $\map {\antipode} {H}
{H}$ called the \defi{antipode}, which satisfies for all $h\in H$,
$\comult^{1}_{1,2}\then \antipode^1_1 \then \mult^{1,2}_1 =
\counit^{1}\then\unit_{1} = 
\comult^{1}_{1,2}\then \antipode^2_2 \then \mult^{1,2}_1$.
As a commutative diagram, this looks like \cref{eq:cd_antipode}
\begin{equation}
\begin{tikzcd}[column sep=tiny]\label{eq:cd_antipode}
        H_{\set{1}}
                \arrow[rr, "\counit^{1}"] \arrow[rd, "\comult^{1}_{1,2}"']
        && H_{\emptyset}
                \arrow[rr, "\unit_{1}"]
        && H_{\set{1}} \\
        & H_{\set{1,2}}
                \arrow[rr, "\antipode^{2}_{2}", shift left]
                \arrow[rr, "\antipode^{1}_{1}"', shift right]
        && H_{\set{1,2}} \arrow[ru, "\mult^{1,2}_{1}"']
\end{tikzcd}
\end{equation}
\end{definition}

In order to do knot theory, we need an algebraic way to represent a crossing of
two strands. This is accomplished by the so-called $\Rmat$-matrix:

\begin{definition}[quasitriangular Hopf algebra]
A \defi{quasitriangular Hopf algebra} is a Hopf algebra $H$, together with an
invertible element $\Rmat \in H\otimes H$, called the \defi{$\Rmat$-matrix},
which satisfies the following properties: (we will denote the inverse by
$\Rmati$)
\begin{align}
        \label{eq:Rmat_overstrand}
        \Rmat_{12}\then\comult^{2}_{23}&=\Rmat_{a2}\Rmat_{b3}\then\mult^{ab}_1\\
        \Rmat_{13}\then\comult^{1}_{12}&=\Rmat_{1b}\Rmat_{2a}\then\mult^{ab}_3\\
        \comult^{1}_{21} &= 
                \comult^{1}_{12} \Rmat_{2_i,1_i}\Rmati_{2_f,1_f}\then
                \mult^{1_i,1,1_f}_{1}\then \mult^{2_i,2,2_f}_{2}
\end{align}
\end{definition}

\begin{definition}[Drinfeld element]
        In a quasitriangular Hopf algebra $H$, the \defi{Drinfeld element},
        $\dfe \in H$ is given by:
        \begin{equation}
                \dfe \defeq \Rmat_{21}\then\antipode^2_2 \then \mult^{12}
        \end{equation}
\end{definition}

\begin{definition}[monodromy]
        The \defi{monodromy}
        $\monodromy_{12} \defeq
        \Rmat_{12}\Rmat_{34}\then\mult^{14}_{1}\then\mult^{23}_{2}$. It's
        inverse will be denoted
        $\invb\monodromy =
        \Rmati_{12}\Rmati_{34}\then\mult^{14}_{1}\then\mult^{23}_{2}$.
\end{definition}

\begin{lemma}
        The Drinfeld element $\dfe$ satisfies for all $h\in H$:
        \begin{align}
                \dfe_{1}h_2\dfe_{3} \then \mult^{1,2,3}
                &= h \then S \then S\\
                \dfe \then \comult_{12} 
                &= \dfe_1\dfe_2\invb\monodromy_{34}
                \then\mult^{13}_{1}\then\mult^{24}_2
        \end{align}
\end{lemma}
\begin{proof}
        See \cite{SM} or \cite{ES} %or an original reference?
        for more details on this standard result. Note that the proof does not
        rely on the additive structure of the Hopf algebra, which allows us to
        extend this result to the realm of meta-Hopf algebras.
        %TODO: finish
\end{proof}

\begin{definition}[ribbon Hopf algebra]
        A quasitriangular Hopf algebra $H$ is called \defi{ribbon} if it has an
        element $\ribbon\in \centre(H)$ such that:
        \begin{align}
                \ribbon_1\ribbon_2\then\mult^{12}
                &= \dfe_1 \dfe_2 \then \antipode^{2}_{2} \then \mult^{12}\\
                \ribbon_1 \then \comult^1_{12}
                &=      \ribbon_1\ribbon_2
                        \then\invb\monodromy_{34}
                        \then\mult^{13}_{1}
                        \then\mult^{24}_{2} \\
                \ribbon \then \antipode &= \ribbon\\
                \ribbon \then \counit &= \unit \then \counit = 1
        \end{align}
\end{definition}

\begin{definition}[distinguished grouplike element (spinner)]
        A \defi{distinguished grouplike element} (or \defi{spinner}) in a
        quasitriangular Hopf algebra $H$ is an invertible element $\spin\in H$
        (with inverse $\invb\spin$) such that for all $x\in H$:
        \begin{align}
                \spin_1\ribbon_2\spin_3 \then \antipode^2_2 \then \mult^{123} &=
                \ribbon\\
                \spin_1\then\comult^{1}_{12} &=\spin_1\spin_2\\
                \spin \then\antipode &= \invb\spin\\
                \spin_{1}x_2\invb\spin_{3}\then\mult^{1,2,3} &=
                x \then \antipode \then \antipode\\
                \spin \then \counit &= \unit \then \counit = 1
        \end{align}
\end{definition}

\begin{lemma}[spinners and ribbon Hopf algebras]
        If a Hopf algebra has either a ribbon element $\ribbon$ or a spinner
        $\spin$, then it must have the other as well, given by the formula:
        $\spin_1 \ribbon_2 \then \mult^{12} = \dfe$.
\end{lemma}

Both the ribbon and the spinner element relevant have topological
interpretations in the context of tangles, which are outlined in
\cref{sec:topological_interpretations}.

% TODO:
% braided categories

% define graphical calculus

\begin{remark}
        By insisting that the ends of all strands in the diagram point up, and
        that only upward-pointing portions of strands participate in crossings,
        we may replace the (co)evaluation operations with the \enquote{spinner}
        element (also called the distinguished grouplike element), derived from
        the ribbon element and the Drinfeld element.
\end{remark}
In \cite{BV}, Bar-Natan and Van der Veen define an invariant of tangles valued
in tensor powers of a certain Hopf algebra $\CU$. Their work expresses the
algebra operations as perturbed Gaußian generating functions so as to produce a
strong polynomial-time tangle invariant.

\subsection{Defining the algebra}
Here we define the Hopf algebra $U$, it's quasitriangular structure, and its
ribbon structure:

We begin by defining the algebra $\CU$. Denote by $\fa$ the non-commutative
$2$-dimensional cocommutative Lie bialgebra spanned by $a$ and $x$ with relation
$\liebk{a}{x} = x$. (This is also a Borel subalgebra of $\Sl_2$.)

Next, we use the Drinfeld double construction (outlined in \cite{ES}) to obtain
a quasitriangular Lie algebra $\fg$. As a vector space,
$\fg = \fa \oplus \dual\fa$. Given $u \in \fa$ and $v\in \dual \fa$, we have
$\liebk{u}{v}_{\fg} \defeq \dual\ad_u(v) - \dual\ad_v(u)$, extended bilinearly
and anticommutatively to all of $\fg$.
Then the algebra $\CU$ is defined to be the universal enveloping algebra
$\uea{\fg}$.

\begin{remark}
        For convenience, we define $b \defeq \dual a \in \dual\fa$ and
        $y \defeq \dual x \in \dual\fa$, so that
        \begin{equation}
                U = \genbuilder*{y, b, a, x}{
                        \liebk{a}{x} = x,
                        \liebk{a}{y} = -y,
                        \liebk{x}{y} = b,
                        \liebk{b}{ } = 0
                }
        \end{equation}
        as an algebra.
\end{remark}


\subsection{Expressing morphisms as generating functions}

When defining a morphism-valued tangle invariant, one needs a compact way of
encoding the morphism. In \cite{BV} this is achieved through the use of
generating functions, whose definition we reproduce below:

For $A$ and $B$ finite sets, consider the set $\hom(\polyring{\Q}{z_A},
\polyring{\Q}{z_B})$ of linear maps between multivariate polynomial rings. Such
a map is determined by its values on the monomials $z_A^\nn$ for each
multi-index $\nn \in \N^A$.

\begin{definition}[Exponential generating function]
        The \defi{exponential generating function} of a map
        $\map {Φ} {\polyring{\Q}{z_A}} {\polyring{\Q}{z_B}}$ between polynomial
        spaces is
        \begin{equation}
                \Gen{Φ} \defeq
                \Sum[\nn\in\N^A] \frac{Φ(z_A^\nn)}{\nn!}ζ_A^\nn
                \in \powerseries{\polyring{\Q}{z_B}}{ζ_A}
        \end{equation}
\end{definition}
\begin{remark}
        Extending the definition of $Φ$ to
        $\powerseries{\polyring{\Q}{z_B}}{ζ_A}$ by the extending scalars to
        $\powerseries{\Q}{ζ_A}$ gives us an equivalent formulation:
        \begin{equation}
                \Gen{Φ}
                = Φ\pn*{\Sum[\nn\in\N^A] \frac{(z_Aζ_A)^\nn}{\nn!}}
                = Φ\pn*{\Gen[\big]{\id_{\polyring{\Q}{z_A}}}}
        \end{equation}
\end{remark}

By the PBW theorem, we know that $\CU$ is isomorphic as a vector space to the
polynomial ring $\polyring{\Q}{y, b, a, x}$ by choosing an ordering of the
generators (following \cite{BV}, we use $(y, b, a, x)$):
\begin{equation}
        \mapdef {\Order} {\polyring{\Q}{y, b, a, x}} [\toiso] {\CU}
        {y^{n_1}b^{n_2}a^{n_3}x^{n_4}} {y^{n_1}b^{n_2}a^{n_3}x^{n_4}}
\end{equation}

Using this vector space isomorphism, \cite{BV} expresses all Hopf algebra
operations as perturbed Gaußians. To extend the resulting tangle invariant to
one on links, one would need to define a trace operator on $\CU$. The first
natural place to look is the coinvariants,
$\CU_\CU = \fracl{\CU}{\liebk{\CU}{\CU}}$. In what follows, we will compute
$\CU_\CU$, determine a vector space isomorphism to a suitable polynomial ring,
and compute the corresponding generating function of the quotient map $\map
{\trace} {\CU} {\CU_\CU}$.

\subsection{Pure tangles as a meta-Hopf algebra}
\label{sec:topological_interpretations}

Tangled objects also have the structure of a meta-Hopf algebra. In this section,
we follow the definitions laid out by Bar-Natan and van der Veen in \cite{BV}.

\begin{definition}[pure tangle]
        A \defi{pure tangle} is an embedding of line segments (called
        \defi{strands}) into the thickened unit disk $D \times [-1,1]$ (or a
        disjoint union of such disks) such that the endpoints of the line
        segments are fixed along $\del D \times \set0$. Two pure tangles are
        considered equivalent if there exists an isotopy of the embedding which
        fixes the endpoints of the strands.
\end{definition}

To work with pure tangles, we define an equivalent notion with a combinatorial
flair:

\begin{definition}[pure tangle diagram]\label{def:pure_tangle_diagram}
        A \defi{pure tangle diagram} is a finite planar graph with distinguished
        circles called \defi{boundary circles}. The remainder of the edges are
        contained inside the boundary circles, either meeting a boundary circle
        at a trivalent vertex, or meeting other internal edges at a tetravalent
        vertex (called a \defi{crossing}). Each crossing is marked with a
        sign---positive or negative. Collections of edges which meet at
        opposite sides of an edge are called \defi{strands}. Each strand must
        meet a boundary circle twice (that is, strands may not form loops).
        Two pure tangle diagrams are equivalent if one can be transformed into
        the other by a sequence of Reidemeister moves on the crossings.
\end{definition}

\begin{theorem}[pure tangles are tangle diagrams]\label{thm:pure_tangle}
        To each pure tangle one associates exactly one pure tangle diagram
        equivalence class. Further, a generic projection of a pure tangle to the
        flattened disks $\Dunion D\times\set0$ allows one to construct a tangle
        diagram corresponding to it.
\end{theorem}
\begin{proof}
        The projection of a generic perturbation of a pure tangle has the
        following properties:
        \begin{itemize}
                \item all intersections of strands with
                                themselves,
                                other strands, or
                                a boundary circle
                are transverse.
                \item all intersections involve at most two strand components,
                        so that no triple intersections appear.
                \item each projected strand is an immersion, so that no cusps
                        appear.
        \end{itemize}
        From these data, we may construct a pure tangle diagram, assigning one
        crossing to each double intersection, with the sign selected based on
        which strand lay above the other before projecting. Extending
        Reidemeister's theorem to objects of this form is straightforward.
\end{proof}

For the sake of algebraic closure, the notion of virtual tangles will be useful:
\begin{definition}[virtual tangle diagram]
        A \defi{virtual tangle diagram} is a diagram satisfying all conditions
        laid out in \cref{def:pure_tangle_diagram} except planarity, taken again
        up to a sequence of Reidemeister moves. We denote by $\tangle_S$ the
        collection of virtual tangle diagrams with strands indexed by the finite
        set $S$.
\end{definition}

\begin{theorem}[virtual tangles are just crossings]
        The data of a virtual tangle is equivalent to:
        \begin{itemize}
                \item a collection of labels for strands
                \item a collection of crossings between strands
                \item the orders in which each strand interacts with each
                  crossing.
        \end{itemize}
\end{theorem}
\begin{proof}
        
\end{proof}

\begin{theorem}[virtual tangles form a quasitriangular meta-Hopf algebra]
        \label{thm:vt_qtmha}
        The collection $\tangle_X$ forms a quasitriangular meta-Hopf algebra
        with the following operations:
        \begin{itemize}
                \item multiplication $\mult^{ij}_{k}[X]$ takes a tangle with
                        strands $X\sqcup\set{i,j}$ and glues the end of strand
                        $i$ to strand $j$, labelling the resulting strand $k$.
                \item the unit $\unit_{i}[X]$ takes a tangle diagram with
                        strands $X$ and introduces a new strand $i$ which does
                        not touch any of the other strands.
                \item the comultiplication $\comult^{i}_{jk}[X]$ takes a tangle
                        with strands $X\sqcup \set{i}$ and doubles strand $i$,
                        separating the two strands along the framing of strand
                        $i$, calling the two resulting strands $j$ and $k$.
                \item the counit $\counit^{i}[X]$ takes a tangle with strands
                        indexed by $X\sqcup \set{i}$ and returns the tangle with
                        strand labelled by $i$ deleted.
                \item the antipode $\antipode^{i}_{j}[X]$ takes a tangle with
                        strands labelled by $X \sqcup \set{i}$ and reverses the
                        direction of strand $i$ (calling the new strand $j$).
        \end{itemize}
\end{theorem}
\begin{proof}
        % todo: prove
        % todo: does the antipode work for non-rotational tangles?
\end{proof}

The invariants we deal with keep track not only of crossing data, but also
rotation of strands between crossings. We introduce an object which monitors
these additional data:
\begin{definition}[\acf{RVT} diagrams]
        A \defi{\acf{RVT} diagram} is a virtual tangle diagram, together with an
        assignment of an integer to each arc, called the \defi{rotation number}
        of the arc. This is visualised by requiring that each strand's
        intersection with the boundary is pointing upwards, and that each
        crossing is between curves whose tangent deviates less than $π/2$ from
        the vertical direction.

        Equivalence between \ac{RVT} diagrams is determined by extending the
        traditional Reidemeister moves with the whirling relation: any crossing
        may be rotated by full rotations. This amounts to increasing both
        outgoing strands' rotation number by some $n\in\Z$, and adding $-n$ to
        the rotation number of the incoming strands.
        %todo: add picture of whirling relation
        Additionally, we must take care that framed Reidemeister 1 and the
        cyclic Reidemeister 2 include appropriate rotation numbers on their
        arcs. %todo: add picture of situation
        The set of \acp{RVT} with strands indexed by a set $X$ will be denoted
        $\RVT_X$.
\end{definition}

\begin{theorem}[\acp{RVT} form a meta-ribbon Hopf algebra]
        The collections $\RVT_X$ form a meta-ribbon Hopf algebra, with the same
        definitions as in \cref{thm:vt_qtmha}, except:
        \begin{itemize}
                \item the antipode $\antipode^{i}_{j}[X]$ takes a tangle with
                        strands labelled by $X \sqcup \set{i}$ and reverses the
                        direction of strand $i$, then adds a counter-clockwise
                        cap to the new beginning, and a clockwise cup to the
                        end. This new strand is called $j$.
                \item the spinner $\spin_i[X]$ takes a tangle in $\RVT_X$ and
                        adds a new strand with rotation number $1$ which has no
                        interactions with any other strands.
        \end{itemize}
\end{theorem}
\begin{proof}
        % todo
\end{proof}

\subsection{Rotational tangle invariants from a ribbon Hopf algebra}
Here we describe the morphism from the category of pure rotational virtual
tangles to a ribbon Hopf algebra.

\section{Extending a pure tangle invariant to links and general tangles}
Thus far, the algebraic setting we have defined allows us to describe invariants
of tangles with no closed components. We now extend the notion of a meta-Hopf
algebra to include closed components.

\begin{definition}[traced meta-algebra]
        A \defi{traced meta-algebra} is a family of meta-algebras: for each
        finite set $L$, we assign one meta-algebra $\set{A_{L,
        S}}_S$.\footnote{These sets index the \enquote{strands} $S$ and the
        \enquote{loops} $L$.} The multiplication maps $\mult^{i,j}_k[L]$ then
        take the form:
        \begin{equation}
                \map {\mult^{i,j}_k[L][S]}
                        {A_{\set{i,j}\sqcup S, L}}
                        {A_{\set{k}\sqcup S, L}} 
        \end{equation}
        for $i$, $j$, $k$ disjoint from both $S$ and $L$.

        The compatibility of the families of meta-algebras is governed by a
        \defi{trace}
        $\map {\trace^i} {A_{\set{i}\sqcup S, L}} {A_{S,\set{i}\sqcup L}}$ which
        satisfies the cyclic property:
        \begin{equation}
                \mult^{i,j}_{k}\then\trace^k
                =
                \mult^{j,i}_{k}\then\trace^k
        \end{equation}
        Further, for each fixed $L$, the collection $\set{A_{S, L}}_S$ must be a
        meta-algebra.
\end{definition}

The first example we give is that of impure tangles.

\begin{definition}[Impure Rotational Virtual Tangles]
        Let $\RVT_{L, S}$ be the set of rotational virtual tangles with open
        strands indexed by $S$ and closed strands indexed by $L$.
\end{definition}

\begin{lemma}[tangles as a traced algebra]
        The collection of all $\RVT_{L, S}$ is a traced ribbon meta-Hopf
        algebra, with trace map given by closing a strand into a loop.
\end{lemma}
\begin{proof}
        When $L = \emptyset$, the situation is exactly the case of
        \cref{thm:pure_tangle}, so $\RVT_{\emptyset, S} = \tangle_S$ is a
        meta-Hopf algebra. Furthermore, since the Reidemeister moves are local
        operations, the presence of closed components does not affect our
        ability to verify the identities on the Hopf-algebra operations.

        The last point to verify is that closing a strand into a loop is a
        cyclic operation. Given two strands, we must verify that stitching one
        end together, then tracing the other yields the same diagram as
        stitching the other ends together, then taking the trace. However, by
        definition of trace, these two actions yield identical diagrams, the two
        strands replaced by the same closed loop.
\end{proof}

\begin{lemma}[coinvariants as a trace map]
        Let $A$ be an algebra, and denote by $A_A$ its set of coinvariants. Then
        define $A_{S, L} \defeq A^{\otimes S} \otimes A_A^{\otimes L}$. Then $A$
        defines a traced meta-algebra with trace map given by $\map {\trace^i_j}
        {A_i}
        {(A_A)_j}$.
\end{lemma}
\begin{proof}
       Observe that for any choice of $L$, extending morphisms by the identity
       yield an isomorphism of traced meta-Hopf algebras:
       \begin{equation}
               \mapdef {ϕ_L}
                       {\set[\big]{A^{\otimes S}}_S} [\toiso]
                       {\set[\big]{A^{\otimes S}\otimes A_A^{\otimes L}}_S}
                       {A^{\otimes S}}
                       {A^{\otimes S}\otimes A_A^{\otimes L} \\
                       f &\mapsto f \otimes \id_{A_A}^{\otimes L}
               }
       \end{equation}
       Next, we wish to show that the two maps with shape
       $\map {f}
               {A^{\otimes \set {i, j} \sqcup S}\otimes A_A^{\otimes L}}
               {A^{\otimes S}\otimes A_A^{\otimes \set k \sqcup L}}$
       are equivalent. This amounts to showing that, given $u, v\in A$, that
       $\overline{uv} = \overline{vu} \in A_A$.
       However, by the construction of the coinvariants,
       $\overline{uv}-\overline{vu} = \overline{uv-vu} = \overline{0} \in A$,
       and we are done.
\end{proof}

\section{The coinvariants of $U$}

We start with a result which simplifies working with coinvariants:

\begin{lemma}[Coinvariant simplification]\label{lem:coinvLieAlg}
        Let $\fh$ be a Lie algebra. Then $\uea{\fh}_{\uea{\fh}} =
        \uea{\fh}_\fh$.
\end{lemma}
\begin{proof}
First, observe that for any $u$, $v$, $f\in\uea{\fh}$,
$\ad_{uv}(f) = \ad_u(vf) + \ad_v(fu)$. Proceeding inductively, for any monomial
$μ\in\uea{\fh}$, $\ad_{μ}(u)$ is a linear combination of elements of
$\liebk*{\fh}{\uea{\fh}}$. By linearity of $\ad$, we conclude
$\liebk*{\uea{\fh}}{\uea{\fh}} = \liebk*{\fh}{\uea{\fh}}$.
\end{proof}

\begin{theorem}
        The coinvariants of $U$, $U_U$, has basis
        $\set{y^n a^k x^n}_{n, k\ge 0}$.
\end{theorem}
\begin{proof}
Using \cref{lem:coinvLieAlg}, we need only compute $\liebk{\Alg}{\CU}$ to
determine $\CU_\CU$. Given a polynomial $f$, we have the
following relations in $U$:
\begin{align}
        f(a)y^r &= y^rf(a-r) &
        x^rf(a) &= f(a-r)x^r
\end{align}
Next we compute the adjoint actions of $y$, $a$, and $x$. (Recall $b$ is
central.)
\begin{align}
  \ad_a f(x) &= xf'(x)&
  \ad_a f(y) &= -yf'(y)\label{eq:ada}\\
  \ad_x f(y) &= bf'(y) &
  \ad_x f(a) &= -\nabla[f](a)x\label{eq:adx}\\
  \ad_y f(x) &= -bf'(x) &
  \ad_y f(a) &= y\nabla[f](a)\label{eq:ady}
\end{align}
(Here $\nabla$ is the backwards finite difference operator $\nabla[f](x) \defeq
f(x) - f(x-1)$.) Observe for any $n$, $m$, $k$, and polynomials $f$ and $g$:
\begin{align}
        \ad_a \pn*{y^m g(b, a) x^n } &= (n-m)y^mg(b, a) x^n
        \label{eq:ada_rel}\\
        \ad_{x}\pn*{y^{n+1}b^{m-1}f(a)x^{k}} &=
                (n+1)y^{n}b^{m}f(a)x^{k} - y^{n+1}b^{m-1}\nabla[f](a)x^{k+1}
        \label{eq:adx_rel}\\
        \ad_{y}\pn*{y^n b^{m-1} f(a) x^{k+1}} &=
                - (k+1)y^n b^m f(a) x^k + y^{n+1} b^{m-1} \nabla[f](a)x^{k+1}
        \label{eq:ady_rel}
\end{align}
By \cref{eq:ada_rel}, any monomial whose powers of $y$ and $x$ differ vanish in
$\CU_{\Alg}$. As a consequence, in \cref{eq:adx_rel,eq:ady_rel}, the only
nontrivial case is when $n=k$, resulting in the same relation. By induction on
$n$, we conclude that:
\begin{equation}\label{eq:coinv_reduction}
        y^n b^m f(a) x^k \sim δ_{nk}\frac{n!}{(n+m)!}y^{n+m}\nabla^m[f](a)x^{n+m}
\end{equation}
Observing when $f$ is a monomial in \cref{eq:coinv_reduction}, we see
$\CU_{\Alg}$ is spanned by $\set{y^n a^k x^n}_{n, k \ge 0}$. Since all relations are
accounted for, setting $m=0$  demonstrates this set is linearly independent, and
we have a basis.
\end{proof}

\subsection{A generating function for the coinvariants}

\ProvideDocumentCommand{\COrder}{}{\Order}
In order to define a generating function, we need to choose an appropriate basis
for the space of coinvariants. We define an isomorphism from the space of
coinvariants to a polynomial space, tweaking the earlier-defined basis by scalar
multiples. Since it plays the role of the ordering map, we also name it
$\COrder$.
\begin{equation}
        \mapdef {\COrder} {\polyring{\Q}{a, z}} [\toiso] {\CU_\CU}
        {a^{n}z^{k}} {\frac{1}{k!}y^{k}a^{n}x^{k} \\
                k!\nabla^m[f](a)z^{k+m} &\mapsfrom y^kb^mf(a)x^k
        }
\end{equation}

This defines a commutative square upon whose bottom edge
$τ = \Order \then {\trace} \then {\inv{\COrder}}$ we compute the generating
function:
\begin{equation}
\begin{tikzcd}
        \CU
                \rar["\trace"]
        & \CU_\CU \\
        \polyring{\Q}{y, b, a, x}
                \uar["\Order"]
                \rar["τ"]
        &
        \polyring{\Q}{a, z}
                \uar["\COrder"]
\end{tikzcd}
\end{equation}

We begin with a result on finite differences:
\begin{lemma}\label{eq:findiffexp}
        The finite difference operator acts in the following way on
        exponentials:
        \begin{equation}
                \nabla^n[\Exp{αa}](a) = (1-\Exp{-α})^n\Exp{αa}
        \end{equation}
\end{lemma}
\begin{proof}
Using the fact that $\nabla^n[f](x) = \Sum[k=0][n]\binom n k(-1)^kf(x-k)$, we
see that
$\nabla^n[\Exp{αa}](a)
        = \Sum[k=0][n]\binom n k(-1)^k\Exp{αa-αk}
        % = \Exp{αa}\Sum[k=0][n]\binom n k(-1)^k\Exp{-αk}
        = (1-\Exp{-α})^n\Exp{αa}$.
\end{proof}
We now are ready to compute the generating function for the trace:
\begin{theorem}[Generating function for the trace of $\CU$]
\begin{equation}\label{eq:trace_formula}
        \Gen\trace = \Exp{αa+\pn*{ηξ+β(1-\Exp{-α})}z}
\end{equation}
\end{theorem}
\begin{proof}
        Using \cref{eq:findiffexp} and the extension of scalars of $\trace$ to
        $\powerseries{\Q}{η, β, α, ξ}$, we see
        \begin{equation}
        \begin{aligned}
                &\Gen[\big]{\Order \then {\trace} \then {\inv{\COrder}}}
                = \pn[\big]{\Exp{η y} \Exp{β b} \Exp{α a} \Exp{ξ x}} \then
                        {\trace} \then {\inv\COrder}\\
                &= \inv\COrder\Sum[i, j, k]\trace\pn*{
                        \frac{(η y)^i}{i!}
                        \frac{(β b)^j}{j!}
                        \Exp{α a}
                        \frac{(ξ x)^k}{k!}
                }\\
                % &=\Sum[i, j, k]\frac{η^i β^j ξ^k}{i! j! k!}
                        % \trace\pn*{ y^i b^j \Exp{α a} x^k }\\
                % &=\Sum[i, j, k]\frac{η^i β^j ξ^k}{i! j! k!}
                        % δ_{ik}i!\nabla^j[\Exp{αa}](a)t^{i+j}\\
                &=\Sum[i, j]\frac{η^i β^j ξ^i}{i! j!}
                        (1-\Exp{-α})^j\Exp{αa}z^{i+j}
                =\Exp{αa+\pn*{ηξ+β(1-\Exp{-α})}z}\qedhere
        \end{aligned}
        \end{equation}
\end{proof}

\subsection{Evaluation of the trace on a generic element}
Here we will outline a computation involving the trace by using Bar-Natan and
van der Veen's Contraction Theorem.

A typical value for a tangle invariant that arises is of the form:
\begin{equation}\label{eq:sample_GDO}
        P\Exp{c + α a_i + β b_i + ξ(b_i) x_i + η(b_i)y_i + λ(b_i)x_i y_i}
\end{equation}
Here, $c$, $α$, and $β$ denotes a constant with respect to the variables $y_i$,
$b_i$, $a_i$, and $x_i$ (collectively referred to as \enquote{$v_i$}s), while
$ξ$, $η$, and $λ$ are potentially $b_i$-dependent, and $P$ is a (rational)
function in (the square root of) $B_i$.

Let us compute the trace of \cref{eq:sample_GDO}. For clarity, we will put bars
over the coinvariants variables $a_i$ and $z_i$, as they do not play a role in
the contraction.
\ProvideDocumentCommand{\ba}{}{\bar{a}}
\ProvideDocumentCommand{\bz}{}{\bar{z}}
\begin{equation}\begin{aligned}\label{eq:trace_on_gaussian}
        &\contraction{
                P(B_i)
                \Exp{c + α a_i + β b_i + ξ(b_i) x_i + η(b_i)y_i + λ(b_i)x_i y_i}
                \trace^i
        }_{v_i}\\
        &= \contraction{
                P(B_i)
                \Exp{c + β b_i + ξ(b_i) x_i + η(b_i)y_i + λ(b_i)x_i y_i
                +η_iξ_i\bz_i+β_i(1-\Exp{-α_i})\bz_i}
                \Exp{α a_i + α_i\ba_i}
        }_{v_i}\\
        &= \Exp{α\ba_i}\contraction{
                P(B_i)
                \Exp{c + ξ(b_i) x_i + η(b_i)y_i + λ(b_i)x_i y_i
                +η_iξ_i\bz_i}
                \Exp{β b_i +β_i(1-\Exp{-α})\bz_i}
        }_{b_i, x_i, y_i}\\
        \alignedintertext{In what follows, we let $μ\defeq (1-\Exp{-α})\bz_i$:}
        &= \Exp{c+α\ba_i+βμ}P(\Exp{-μ})\contraction{
                \Exp{η(μ)y_i}
                \Exp{(ξ(μ) + λ(μ)y_i)x_i + ξ_iη_i\bz_i}
        }_{x_i, y_i}\\
        &= \Exp{c+α\ba_i+βμ}P(\Exp{-μ})\contraction{
                \Exp{η(μ)y_i+ξ(μ)\bz_iη_i + λ(μ)\bz_iη_iy_i}
        }_{y_i}\\
        &= \frac{P(\Exp{-μ})}
                {1-λ(μ)\bz_i}\Exp{c+α\ba_i+βμ+\frac{η(μ)ξ(μ)\bz_i}{1-λ(μ)\bz_i}}
\end{aligned}\end{equation}

\subsection{Computational examples}

Using the formula given in \cref{eq:trace_on_gaussian}, let us do some
preliminary examples:

\begin{align}
        \trace^i(R_{ij}) &= 1\label{eq:trR}\\
        \trace^j(R_{ij}) &= \Exp{b_i\ba_j}\label{eq:trR2}\\
        \begin{split}\label{eq:trHopf}
                \trace^2\pn[\Big]{\sqrt{B_2}\Exp{
                                -a_2 b_1-a_1 b_2 +
                                \frac{(B_1-1) x_2 y_1}{b_1 B_1}+
                        \frac{(B_2-1) x_1 y_2}{b_2 B_2}}
                } &= \\
                \Exp{
                        \frac{a_1 (\bz_2-B_1 \bz_2)}{B_1}-b_1 \ba_2+
                        \frac{e^{B_1 \bz_2}
                        (x_1 y_1 e^{\inv{B_1} \bz_2}-x_1 y_1 e^{\bz_2})}{b_1}+
                        \frac{1}{2} \inv{B_1} \bz_2-\frac{\bz_2}{2}
                }&%\\
        \end{split}
        % \trace^i(
                % P\Exp{c + α a_i + β b_i + ξ(b_i) x_i + η(b_i)y_i + λ(b_i)x_i y_i}
        % )
\end{align}

\Cref{eq:trR,eq:trR2} are the values one obtains for the two (virtual)
one-crossing, two-component link, while \cref{eq:trHopf} is the value of the
invariant on the Hopf link.

\subsection{Limitations of this definition}
For some inputs to the trace, expressions involving the Lambert $W$-function
appear, which complicates attempts to keep the invariant valued in the space of
perturbed Gaußians.

\section{Generating \acp{RVT} for links}

\subsection{\acp{RVT} for knots}
Describe algorithm previously developed for knots

\subsection{Extending the algorithm to multiple components}

Given a classical link, there is a unique \ac{RVL} corresponding to it. Given a
classical link diagram, one may obtain the corresponding \ac{RVL} by attaching
an appropriate rotation number to each arc. However, there is not a unique way
to do so.

The situation becomes more complicated when one considers the case where the
tangle has an open component. In this case, two \ac{RVT} diagrams which
correspond to the same classical link exactly when they differ only by a
sequence of rotational Reidemeister moves \emph{and} a modification of the
rotation numbers of the (two) unbounded arcs. Equivalently, we have the
statement:

\begin{lemma}
        For each classical tangle with one open component, there exists a unique
        \ac{RVT} whose unbounded arcs have rotation numbers $0$.
\end{lemma}
\begin{proof}
        See \cite{BV}.
\end{proof}

Bar-Natan and van der Veen develop an algorithm to convert a classical long knot
into an \ac{RVT}. As we are interested in links, we must extend this algorithm
to include so-called \enquote{long links}, which we outline below:
\begin{verbatim}
        1. Pass a front over the beginning of the open strand.
        2. Progressively absorb the leftmost crossings
                2a. As crossings are absorbed,
                    take into account any rotations of arcs.
        3. If an arc passes through the front twice, absorb it,
           taking into account any rotations of that arc as a result.
\end{verbatim}

\section{Comparison with the \ac{MVA}}

Given that the long knot (i.e. one-component) case of this invariant
encodes the Alexander Polynomial, it was suspected that the invariant on long
links (i.e. multiple components, one of which is long) formed by adding the
trace would encode the \ac{MVA}. However, there are links which the \ac{MVA}
separates which this invariant does not.

On all two-component links with at most $11$ crossings (a collection of size
$914$), the trace map attains $878$ distinct values, while the MVA attains only
$778$. However, the two invariants are incomparable in terms of their strength.

\section{Further work}
While all other Hopf algebra operations in $U$ are expressed by \cite{BV} as
perturbed Gaußians, the form in \cref{eq:trace_formula} does not to conform to
the same structure. Further work is needed to either implement this operation
into the established framework, or to suitably extend the framework.

\appendix

\section{Code implementation}

\subsection{Implementation of $Z$ invariant (from \cite{BV})}

This is a Mathematica™ implementation by Bar-Natan and van der Veen, modified by
the author.

\subsection{Implementation of $\trace$}

This is a Mathematica™ implementation.

\subsection{Implementation of \lstinline|toRVT|}

This is a Haskell implementation.

\section{Acronyms}
\begin{acronym}
        \acro{RVT}{Rotational Virtual Tangle}
        \acro{RVK}{Rotational Virtual Knot}
        \acro{RVL}{Rotational Virtual Link}
        \acroindefinite{RVT}{an}{a}
        \acroindefinite{RVK}{an}{a}
        \acroindefinite{RVL}{an}{a}
        \acro{MVA}{Multivariable Alexander polynomial}
\end{acronym}

\printbibliography
\end{document}
