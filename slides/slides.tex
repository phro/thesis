\documentclass{beamer}

\usetheme{Berlin}

\usepackage{mathtools}
\usepackage{newunicodechar}
% Thesis version
\newcommand{\thesisVersion}{1.0.0}

% Personal data (insert your own data here)
\newcommand{\myTitle}{Computing the generating function of a coinvariants map\xspace}
\newcommand{\myName}{Jesse Frohlich\xspace}
\newcommand{\myDegree}{Doctor of Philosophy\xspace}
\newcommand{\myDepartment}{Graduate Department of Mathematics\xspace}
\newcommand{\myUni}{University of Toronto\xspace}
\newcommand{\myTime}{2023\xspace}

% Symbols

% Paired Delimiters

\ProvideDocumentCommand{\curry}{m}{\ifblank{#1}{\:\cdot\:}{#1}}

%% Parenthetical constructs
\DeclarePairedDelimiter{\pn}\lparen\rparen
\DeclarePairedDelimiter{\set}\{\}
\DeclarePairedDelimiter{\bk}\lbrack\rbrack
\DeclarePairedDelimiter{\bkk}\llbracket\rrbracket
\DeclarePairedDelimiter{\gen}\langle\rangle

%% Operators

\DeclarePairedDelimiterX{\commutator}[2]{[}{]}{\curry{#1}, \curry{#2}}

\DeclarePairedDelimiterX{\liebk}[2]{[}{]}{\curry{#1}, \curry{#2}}

\DeclarePairedDelimiterXPP{\normHelper}[2]{}\lVert\rVert
        {\IfNoValueTF{#2}{}{_{#2}}}
        {\curry{#1}}
\ProvideDocumentCommand{\norm}{s O{} m o}{% Optional subscript token at the end.
        \IfBooleanTF{#1}
                {\normHelper*{#3}{#4}}
                {\normHelper[#2]{#3}{#4}}
}

\DeclarePairedDelimiterXPP{\absHelper}[2]{}\lvert\rvert
        {\IfNoValueTF{#2}{}{_{#2}}}
        {\curry{#1}}
\ProvideDocumentCommand{\abs}{s O{} m o}{% Optional subscript token at the end.
        \IfBooleanTF{#1}
                {\absHelper*{#3}{#4}}
                {\absHelper[#2]{#3}{#4}}
}

\DeclarePairedDelimiterX{\pair}[2]\langle\rangle{%
        \nonscript\,\curry{#1}\PairSymbol[\delimsize]{\vert}\curry{#2}\nonscript\,%
}

\DeclarePairedDelimiterX{\inner}[2]\langle\rangle{%
        \nonscript\,\curry{#1}\PairSymbol[\delimsize]{,}\curry{#2}\nonscript\,%
}

\DeclarePairedDelimiterX{\card}[1]\lvert\rvert{\curry{#1}}
% \ProvideDocumentCommand{\card}{\#}

\DeclarePairedDelimiterX\ceil [1]{\lceil}{\rceil}{\curry{#1}}

\DeclarePairedDelimiterX\floor[1]{\lfloor}{\rfloor}{\curry{#1}}

\DeclarePairedDelimiterX\ord  [1]{\lvert}{\rvert}{\curry{#1}}
\DeclareMathOperator{\Ord}{ord}


%% Building constructs

\DeclarePairedDelimiterX{\setbuilder}[2]\{\}{\,#1\SetSymbol[\delimsize]#2\,}

\DeclarePairedDelimiterX{\genbuilder}[2]\langle\rangle{\,#1\SetSymbol[\delimsize]#2\,}

\newcommand{\grppres}{\genbuilder}

\DeclareMathOperator{\HomOp}{Hom} 
\DeclarePairedDelimiterXPP{\HomHelper}[3]
        {\IfNoValueTF{#1}
                {\HomOp}
                {\HomOp_{#1}}
        }
        ()
        {}{\curry{#2}, \curry{#3}}
\ProvideDocumentCommand{\Hom}{s O{} m m}{
        \IfBooleanTF{#1}
                {\HomHelper*{#2}{#3}{#4}}
                {\HomHelper{#2}{#3}{#4}}
}

% Quantum groups / Hopf algebras

\newcommand{\unit}{\eta}           % unit
\newcommand{\one}{\mbf 1}          % another form of the unit
\newcommand{\counit}{\epsilon}     % counit

\newcommand{\mult}{m}              % multiplication
\newcommand{\comult}{\Delta}       % comultiplication
\ProvideDocumentCommand{\cmf}{mm}  % comultiplication factor
  {#2_{(#1)}}

\newcommand{\antipode}{S}          % antipode
\ProvideDocumentCommand{\bap}{m}   % bar-antipode
  {\overline{#1}}
\ProvideDocumentCommand{\baap}{m}  % bar-inverse(i.e. anti)-antipode
  {\underline{#1}}

\newcommand{\rmat}{r}         % Lie-algebraic r-matrix

\newcommand{\Rmat}{\mcl R}         % R-matrix
\ProvideDocumentCommand{\rmf}{O{}} % R-matrix First factor
  {\Rmat^{(1)}_{#1}}
\ProvideDocumentCommand{\rms}{O{}} % R-matrix Second factor
  {\Rmat^{(2)}_{#1}}

\ProvideDocumentCommand{\coevf}{m} % Coevaluation First factor
  {r_{#1}}
\ProvideDocumentCommand{\coevd}{m} % Coevaluation Dual factor
  {ρ_{#1}}



\title{\myTitle}
\author{\myName}
\date{\myTime}
\institute{\myUni\\\myDepartment}

\begin{document}
\begin{frame}
        \titlepage
\end{frame}
\begin{frame}
        \frametitle{Outline}
        \tableofcontents
\end{frame}
\section{Introduction}
\subsection{Knots, links, and tangles}
\begin{frame}
\begin{figure}
        \centering
        \includegraphics[height=0.8\textheight]{../figures/tangle_example.pdf}
        \caption{An example of a tangle}
        \label{fig:tangle_example}
\end{figure}
\end{frame}
\begin{frame}
\begin{figure}
        \centering
        \includegraphics[width=0.9\textwidth]{../figures/tangle_intro.pdf}
        \caption{Can you arrange the left-hand side to look like the right?}
        \label{fig:tangle_intro}
\end{figure}
\end{frame}
\begin{frame}
\begin{figure}
        \centering
        \includegraphics[width=0.9\textwidth]{../figures/tangle_comult.pdf}
        \caption{Doubling strand $i$ into two strands $j$ and $k$}
        \label{fig:tangle_comult}
\end{figure}
\end{frame}
\section{Quantum invariants}
\begin{frame}
\begin{figure}
        \centering
        \includegraphics[height=0.8\textheight]{../figures/invariant_example.pdf}
        \caption{The first step in constructing an invariant}
        \label{fig:}
\end{figure}
\end{frame}
\begin{frame}
        \begin{definition}[indexed tensor powers]\label{def:indexed_tensor_powers}
        Let $V$ be a vector space and $S = \set{s_1,…, s_n}$ be a finite
        set. We define the \defi{indexed tensor power} of $V$ to be the
        collection of formal linear combinations of functions from $S$ to $V$
        \begin{equation}\label{eq:indexed_tensor}
                V_S \defeq V^{\otimes S} \defeq \Span\set{\map {f} {S} {V}}/\sim
        \end{equation}
        subject to multi-additivity and the factoring of scalars.
\end{definition}
\pause
We will write such functions $\map {f} {S} {V}$ with $f(s_i) = v_i$
with the following notation:
\begin{equation}\label{eq:product_notation}
        \pn{v_{1}}_{s_1}
        \pn{v_{2}}_{s_2} ⋯
        \pn{v_{n}}_{s_n}
        \defeq f
\end{equation}
\end{frame}
\begin{frame}
        \centering
        \Huge Thank you!
\end{frame}
\end{document}
