\documentclass{beamer}

\usetheme{Berlin}

\usepackage{mathtools}
\usepackage{newunicodechar}
\usepackage{tikz-cd}
% Thesis version
\newcommand{\thesisVersion}{1.0.0}

% Personal data (insert your own data here)
\newcommand{\myTitle}{Computing the generating function of a coinvariants map\xspace}
\newcommand{\myName}{Jesse Frohlich\xspace}
\newcommand{\myDegree}{Doctor of Philosophy\xspace}
\newcommand{\myDepartment}{Graduate Department of Mathematics\xspace}
\newcommand{\myUni}{University of Toronto\xspace}
\newcommand{\myTime}{2023\xspace}

% Symbols

% Paired Delimiters

\ProvideDocumentCommand{\curry}{m}{\ifblank{#1}{\:\cdot\:}{#1}}

%% Parenthetical constructs
\DeclarePairedDelimiter{\pn}\lparen\rparen
\DeclarePairedDelimiter{\set}\{\}
\DeclarePairedDelimiter{\bk}\lbrack\rbrack
\DeclarePairedDelimiter{\bkk}\llbracket\rrbracket
\DeclarePairedDelimiter{\gen}\langle\rangle

%% Operators

\DeclarePairedDelimiterX{\commutator}[2]{[}{]}{\curry{#1}, \curry{#2}}

\DeclarePairedDelimiterX{\liebk}[2]{[}{]}{\curry{#1}, \curry{#2}}

\DeclarePairedDelimiterXPP{\normHelper}[2]{}\lVert\rVert
        {\IfNoValueTF{#2}{}{_{#2}}}
        {\curry{#1}}
\ProvideDocumentCommand{\norm}{s O{} m o}{% Optional subscript token at the end.
        \IfBooleanTF{#1}
                {\normHelper*{#3}{#4}}
                {\normHelper[#2]{#3}{#4}}
}

\DeclarePairedDelimiterXPP{\absHelper}[2]{}\lvert\rvert
        {\IfNoValueTF{#2}{}{_{#2}}}
        {\curry{#1}}
\ProvideDocumentCommand{\abs}{s O{} m o}{% Optional subscript token at the end.
        \IfBooleanTF{#1}
                {\absHelper*{#3}{#4}}
                {\absHelper[#2]{#3}{#4}}
}

\DeclarePairedDelimiterX{\pair}[2]\langle\rangle{%
        \nonscript\,\curry{#1}\PairSymbol[\delimsize]{\vert}\curry{#2}\nonscript\,%
}

\DeclarePairedDelimiterX{\inner}[2]\langle\rangle{%
        \nonscript\,\curry{#1}\PairSymbol[\delimsize]{,}\curry{#2}\nonscript\,%
}

\DeclarePairedDelimiterX{\card}[1]\lvert\rvert{\curry{#1}}
% \ProvideDocumentCommand{\card}{\#}

\DeclarePairedDelimiterX\ceil [1]{\lceil}{\rceil}{\curry{#1}}

\DeclarePairedDelimiterX\floor[1]{\lfloor}{\rfloor}{\curry{#1}}

\DeclarePairedDelimiterX\ord  [1]{\lvert}{\rvert}{\curry{#1}}
\DeclareMathOperator{\Ord}{ord}


%% Building constructs

\DeclarePairedDelimiterX{\setbuilder}[2]\{\}{\,#1\SetSymbol[\delimsize]#2\,}

\DeclarePairedDelimiterX{\genbuilder}[2]\langle\rangle{\,#1\SetSymbol[\delimsize]#2\,}

\newcommand{\grppres}{\genbuilder}

\DeclareMathOperator{\HomOp}{Hom} 
\DeclarePairedDelimiterXPP{\HomHelper}[3]
        {\IfNoValueTF{#1}
                {\HomOp}
                {\HomOp_{#1}}
        }
        ()
        {}{\curry{#2}, \curry{#3}}
\ProvideDocumentCommand{\Hom}{s O{} m m}{
        \IfBooleanTF{#1}
                {\HomHelper*{#2}{#3}{#4}}
                {\HomHelper{#2}{#3}{#4}}
}

% Quantum groups / Hopf algebras

\newcommand{\unit}{\eta}           % unit
\newcommand{\one}{\mbf 1}          % another form of the unit
\newcommand{\counit}{\epsilon}     % counit

\newcommand{\mult}{m}              % multiplication
\newcommand{\comult}{\Delta}       % comultiplication
\ProvideDocumentCommand{\cmf}{mm}  % comultiplication factor
  {#2_{(#1)}}

\newcommand{\antipode}{S}          % antipode
\ProvideDocumentCommand{\bap}{m}   % bar-antipode
  {\overline{#1}}
\ProvideDocumentCommand{\baap}{m}  % bar-inverse(i.e. anti)-antipode
  {\underline{#1}}

\newcommand{\rmat}{r}         % Lie-algebraic r-matrix

\newcommand{\Rmat}{\mcl R}         % R-matrix
\ProvideDocumentCommand{\rmf}{O{}} % R-matrix First factor
  {\Rmat^{(1)}_{#1}}
\ProvideDocumentCommand{\rms}{O{}} % R-matrix Second factor
  {\Rmat^{(2)}_{#1}}

\ProvideDocumentCommand{\coevf}{m} % Coevaluation First factor
  {r_{#1}}
\ProvideDocumentCommand{\coevd}{m} % Coevaluation Dual factor
  {ρ_{#1}}



\title{\myTitle}
\author{\myName}
\date{\myTime}
\institute{\myUni\\\myDepartment}

\begin{document}
\begin{frame}
        \titlepage
\end{frame}
\begin{frame}
        \frametitle{Outline}
        \tableofcontents
\end{frame}
\section{Introduction}
\subsection{Knots, links, and tangles}
\begin{frame}
\begin{figure}
        \centering
        \includegraphics[height=0.8\textheight]{../figures/tangle_example.pdf}
        \caption{An example of a tangle}
        \label{fig:tangle_example}
\end{figure}
\end{frame}
\begin{frame}
\begin{figure}
        \centering
        \includegraphics[width=0.9\textwidth]{../figures/tangle_intro.pdf}
        \caption{Can you arrange the left-hand side to look like the right?}
        \label{fig:tangle_intro}
\end{figure}
\end{frame}
\begin{frame}
\begin{figure}
        \centering
        \includegraphics[width=0.9\textwidth]{../figures/tangle_comult.pdf}
        \caption{Doubling strand $i$ into two strands $j$ and $k$}
        \label{fig:tangle_comult}
\end{figure}
\end{frame}
\section{Quantum invariants}
\begin{frame}
\begin{figure}
        \centering
        \includegraphics[height=0.8\textheight]{../figures/invariant_example.pdf}
        \caption{The first step in constructing an invariant}
        \label{fig:}
\end{figure}
\end{frame}
\begin{frame}
        \begin{definition}[indexed tensor powers]\label{def:indexed_tensor_powers}
        Let $V$ be a vector space and $S = \set{s_1,…, s_n}$ be a finite
        set. We define the \defi{indexed tensor power} of $V$ to be the
        collection of formal linear combinations of functions from $S$ to $V$
        \begin{equation}\label{eq:indexed_tensor}
                V_S \defeq V^{\otimes S} \defeq \Span\set{\map {f} {S} {V}}/\sim
        \end{equation}
        subject to multi-additivity and the factoring of scalars.
\end{definition}
\pause
We will write such functions $\map {f} {S} {V}$ with $f(s_i) = v_i$
with the following notation:
\begin{equation}\label{eq:product_notation}
        \pn{v_{1}}_{s_1}
        \pn{v_{2}}_{s_2} ⋯
        \pn{v_{n}}_{s_n}
        \defeq f
\end{equation}
\end{frame}
\begin{frame}
        \begin{definition}[meta-algebra]\label{def:meta_algebra}
                A \defi{meta-algebra} (or \defi{meta-monoid}) is a meta-object
                $\set{A_X}_X$ in $\CC$ with operations an associative
                multiplication $\map {\mult^{i,j}_{k}} {A_{\set{i,j}}}
                {A_{\set{k}}}$  and a unit
                $\map{\unit_{i}}{A_{\emptyset}}{A_{\set{i}}}$.
        \end{definition}
        \begin{columns}
                \column{0.5\textwidth}
                \begin{equation*}\label{eq:cd_mult}
                        \begin{tikzcd}[ampersand replacement=\&]
                                A_{\set{1,2,3}}
                                \rar["\mult^{1,2}_{1}"]
                                \dar["\mult^{2,3}_{2}"']
                                \&A_{\set{1,3}}
                                \dar["\mult^{1,3}_{1}"] \\
                                A_{\set{1,2}}
                                \rar["\mult^{1,2}_{1}"']
                                \&A_{\set{1}}
                        \end{tikzcd}
                \end{equation*}
                \column{0.5\textwidth}
                \begin{equation*}\label{eq:cd_unit}
                \begin{tikzcd}[ampersand replacement=\&,column sep=large]
                        A_{\set{1}}
                                \rar["\unit_{2}"]
                                \drar["\id"']
                        \&A_{\set{1,2}}
                                \dar["\mult^{1,2}_{1}", shift left]
                                \dar["\mult^{2,1}_{1}"', shift right] \\
                        \&A_{\set{1}}
                \end{tikzcd}
                \end{equation*}
        \end{columns}
\end{frame}

\begin{frame}
        \begin{definition}[meta-coalgebra]
                A \defi{meta-coalgebra} (or \defi{meta-comonoid}) is a meta-object
                $\set{C_X}_X$ with operations a \defi{comultiplication}
                $\map {\comult^{i}_{jk}} {C_{\set{i}}} {C_{\set{j,k}}}$ which is
                \defi{coassociative} and a \defi{counit}, which is a
                map $\map {\counit^{i}} {A_{i}} {A_{\emptyset}}$.
        \end{definition}
        \begin{columns}
                \column{0.5\textwidth}
        \begin{equation*}\label{eq:cd_comult}
        \begin{tikzcd}[ampersand replacement=\&]
                C_{\set{1,2,3}}
                \&C_{\set{2,3}}
                        \lar["\comult^{1}_{{1,2}}"'] \\
                C_{\set{1,2}}
                        \uar["\comult^{2}_{2,3}"]
                \&C_{\set{1}}
                        \lar["\comult^{1}_{1,2}"]
                        \uar["\comult^{1}_{1,3}"']
        \end{tikzcd}
        \end{equation*}
        \column{0.5\textwidth}
        \begin{equation*}\label{eq:cd_counit}
        \begin{tikzcd}[ampersand replacement=\&]
                C_{\set{1}}
                \&C_{\set{1,2}}
                        \lar["\counit^{2}"']\\
                \&C_{\set{1}}
                        \ular["\id", shift left]
                        \uar["\comult^{1}_{1,2}", shift left]
                        \uar["\comult^{1}_{2,1}"', shift right] \\
        \end{tikzcd}
        \end{equation*}
        \end{columns}
\end{frame}

\begin{frame}
        \begin{definition}[meta-bialgebra]
                A \defi{meta-bialgebra} (or \defi{meta-bimonoid}) is a
                meta-algebra $(B,\mult,\unit)$ and a meta-coalgebra
                $(B,\comult,\counit)$, such that $\comult$ and $\counit$ are
                meta-algebra morphisms.
        \end{definition}
\end{frame}
\begin{frame}
        \frametitle{Bialgebra diagrams}
        \framesubtitle{
                \onslide*<2,3,4>{Colgebra: %
                        \onslide*<3>{Multiplication is a coalgebra morphism.}%
                        \onslide*<4>{The unit is a coalgebra morphism.}%
                }
                \onslide*<5,6,7>{Algebra: %
                        \onslide*<6>{Comultiplication is an algebra morphism.}%
                        \onslide*<7>{The counit is a algebra morphism.}%
                }
        }
        \begin{columns}
        \column{0.45\linewidth}
        \onslide<1,2,3,5,6>
        \begin{equation*}\label{eq:cd_mult_comult}
        \begin{tikzcd}[ampersand replacement=\&,column sep=large]
                B_{\set{1,2}}
                        \rar["\mult^{1,2}_{1}"]
                        \dar["\comult^{1}_{1,3}\then\comult^{2}_{2,4}"]
                \&B_{\set{1}}
                        \dar["\comult^{1}_{1,2}"] \\
                B_{\set{1,2,3,4}}
                        \rar["\mult^{1,2}_{1}\then\mult^{3,4}_{2}"']
                \&B_{\set{1,2}}
        \end{tikzcd}
        \end{equation*}
        \onslide<2,3,4>\hline
        \onslide<1,2,4,5,6>
        \begin{equation*}\label{eq:cd_unit_comult}
        \begin{tikzcd}[ampersand replacement=\&,row sep=tiny]
                \&B_{\set{1}}
                        \ar[dd,"\comult^{1}_{1,2}"] \\
                B_{\emptyset}
                        \urar["\unit_{1}"]
                        \drar["\unit_{1}\then\unit_{2}"',near end]\\
                \&B_{\set{1,2}}
        \end{tikzcd}
        \end{equation*}
        \column{0.1\linewidth}
        \onslide<5,6,7>\rule{0.1mm}{0.7\textheight}
        \column{0.45\linewidth}
        \onslide<1,2,3,5,7>
        \begin{equation*}\label{eq:cd_mult_counit}
        \begin{tikzcd}[ampersand replacement=\&,column sep=tiny]
                B_{\set{1,2}}
                        \ar[rr,"\mult^{1,2}_{1}"]
                        \drar["\counit^{1}\then\counit^{2}"']
                \&\&B_{\set{1}}
                        \dlar["\counit^{1}"] \\
                \&B_{\emptyset}
        \end{tikzcd}
        \end{equation*}
        \onslide<2,3,4>\hline
        \onslide<1,2,4,5,7>
        \begin{equation*}\label{eq:cd_unit_counit}
        \begin{tikzcd}[ampersand replacement=\&]
                B_{\emptyset}
                        \rar["\unit_{1}"]
                        \drar["\id"']
                \&B_{\set{1}}
                        \dar["\counit^{1}"] \\
                \&B_{\emptyset}
        \end{tikzcd}
        \end{equation*}
        \end{columns}
\end{frame}

\begin{frame}
        \begin{definition}[meta-Hopf algebra]
        A \defi{meta-Hopf algebra} (or \defi{meta-Hopf monoid}) is a
        meta-bialgebra $H$ together with a map $\map {\antipode} {H} {H}$ called
        the \defi{antipode}, which satisfies
        $\comult^{1}_{1,2}\then \antipode^1_1 \then \mult^{1,2}_1 =
        \counit^{1}\then\unit_{1} =
        \comult^{1}_{1,2}\then \antipode^2_2 \then \mult^{1,2}_1$.
        \end{definition}
        \begin{equation*}
        \begin{tikzcd}[ampersand replacement=\&,column sep=tiny]
                \label{eq:cd_antipode}
                H_{\set{1}}
                        \arrow[rr, "\counit^{1}"] \arrow[rd, "\comult^{1}_{1,2}"']
                \&\& H_{\emptyset}
                        \arrow[rr, "\unit_{1}"]
                \&\& H_{\set{1}} \\
                \& H_{\set{1,2}}
                        \arrow[rr, "\antipode^{2}_{2}", shift left]
                        \arrow[rr, "\antipode^{1}_{1}"', shift right]
                \&\& H_{\set{1,2}} \arrow[ru, "\mult^{1,2}_{1}"']
        \end{tikzcd}
        \end{equation*}
\end{frame}

\begin{frame}
        \centering
        \Huge Thank you!
\end{frame}
\end{document}
