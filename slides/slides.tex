\documentclass{beamer}

\usetheme{Berlin}

\usepackage{mathtools}
\usepackage{newunicodechar}
\usepackage{unicode-math}
\usepackage{tikz-cd}
\usepackage{subcaption}
\usepackage{graphicx}
\usepackage{cleveref}
\usepackage{stmaryrd}
% Thesis version
\newcommand{\thesisVersion}{1.0.0}

% Personal data (insert your own data here)
\newcommand{\myTitle}{Computing the generating function of a coinvariants map\xspace}
\newcommand{\myName}{Jesse Frohlich\xspace}
\newcommand{\myDegree}{Doctor of Philosophy\xspace}
\newcommand{\myDepartment}{Graduate Department of Mathematics\xspace}
\newcommand{\myUni}{University of Toronto\xspace}
\newcommand{\myTime}{2023\xspace}

% Symbols

% Paired Delimiters

\ProvideDocumentCommand{\curry}{m}{\ifblank{#1}{\:\cdot\:}{#1}}

%% Parenthetical constructs
\DeclarePairedDelimiter{\pn}\lparen\rparen
\DeclarePairedDelimiter{\set}\{\}
\DeclarePairedDelimiter{\bk}\lbrack\rbrack
\DeclarePairedDelimiter{\bkk}\llbracket\rrbracket
\DeclarePairedDelimiter{\gen}\langle\rangle

%% Operators

\DeclarePairedDelimiterX{\commutator}[2]{[}{]}{\curry{#1}, \curry{#2}}

\DeclarePairedDelimiterX{\liebk}[2]{[}{]}{\curry{#1}, \curry{#2}}

\DeclarePairedDelimiterXPP{\normHelper}[2]{}\lVert\rVert
        {\IfNoValueTF{#2}{}{_{#2}}}
        {\curry{#1}}
\ProvideDocumentCommand{\norm}{s O{} m o}{% Optional subscript token at the end.
        \IfBooleanTF{#1}
                {\normHelper*{#3}{#4}}
                {\normHelper[#2]{#3}{#4}}
}

\DeclarePairedDelimiterXPP{\absHelper}[2]{}\lvert\rvert
        {\IfNoValueTF{#2}{}{_{#2}}}
        {\curry{#1}}
\ProvideDocumentCommand{\abs}{s O{} m o}{% Optional subscript token at the end.
        \IfBooleanTF{#1}
                {\absHelper*{#3}{#4}}
                {\absHelper[#2]{#3}{#4}}
}

\DeclarePairedDelimiterX{\pair}[2]\langle\rangle{%
        \nonscript\,\curry{#1}\PairSymbol[\delimsize]{\vert}\curry{#2}\nonscript\,%
}

\DeclarePairedDelimiterX{\inner}[2]\langle\rangle{%
        \nonscript\,\curry{#1}\PairSymbol[\delimsize]{,}\curry{#2}\nonscript\,%
}

\DeclarePairedDelimiterX{\card}[1]\lvert\rvert{\curry{#1}}
% \ProvideDocumentCommand{\card}{\#}

\DeclarePairedDelimiterX\ceil [1]{\lceil}{\rceil}{\curry{#1}}

\DeclarePairedDelimiterX\floor[1]{\lfloor}{\rfloor}{\curry{#1}}

\DeclarePairedDelimiterX\ord  [1]{\lvert}{\rvert}{\curry{#1}}
\DeclareMathOperator{\Ord}{ord}


%% Building constructs

\DeclarePairedDelimiterX{\setbuilder}[2]\{\}{\,#1\SetSymbol[\delimsize]#2\,}

\DeclarePairedDelimiterX{\genbuilder}[2]\langle\rangle{\,#1\SetSymbol[\delimsize]#2\,}

\newcommand{\grppres}{\genbuilder}

\DeclareMathOperator{\HomOp}{Hom} 
\DeclarePairedDelimiterXPP{\HomHelper}[3]
        {\IfNoValueTF{#1}
                {\HomOp}
                {\HomOp_{#1}}
        }
        ()
        {}{\curry{#2}, \curry{#3}}
\ProvideDocumentCommand{\Hom}{s O{} m m}{
        \IfBooleanTF{#1}
                {\HomHelper*{#2}{#3}{#4}}
                {\HomHelper{#2}{#3}{#4}}
}

% Quantum groups / Hopf algebras

\newcommand{\unit}{\eta}           % unit
\newcommand{\one}{\mbf 1}          % another form of the unit
\newcommand{\counit}{\epsilon}     % counit

\newcommand{\mult}{m}              % multiplication
\newcommand{\comult}{\Delta}       % comultiplication
\ProvideDocumentCommand{\cmf}{mm}  % comultiplication factor
  {#2_{(#1)}}

\newcommand{\antipode}{S}          % antipode
\ProvideDocumentCommand{\bap}{m}   % bar-antipode
  {\overline{#1}}
\ProvideDocumentCommand{\baap}{m}  % bar-inverse(i.e. anti)-antipode
  {\underline{#1}}

\newcommand{\rmat}{r}         % Lie-algebraic r-matrix

\newcommand{\Rmat}{\mcl R}         % R-matrix
\ProvideDocumentCommand{\rmf}{O{}} % R-matrix First factor
  {\Rmat^{(1)}_{#1}}
\ProvideDocumentCommand{\rms}{O{}} % R-matrix Second factor
  {\Rmat^{(2)}_{#1}}

\ProvideDocumentCommand{\coevf}{m} % Coevaluation First factor
  {r_{#1}}
\ProvideDocumentCommand{\coevd}{m} % Coevaluation Dual factor
  {ρ_{#1}}


\RenewDocumentCommand{\defi}{}{\emph}

\title{\myTitle}
\author{\myName}
\date{\myTime}
\institute{\myUni~\\\myDepartment}

\AtBeginSection[ ]
{
\begin{frame}{Outline}
        \tableofcontents[
                currentsection,
                hideallsubsections
        ]
\end{frame}
}

\begin{document}

\begin{frame}
        \titlepage
\end{frame}

\begin{frame}
        \frametitle{Outline}
        \tableofcontents[hideallsubsections]
\end{frame}

\section{Knots, links, and tangles}

\subsection{Examples of tangles}

\begin{frame}
        \begin{figure}
                \centering
                \includegraphics[height=0.8\textheight]{../figures/pure_tangle_example.pdf}
                \caption{An open tangle. All components intersect the boundary.}
                \label{fig:open_tangle}
        \end{figure}
\end{frame}

\begin{frame}
        \begin{figure}
        \centering
        \includegraphics[height=0.8\textheight]{../figures/impure_tangle_example.pdf}
        \caption{A tangle with a closed component.}
        \label{fig:impure_tangle}
\end{figure}
\end{frame}

\begin{frame}
        \begin{figure}
                \centering
                \includegraphics[height=0.6\textheight]{../figures/hopf_closed.pdf}
                \caption{A \defi{link} is a tangle with no open components.}
                \label{fig:hopf_closed}
        \end{figure}
\end{frame}

\subsection{Working with tangles}

\begin{frame}
\begin{figure}
        \centering
        \includegraphics[width=0.9\textwidth]{../figures/tangle_intro.pdf}
        \caption{Can you arrange the left-hand side to look like the right?}
        \label{fig:tangle_intro}
\end{figure}
\end{frame}

\begin{frame}
\begin{figure}
        \centering
        \includegraphics[width=0.9\textwidth]{../figures/tangle_comult.pdf}
        \caption{A topological operation on tangles}
        \label{fig:tangle_comult_example}
\end{figure}
\end{frame}

\begin{frame}
        \begin{figure}
                \centering
                \includegraphics[height=0.8\textheight]{../figures/invariant_example.pdf}
                \caption{One way to construct an invariant}
                \label{fig:invariant_example}
        \end{figure}
\end{frame}

\subsection{Definitions}

\begin{frame}
        \begin{definition}[open tangle]
                An \defi{open tangle} is an embedding of line segments (called
                \defi{components} or \defi{strands}) into a thickened
                topological disk $D \times [-1,1]$ (or a disjoint union of such
                disks) such that the endpoints of the line segments are fixed
                along $\boundary D \times \set0$, taken up to endpoint-fixing
                isotopy.
                \pause
                The set of all tangles with strands indexed by $X$ will be
                denoted $\tangleDown X$.
        \end{definition}
        \pause
        (Here, \enquote{open} refers to the absence of closed loops.)
\end{frame}

\begin{frame}
        Instead of strings, we will use thin bands:
        \pause
        \begin{definition}[framed tangle]
                A \defi{framed tangle} is an open tangle together with a choice
                of section of the normal bundle for each component, with
                endpoints of the section fixed pointing to the right of the
                tangent vector. The section is taken up to endpoint-fixing
                homotopy.
        \end{definition}
        \pause
        From now on all tangles are framed.
\end{frame}

\begin{frame}
        \begin{theorem}[Reidemeister]
                Two diagrams of a framed tangle are isotopic to one another
                exactly when they differ by a finite number of
                \defi{Reidemeister moves}.
        \end{theorem}
\end{frame}

\begin{frame}
        \begin{figure}
                \centering
                \includegraphics[
                        width=0.8\textwidth,
                        height=0.8\textheight,
                        keepaspectratio,
                ]{../figures/R1p.pdf}
                \caption{(Framed) Reidemeister move $R1'$}
                \label{fig:R1p}
        \end{figure}
\end{frame}

\begin{frame}
        \begin{figure}
                \centering
                \includegraphics[
                        width=0.8\textwidth,
                        height=0.8\textheight,
                        keepaspectratio,
                ]{../figures/R2.pdf}
                \caption{Reidemeister move $R2$}
                \label{fig:R2}
        \end{figure}
\end{frame}

\begin{frame}
        \begin{figure}
                \centering
                \includegraphics[
                        width=0.8\textwidth,
                        height=0.8\textheight,
                        keepaspectratio,
                ]{../figures/R3.pdf}
                \caption{Reidemeister move $R3$}
                \label{fig:R3}
        \end{figure}
\end{frame}

\begin{frame}
        If the endpoints of strands must always point upwards, we get two more types of
        moves:
        \pause
        \begin{figure}
                \centering
                \includegraphics[
                        width=0.8\textwidth,
                        height=0.7\textheight,
                        keepaspectratio,
                ]{../figures/R2rot.pdf}
                \caption{The (cyclic) rotational Reidemeister move $R2_{\text{rot}}$}
                \label{fig:R2rot}
        \end{figure}
\end{frame}

\begin{frame}
        \begin{figure}
        \centering
        \includegraphics[
                width=0.8\textwidth,
                height=0.8\textheight,
                keepaspectratio,
        ]{../figures/whirl.pdf}
        \caption{The whirling move}
        \label{fig:whirl}
\end{figure}
\end{frame}

\section{Algebraic structures}

\subsection{Index notation}

\begin{frame}
        \begin{definition}[indexed tensor powers]\label{def:indexed_tensor_powers}
        Let $V$ be a vector space and $S = \set{s_1,…, s_n}$ be a finite
        set. We define the \defi{indexed tensor power} of $V$ to be the
        collection of formal linear combinations of functions from $S$ to $V$
        \begin{equation}\label{eq:indexed_tensor}
                V_S \defeq V^{\otimes S} \defeq \Span\set{\map {f} {S} {V}}/\sim
        \end{equation}
        subject to multi-additivity and the factoring of scalars.
\end{definition}
\pause
We will write such functions $\map {f} {S} {V}$ with $f(s_i) = v_i$
with the following notation:
\begin{equation}\label{eq:product_notation}
        \pn{v_{1}}_{s_1}
        \pn{v_{2}}_{s_2} ⋯
        \pn{v_{n}}_{s_n}
        \defeq f
\end{equation}
\end{frame}

\begin{frame}
        Let $D$ and $C$ be finite sets, and
        $\map {T} {V_{D}} {V_{C}}$.

        We will write $T$ alternatively as $T^{D}_{C}$.
\end{frame}

\begin{frame}
        \begin{example}
                Let $V = \R^3$, and $\map {T} {V_{\set{a, b}}} {V_{\set{c}}}$
                with
                $T^{a, b}_{c}\pn*{\vec v_a\vec w_b} = (\vec v \times \vec w)_c$
                be the cross product. Stating that the cross product is
                antisymmetric becomes:
                \begin{equation}
                        T^{a, b}_{c} = -T^{b, a}_{c}
                \end{equation}
        \end{example}
\end{frame}

\subsection{Meta-algebras}

\begin{frame}
        \begin{definition}[meta-algebra]\label{def:meta_algebra}
                A \defi{meta-algebra} (or \defi{meta-monoid}) is a meta-object
                $\set{A_X}_X$ in $\CC$ with operations an associative
                multiplication $\map {\mult^{i,j}_{k}} {A_{\set{i,j}}}
                {A_{\set{k}}}$  and a unit
                $\map{\unit_{i}}{A_{\emptyset}}{A_{\set{i}}}$.
        \end{definition}
        \begin{columns}
                \column{0.5\textwidth}
                \begin{equation*}\label{eq:cd_mult}
                        \begin{tikzcd}[ampersand replacement=\&]
                                A_{\set{1,2,3}}
                                \rar["\mult^{1,2}_{1}"]
                                \dar["\mult^{2,3}_{2}"']
                                \&A_{\set{1,3}}
                                \dar["\mult^{1,3}_{1}"] \\
                                A_{\set{1,2}}
                                \rar["\mult^{1,2}_{1}"']
                                \&A_{\set{1}}
                        \end{tikzcd}
                \end{equation*}
                \column{0.5\textwidth}
                \begin{equation*}\label{eq:cd_unit}
                \begin{tikzcd}[ampersand replacement=\&,column sep=large]
                        A_{\set{1}}
                                \rar["\unit_{2}"]
                                \drar["\id"']
                        \&A_{\set{1,2}}
                                \dar["\mult^{1,2}_{1}", shift left]
                                \dar["\mult^{2,1}_{1}"', shift right] \\
                        \&A_{\set{1}}
                \end{tikzcd}
                \end{equation*}
        \end{columns}
\end{frame}

\begin{frame}
        \begin{figure}
        \centering
        \includegraphics[width=0.9\textwidth]{../figures/tangle_mult.pdf}
        \caption{
                Multiplication $\mult^{ij}_{k}$ stitches two strands in a tangle
                together.
        }
        \label{fig:tangle_mult}
\end{figure}
\end{frame}

\begin{frame}
        \begin{figure}
        \centering
        \includegraphics[width=0.9\textwidth]{../figures/tangle_unit.pdf}
        \caption{The unit $\unit_i$ introduces a new strand in a tangle.}
        \label{fig:tangle_unit}
\end{figure}
\end{frame}

\subsection{Meta-coalgebras}

\begin{frame}
        \begin{definition}[meta-coalgebra]
                A \defi{meta-coalgebra} (or \defi{meta-comonoid}) is a meta-object
                $\set{C_X}_X$ with operations a \defi{comultiplication}
                $\map {\comult^{i}_{jk}} {C_{\set{i}}} {C_{\set{j,k}}}$ which is
                \defi{coassociative} and a \defi{counit}, which is a
                map $\map {\counit^{i}} {A_{i}} {A_{\emptyset}}$.
        \end{definition}
        \begin{columns}
                \column{0.5\textwidth}
        \begin{equation*}\label{eq:cd_comult}
        \begin{tikzcd}[ampersand replacement=\&]
                C_{\set{1,2,3}}
                \&C_{\set{2,3}}
                        \lar["\comult^{1}_{{1,2}}"'] \\
                C_{\set{1,2}}
                        \uar["\comult^{2}_{2,3}"]
                \&C_{\set{1}}
                        \lar["\comult^{1}_{1,2}"]
                        \uar["\comult^{1}_{1,3}"']
        \end{tikzcd}
        \end{equation*}
        \column{0.5\textwidth}
        \begin{equation*}\label{eq:cd_counit}
        \begin{tikzcd}[ampersand replacement=\&]
                C_{\set{1}}
                \&C_{\set{1,2}}
                        \lar["\counit^{2}"']\\
                \&C_{\set{1}}
                        \ular["\id", shift left]
                        \uar["\comult^{1}_{1,2}", shift left]
                        \uar["\comult^{1}_{2,1}"', shift right] \\
        \end{tikzcd}
        \end{equation*}
        \end{columns}
\end{frame}

\begin{frame}
        \begin{figure}
                \centering
                \includegraphics[width=0.9\textwidth]{../figures/tangle_counit.pdf}
                \caption{The counit $\counit^i$ deletes a strand in a tangle.}
                \label{fig:tangle_counit}
        \end{figure}
\end{frame}

\begin{frame}
        \begin{figure}
                \centering
                \includegraphics[width=0.9\textwidth]{../figures/tangle_comult.pdf}
                \caption{%
                        The comultiplication $\comult^{i}_{jk}$ doubles a
                        strand in a tangle along its framing.%
                }
                \label{fig:tangle_comult}
        \end{figure}
\end{frame}

\subsection{Meta-bialgebras}

\begin{frame}
        \begin{definition}[meta-bialgebra]
                A \defi{meta-bialgebra} (or \defi{meta-bimonoid}) is a
                meta-algebra $(B,\mult,\unit)$ and a meta-coalgebra
                $(B,\comult,\counit)$, such that $\comult$ and $\counit$ are
                meta-algebra morphisms.
        \end{definition}
\end{frame}

\begin{frame}
        \frametitle{Bialgebra diagrams}
        \framesubtitle{
                \onslide*<2,3,4>{Colgebra: %
                        \onslide*<3>{Multiplication is a coalgebra morphism.}%
                        \onslide*<4>{The unit is a coalgebra morphism.}%
                }
                \onslide*<5,6,7>{Algebra: %
                        \onslide*<6>{Comultiplication is an algebra morphism.}%
                        \onslide*<7>{The counit is a algebra morphism.}%
                }
        }
        \begin{columns}
        \column{0.45\linewidth}
        \onslide<1,2,3,5,6>
        \begin{equation*}\label{eq:cd_mult_comult}
        \begin{tikzcd}[ampersand replacement=\&,column sep=large]
                B_{\set{1,2}}
                        \rar["\mult^{1,2}_{1}"]
                        \dar["\comult^{1}_{1,3}\then\comult^{2}_{2,4}"]
                \&B_{\set{1}}
                        \dar["\comult^{1}_{1,2}"] \\
                B_{\set{1,2,3,4}}
                        \rar["\mult^{1,2}_{1}\then\mult^{3,4}_{2}"']
                \&B_{\set{1,2}}
        \end{tikzcd}
        \end{equation*}
        \onslide<2,3,4>\hline
        \onslide<1,2,4,5,6>
        \begin{equation*}\label{eq:cd_unit_comult}
        \begin{tikzcd}[ampersand replacement=\&,row sep=tiny]
                \&B_{\set{1}}
                        \ar[dd,"\comult^{1}_{1,2}"] \\
                B_{\emptyset}
                        \urar["\unit_{1}"]
                        \drar["\unit_{1}\then\unit_{2}"',near end]\\
                \&B_{\set{1,2}}
        \end{tikzcd}
        \end{equation*}
        \column{0.1\linewidth}
        \onslide<5,6,7>\rule{0.1mm}{0.7\textheight}
        \column{0.45\linewidth}
        \onslide<1,2,3,5,7>
        \begin{equation*}\label{eq:cd_mult_counit}
        \begin{tikzcd}[ampersand replacement=\&,column sep=tiny]
                B_{\set{1,2}}
                        \ar[rr,"\mult^{1,2}_{1}"]
                        \drar["\counit^{1}\then\counit^{2}"']
                \&\&B_{\set{1}}
                        \dlar["\counit^{1}"] \\
                \&B_{\emptyset}
        \end{tikzcd}
        \end{equation*}
        \onslide<2,3,4>\hline
        \onslide<1,2,4,5,7>
        \begin{equation*}\label{eq:cd_unit_counit}
        \begin{tikzcd}[ampersand replacement=\&]
                B_{\emptyset}
                        \rar["\unit_{1}"]
                        \drar["\id"']
                \&B_{\set{1}}
                        \dar["\counit^{1}"] \\
                \&B_{\emptyset}
        \end{tikzcd}
        \end{equation*}
        \end{columns}
\end{frame}

\subsection{Meta-Hopf algebras}

\begin{frame}
        \begin{definition}[meta-Hopf algebra]
        A \defi{meta-Hopf algebra} (or \defi{meta-Hopf monoid}) is a
        meta-bialgebra $H$ together with a map $\map {\antipode} {H} {H}$ called
        the \defi{antipode}, which satisfies
        $\comult^{1}_{1,2}\then \antipode^1_1 \then \mult^{1,2}_1 =
        \counit^{1}\then\unit_{1} =
        \comult^{1}_{1,2}\then \antipode^2_2 \then \mult^{1,2}_1$.
        \end{definition}
        \begin{equation*}
        \begin{tikzcd}[ampersand replacement=\&,column sep=tiny]
                \label{eq:cd_antipode}
                H_{\set{1}}
                        \arrow[rr, "\counit^{1}"] \arrow[rd, "\comult^{1}_{1,2}"']
                \&\& H_{\emptyset}
                        \arrow[rr, "\unit_{1}"]
                \&\& H_{\set{1}} \\
                \& H_{\set{1,2}}
                        \arrow[rr, "\antipode^{2}_{2}", shift left]
                        \arrow[rr, "\antipode^{1}_{1}"', shift right]
                \&\& H_{\set{1,2}} \arrow[ru, "\mult^{1,2}_{1}"']
        \end{tikzcd}
        \end{equation*}
\end{frame}

\begin{frame}
        \begin{figure}
        \centering
        \includegraphics[height=0.8\textheight]{../figures/tangle_antipode.pdf}
        \caption{The antipode $\antipode^{i}_{i}$ reverses a strand, rotating the
        endpoints to maintain an upright tangle.}
        \label{fig:tangle_antipode}
\end{figure}
\end{frame}

\subsection{Quasitriangular meta-Hopf algebras}

\begin{frame}
        \begin{definition}[quasitriangular meta-Hopf algebra]
        A \defi{quasitriangular meta-Hopf algebra} (or \defi{quasitriangular meta-Hopf
        monoid}) is a meta-Hopf algebra $H$, together with an invertible element
        $\Rmat_{i,j} \in H_{i,j}$, called the \defi{$\Rmat$-matrix}, which satisfies the
        following properties: (we will denote the inverse by $\Rmati$)
        \begin{align}
                \label{eq:Rmat_understrand}
                \Rmat_{13}\then\comult^{1}_{12}&=\Rmat_{13}\Rmat_{24}\then\mult^{34}_3\\
                \label{eq:Rmat_overstrand}
                \Rmat_{13}\then\comult^{3}_{23}&=\Rmat_{13}\Rmat_{42}\then\mult^{14}_1\\
                \label{eq:Rmat_comult}
                \comult^{1}_{21} &=
                        \comult^{1}_{12} \Rmat_{a_1,a_2}\Rmati_{p_1,p_2}\then
                        \mult^{a_1,1,p_1}_{1}\then \mult^{a_2,2,p_2}_{2}
        \end{align}
        \end{definition}
\end{frame}

\begin{frame}
        \begin{figure}
                \centering
                \begin{subfigure}[b]{0.4\textwidth}
                        \centering
                        \includegraphics{../figures/tangle_rmat.pdf}
                        \caption{A positive crossing, represented by $\Rmat_{ij}$}
                        \label{fig:tangle_rmat}
                \end{subfigure}
                \begin{subfigure}[b]{0.4\textwidth}
                        \centering
                        \includegraphics{../figures/tangle_rmati.pdf}
                        \caption{A negative crossing, represented by $\Rmati_{ij}$}
                        \label{fig:tangle_rmati}
                \end{subfigure}
                \caption{The $\Rmat$-matrix and its inverse represent a tangle with a
                single crossing.}
                \label{fig:tangle_rmats}
        \end{figure}
\end{frame}

\begin{frame}
        \begin{figure}
                \centering
                \includegraphics{../figures/Rmat_understrand.pdf}
                \caption{Example of a tangle satisfying \cref{eq:Rmat_understrand}}
                \label{fig:Rmat_understrand}
        \end{figure}
\end{frame}

\begin{frame}
        \begin{figure}
                \centering
                \includegraphics{../figures/Rmat_overstrand.pdf}
                \caption{Example of a tangle satisfying \cref{eq:Rmat_overstrand}}
                \label{fig:Rmat_overstrand}
        \end{figure}
\end{frame}

\subsection{Ribbon meta-Hopf algebras}

\begin{frame}
        \begin{definition}[Drinfeld element]
                In a quasitriangular meta-Hopf algebra $H$, the \defi{Drinfeld
                element}, $\dfe \in H$ is:
                \begin{equation}
                        \dfe \defeq \Rmat_{21}\then\antipode^1_1 \then \mult^{12}
                \end{equation}
        \end{definition}
        \pause
        \begin{figure}
                \centering
                \includegraphics[
                        width=0.8\textwidth,
                        height=0.4\textheight,
                        keepaspectratio,
                ]{../figures/tangle_drinfeld.pdf}
                \caption{The Drinfeld element $\dfe_i$ in the meta-Hopf algebra of
                tangles.}
                \label{fig:tangle_drinfeld}
        \end{figure}
\end{frame}

\begin{frame}
        \begin{definition}[monodromy]
                Each quasitriangular meta-Hopf algebra has a \defi{monodromy}
                $\monodromy_{12} \defeq
                \Rmat_{12}\Rmat_{34}\then\mult^{14}_{1}\then\mult^{23}_{2}$. Its
                inverse will be denoted
                $\invb\monodromy_{12} =
                \Rmati_{12}\Rmati_{34}\then\mult^{14}_{1}\then\mult^{23}_{2}$.
        \end{definition}
        \pause
        \begin{figure}[h]
                \centering
                \includegraphics[
                        width=0.8\textwidth,
                        height=0.4\textheight,
                        keepaspectratio,
                ]{../figures/monodromy.pdf}
                \caption{The monodromy in the meta-Hopf algebra of tangles.}
                \label{fig:monodromy}
        \end{figure}
\end{frame}

\begin{frame}
        \begin{definition}[spinner]
                A \defi{spinner} in a ribbon meta-Hopf algebra $H$ is an
                invertible element $\spin\in H$ (with inverse $\invb\spin$) such
                that for all $x\in H$:
                \begin{align}
                        \label{eq:spinner_ribbon}
                        \spin_1\ribbon_2\spin_3 \then \antipode^2_2 \then \mult^{123} &=
                        \ribbon\\
                        \label{eq:spinner_comult}
                        \spin_1\then\comult^{1}_{12} &=\spin_1\spin_2\\
                        \label{eq:spinner_antipode}
                        \spin \then\antipode &= \invb\spin\\
                        \label{eq:spinner_conjugate}
                        \spin_{1}x_2\invb\spin_{3}\then\mult^{123} &=
                        x \then \antipode \then \antipode\\
                        \label{eq:spinner_counit}
                        \spin \then \counit &= \unit \then \counit = 1
                \end{align}
        \end{definition}
\end{frame}

\begin{frame}
        \begin{figure}
        \centering
        \begin{subfigure}[b]{0.4\textwidth}
                \centering
                \includegraphics{../figures/tangle_spin.pdf}
                \caption{The spinner $\spin_i$ has rotation number $1$.}
                \label{fig:tangle_spin}
        \end{subfigure}
        \begin{subfigure}[b]{0.4\textwidth}
                \centering
                \includegraphics{../figures/tangle_spini.pdf}
                \caption{The inverse spinner $\spini_i$ has rotation number
                $-1$.}
                \label{fig:tangle_spini}
        \end{subfigure}
        \caption{The spinners represent strands with a unit rotation number.}
        \label{fig:tangle_spinner}
\end{figure}
\end{frame}

\begin{frame}
        \begin{definition}[ribbon meta-Hopf algebra]
                A quasitriangular meta-Hopf algebra $H$ is called \defi{ribbon}
                if it has an element $\ribbon\in \centre(H)$ such that:
                \begin{align}
                        \ribbon_1\ribbon_2\then\mult^{12}
                &= \dfe_1 \dfe_2 \then \antipode^{2}_{2} \then \mult^{12}\\
                \ribbon_1 \then \comult^1_{12}
                &=      \ribbon_1\ribbon_2
                \then\invb\monodromy_{34}
                \then\mult^{13}_{1}
                \then\mult^{24}_{2} \\
                        \ribbon \then \antipode &= \ribbon\\
                        \ribbon \then \counit &= \unit \then \counit = 1
                \end{align}
        \end{definition}
\end{frame}

\begin{frame}
        \begin{figure}
        \centering
        \includegraphics{../figures/tangle_ribbon.pdf}
        \caption{%
                Two equivalent forms of the ribbon element $\ribbon_i$ in the
                meta-Hopf algebra of tangles.%
        }
        \label{fig:tangle_ribbon}
\end{figure}
\end{frame}

\section{The ribbon Hopf algebra $\CU$}

\subsection{Defining $\CU$}

\begin{frame}
        \begin{definition}[The algebra structure of $\CU$]
        Define the Lie algebra
        \begin{equation*}
        \fg \defeq \Span_{\Q}\setbuilder[\Big]{\yo, \bo, \ao, \xo}{
                \liebk{\ao}{\xo} = \xo,
                \liebk{\ao}{\yo} = -\yo,
                \liebk{\xo}{\yo} = \bo,
                \liebk{\bo}{ } = 0
        }
        \end{equation*}
        We put a grading on $\fg$ by $\deg(y) = \deg(b) = 1$ and
        $\deg(a) = \deg(x) = 0$. Then $\CU=\uea*{\fg}$ is the graded completion
        of the universal enveloping algebra of $\fg$.
        \end{definition}
\end{frame}

\begin{frame}
        \begin{definition}[The bialgebra structure of $\CU$]
                With $\Bo \defeq \Exp{-\bo}$, let:
                \begin{equation}\begin{aligned}
                        \comult_{i,j}(\yo) &=
                        \frac{\bo_i+\bo_j}{1-\Bo_i\Bo_j} \pn*{
                                \Bo_j\frac{1-\Bo_i}{\bo_i}\yo_i+
                                \frac{1-\Bo_j}{\bo_j}\yo_j
                        }\\
                        \comult_{i,j}(\bo) &= \bo_i + \bo_j\\
                        \comult_{i,j}(\ao) &= \ao_i + \ao_j\\
                        \comult_{i,j}(\xo) &= \xo_i + \xo_j\\
                \end{aligned}\end{equation}
                For $\zo\in\set{\yo, \bo, \ao, \xo}$, set $\counit(\zo)=0$.
        \end{definition}
\end{frame}

\begin{frame}
        \begin{definition}[The Hopf algebra structure of $\CU$]
                Let $\antipode(\zo) \defeq -\zo$ for each
                $\zo\in\set{\yo, \bo, \ao, \xo}$, extended antimultiplicatively.
        \end{definition}

        \begin{definition}[The ribbon structure of $\CU$]
                \begin{align}
                        \Rmat_{i,j}
                        &\defeq \exp\pn*{\bo_i\ao_j}
                        \exp\pn*{\frac{1-\Bo_i}{\bo_i}\yo_i\xo_j}\\
                        \spin &\defeq \sqrt{\Bo}\\
                        \ribbon
                              &\defeq \Rmati_{31}\spini_{2} \then \mult^{123}
                              = \Rmati_{13}\spin_{2} \then \mult^{123}
                \end{align}
        \end{definition}
\end{frame}

\subsection{Commutation relations in $\CU$}

\begin{frame}
        \begin{lemma}[Commutation relations in $\CU$]\label{lem:xay_relations}
                Given $f\in \polyring{\Q}{\ao}$,
                \begin{align}\label{eq:xay_relations}
                        f(\ao)\yo^r &= \yo^rf(\ao-r) &
                        \xo^rf(\ao) &= f(\ao-r)\xo^r
                \end{align}
                \pause
                Also true for $f(a, b)\in \powerseries{\polyring{\Q}{\ao}}{\bo}$.
        \end{lemma}
        \pause
        \begin{lemma}[Weyl canonical commutation relation]
                In $\powerseries{\CU}{ξ, η}$:
                \begin{equation}\label{eq:weyl_relation}
                        \Exp{ξ\xo}\Exp{η\yo} = \Exp{ξη\bo}\Exp{η\yo}\Exp{ξ\xo}
                \end{equation}
        \end{lemma}
\end{frame}

\begin{frame}
        \begin{lemma}[Exponential commutation relations in $\CU$]
                \label{lem:exp_xay_relations}
                In the ring $\powerseries{\CU}{α, η, ξ}$, let $\A\defeq \Exp{α}$.
                Then:
                \begin{align}
                        \label{eq:ay_commutation}
                        \Exp{α\ao}\Exp{η\yo} &= \Exp{\frac{η}{\A}\yo}\Exp{α\ao}\\
                        \label{eq:ax_commutation}
                        \Exp{ξ\xo}\Exp{α\ao} &= \Exp{α\ao}\Exp{\frac{ξ}{\A}\xo}
                \end{align}
        \end{lemma}
\end{frame}

\begin{frame}
        \centering
        \Huge Thank you!
\end{frame}
\end{document}
