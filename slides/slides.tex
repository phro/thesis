\documentclass{beamer}

\usetheme{Berlin}

\usepackage{mathtools}
\usepackage{newunicodechar}
\usepackage{unicode-math}
\usepackage{tikz-cd}
\usepackage{subcaption}
\usepackage{graphicx}
% Thesis version
\newcommand{\thesisVersion}{0.4.0}

% Personal data (insert your own data here)
\newcommand{\myTitle}{Computing the generating function of a coinvariants map\xspace}
\newcommand{\myName}{Jesse Frohlich\xspace}
\newcommand{\myDegree}{Doctor of Philosophy\xspace}
\newcommand{\myDepartment}{Graduate Department of Mathematics\xspace}
\newcommand{\myUni}{University of Toronto\xspace}
\newcommand{\myTime}{2023\xspace}

% Theorem-like environments

        \ProvideDocumentCommand{\defi}{m}{\uline{#1}} % Item being DEFined

        \newcommand{\alignedintertext}[1]{%
          \noalign{%
            \vskip\belowdisplayshortskip
            \vtop{\hsize=\linewidth#1\par
            \expandafter}%
            \expandafter\prevdepth\the\prevdepth
          }%
        }

% Font changes

        \ProvideDocumentCommand{\mcl}{m}{\mathcal{#1}}
        \ProvideDocumentCommand{\mbb}{m}{\mathbb{#1}}
        \ProvideDocumentCommand{\mbf}{m}{\boldsymbol{\mathbf{#1}}}
        \ProvideDocumentCommand{\msc}{m}{\mathscr{#1}}
        \ProvideDocumentCommand{\mfk}{m}{\mathfrak{#1}}
        \ProvideDocumentCommand{\mit}{m}{\mathit{#1}}

% Symbols

        \ProvideDocumentCommand{\lt}{}{<}
        \ProvideDocumentCommand{\gt}{}{>}
        \ProvideDocumentCommand{\eq}{}{=}
        \ProvideDocumentCommand{\defeq}{}{\coloneq}
        \newcommand{\iso}{\cong} % or \cong
        \newcommand{\homeo}{\cong} % or \cong
        \newcommand{\hty}{\cong} % or \cong
        \ProvideDocumentCommand{\then}{}{\mathbin{/\mkern-6mu/}}
        \newcommand{\fracl}[2]{\left.{#1}\middle/{#2}\right.}
        \renewcommand{\boundary}{\partial}
        \newunicodechar{∂}{\ensuremath\partial}
        \newunicodechar{⋯}{\ensuremath\cdots}
        \newunicodechar{⋮}{\ensuremath\vdots}
        \newunicodechar{…}{\ensuremath\dots}

% Standard number sets.
        \newcommand{\N}{\mbb{N}}   % Natural
        \newcommand{\Z}{\mbb{Z}}   % Integer
        \newcommand{\Q}{\mbb{Q}}   % Rational
        \newcommand{\R}{\mbb{R}}   % Real
        \newcommand{\C}{\mbb{C}}   % Complex
        \newcommand{\F}{\mbb{F}}   % Finite
        \newcommand{\K}{\mbb{k}}   % Generic

% Common operators

        \ProvideDocumentCommand{\inv}{sm}{% Invert something
                \IfBooleanTF{#1}
                        {\frac1{#2}}
                        {#2^{-1}}
        }
        % Exponential Map
        \ProvideDocumentCommand{\Exp}{m}{%
                \mbf e^{#1}
        }
        \ProvideDocumentCommand{\invb}{m}{\overline{#1}} % bar-style inverse
        \DeclareMathOperator{\dom}{dom}        % Domain
        \DeclareMathOperator{\cod}{cod}        % Codomain
        \ProvideDocumentCommand{\id}{o}{
          \IfNoValueTF {#1}
            {\operatorname{id}}
            {\operatorname{id}_{#1}}
        }
        \ProvideDocumentCommand{\Id}{m}{\id^{#1}_{#1}}
        \ProvideDocumentCommand{\dual}{m}{{#1}^*}

        \DeclareMathOperator{\Span}{span}       % span
        \DeclareMathOperator{\nullity}{nullity} % nullity
        \DeclareMathOperator{\rank}{rank}       % rank
        \DeclareMathOperator{\trace}{tr}        % trace
        \DeclareMathOperator{\Null}{Null}      % Nullspace
        \DeclareMathOperator{\Gal}{Gal}        % Galois group
        \DeclareMathOperator{\Spec}{Spec}      % Spectrum
        \DeclareMathOperator{\Frac}{Frac}      % Field of fractions
        \DeclareMathOperator{\Proj}{Proj}      % Proj construction
        \DeclareMathOperator*{\esssup}{ess\,sup} % essential supremum
        \DeclareMathOperator*{\supp}{supp}     % support
        \ProvideDocumentCommand{\centre}{}{%
          Z
          % ζ
        }

        \ProvideDocumentCommand{\Operator}{moo}{
          \IfNoValueTF{#2}
            {#1}
            {\IfNoValueTF{#3}% Don't make parentheses expand beyond the operator.
              {#1_{#2}}
              {#1_{#2}^{#3}}
              % {\vphantom{#1^{-}}\smash{#1_{#2}}}
              % {\vphantom{#1^{-}}\smash{#1_{#2}^{#3}}}
            }
        }


        % There should really be another argument for what the operator is acting on,
        % but that's a lot more typing I really don't care to do. However, semantically
        % this is the way to go, and snippets *are* at my disposal.
        % {oo} means "take two optional arguments"
        \ProvideDocumentCommand{\Sum   }{oo}{\Operator\sum      [#1][#2]}
        \ProvideDocumentCommand{\Dsum  }{oo}{\Operator\bigoplus [#1][#2]}
        % The following code does not work. Why is this?
        \ProvideDocumentCommand{\Dsumc }{oo}{\Operator{\bigoplus}[#1][#2]}
        \ProvideDocumentCommand{\Prod  }{oo}{\Operator\prod     [#1][#2]}
        \ProvideDocumentCommand{\Tprod }{oo}{\Operator\bigotimes[#1][#2]}
        \ProvideDocumentCommand{\Coprod}{oo}{\Operator\coprod   [#1][#2]}
        \ProvideDocumentCommand{\Dunion}{oo}{\Operator\bigsqcup [#1][#2]}
        \ProvideDocumentCommand{\Union }{oo}{\Operator\bigcup   [#1][#2]}
        \ProvideDocumentCommand{\Inter }{oo}{\Operator\bigcap   [#1][#2]}
        \ProvideDocumentCommand{\Int   }{oo}{\Operator\int      [#1][#2]}
        \newcommand{\gint}{\mathrlap{\int}\,G}
        \ProvideDocumentCommand{\GInt  }{}{\Operator\gint     }
        \ProvideDocumentCommand{\IInt  }{oo}{\Operator\iint     [#1][#2]}
        \ProvideDocumentCommand{\IIInt }{oo}{\Operator\iiint    [#1][#2]}
        \ProvideDocumentCommand{\Wedge }{oo}{\Operator\bigwedge [#1][#2]}

% Rings

        \ProvideDocumentCommand{\polyring}{mm}{#1\bk{#2}}
        \ProvideDocumentCommand{\powerseries}{mm}{#1\bkk{#2}}

% Classic Groups
        \DeclareMathOperator{\GL}{GL}
        \DeclareMathOperator{\SL}{SL}
        \DeclareMathOperator{\SP}{Sp}
        \DeclareMathOperator{\SO}{SO}
        \DeclareMathOperator{\Spin}{Spin}
        \DeclareMathOperator{\U}{U}
        \DeclareMathOperator{\SU}{SU}
        \DeclareMathOperator{\Or}{O}

% Lie algebras

        %% Algebras

                \DeclareMathOperator{\Gl}{\mfk{gl}}
                \DeclareMathOperator{\Sp}{\mfk{sp}}
                \DeclareMathOperator{\Sl}{\mfk{sl}}
                \DeclareMathOperator{\So}{\mfk{so}}
                \DeclareMathOperator{\g}{\mfk{g}}
                \newcommand{\fg}{\mfk{g}}
                \newcommand{\fh}{\mfk{h}}
                \newcommand{\fn}{\mfk{n}}
                \newcommand{\fb}{\mfk{b}}

        %% Lie algebra operations

                \DeclareMathOperator{\ad}{ad}   % adjoint
                \DeclareMathOperator{\Ad}{Ad}   % Big Adjoint
                \DeclareMathOperator{\Lie}{Lie} % Lie algebra

        %% Universal enveloping algebra
                \ProvideDocumentCommand{\uea}{sO{}m}{
                        \mathfrak{
                                \IfBooleanTF{#1}{\hat U}{U}
                        }_{#2}\pn{#3}
                }

% Categories
        \ProvideDocumentCommand{\catname}{m}{\mathbf{#1}}
        \DeclareMathOperator{\mfld  }{\catname{Mfld}}
        \DeclareMathOperator{\mfldb }{\catname{Mfld}\pmb\boundary}
        \DeclareMathOperator{\Vect  }{\catname{Vect}}
        \DeclareMathOperator{\Mod   }{\catname{Mod}}
        \DeclareMathOperator{\Set   }{\catname{Set}}
        \DeclareMathOperator{\Ring  }{\catname{Ring}}
        \DeclareMathOperator{\Top   }{\catname{Top}}
        \DeclareMathOperator{\FinSet}{\catname{FinSet}}

% Paired Delimiters

\ProvideDocumentCommand{\curry}{m}{\ifblank{#1}{\:\cdot\:}{#1}}

%% Parenthetical constructs
\DeclarePairedDelimiter{\pn}\lparen\rparen
\DeclarePairedDelimiter{\set}\{\}
\DeclarePairedDelimiter{\bk}\lbrack\rbrack
\DeclarePairedDelimiter{\bkk}\llbracket\rrbracket
\DeclarePairedDelimiter{\gen}\langle\rangle

%% Operators

\DeclarePairedDelimiterX{\commutator}[2]{[}{]}{\curry{#1}, \curry{#2}}

\DeclarePairedDelimiterX{\liebk}[2]{[}{]}{\curry{#1}, \curry{#2}}

\DeclarePairedDelimiterXPP{\normHelper}[2]{}\lVert\rVert
        {\IfNoValueTF{#2}{}{_{#2}}}
        {\curry{#1}}
\ProvideDocumentCommand{\norm}{s O{} m o}{% Optional subscript token at the end.
        \IfBooleanTF{#1}
                {\normHelper*{#3}{#4}}
                {\normHelper[#2]{#3}{#4}}
}

\DeclarePairedDelimiterXPP{\absHelper}[2]{}\lvert\rvert
        {\IfNoValueTF{#2}{}{_{#2}}}
        {\curry{#1}}
\ProvideDocumentCommand{\abs}{s O{} m o}{% Optional subscript token at the end.
        \IfBooleanTF{#1}
                {\absHelper*{#3}{#4}}
                {\absHelper[#2]{#3}{#4}}
}

\NewDocumentCommand{\restrict}{sO{}mO{}}{%
        \IfBooleanTF{#1}{% star
                \mleft.\kern-\nulldelimiterspace
                #3
                \mright|%
        }{% no star
                #3#2|%
        }%
        _{#4}%
}

\DeclarePairedDelimiterX{\pair}[2]\langle\rangle{%
        \nonscript\,\curry{#1}\PairSymbol[\delimsize]{\vert}\curry{#2}\nonscript\,%
}

\DeclarePairedDelimiterX{\inner}[2]\langle\rangle{%
        \nonscript\,\curry{#1}\PairSymbol[\delimsize]{,}\curry{#2}\nonscript\,%
}

\DeclarePairedDelimiterXPP{\contractionHelper}[2]{}\langle\rangle
        {\IfNoValueTF{#2}{}{_{#2}}}
        {\curry{#1}}
\ProvideDocumentCommand{\contraction}{s O{} m o}{% Optional subscript token at the end.
        \IfBooleanTF{#1}
                {\contractionHelper*{#3}{#4}}
                {\contractionHelper[#2]{#3}{#4}}
}

\DeclarePairedDelimiterX{\card}[1]\lvert\rvert{\curry{#1}}
% \ProvideDocumentCommand{\card}{\#}

\DeclarePairedDelimiterX\ceil [1]{\lceil}{\rceil}{\curry{#1}}

\DeclarePairedDelimiterX\floor[1]{\lfloor}{\rfloor}{\curry{#1}}

\DeclarePairedDelimiterX\ord  [1]{\lvert}{\rvert}{\curry{#1}}
\DeclareMathOperator{\Ord}{ord}


%% Building constructs

% can be useful to refer to this outside \Set
\ProvideDocumentCommand{\innerspacing}{}{\mathchoice{\:}{\:}{\,}{\,}}
\newcommand\SetSymbol[1][]{%
  \innerspacing%
  %:%
  #1\vert%
  \allowbreak\innerspacing\mathopen{}%
}
\newcommand\PairSymbol[2][]{%
  \nonscript\, #1 #2 \allowbreak\nonscript\,\mathopen{}%
}

\DeclarePairedDelimiterX{\setbuilder}[2]\{\}{\,#1\SetSymbol[\delimsize]#2\,}

\DeclarePairedDelimiterX{\genbuilder}[2]\langle\rangle{\,#1\SetSymbol[\delimsize]#2\,}

\newcommand{\grppres}{\genbuilder}

\DeclareMathOperator{\Object}{Ob} 
\DeclareMathOperator{\HomOp}{Hom} 
\DeclarePairedDelimiterXPP{\HomHelper}[3]
        {\IfNoValueTF{#1}
                {\HomOp}
                {\HomOp_{#1}}
        }
        ()
        {}{\curry{#2}, \curry{#3}}
\ProvideDocumentCommand{\Hom}{s O{} m m}{
        \IfBooleanTF{#1}
                {\HomHelper*{#2}{#3}{#4}}
                {\HomHelper{#2}{#3}{#4}}
}

% Maps
        \ProvideDocumentCommand{\map}{mmO{\to}m}{#1\colon#2#3#4}
        \ProvideDocumentCommand{\nodomainmap}{mmm}{#1\colon#2\mapsto#3}
        \ProvideDocumentCommand{\selfmap}{mmO{\to}}{#1\colon#2#3#2}
        \ProvideDocumentCommand{\selfmapdef}{mmO{\to}m m }{%
                \begin{aligned}
                        #1\colon#2&#3#2\\
                        #4&\mapsto#5
                \end{aligned}%
        }
        \ProvideDocumentCommand{\mapdef}{mmO{\to}m m m o}{%
                \begin{aligned}
                        #1\colon#2&#3#4\\
                        #5&\mapsto#6
                        \IfNoValueTF{#7}{}{\\ #7}
                \end{aligned}%
        }

        \newcommand{\mono}{\hookrightarrow} % monomorphism
        \newcommand{\inc }{\hookrightarrow} % inclusion % DEPRECATE?
        \newcommand{\inj }{\hookrightarrow} % injection
        \newcommand{\emb }{\hookrightarrow} % embedding
        \newcommand{\sur }{\twoheadrightarrow} % epimorphism
        \newcommand{\epi }{\twoheadrightarrow} % epimorphism
        \newcommand{\toiso }{\xrightarrow{\sim}} % isomorphism

% Quantum groups / Hopf algebras

\newcommand{\unit}{\eta}           % unit
\newcommand{\one}{\mbf 1}          % another form of the unit
\newcommand{\counit}{\epsilon}     % counit

\newcommand{\mult}{m}              % multiplication
\newcommand{\comult}{\Delta}       % comultiplication
\ProvideDocumentCommand{\cmf}{mm}  % comultiplication factor
  {#2_{(#1)}}

\newcommand{\antipode}{S}          % antipode
\ProvideDocumentCommand{\bap}{m}   % bar-antipode
  {\overline{#1}}
\ProvideDocumentCommand{\baap}{m}  % bar-inverse(i.e. anti)-antipode
  {\underline{#1}}

\newcommand{\rmat}{r}         % Lie-algebraic r-matrix

\newcommand{\Rmat}{\mcl R}         % R-matrix
\newcommand{\Rmati}{\invb\Rmat}
\ProvideDocumentCommand{\rmf}{O{}} % R-matrix First factor
  {\Rmat^{(1)}_{#1}}
\ProvideDocumentCommand{\rms}{O{}} % R-matrix Second factor
  {\Rmat^{(2)}_{#1}}

\ProvideDocumentCommand{\coevf}{m} % Coevaluation First factor
  {r_{#1}}
\ProvideDocumentCommand{\coevd}{m} % Coevaluation Dual factor
  {ρ_{#1}}

\ProvideDocumentCommand{\spin}{}{C}
\ProvideDocumentCommand{\ribbon}{}{ν}
\ProvideDocumentCommand{\dfe}{}{\mfk u}
\ProvideDocumentCommand{\monodromy}{}{Q}

% Thesis-specific macros

\ProvideDocumentCommand{\ptr}{}{Z^\trace}
\ProvideDocumentCommand{\fa}{}{\mfk{a}}
\ProvideDocumentCommand{\Alg}{}{\fg}
\ProvideDocumentCommand{\CUlong}{}{\uea*{\Sl_{2+}^0}}
\ProvideDocumentCommand{\CU}{}{U}
\ProvideDocumentCommand{\nn}{}{\mathbf{n}}
\ProvideDocumentCommand{\Order}{}{\mathbb O}
\RenewDocumentCommand{\k}{}{\mbf k}
\DeclareMathOperator{\GenMap}{\msc G} 
\DeclarePairedDelimiterXPP{\Gen}[1]{\GenMap}\lparen\rparen{}{#1}
\ProvideDocumentCommand{\tangle}{}{\msc T}
\ProvideDocumentCommand{\RVT}{}{\msc{T}^{\text{rv}}}
\ProvideDocumentCommand{\A}{}{\msc{A}}
\ProvideDocumentCommand{\tanglename}{momm}{#1_{
                #3%
                \IfValueTF{#2}{\text{#2}}{,}%
                #4
}}
\newcommand{\knot}{\tanglename{K}} 
\newcommand{\link}{\tanglename{L}} 

\RenewDocumentCommand{\defi}{}{\emph}

\title{\myTitle}
\author{\myName}
\date{\myTime}
\institute{\myUni~\\\myDepartment}

\begin{document}

\begin{frame}
        \titlepage
\end{frame}

\begin{frame}
        \frametitle{Outline}
        \tableofcontents
\end{frame}
\section{Algebraic structures}

\begin{frame}
\begin{figure}
        \centering
        \includegraphics[height=0.8\textheight]{../figures/tangle_example.pdf}
        \caption{An example of a tangle}
        \label{fig:tangle_example}
\end{figure}
\end{frame}

\begin{frame}
\begin{figure}
        \centering
        \includegraphics[width=0.9\textwidth]{../figures/tangle_intro.pdf}
        \caption{Can you arrange the left-hand side to look like the right?}
        \label{fig:tangle_intro}
\end{figure}
\end{frame}

\begin{frame}
\begin{figure}
        \centering
        \includegraphics[width=0.9\textwidth]{../figures/tangle_comult.pdf}
        \caption{Doubling strand $i$ into two strands $j$ and $k$}
        \label{fig:tangle_comult}
\end{figure}
\end{frame}

\subsection{Meta-algebras}

\begin{frame}
        \begin{definition}[meta-algebra]\label{def:meta_algebra}
                A \defi{meta-algebra} (or \defi{meta-monoid}) is a meta-object
                $\set{A_X}_X$ in $\CC$ with operations an associative
                multiplication $\map {\mult^{i,j}_{k}} {A_{\set{i,j}}}
                {A_{\set{k}}}$  and a unit
                $\map{\unit_{i}}{A_{\emptyset}}{A_{\set{i}}}$.
        \end{definition}
        \begin{columns}
                \column{0.5\textwidth}
                \begin{equation*}\label{eq:cd_mult}
                        \begin{tikzcd}[ampersand replacement=\&]
                                A_{\set{1,2,3}}
                                \rar["\mult^{1,2}_{1}"]
                                \dar["\mult^{2,3}_{2}"']
                                \&A_{\set{1,3}}
                                \dar["\mult^{1,3}_{1}"] \\
                                A_{\set{1,2}}
                                \rar["\mult^{1,2}_{1}"']
                                \&A_{\set{1}}
                        \end{tikzcd}
                \end{equation*}
                \column{0.5\textwidth}
                \begin{equation*}\label{eq:cd_unit}
                \begin{tikzcd}[ampersand replacement=\&,column sep=large]
                        A_{\set{1}}
                                \rar["\unit_{2}"]
                                \drar["\id"']
                        \&A_{\set{1,2}}
                                \dar["\mult^{1,2}_{1}", shift left]
                                \dar["\mult^{2,1}_{1}"', shift right] \\
                        \&A_{\set{1}}
                \end{tikzcd}
                \end{equation*}
        \end{columns}
\end{frame}

\begin{frame}
        \begin{figure}
        \centering
        \includegraphics[width=0.9\textwidth]{../figures/tangle_mult.pdf}
        \caption{
                Multiplication $\mult^{ij}_{k}$ stitches two strands in a tangle
                together.
        }
        \label{fig:tangle_mult}
\end{figure}
\end{frame}

\begin{frame}
        \begin{figure}
        \centering
        \includegraphics[width=0.9\textwidth]{../figures/tangle_unit.pdf}
        \caption{The unit $\unit_i$ introduces a new strand in a tangle.}
        \label{fig:tangle_unit}
\end{figure}
\end{frame}

\subsection{Meta-coalgebras}

\begin{frame}
        \begin{definition}[meta-coalgebra]
                A \defi{meta-coalgebra} (or \defi{meta-comonoid}) is a meta-object
                $\set{C_X}_X$ with operations a \defi{comultiplication}
                $\map {\comult^{i}_{jk}} {C_{\set{i}}} {C_{\set{j,k}}}$ which is
                \defi{coassociative} and a \defi{counit}, which is a
                map $\map {\counit^{i}} {A_{i}} {A_{\emptyset}}$.
        \end{definition}
        \begin{columns}
                \column{0.5\textwidth}
        \begin{equation*}\label{eq:cd_comult}
        \begin{tikzcd}[ampersand replacement=\&]
                C_{\set{1,2,3}}
                \&C_{\set{2,3}}
                        \lar["\comult^{1}_{{1,2}}"'] \\
                C_{\set{1,2}}
                        \uar["\comult^{2}_{2,3}"]
                \&C_{\set{1}}
                        \lar["\comult^{1}_{1,2}"]
                        \uar["\comult^{1}_{1,3}"']
        \end{tikzcd}
        \end{equation*}
        \column{0.5\textwidth}
        \begin{equation*}\label{eq:cd_counit}
        \begin{tikzcd}[ampersand replacement=\&]
                C_{\set{1}}
                \&C_{\set{1,2}}
                        \lar["\counit^{2}"']\\
                \&C_{\set{1}}
                        \ular["\id", shift left]
                        \uar["\comult^{1}_{1,2}", shift left]
                        \uar["\comult^{1}_{2,1}"', shift right] \\
        \end{tikzcd}
        \end{equation*}
        \end{columns}
\end{frame}

\begin{frame}
        \begin{figure}
                \centering
                \includegraphics[width=0.9\textwidth]{../figures/tangle_counit.pdf}
                \caption{The counit $\counit^i$ deletes a strand in a tangle.}
                \label{fig:tangle_counit}
        \end{figure}
\end{frame}

\begin{frame}
        \begin{figure}
                \centering
                \includegraphics[width=0.9\textwidth]{../figures/tangle_comult.pdf}
                \caption{%
                        The comultiplication $\comult^{i}_{jk}$ doubles a
                        strand in a tangle along its framing.%
                }
                \label{fig:tangle_comult}
        \end{figure}
\end{frame}

\subsection{Meta-bialgebras}

\begin{frame}
        \begin{definition}[meta-bialgebra]
                A \defi{meta-bialgebra} (or \defi{meta-bimonoid}) is a
                meta-algebra $(B,\mult,\unit)$ and a meta-coalgebra
                $(B,\comult,\counit)$, such that $\comult$ and $\counit$ are
                meta-algebra morphisms.
        \end{definition}
\end{frame}

\begin{frame}
        \frametitle{Bialgebra diagrams}
        \framesubtitle{
                \onslide*<2,3,4>{Colgebra: %
                        \onslide*<3>{Multiplication is a coalgebra morphism.}%
                        \onslide*<4>{The unit is a coalgebra morphism.}%
                }
                \onslide*<5,6,7>{Algebra: %
                        \onslide*<6>{Comultiplication is an algebra morphism.}%
                        \onslide*<7>{The counit is a algebra morphism.}%
                }
        }
        \begin{columns}
        \column{0.45\linewidth}
        \onslide<1,2,3,5,6>
        \begin{equation*}\label{eq:cd_mult_comult}
        \begin{tikzcd}[ampersand replacement=\&,column sep=large]
                B_{\set{1,2}}
                        \rar["\mult^{1,2}_{1}"]
                        \dar["\comult^{1}_{1,3}\then\comult^{2}_{2,4}"]
                \&B_{\set{1}}
                        \dar["\comult^{1}_{1,2}"] \\
                B_{\set{1,2,3,4}}
                        \rar["\mult^{1,2}_{1}\then\mult^{3,4}_{2}"']
                \&B_{\set{1,2}}
        \end{tikzcd}
        \end{equation*}
        \onslide<2,3,4>\hline
        \onslide<1,2,4,5,6>
        \begin{equation*}\label{eq:cd_unit_comult}
        \begin{tikzcd}[ampersand replacement=\&,row sep=tiny]
                \&B_{\set{1}}
                        \ar[dd,"\comult^{1}_{1,2}"] \\
                B_{\emptyset}
                        \urar["\unit_{1}"]
                        \drar["\unit_{1}\then\unit_{2}"',near end]\\
                \&B_{\set{1,2}}
        \end{tikzcd}
        \end{equation*}
        \column{0.1\linewidth}
        \onslide<5,6,7>\rule{0.1mm}{0.7\textheight}
        \column{0.45\linewidth}
        \onslide<1,2,3,5,7>
        \begin{equation*}\label{eq:cd_mult_counit}
        \begin{tikzcd}[ampersand replacement=\&,column sep=tiny]
                B_{\set{1,2}}
                        \ar[rr,"\mult^{1,2}_{1}"]
                        \drar["\counit^{1}\then\counit^{2}"']
                \&\&B_{\set{1}}
                        \dlar["\counit^{1}"] \\
                \&B_{\emptyset}
        \end{tikzcd}
        \end{equation*}
        \onslide<2,3,4>\hline
        \onslide<1,2,4,5,7>
        \begin{equation*}\label{eq:cd_unit_counit}
        \begin{tikzcd}[ampersand replacement=\&]
                B_{\emptyset}
                        \rar["\unit_{1}"]
                        \drar["\id"']
                \&B_{\set{1}}
                        \dar["\counit^{1}"] \\
                \&B_{\emptyset}
        \end{tikzcd}
        \end{equation*}
        \end{columns}
\end{frame}

\subsection{meta-Hopf algebras}

\begin{frame}
        \begin{definition}[meta-Hopf algebra]
        A \defi{meta-Hopf algebra} (or \defi{meta-Hopf monoid}) is a
        meta-bialgebra $H$ together with a map $\map {\antipode} {H} {H}$ called
        the \defi{antipode}, which satisfies
        $\comult^{1}_{1,2}\then \antipode^1_1 \then \mult^{1,2}_1 =
        \counit^{1}\then\unit_{1} =
        \comult^{1}_{1,2}\then \antipode^2_2 \then \mult^{1,2}_1$.
        \end{definition}
        \begin{equation*}
        \begin{tikzcd}[ampersand replacement=\&,column sep=tiny]
                \label{eq:cd_antipode}
                H_{\set{1}}
                        \arrow[rr, "\counit^{1}"] \arrow[rd, "\comult^{1}_{1,2}"']
                \&\& H_{\emptyset}
                        \arrow[rr, "\unit_{1}"]
                \&\& H_{\set{1}} \\
                \& H_{\set{1,2}}
                        \arrow[rr, "\antipode^{2}_{2}", shift left]
                        \arrow[rr, "\antipode^{1}_{1}"', shift right]
                \&\& H_{\set{1,2}} \arrow[ru, "\mult^{1,2}_{1}"']
        \end{tikzcd}
        \end{equation*}
\end{frame}

\begin{frame}
        \begin{figure}
        \centering
        \includegraphics[height=0.8\textheight]{../figures/tangle_antipode.pdf}
        \caption{The antipode $\antipode^{i}_{i}$ reverses a strand, rotating the
        endpoints to maintain an upright tangle.}
        \label{fig:tangle_antipode}
\end{figure}
\end{frame}

\subsection{Quasitriangular meta-Hopf algebras}

\begin{frame}
        \begin{definition}[quasitriangular meta-Hopf algebra]
        A \defi{quasitriangular meta-Hopf algebra} (or \defi{quasitriangular meta-Hopf
        monoid}) is a meta-Hopf algebra $H$, together with an invertible element
        $\Rmat_{i,j} \in H_{i,j}$, called the \defi{$\Rmat$-matrix}, which satisfies the
        following properties: (we will denote the inverse by $\Rmati$)
        \begin{align}
                \label{eq:Rmat_understrand}
                \Rmat_{13}\then\comult^{1}_{12}&=\Rmat_{13}\Rmat_{24}\then\mult^{34}_3\\
                \label{eq:Rmat_overstrand}
                \Rmat_{13}\then\comult^{3}_{23}&=\Rmat_{13}\Rmat_{42}\then\mult^{14}_1\\
                \label{eq:Rmat_comult}
                \comult^{1}_{21} &=
                        \comult^{1}_{12} \Rmat_{a_1,a_2}\Rmati_{p_1,p_2}\then
                        \mult^{a_1,1,p_1}_{1}\then \mult^{a_2,2,p_2}_{2}
        \end{align}
        \end{definition}
\end{frame}

\begin{frame}
        \begin{figure}
                \centering
                \begin{subfigure}[b]{0.4\textwidth}
                        \centering
                        \includegraphics{../figures/tangle_rmat.pdf}
                        \caption{A positive crossing, represented by $\Rmat_{ij}$}
                        \label{fig:tangle_rmat}
                \end{subfigure}
                \begin{subfigure}[b]{0.4\textwidth}
                        \centering
                        \includegraphics{../figures/tangle_rmati.pdf}
                        \caption{A negative crossing, represented by $\Rmati_{ij}$}
                        \label{fig:tangle_rmati}
                \end{subfigure}
                \caption{The $\Rmat$-matrix and its inverse represent a tangle with a
                single crossing.}
                \label{fig:tangle_rmats}
        \end{figure}
\end{frame}

\subsection{Ribbon meta-Hopf algebras}

\begin{frame}
        \begin{definition}[Drinfeld element]
                In a quasitriangular meta-Hopf algebra $H$, the \defi{Drinfeld
                element}, $\dfe \in H$ is:
                \begin{equation}
                        \dfe \defeq \Rmat_{21}\then\antipode^1_1 \then \mult^{12}
                \end{equation}
        \end{definition}
        \pause
        \begin{figure}
                \centering
                \includegraphics[
                        width=0.8\textwidth,
                        height=0.4\textheight,
                        keepaspectratio,
                ]{../figures/tangle_drinfeld.pdf}
                \caption{The Drinfeld element $\dfe_i$ in the meta-Hopf algebra of
                tangles.}
                \label{fig:tangle_drinfeld}
        \end{figure}
\end{frame}

\begin{frame}
        \begin{definition}[monodromy]
                Each quasitriangular meta-Hopf algebra has a \defi{monodromy}
                $\monodromy_{12} \defeq
                \Rmat_{12}\Rmat_{34}\then\mult^{14}_{1}\then\mult^{23}_{2}$. Its
                inverse will be denoted
                $\invb\monodromy_{12} =
                \Rmati_{12}\Rmati_{34}\then\mult^{14}_{1}\then\mult^{23}_{2}$.
        \end{definition}
\end{frame}

\begin{frame}
        \begin{definition}[ribbon meta-Hopf algebra]
                A quasitriangular meta-Hopf algebra $H$ is called \defi{ribbon}
                if it has an element $\ribbon\in \centre(H)$ such that:
                \begin{align}
                        \ribbon_1\ribbon_2\then\mult^{12}
                &= \dfe_1 \dfe_2 \then \antipode^{2}_{2} \then \mult^{12}\\
                \ribbon_1 \then \comult^1_{12}
                &=      \ribbon_1\ribbon_2
                \then\invb\monodromy_{34}
                \then\mult^{13}_{1}
                \then\mult^{24}_{2} \\
                        \ribbon \then \antipode &= \ribbon\\
                        \ribbon \then \counit &= \unit \then \counit = 1
                \end{align}
        \end{definition}
\end{frame}

\begin{frame}
        \begin{definition}[spinner]
        A \defi{spinner} in a ribbon meta-Hopf algebra $H$ is an invertible element
        $\spin\in H$ (with inverse $\invb\spin$) such that for all $x\in H$:
        \begin{align}
                \label{eq:spinner_ribbon}
                \spin_1\ribbon_2\spin_3 \then \antipode^2_2 \then \mult^{123} &=
                \ribbon\\
                \label{eq:spinner_comult}
                \spin_1\then\comult^{1}_{12} &=\spin_1\spin_2\\
                \label{eq:spinner_antipode}
                \spin \then\antipode &= \invb\spin\\
                \label{eq:spinner_conjugate}
                \spin_{1}x_2\invb\spin_{3}\then\mult^{123} &=
                x \then \antipode \then \antipode\\
                \label{eq:spinner_counit}
                \spin \then \counit &= \unit \then \counit = 1
        \end{align}
\end{definition}
\end{frame}
\section{Knots, links, and tangles}

\begin{frame}
        \begin{figure}
                \centering
                \includegraphics[height=0.8\textheight]{../figures/pure_tangle_example.pdf}
                \caption{An open tangle. All components intersect the boundary.}
                \label{fig:open_tangle}
        \end{figure}
\end{frame}

\begin{frame}
        \begin{figure}
        \centering
        \includegraphics[height=0.8\textheight]{../figures/impure_tangle_example.pdf}
        \caption{A tangle with a closed component.}
        \label{fig:impure_tangle}
\end{figure}
\end{frame}

\begin{frame}
        \begin{figure}
                \centering
                \includegraphics[height=0.4\textheight]{../figures/hopf_closed.pdf}
                \caption{A \defi{link} is a tangle with no open components.}
                \label{fig:hopf_closed}
        \end{figure}
\end{frame}
\section{Quantum invariants}

\begin{frame}
\begin{figure}
        \centering
        \includegraphics[height=0.8\textheight]{../figures/invariant_example.pdf}
        \caption{The first step in constructing an invariant}
        \label{fig:}
\end{figure}
\end{frame}

\begin{frame}
        \begin{definition}[indexed tensor powers]\label{def:indexed_tensor_powers}
        Let $V$ be a vector space and $S = \set{s_1,…, s_n}$ be a finite
        set. We define the \defi{indexed tensor power} of $V$ to be the
        collection of formal linear combinations of functions from $S$ to $V$
        \begin{equation}\label{eq:indexed_tensor}
                V_S \defeq V^{\otimes S} \defeq \Span\set{\map {f} {S} {V}}/\sim
        \end{equation}
        subject to multi-additivity and the factoring of scalars.
\end{definition}
\pause
We will write such functions $\map {f} {S} {V}$ with $f(s_i) = v_i$
with the following notation:
\begin{equation}\label{eq:product_notation}
        \pn{v_{1}}_{s_1}
        \pn{v_{2}}_{s_2} ⋯
        \pn{v_{n}}_{s_n}
        \defeq f
\end{equation}
\end{frame}

\begin{frame}
        \centering
        \Huge Thank you!
\end{frame}
\end{document}
