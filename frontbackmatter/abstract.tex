\pdfbookmark[1]{Abstract}{Abstract}

\begingroup
\let\clearpage\relax
\let\cleardoublepage\relax
\let\cleardoublepage\relax

\chapter*{Abstract}

\doublespacing

% Do NOT edit the title info here. There is a central place to set Title, Author etc. in the thesisconfig.tex
\begin{center}
	\myTitle\\
\myName\\
\myDegree\\
\myDepartment\\
\myUni\\
\myTime
\end{center}

A well-known source of link invariants is representations of quantum
groups. These invariants require exponential time in the number of crossings to
compute. Recent work has allowed for polynomial-time computations within the
quantum groups themselves, using perturbed Gaußian differential operators. This
thesis introduces and explores a partial expansion of the tangle invariant $Z$
introduced by Bar-Natan and van der Veen \cite{BV}. We expand the use of the
quantum group $\uea{\Sl_{2+}^0}$ to include its space of coinvariants, providing
an extension $\ptr$ of $Z$ from open tangles to links.

We compute a basis for the space of coinvariants of $\uea{\Sl_{2+}^0}$ and a
closed-form expression for the exponential generating function of the
corresponding trace map. The resulting function is not directly compatible with
the previous research. To respond to this limitation, we find a method of
computing $\ptr$ on a subclass of links and write a program to compute $\ptr$ on
two-component links. Contrary to expectations, we find that $\ptr$ is neither
stronger nor weaker than the \acl{MVA}. This unexplained behaviour warrants
further study into $\ptr$ and its relationship with other invariants.

\endgroup
