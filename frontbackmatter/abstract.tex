\pdfbookmark[1]{Abstract}{Abstract}

\begingroup
\let\clearpage\relax
\let\cleardoublepage\relax
\let\cleardoublepage\relax

\chapter*{Abstract}

\doublespacing

% Do NOT edit the title info here. There is a central place to set Title, Author etc. in the thesisconfig.tex
\begin{center}
	\myTitle\\
\myName\\
\myDegree\\
\myDepartment\\
\myUni\\
\myTime
\end{center}

A well-known source of strong link invariants comes from quantum groups.
Typically, one uses a representation of a quantum group to build a computable
invariant, though these computations require exponential time in the number of
crossings. Recent work has allowed for direct and efficient computations within
the quantum groups themselves, through the use of perturbed Gaußian
differential operators. This thesis introduces and explores a partial expansion
of the tangle-theoretic computations performed by Bar-Natan and van der Veen
\cite{BV} in the quantum group $\uea{\Sl_{2+}^0}$ to its space of coinvariants,
providing an extension of this computational method from open tangles to links.
We compute a basis for the space of coinvariants, then compute a closed-form
expression for the corresponding trace map in the form of a generating function.
The resulting function is not a compatible perturbed Gaußian with respect to the
previous research, so in addition, we find a method of computing the link
invariant for a subclass of links. After writing a program to compute the
invariant on two-component links, we found unexpectedly that this extension is
neither stronger nor weaker than the \acl{MVA}. 
\endgroup
