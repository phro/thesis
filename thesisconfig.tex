% This is the file where all your settings, packages, newcommands, ... should go

\PassOptionsToPackage{T1}{fontenc}
  \usepackage{fontenc}

% Configure classicthesis for your needs here.
% (see ClassicThesis.pdf for more information)
\PassOptionsToPackage{
  drafting=false,    % print version information on the bottom of the pages
  tocaligned=false, % the left column of the toc will be aligned (no indentation)
  dottedtoc=true,  % page numbers in ToC flushed right
  eulerchapternumbers=true, % use AMS Euler for chapter font (otherwise Palatino)
  linedheaders=false,       % chaper headers will have line above and beneath
  floatperchapter=true,     % numbering per chapter for all floats (i.e., Figure 1.1)
  eulermath=false,  % use awesome Euler fonts for mathematical formulae (only with pdfLaTeX)
  beramono=true,    % toggle a nice monospaced font (w/ bold)
  palatino=false,    % deactivate standard font for loading another one, see the last section at the end of this file for suggestions
  parts=false,
  style=classicthesis
}{classicthesis}

% Thesis version
\newcommand{\thesisVersion}{0.3.1}

% Personal data (insert your own data here)
\newcommand{\myTitle}{Computing the generating function of a coinvariants map\xspace}
\newcommand{\myName}{Jesse Frohlich\xspace}
\newcommand{\myDegree}{Doctor of Philosophy\xspace}
\newcommand{\myDepartment}{Graduate Department of Mathematics\xspace}
\newcommand{\myUni}{University of Toronto\xspace}
\newcommand{\myTime}{2023\xspace}

% Load all the packages you want here
% Probably you will need the following
\PassOptionsToPackage{english}{babel}
\usepackage{fontsetup}
\usepackage{babel} % language support
\usepackage{enumitem} % for better itemize and enumerate
\usepackage{mathtools,amsthm,amssymb} % because math
\usepackage{mleftright}
\usepackage{etoolbox}
\usepackage{thmtools} % for correct autorefs
\usepackage[onehalfspacing]{setspace} % as required by SGS
\usepackage{graphicx} % to include graphics
\usepackage{xspace} % to get the spacing after macros right
\usepackage{calc} % to allow adding lengths
\usepackage{multicol}
\usepackage{ulem}
\usepackage{stmaryrd}
\usepackage{acronym}
\usepackage{newunicodechar}
\usepackage{ltablex}
\usepackage{ragged2e}
\usepackage{pdflscape}
\usepackage{caption}
\usepackage{subcaption}

% Footnote customization; it is now easier to distinguish superscripts from
% footnotes.
\renewcommand{\thefootnote}{\dag\arabic{footnote}}
% Theorem-like environments

        \ProvideDocumentCommand{\defi}{m}{\uline{#1}} % Item being DEFined

        \newcommand{\alignedintertext}[1]{%
          \noalign{%
            \vskip\belowdisplayshortskip
            \vtop{\hsize=\linewidth#1\par
            \expandafter}%
            \expandafter\prevdepth\the\prevdepth
          }%
        }

% Font changes

        \ProvideDocumentCommand{\mcl}{m}{\mathcal{#1}}
        \ProvideDocumentCommand{\mbb}{m}{\mathbb{#1}}
        \ProvideDocumentCommand{\mbf}{m}{\boldsymbol{\mathbf{#1}}}
        \ProvideDocumentCommand{\msc}{m}{\mathscr{#1}}
        \ProvideDocumentCommand{\mfk}{m}{\mathfrak{#1}}
        \ProvideDocumentCommand{\mit}{m}{\mathit{#1}}

% Symbols

        \ProvideDocumentCommand{\lt}{}{<}
        \ProvideDocumentCommand{\gt}{}{>}
        \ProvideDocumentCommand{\eq}{}{=}
        \ProvideDocumentCommand{\defeq}{}{\coloneq}
        \newcommand{\iso}{\cong} % or \cong
        \newcommand{\homeo}{\cong} % or \cong
        \newcommand{\hty}{\cong} % or \cong
        \ProvideDocumentCommand{\then}{}{\mathbin{/\mkern-6mu/}}
        \newcommand{\fracl}[2]{\left.{#1}\middle/{#2}\right.}
        \renewcommand{\boundary}{\partial}
        \newunicodechar{∂}{\ensuremath\partial}
        \newunicodechar{⋯}{\ensuremath\cdots}
        \newunicodechar{⋮}{\ensuremath\vdots}
        \newunicodechar{…}{\ensuremath\dots}

% Standard number sets.
        \newcommand{\N}{\mbb{N}}   % Natural
        \newcommand{\Z}{\mbb{Z}}   % Integer
        \newcommand{\Q}{\mbb{Q}}   % Rational
        \newcommand{\R}{\mbb{R}}   % Real
        \newcommand{\C}{\mbb{C}}   % Complex
        \newcommand{\F}{\mbb{F}}   % Finite
        \newcommand{\K}{\mbb{k}}   % Generic

% Common operators

        \ProvideDocumentCommand{\inv}{sm}{% Invert something
                \IfBooleanTF{#1}
                        {\frac1{#2}}
                        {#2^{-1}}
        }
        % Exponential Map
        \ProvideDocumentCommand{\Exp}{m}{%
                \mbf e^{#1}
        }
        \ProvideDocumentCommand{\invb}{m}{\overline{#1}} % bar-style inverse
        \DeclareMathOperator{\dom}{dom}        % Domain
        \DeclareMathOperator{\cod}{cod}        % Codomain
        \ProvideDocumentCommand{\id}{o}{
          \IfNoValueTF {#1}
            {\operatorname{id}}
            {\operatorname{id}_{#1}}
        }
        \ProvideDocumentCommand{\Id}{m}{\id^{#1}_{#1}}
        \ProvideDocumentCommand{\dual}{m}{{#1}^*}

        \DeclareMathOperator{\Span}{span}       % span
        \DeclareMathOperator{\nullity}{nullity} % nullity
        \DeclareMathOperator{\rank}{rank}       % rank
        \DeclareMathOperator{\trace}{tr}        % trace
        \DeclareMathOperator{\Null}{Null}      % Nullspace
        \DeclareMathOperator{\Gal}{Gal}        % Galois group
        \DeclareMathOperator{\Spec}{Spec}      % Spectrum
        \DeclareMathOperator{\Frac}{Frac}      % Field of fractions
        \DeclareMathOperator{\Proj}{Proj}      % Proj construction
        \DeclareMathOperator*{\esssup}{ess\,sup} % essential supremum
        \DeclareMathOperator*{\supp}{supp}     % support
        \ProvideDocumentCommand{\centre}{}{%
          Z
          % ζ
        }

        \ProvideDocumentCommand{\Operator}{moo}{
          \IfNoValueTF{#2}
            {#1}
            {\IfNoValueTF{#3}% Don't make parentheses expand beyond the operator.
              {#1_{#2}}
              {#1_{#2}^{#3}}
              % {\vphantom{#1^{-}}\smash{#1_{#2}}}
              % {\vphantom{#1^{-}}\smash{#1_{#2}^{#3}}}
            }
        }


        % There should really be another argument for what the operator is acting on,
        % but that's a lot more typing I really don't care to do. However, semantically
        % this is the way to go, and snippets *are* at my disposal.
        % {oo} means "take two optional arguments"
        \ProvideDocumentCommand{\Sum   }{oo}{\Operator\sum      [#1][#2]}
        \ProvideDocumentCommand{\Dsum  }{oo}{\Operator\bigoplus [#1][#2]}
        % The following code does not work. Why is this?
        \ProvideDocumentCommand{\Dsumc }{oo}{\Operator{\bigoplus}[#1][#2]}
        \ProvideDocumentCommand{\Prod  }{oo}{\Operator\prod     [#1][#2]}
        \ProvideDocumentCommand{\Tprod }{oo}{\Operator\bigotimes[#1][#2]}
        \ProvideDocumentCommand{\Coprod}{oo}{\Operator\coprod   [#1][#2]}
        \ProvideDocumentCommand{\Dunion}{oo}{\Operator\bigsqcup [#1][#2]}
        \ProvideDocumentCommand{\Union }{oo}{\Operator\bigcup   [#1][#2]}
        \ProvideDocumentCommand{\Inter }{oo}{\Operator\bigcap   [#1][#2]}
        \ProvideDocumentCommand{\Int   }{oo}{\Operator\int      [#1][#2]}
        \newcommand{\gint}{\mathrlap{\int}\,G}
        \ProvideDocumentCommand{\GInt  }{}{\Operator\gint     }
        \ProvideDocumentCommand{\IInt  }{oo}{\Operator\iint     [#1][#2]}
        \ProvideDocumentCommand{\IIInt }{oo}{\Operator\iiint    [#1][#2]}
        \ProvideDocumentCommand{\Wedge }{oo}{\Operator\bigwedge [#1][#2]}

% Rings

        \ProvideDocumentCommand{\polyring}{mm}{#1\bk{#2}}
        \ProvideDocumentCommand{\powerseries}{mm}{#1\bkk{#2}}

% Classic Groups
        \DeclareMathOperator{\GL}{GL}
        \DeclareMathOperator{\SL}{SL}
        \DeclareMathOperator{\SP}{Sp}
        \DeclareMathOperator{\SO}{SO}
        \DeclareMathOperator{\Spin}{Spin}
        \DeclareMathOperator{\U}{U}
        \DeclareMathOperator{\SU}{SU}
        \DeclareMathOperator{\Or}{O}

% Lie algebras

        %% Algebras

                \DeclareMathOperator{\Gl}{\mfk{gl}}
                \DeclareMathOperator{\Sp}{\mfk{sp}}
                \DeclareMathOperator{\Sl}{\mfk{sl}}
                \DeclareMathOperator{\So}{\mfk{so}}
                \DeclareMathOperator{\g}{\mfk{g}}
                \newcommand{\fg}{\mfk{g}}
                \newcommand{\fh}{\mfk{h}}
                \newcommand{\fn}{\mfk{n}}
                \newcommand{\fb}{\mfk{b}}

        %% Lie algebra operations

                \DeclareMathOperator{\ad}{ad}   % adjoint
                \DeclareMathOperator{\Ad}{Ad}   % Big Adjoint
                \DeclareMathOperator{\Lie}{Lie} % Lie algebra

        %% Universal enveloping algebra
                \ProvideDocumentCommand{\uea}{sO{}m}{
                        \mathfrak{
                                \IfBooleanTF{#1}{\hat U}{U}
                        }_{#2}\pn{#3}
                }

% Categories
        \ProvideDocumentCommand{\catname}{m}{\mathbf{#1}}
        \DeclareMathOperator{\mfld  }{\catname{Mfld}}
        \DeclareMathOperator{\mfldb }{\catname{Mfld}\pmb\boundary}
        \DeclareMathOperator{\Vect  }{\catname{Vect}}
        \DeclareMathOperator{\Mod   }{\catname{Mod}}
        \DeclareMathOperator{\Set   }{\catname{Set}}
        \DeclareMathOperator{\Ring  }{\catname{Ring}}
        \DeclareMathOperator{\Top   }{\catname{Top}}
        \DeclareMathOperator{\FinSet}{\catname{FinSet}}

% Paired Delimiters

\ProvideDocumentCommand{\curry}{m}{\ifblank{#1}{\:\cdot\:}{#1}}

%% Parenthetical constructs
\DeclarePairedDelimiter{\pn}\lparen\rparen
\DeclarePairedDelimiter{\set}\{\}
\DeclarePairedDelimiter{\bk}\lbrack\rbrack
\DeclarePairedDelimiter{\bkk}\llbracket\rrbracket
\DeclarePairedDelimiter{\gen}\langle\rangle

%% Operators

\DeclarePairedDelimiterX{\commutator}[2]{[}{]}{\curry{#1}, \curry{#2}}

\DeclarePairedDelimiterX{\liebk}[2]{[}{]}{\curry{#1}, \curry{#2}}

\DeclarePairedDelimiterXPP{\normHelper}[2]{}\lVert\rVert
        {\IfNoValueTF{#2}{}{_{#2}}}
        {\curry{#1}}
\ProvideDocumentCommand{\norm}{s O{} m o}{% Optional subscript token at the end.
        \IfBooleanTF{#1}
                {\normHelper*{#3}{#4}}
                {\normHelper[#2]{#3}{#4}}
}

\DeclarePairedDelimiterXPP{\absHelper}[2]{}\lvert\rvert
        {\IfNoValueTF{#2}{}{_{#2}}}
        {\curry{#1}}
\ProvideDocumentCommand{\abs}{s O{} m o}{% Optional subscript token at the end.
        \IfBooleanTF{#1}
                {\absHelper*{#3}{#4}}
                {\absHelper[#2]{#3}{#4}}
}

\NewDocumentCommand{\restrict}{sO{}mO{}}{%
        \IfBooleanTF{#1}{% star
                \mleft.\kern-\nulldelimiterspace
                #3
                \mright|%
        }{% no star
                #3#2|%
        }%
        _{#4}%
}

\DeclarePairedDelimiterX{\pair}[2]\langle\rangle{%
        \nonscript\,\curry{#1}\PairSymbol[\delimsize]{\vert}\curry{#2}\nonscript\,%
}

\DeclarePairedDelimiterX{\inner}[2]\langle\rangle{%
        \nonscript\,\curry{#1}\PairSymbol[\delimsize]{,}\curry{#2}\nonscript\,%
}

\DeclarePairedDelimiterXPP{\contractionHelper}[2]{}\langle\rangle
        {\IfNoValueTF{#2}{}{_{#2}}}
        {\curry{#1}}
\ProvideDocumentCommand{\contraction}{s O{} m o}{% Optional subscript token at the end.
        \IfBooleanTF{#1}
                {\contractionHelper*{#3}{#4}}
                {\contractionHelper[#2]{#3}{#4}}
}

\DeclarePairedDelimiterX{\card}[1]\lvert\rvert{\curry{#1}}
% \ProvideDocumentCommand{\card}{\#}

\DeclarePairedDelimiterX\ceil [1]{\lceil}{\rceil}{\curry{#1}}

\DeclarePairedDelimiterX\floor[1]{\lfloor}{\rfloor}{\curry{#1}}

\DeclarePairedDelimiterX\ord  [1]{\lvert}{\rvert}{\curry{#1}}
\DeclareMathOperator{\Ord}{ord}


%% Building constructs

% can be useful to refer to this outside \Set
\ProvideDocumentCommand{\innerspacing}{}{\mathchoice{\:}{\:}{\,}{\,}}
\newcommand\SetSymbol[1][]{%
  \innerspacing%
  %:%
  #1\vert%
  \allowbreak\innerspacing\mathopen{}%
}
\newcommand\PairSymbol[2][]{%
  \nonscript\, #1 #2 \allowbreak\nonscript\,\mathopen{}%
}

\DeclarePairedDelimiterX{\setbuilder}[2]\{\}{\,#1\SetSymbol[\delimsize]#2\,}

\DeclarePairedDelimiterX{\genbuilder}[2]\langle\rangle{\,#1\SetSymbol[\delimsize]#2\,}

\newcommand{\grppres}{\genbuilder}

\DeclareMathOperator{\Object}{Ob} 
\DeclareMathOperator{\HomOp}{Hom} 
\DeclarePairedDelimiterXPP{\HomHelper}[3]
        {\IfNoValueTF{#1}
                {\HomOp}
                {\HomOp_{#1}}
        }
        ()
        {}{\curry{#2}, \curry{#3}}
\ProvideDocumentCommand{\Hom}{s O{} m m}{
        \IfBooleanTF{#1}
                {\HomHelper*{#2}{#3}{#4}}
                {\HomHelper{#2}{#3}{#4}}
}

% Maps
        \ProvideDocumentCommand{\map}{mmO{\to}m}{#1\colon#2#3#4}
        \ProvideDocumentCommand{\nodomainmap}{mmm}{#1\colon#2\mapsto#3}
        \ProvideDocumentCommand{\selfmap}{mmO{\to}}{#1\colon#2#3#2}
        \ProvideDocumentCommand{\selfmapdef}{mmO{\to}m m }{%
                \begin{aligned}
                        #1\colon#2&#3#2\\
                        #4&\mapsto#5
                \end{aligned}%
        }
        \ProvideDocumentCommand{\mapdef}{mmO{\to}m m m o}{%
                \begin{aligned}
                        #1\colon#2&#3#4\\
                        #5&\mapsto#6
                        \IfNoValueTF{#7}{}{\\ #7}
                \end{aligned}%
        }

        \newcommand{\mono}{\hookrightarrow} % monomorphism
        \newcommand{\inc }{\hookrightarrow} % inclusion % DEPRECATE?
        \newcommand{\inj }{\hookrightarrow} % injection
        \newcommand{\emb }{\hookrightarrow} % embedding
        \newcommand{\sur }{\twoheadrightarrow} % epimorphism
        \newcommand{\epi }{\twoheadrightarrow} % epimorphism
        \newcommand{\toiso }{\xrightarrow{\sim}} % isomorphism

% Quantum groups / Hopf algebras

\newcommand{\unit}{\eta}           % unit
\newcommand{\one}{\mbf 1}          % another form of the unit
\newcommand{\counit}{\epsilon}     % counit

\newcommand{\mult}{m}              % multiplication
\newcommand{\comult}{\Delta}       % comultiplication
\ProvideDocumentCommand{\cmf}{mm}  % comultiplication factor
  {#2_{(#1)}}

\newcommand{\antipode}{S}          % antipode
\ProvideDocumentCommand{\bap}{m}   % bar-antipode
  {\overline{#1}}
\ProvideDocumentCommand{\baap}{m}  % bar-inverse(i.e. anti)-antipode
  {\underline{#1}}

\newcommand{\rmat}{r}         % Lie-algebraic r-matrix

\newcommand{\Rmat}{\mcl R}         % R-matrix
\newcommand{\Rmati}{\invb\Rmat}
\ProvideDocumentCommand{\rmf}{O{}} % R-matrix First factor
  {\Rmat^{(1)}_{#1}}
\ProvideDocumentCommand{\rms}{O{}} % R-matrix Second factor
  {\Rmat^{(2)}_{#1}}

\ProvideDocumentCommand{\coevf}{m} % Coevaluation First factor
  {r_{#1}}
\ProvideDocumentCommand{\coevd}{m} % Coevaluation Dual factor
  {ρ_{#1}}

\ProvideDocumentCommand{\spin}{}{C}
\ProvideDocumentCommand{\ribbon}{}{ν}
\ProvideDocumentCommand{\dfe}{}{\mfk u}
\ProvideDocumentCommand{\monodromy}{}{Q}

% Thesis-specific macros

\ProvideDocumentCommand{\ptr}{}{Z^\trace}
\ProvideDocumentCommand{\fa}{}{\mfk{a}}
\ProvideDocumentCommand{\Alg}{}{\fg}
\ProvideDocumentCommand{\CUlong}{}{\uea*{\Sl_{2+}^0}}
\ProvideDocumentCommand{\CU}{}{U}
\ProvideDocumentCommand{\nn}{}{\mathbf{n}}
\ProvideDocumentCommand{\Order}{}{\mathbb O}
\RenewDocumentCommand{\k}{}{\mbf k}
\DeclareMathOperator{\GenMap}{\msc G} 
\DeclarePairedDelimiterXPP{\Gen}[1]{\GenMap}\lparen\rparen{}{#1}
\ProvideDocumentCommand{\tangle}{}{\msc T}
\ProvideDocumentCommand{\RVT}{}{\msc{T}^{\text{rv}}}
\ProvideDocumentCommand{\A}{}{\msc{A}}
\ProvideDocumentCommand{\tanglename}{momm}{#1_{
                #3%
                \IfValueTF{#2}{\text{#2}}{,}%
                #4
}}
\newcommand{\knot}{\tanglename{K}} 
\newcommand{\link}{\tanglename{L}} 


% Her Majesty herself
\usepackage{classicthesis}

\usepackage[langlinenos=true]{minted}
\usepackage[most, minted]{tcolorbox}
\setminted{breaklines=true,linenos=true}

\usepackage{xpatch}
\makeatletter
\ifminted@draft
  \xpatchcmd\inputminted
    {\VerbatimInput{#3}}
    {\minted@langlinenoson
     \VerbatimInput{#3}%
     \minted@langlinenosoff}
    {}{\PatchFailed}
\else
  \xpatchcmd\inputminted
    {\minted@pygmentize[#3]{#2}}
    {\minted@langlinenoson
     \minted@pygmentize[#3]{#2}%
     \minted@langlinenosoff}
    {}{\PatchFailed}
\fi
\makeatother

\newtcblisting{code}{%
    listing engine=minted,
    minted language=haskell,
    listing only,
    breakable,
    enhanced,
    minted options = {
        breaklines=true,
        breakbefore=.,
        fontsize=\footnotesize,
        numbersep=1mm,
        linenos=true,
        langlinenos=true,
        firstnumber=last
    },
    overlay={%
        \begin{tcbclipinterior}
            \fill[gray!25] (frame.south west) rectangle ([xshift=5mm]frame.north west);
        \end{tcbclipinterior}
    }
}
\newtcblisting{mathematica}{%
    listing engine=minted,
    minted language=mathematica,
    listing only,
    breakable,
    enhanced,
    minted options = {
        breaklines=true,
        breakbefore=.,
        fontsize=\footnotesize,
        numbersep=1mm,
        langlinenos=true,
        firstnumber=last
    },
    overlay={%
        \begin{tcbclipinterior}
            \fill[gray!25] (frame.south west) rectangle ([xshift=5mm]frame.north west);
        \end{tcbclipinterior}
    }
}

\setminted[mathematica]{firstnumber=last}
\usemintedstyle[wolfram]{mathematica}
\newminted[spec]{haskell}{}
\newmintinline[hs]{haskell}{}
\newmintinline[mma]{mathematica}{}

% Disable fcolorbox.
\makeatletter
\AtBeginEnvironment{minted}{\dontdofcolorbox}
\def\dontdofcolorbox{\renewcommand\fcolorbox[4][]{##4}}
\xpatchcmd{\inputminted}{\minted@fvset}{\minted@fvset\dontdofcolorbox}{}{}
\makeatother

% Fine-tune hyperreferences (hyperref should be called last)
\iftoggle{paper}{%
        \hypersetup{colorlinks=false, linktocpage=false, pdfborder={0 0 0}}
}{%
        \hypersetup{colorlinks=true, linktocpage=true}
}
\hypersetup{%
  %draft, % hyperref's draft mode, for printing see below
  pdfstartpage=3, pdfstartview=FitV,%
  %pdfstartpage=3, pdfstartview=FitV,
  breaklinks=true, pageanchor=true,%
  pdfpagemode=UseNone, %
  % pdfpagemode=UseOutlines,%
  plainpages=false, bookmarksnumbered, bookmarksopen=true, bookmarksopenlevel=1,%
  hypertexnames=true, pdfhighlight=/O,%nesting=true,%frenchlinks,%
  urlcolor=CTurl, linkcolor=CTlink, citecolor=CTcitation, %pagecolor=RoyalBlue,%
  %urlcolor=Black, linkcolor=Black, citecolor=Black, %pagecolor=Black,%
  pdftitle={\myTitle},%
  pdfauthor={\textcopyright\ \myName, \myDepartment, \myUni},%
  pdfsubject={},%
  pdfkeywords={},%
  pdfcreator={pdfLaTeX},%
  pdfproducer={LaTeX with hyperref and classicthesis}%
}
\usepackage[noabbrev]{cleveref}

% Setup autoreferences (hyperref and babel)
% In the document, don't use \ref{...}
% Instead, use \autoref{...}
% That changes the reference from "3.1" to "Definition 3.1"
\makeatletter
\@ifpackageloaded{babel}%
  {%
    \addto\extrasenglish{%
      \renewcommand*{\figureautorefname}{Figure}%
      \renewcommand*{\tableautorefname}{Table}%
      \renewcommand*{\partautorefname}{Part}%
      \renewcommand*{\chapterautorefname}{Chapter}%
      \renewcommand*{\sectionautorefname}{Section}%
      \renewcommand*{\subsectionautorefname}{Section}%
      \renewcommand*{\subsubsectionautorefname}{Section}%
      \renewcommand*{\theoremautorefname}{Theorem}%
      \renewcommand*{\lemmaautorefname}{Lemma}%
      \renewcommand*{\conjectureautorefname}{Conjecture}%
      \renewcommand*{\corollaryautorefname}{Corollary}%
      \renewcommand*{\definitionautorefname}{Definition}%
      \renewcommand*{\methodautorefname}{Method}%
      \renewcommand*{\factautorefname}{Fact}%
      \renewcommand*{\problemautorefname}{Problem}%
      \renewcommand*{\questionautorefname}{Question}%
      \renewcommand*{\exampleautorefname}{Example}%
      \renewcommand*{\remarkautorefname}{Remark}%
    }%
    }{\relax}
\makeatother

% This changes the regular classicthesis chapter header to one that is a bit larger and uses red as a colour
\makeatletter
\ifthenelse{\boolean{ct@linedheaders}}%
{% lines above and below, number right
    \titleformat{\chapter}[display]%
    {\relax}{\raggedleft{\color{CTsemi}\chapterNumber\thechapter} \\ }{0pt}%
    {\titlerule\vspace*{.9\baselineskip}\raggedright\Large\color{CTtitle}\spacedallcaps}[\normalsize\vspace*{.8\baselineskip}\titlerule]%
}{% something like Bringhurst
    \titleformat{\chapter}[display]%
    {\relax}{\mbox{}\oldmarginpar{\vspace*{-3\baselineskip}\color{CTsemi}\chapterNumber\thechapter}}{0pt}%
    {\raggedright\huge\color{CTtitle}\spacedallcaps}[\normalsize\vspace*{.8\baselineskip}\titlerule]%
}
\makeatother

% Load tikz-cd after chapter title redefinition.
\usepackage{tikz-cd}
\usepackage{csquotes}

% Add any other environments you might need here
% Note: You should also add any new environments to the autorefnames above
\newtheorem{theorem}{Theorem}[chapter]
\newtheorem{lemma}[theorem]{Lemma}
\newtheorem{conjecture}[theorem]{Conjecture}
\newtheorem{corollary}[theorem]{Corollary}
\newtheorem{proposition}[theorem]{Proposition}

\theoremstyle{definition}
\newtheorem{definition}[theorem]{Definition}
\newtheorem{method}[theorem]{Method}
\newtheorem{fact}[theorem]{Fact}
\newtheorem{problem}[theorem]{Problem}
\newtheorem{question}[theorem]{Question}
\newtheorem{example}[theorem]{Example}

\theoremstyle{remark}
\newtheorem{remark}[theorem]{Remark}

% Better spacing than regular itemize
\newenvironment{itemize*}
  {\begin{itemize}[topsep=-\parskip+\jot,itemsep=-\parskip-\jot]}
  {\end{itemize}}

% Better spacing than regular enumerate, (a), (b), ...
\newenvironment{enumerate*}
  {\begin{enumerate}[label=(\alph*),topsep=-\parskip+\jot,itemsep=-\parskip-\jot]}
  {\end{enumerate}}

% Better spacing than regular enumerate, (i), (ii), ...
\newenvironment{enumerate**}
  {\begin{enumerate}[label=(\roman*),topsep=-\parskip+\jot,itemsep=-\parskip-\jot]}
  {\end{enumerate}}

% Better spacing than regular enumerate, (a'), (b'), ...
\newenvironment{enumerate***}
  {\begin{enumerate}[label=(\alph*'),topsep=-\parskip+\jot,itemsep=-\parskip-\jot]}
  {\end{enumerate}}
