% This is the file where all your settings, packages, newcommands, ... should go

\PassOptionsToPackage{T1}{fontenc}
  \usepackage{fontenc}

% Configure classicthesis for your needs here.
% (see ClassicThesis.pdf for more information)
\PassOptionsToPackage{
  drafting=false,    % print version information on the bottom of the pages
  tocaligned=false, % the left column of the toc will be aligned (no indentation)
  dottedtoc=true,  % page numbers in ToC flushed right
  eulerchapternumbers=true, % use AMS Euler for chapter font (otherwise Palatino)
  linedheaders=false,       % chaper headers will have line above and beneath
  floatperchapter=true,     % numbering per chapter for all floats (i.e., Figure 1.1)
  eulermath=false,  % use awesome Euler fonts for mathematical formulae (only with pdfLaTeX)
  beramono=true,    % toggle a nice monospaced font (w/ bold)
  palatino=true,    % deactivate standard font for loading another one, see the last section at the end of this file for suggestions
  parts=false,
  style=classicthesis
}{classicthesis}

% Personal data (insert your own data here)
\newcommand{\myTitle}{Computing the generating function of a coinvariants map\xspace}
\newcommand{\myName}{Jesse Frohlich\xspace}
\newcommand{\myDegree}{Doctor of Philosophy\xspace}
\newcommand{\myDepartment}{Graduate Department of Mathematics\xspace}
\newcommand{\myUni}{University of Toronto\xspace}
\newcommand{\myTime}{2023\xspace}

% Load all the packages you want here
% Probably you will need the following
\PassOptionsToPackage{english}{babel}
\usepackage{babel} % language support
\usepackage{enumitem} % for better itemize and enumerate
\usepackage{amsmath,amsthm,amssymb} % because math
\usepackage{thmtools} % for correct autorefs
\usepackage[onehalfspacing]{setspace} % as required by SGS
\usepackage{graphicx} % to include graphics
\usepackage{xspace} % to get the spacing after macros right
\usepackage{calc} % to allow adding lengths

% Her Majesty herself
\usepackage{classicthesis}

% Fine-tune hyperreferences (hyperref should be called last)
\hypersetup{%
  %draft, % hyperref's draft mode, for printing see below
  colorlinks=true, linktocpage=true, pdfstartpage=3, pdfstartview=FitV,%
  % uncomment the following line if you want to have black links (e.g., for printing)
  %colorlinks=false, linktocpage=false, pdfstartpage=3, pdfstartview=FitV, pdfborder={0 0 0},%
  breaklinks=true, pageanchor=true,%
  pdfpagemode=UseNone, %
  % pdfpagemode=UseOutlines,%
  plainpages=false, bookmarksnumbered, bookmarksopen=true, bookmarksopenlevel=1,%
  hypertexnames=true, pdfhighlight=/O,%nesting=true,%frenchlinks,%
  urlcolor=CTurl, linkcolor=CTlink, citecolor=CTcitation, %pagecolor=RoyalBlue,%
  %urlcolor=Black, linkcolor=Black, citecolor=Black, %pagecolor=Black,%
  pdftitle={\myTitle},%
  pdfauthor={\textcopyright\ \myName, \myDepartment, \myUni},%
  pdfsubject={},%
  pdfkeywords={},%
  pdfcreator={pdfLaTeX},%
  pdfproducer={LaTeX with hyperref and classicthesis}%
}

% Setup autoreferences (hyperref and babel)
% In the document, don't use \ref{...}
% Instead, use \autoref{...}
% That changes the reference from "3.1" to "Definition 3.1"
\makeatletter
\@ifpackageloaded{babel}%
  {%
    \addto\extrasenglish{%
      \renewcommand*{\figureautorefname}{Figure}%
      \renewcommand*{\tableautorefname}{Table}%
      \renewcommand*{\partautorefname}{Part}%
      \renewcommand*{\chapterautorefname}{Chapter}%
      \renewcommand*{\sectionautorefname}{Section}%
      \renewcommand*{\subsectionautorefname}{Section}%
      \renewcommand*{\subsubsectionautorefname}{Section}%
      \renewcommand*{\theoremautorefname}{Theorem}%
      \renewcommand*{\lemmaautorefname}{Lemma}%
      \renewcommand*{\conjectureautorefname}{Conjecture}%
      \renewcommand*{\corollaryautorefname}{Corollary}%
      \renewcommand*{\definitionautorefname}{Definition}%
      \renewcommand*{\methodautorefname}{Method}%
      \renewcommand*{\factautorefname}{Fact}%
      \renewcommand*{\problemautorefname}{Problem}%
      \renewcommand*{\questionautorefname}{Question}%
      \renewcommand*{\exampleautorefname}{Example}%
      \renewcommand*{\remarkautorefname}{Remark}%    
    }%
    }{\relax}
\makeatother

% This changes the regular classicthesis chapter header to one that is a bit larger and uses red as a colour
\makeatletter
\ifthenelse{\boolean{ct@linedheaders}}%
{% lines above and below, number right
    \titleformat{\chapter}[display]%
    {\relax}{\raggedleft{\color{CTsemi}\chapterNumber\thechapter} \\ }{0pt}%
    {\titlerule\vspace*{.9\baselineskip}\raggedright\Large\color{CTtitle}\spacedallcaps}[\normalsize\vspace*{.8\baselineskip}\titlerule]%
}{% something like Bringhurst
    \titleformat{\chapter}[display]%
    {\relax}{\mbox{}\oldmarginpar{\vspace*{-3\baselineskip}\color{CTsemi}\chapterNumber\thechapter}}{0pt}%
    {\raggedright\huge\color{CTtitle}\spacedallcaps}[\normalsize\vspace*{.8\baselineskip}\titlerule]%
}
\makeatother

% Add any other environments you might need here
% Note: You should also add any new environments to the autorefnames above
\newtheorem{theorem}{Theorem}[chapter]
\newtheorem{lemma}[theorem]{Lemma}
\newtheorem{conjecture}[theorem]{Conjecture}
\newtheorem{corollary}[theorem]{Corollary}

\theoremstyle{definition}
\newtheorem{definition}[theorem]{Definition}
\newtheorem{method}[theorem]{Method}
\newtheorem{fact}[theorem]{Fact}
\newtheorem{problem}[theorem]{Problem}
\newtheorem{question}[theorem]{Question}
\newtheorem{example}[theorem]{Example}

\theoremstyle{remark}
\newtheorem{remark}[theorem]{Remark}

% Better spacing than regular itemize
\newenvironment{itemize*}
  {\begin{itemize}[topsep=-\parskip+\jot,itemsep=-\parskip-\jot]}
  {\end{itemize}}
  
% Better spacing than regular enumerate, (a), (b), ...
\newenvironment{enumerate*}
  {\begin{enumerate}[label=(\alph*),topsep=-\parskip+\jot,itemsep=-\parskip-\jot]}
  {\end{enumerate}}
  
% Better spacing than regular enumerate, (i), (ii), ...
\newenvironment{enumerate**}
  {\begin{enumerate}[label=(\roman*),topsep=-\parskip+\jot,itemsep=-\parskip-\jot]}
  {\end{enumerate}}
  
% Better spacing than regular enumerate, (a'), (b'), ...
\newenvironment{enumerate***}
  {\begin{enumerate}[label=(\alph*'),topsep=-\parskip+\jot,itemsep=-\parskip-\jot]}
  {\end{enumerate}}
