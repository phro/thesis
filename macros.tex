% Theorem-like environments

\ProvideDocumentCommand{\defi}{m}{\uline{#1}} % the Item being DEFined

% Font changes

        \ProvideDocumentCommand{\mcl}{m}{\mathcal{#1}}
        \ProvideDocumentCommand{\mbb}{m}{\mathbb{#1}}
        \ProvideDocumentCommand{\mbf}{m}{\boldsymbol{\mathbf{#1}}}
        \ProvideDocumentCommand{\msc}{m}{\mathscr{#1}}
        \ProvideDocumentCommand{\mfk}{m}{\mathfrak{#1}}
        \ProvideDocumentCommand{\mit}{m}{\mathit{#1}}

% Symbols

        \ProvideDocumentCommand{\lt}{}{<}
        \ProvideDocumentCommand{\gt}{}{>}
        \ProvideDocumentCommand{\eq}{}{=}
        \ProvideDocumentCommand{\defeq}{}{\coloneq}
        \newcommand{\iso}{\cong} % or \cong
        \newcommand{\homeo}{\cong} % or \cong
        \newcommand{\hty}{\cong} % or \cong
        \ProvideDocumentCommand{\then}{}{\mathbin{/\mkern-6mu/}}

% Standard number sets.
        \newcommand{\N}{\mbb{N}}   % Natural
        \newcommand{\Z}{\mbb{Z}}   % Integer
        \newcommand{\Q}{\mbb{Q}}   % Rational
        \newcommand{\R}{\mbb{R}}   % Real
        \newcommand{\C}{\mbb{C}}   % Complex
        \newcommand{\F}{\mbb{F}}   % Finite
        \newcommand{\K}{\mbb{k}}   % Generic

% Common operators

        \ProvideDocumentCommand{\invb}{m}{\overline{#1}} % bar-style inverse
        \DeclareMathOperator{\dom}{dom}        % Domain
        \DeclareMathOperator{\cod}{cod}        % Codomain
        \ProvideDocumentCommand{\id}{o}{
          \IfNoValueTF {#1}
            {\operatorname{id}}
            {\operatorname{id}_{#1}}
        }
        \ProvideDocumentCommand{\Id}{m}{\id^{#1}_{#1}}

        \DeclareMathOperator{\Null}{Null}      % Nullspace
        \DeclareMathOperator{\Gal}{Gal}        % Galois group
        \DeclareMathOperator{\Spec}{Spec}      % Spectrum
        \DeclareMathOperator{\Frac}{Frac}      % Field of fractions
        \DeclareMathOperator{\Proj}{Proj}      % Proj construction
        \DeclareMathOperator*{\esssup}{ess\,sup} % essential supremum
        \DeclareMathOperator*{\supp}{supp}     % support

% Classic Groups
        \DeclareMathOperator{\GL}{GL}
        \DeclareMathOperator{\SL}{SL}
        \DeclareMathOperator{\SP}{Sp}
        \DeclareMathOperator{\SO}{SO}
        \DeclareMathOperator{\Spin}{Spin}
        \DeclareMathOperator{\U}{U}
        \DeclareMathOperator{\SU}{SU}
        \DeclareMathOperator{\Or}{O}

% Lie algebras

        %% Algebras

                \DeclareMathOperator{\Gl}{\mfk{gl}}
                \DeclareMathOperator{\Sp}{\mfk{sp}}
                \DeclareMathOperator{\Sl}{\mfk{sl}}
                \DeclareMathOperator{\So}{\mfk{so}}
                \DeclareMathOperator{\g}{\mfk{g}}

        %% Lie algebra operations

                \DeclareMathOperator{\ad}{ad}   % adjoint
                \DeclareMathOperator{\Ad}{Ad}   % Big Adjoint
                \DeclareMathOperator{\Lie}{Lie} % Lie algebra

        %% Universal enveloping algebra
                \ProvideDocumentCommand{\uea}{O{}m}{\mathfrak{U}_{#1}\pn{#2}}

% Categories
        \ProvideDocumentCommand{\catname}{m}{\mathbf{#1}}
        \DeclareMathOperator{\mfld  }{\catname{Mfld}}
        \DeclareMathOperator{\mfldb }{\catname{Mfld}\pmb\del}
        \DeclareMathOperator{\Vect  }{\catname{Vect}}
        \DeclareMathOperator{\Mod   }{\catname{Mod}}
        \DeclareMathOperator{\Set   }{\catname{Set}}
        \DeclareMathOperator{\Ring  }{\catname{Ring}}
        \DeclareMathOperator{\Top   }{\catname{Top}}
        \DeclareMathOperator{\FinSet}{\catname{FinSet}}

% Paired Delimiters

\ProvideDocumentCommand{\curry}{m}{\ifblank{#1}{\:\cdot\:}{#1}}

%% Parenthetical constructs
\DeclarePairedDelimiter{\pn}\lparen\rparen
\DeclarePairedDelimiter{\set}\{\}
\DeclarePairedDelimiter{\bk}\lbrack\rbrack
\DeclarePairedDelimiter{\bkk}\llbracket\rrbracket
\DeclarePairedDelimiter{\gen}\langle\rangle

%% Operators

\DeclarePairedDelimiterX{\commutator}[2]{[}{]}{\curry{#1}, \curry{#2}}

\DeclarePairedDelimiterX{\liebk}[2]{[}{]}{\curry{#1}, \curry{#2}}

\DeclarePairedDelimiterXPP{\normHelper}[2]{}\lVert\rVert
        {\IfNoValueTF{#2}{}{_{#2}}}
        {\curry{#1}}
\ProvideDocumentCommand{\norm}{s O{} m o}{% Optional subscript token at the end.
        \IfBooleanTF{#1}
                {\normHelper*{#3}{#4}}
                {\normHelper[#2]{#3}{#4}}
}

\DeclarePairedDelimiterXPP{\absHelper}[2]{}\lvert\rvert
        {\IfNoValueTF{#2}{}{_{#2}}}
        {\curry{#1}}
\ProvideDocumentCommand{\abs}{s O{} m o}{% Optional subscript token at the end.
        \IfBooleanTF{#1}
                {\absHelper*{#3}{#4}}
                {\absHelper[#2]{#3}{#4}}
}

\DeclarePairedDelimiterX{\pair}[2]\langle\rangle{%
        \nonscript\,\curry{#1}\PairSymbol[\delimsize]{\vert}\curry{#2}\nonscript\,%
}

\DeclarePairedDelimiterX{\inner}[2]\langle\rangle{%
        \nonscript\,\curry{#1}\PairSymbol[\delimsize]{,}\curry{#2}\nonscript\,%
}

\DeclarePairedDelimiterX{\card}[1]\lvert\rvert{\curry{#1}}
% \ProvideDocumentCommand{\card}{\#}

\DeclarePairedDelimiterX\ceil [1]{\lceil}{\rceil}{\curry{#1}}

\DeclarePairedDelimiterX\floor[1]{\lfloor}{\rfloor}{\curry{#1}}

\DeclarePairedDelimiterX\ord  [1]{\lvert}{\rvert}{\curry{#1}}
\DeclareMathOperator{\Ord}{ord}


%% Building constructs

\DeclarePairedDelimiterX{\setbuilder}[2]\{\}{\,#1\SetSymbol[\delimsize]#2\,}

\DeclarePairedDelimiterX{\genbuilder}[2]\langle\rangle{\,#1\SetSymbol[\delimsize]#2\,}

\newcommand{\grppres}{\genbuilder}

\DeclareMathOperator{\HomOp}{Hom} 
\DeclarePairedDelimiterXPP{\HomHelper}[3]
        {\IfNoValueTF{#1}
                {\HomOp}
                {\HomOp_{#1}}
        }
        ()
        {}{\curry{#2}, \curry{#3}}
\ProvideDocumentCommand{\Hom}{s O{} m m}{
        \IfBooleanTF{#1}
                {\HomHelper*{#2}{#3}{#4}}
                {\HomHelper{#2}{#3}{#4}}
}

% Maps
        \ProvideDocumentCommand{\map}{mmO{\to}m}{#1\colon#2#3#4}
        \ProvideDocumentCommand{\nodomainmap}{mmm}{#1\colon#2\mapsto#3}
        \ProvideDocumentCommand{\selfmap}{mmO{\to}}{#1\colon#2#3#2}
        \ProvideDocumentCommand{\selfmapdef}{mmO{\to}m m }{%
                \begin{aligned}
                        #1\colon#2&#3#2\\
                        #4&\mapsto#5
                \end{aligned}%
        }
        \ProvideDocumentCommand{\mapdef}{mmO{\to}m m m o}{%
                \begin{aligned}
                        #1\colon#2&#3#4\\
                        #5&\mapsto#6
                        \IfNoValueTF{#7}{}{\\ #7}
                \end{aligned}%
        }

% Quantum groups / Hopf algebras

\newcommand{\unit}{\eta}           % unit
\newcommand{\one}{\mbf 1}          % another form of the unit
\newcommand{\counit}{\epsilon}     % counit

\newcommand{\mult}{m}              % multiplication
\newcommand{\comult}{\Delta}       % comultiplication
\ProvideDocumentCommand{\cmf}{mm}  % comultiplication factor
  {#2_{(#1)}}

\newcommand{\antipode}{S}          % antipode
\ProvideDocumentCommand{\bap}{m}   % bar-antipode
  {\overline{#1}}
\ProvideDocumentCommand{\baap}{m}  % bar-inverse(i.e. anti)-antipode
  {\underline{#1}}

\newcommand{\rmat}{r}         % Lie-algebraic r-matrix

\newcommand{\Rmat}{\mcl R}         % R-matrix
\newcommand{\Rmati}{\invb\Rmat}
\ProvideDocumentCommand{\rmf}{O{}} % R-matrix First factor
  {\Rmat^{(1)}_{#1}}
\ProvideDocumentCommand{\rms}{O{}} % R-matrix Second factor
  {\Rmat^{(2)}_{#1}}

\ProvideDocumentCommand{\coevf}{m} % Coevaluation First factor
  {r_{#1}}
\ProvideDocumentCommand{\coevd}{m} % Coevaluation Dual factor
  {ρ_{#1}}

\ProvideDocumentCommand{\spin}{}{C}
\ProvideDocumentCommand{\ribbon}{}{ν}
\ProvideDocumentCommand{\dfe}{}{\mfk u}
\ProvideDocumentCommand{\monodromy}{}{Q}
